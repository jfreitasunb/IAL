%!TEX root = IAL.tex

\chapter{Sistemas Lineares}

Ao longo de todo esse cap{\'\i}tulo o conjunto $\cp{K}$ ser\'a considerado como sendo um dos conjuntos: $\rac$, $\real$ ou $\complex$.

\section{Sistemas Lineares}\label{ssub:sistemas_lineares}
Vamos trabalhar com \textbf{equa\c{c}\~oes lineares}\index{Sistemas Lineares!Equa\c{c}\~ao linear} em $n$ vari\'aveis $x_1$, $x_2$, \dots, $x_n$ que s\~ao equa\c{c}\~oes que podem ser escritas na forma
\begin{equation}\label{equacao_linear}
    a_1x_1  + a_2x_2 +  \cdots + a_qx_q  = b,
\end{equation}
onde $a_1$, $a_2$, \dots, $a_q$, $b$ s\~ao escalares no corpo $\cp{K}$, sendo que nem todos os coeficientes $a_i$ s\~ao nulos.

No caso em que $b = 0$,  a equa\c{c}\~ao \eqref{equacao_linear} tem a forma
\begin{equation}
    a_1x_1 + a_2x_2 + \cdots + a_qx_q = 0
\end{equation}
e \'e chamada de uma \textbf{equa\c{c}\~ao linear homog\^enea}\index{Sistemas Lineares!Equa\c{c}\~ao linear homog\^enea}.

\begin{exemplos}
    \begin{enumerate}[label={\arabic*})]
        \item Equa\c{c}\~oes da forma
            \begin{enumerate}
                \item $x_1 + 2x_2 - 3x_3 = 10$, em $\cp{K} = \rac$
                \item $2x_1 + 0x_2 + \sqrt{2}x_3 - 4x_4 = 0$, em $\cp{K} = \real$, que ser\'a escrita como $2x_1 + 0x_2 + \sqrt{2}x_3 - 4x_4 = 0$
                \item $ix_1 + (2 +i)x_2 = 3 + 2i$, em $\cp{K} = \complex$
            \end{enumerate}
        s\~ao exemplos de \textbf{equa\c{c}\~oes lineares}.

        \item J\'a equa\c{c}\~oes na forma
            \begin{enumerate}
                \item $x_1^2 + 2x_2 - 3x_3 = 0$ em $\cp{K} = \real$
                \item $x_1 + 2\sin(x_2) + 3x_3 + x_ 4 = 1$ em $\cp{K} = \real$
                \item $2x_1 + 3x_2 - 4x_3^{-1} = 2$ em $\cp{K} = \rac$
            \end{enumerate}
        n\~ao s\~ao equa\c{c}\~oes lineares e portanto n\~ao ser\~ao abordadas nesse curso.
    \end{enumerate}
\end{exemplos}

Considerenado $\cp{K} = \real$, queremos analisar um conjuto de equa\c{c}\~oes lineares do tipo:

\begin{align}
    \begin{cases}\label{sistema_linear_2x2}
        ax + by = b_1\\
        cx + dy = b_2
    \end{cases}
\end{align}
onde $a$, $b$, $c$, $d$, $b_1$, $b_2 \in \cp{K}$.

Queremos saber se \'e poss{\'\i}vel encontrar valores para $x$, $y$ no conjunto $\real$ de modo que essas duas equa\c{c}\~oes sejam simultaneamente verdadeiras? Caso existam, gostar{\'\i}amos de determinar todos os poss{\'\i}veis valores que $x$ e $y$ podem assumir.

Observe que cada equa\c{c}\~ao linear em \eqref{sistema_linear_2x2} representa uma reta no plano cartesiano. Assim queremos determinar se essas restas possuem alguma interse\c{c}\~ao.

Mas sabemos que duas retas no plano podem ser coincidentes, paralelas ou se interceptam em um \'unico ponto.

Assim se as retas do sistema \eqref{sistema_linear_2x2} s\~ao \textbf{coincidentes}, existem infinitos valores de $x$, $y \in \cp{K}$ que satisfazem as duas equa\c{c}\~oes simultanemente. Neste caso dizemos que o sistema \eqref{sistema_linear_2x2} \'e \textbf{poss{\'\i}vel e indeterminado}.

\definecolor{qqqqff}{rgb}{0,0,1}
\definecolor{ffqqqq}{rgb}{1,0,0}
\begin{center}
    \begin{tikzpicture}[line cap=round,line join=round,>=triangle 45,x=1cm,y=1cm]
        \begin{axis}[
            x=1cm,y=1cm,
            axis lines=middle,
            xticklabels={,,},
            yticklabels={,,},
            xlabel=$x$,
            ylabel=$y$,
            xmin=-2,
            xmax=5,
            ymin=-1.3,
            ymax=3,]
            \draw [line width=5.2pt,dash pattern=on 1pt off 1pt,color=ffqqqq,domain=-1.348257839721257:5.807994044102577] plot(\x,{(--4-1*\x)/2});
            \draw [line width=2pt,color=qqqqff,domain=-1.348257839721257:5.807994044102577] plot(\x,{(--8-2*\x)/4});
            \begin{scriptsize}
                \draw[color=ffqqqq] (1.8,2) node {$a_{11}x + a_{12}y = b_1$};
                \draw[color=qqqqff] (3,1.5) node {$a_{21}x + a_{22}y = b_2$};
            \end{scriptsize}
        \end{axis}
    \end{tikzpicture}
\end{center}

Agora se as retas em \eqref{sistema_linear_2x2} s\~ao \textbf{paralelas}, n\~ao existem valores de $x$, $y \in \cp{K}$ satisfazendo as duas equa\c{c}\~oes simultaneamente. Nesse caso dizemos que o sistema \eqref{sistema_linear_2x2} \'e \textbf{imposs{\'\i}vel}.

\definecolor{ffqqqq}{rgb}{1,0,0}
\definecolor{qqqqff}{rgb}{0,0,1}
\begin{center}
    \begin{tikzpicture}[line cap=round,line join=round,>=triangle 45,x=1cm,y=1cm]
        \begin{axis}[
            x=1cm,y=1cm,
            axis lines=middle,
            xticklabels={,,},
            yticklabels={,,},
            xlabel=$x$,
            ylabel=$y$,
            xmin=-5.3,
            xmax=6.3,
            ymin=-1,
            ymax=4.3,]
            \draw [line width=2pt,color=qqqqff,domain=-5:6] plot(\x,{(--4-1*\x)/5});
            \draw [line width=2pt,color=ffqqqq,domain=-5:6] plot(\x,{(--16-1*\x)/5});
            \begin{scriptsize}
                \draw[color=qqqqff] (-3.8,2.2) node {$a_{21}x + a_{22}y = b_2$};
                \draw[color=ffqqqq] (-3.8,3.4) node {$a_{11}x + a_{12}y = b_1$};
            \end{scriptsize}
        \end{axis}
    \end{tikzpicture}
\end{center}

Por fim, as retas em \eqref{sistema_linear_2x2} podem se \textbf{interceptar}. Nesse caso existe um \'unico conjunto de valores $x = \alpha \in \cp{K}$ e $y = \lambda \in \cp{K}$ que satisfaz as duas equa\c{c}\~oes simultaneamente. Neste caso dizemos que o sistema \eqref{sistema_linear_2x2} \'e \textbf{poss{\'\i}vel e determinado}.

\definecolor{qqqqff}{rgb}{0,0,1}
\definecolor{ffqqqq}{rgb}{1,0,0}
\begin{center}
    \begin{tikzpicture}[line cap=round,line join=round,>=triangle 45,x=1cm,y=1cm]
        \begin{axis}[
            x=1cm,y=1cm,
            axis lines=middle,
            xticklabels={,,},
            yticklabels={,,},
            xlabel=$x$,
            ylabel=$y$,
            xmin=-2,
            xmax=4.3,
            ymin=-2,
            ymax=3.5,]
            \draw [line width=2pt,color=ffqqqq,domain=-4:5] plot(\x,{(--4.94--7.92*\x)/7.58});
            \draw [line width=2pt,color=qqqqff,domain=-4:5] plot(\x,{(--4.824-8.66*\x)/8.24});
            \begin{scriptsize}
                \draw[color=ffqqqq] (2.9,2) node {$a_{11}x + a_{12}y = b_1$};
                \draw[color=qqqqff] (2.7,-0.5) node {$a_{21}x + a_{22}y = b_2$};
            \end{scriptsize}
        \end{axis}
    \end{tikzpicture}
\end{center}

Esse mesmo tipo de an\'alise pode ser aplicada num sistema da forma
\begin{align}
    \begin{cases}\label{sistema_linear_3x3}
        a_1x + a_2y + a_3z = b_1\\
        a_4x + a_5y + a_6z = b_2\\
        a_7x + a_8y + a_9z = b_3
    \end{cases}
\end{align}
onde $a_i \in \cp{K}$ para $1 \le i \le 9$ e $b_j \in \cp{K}$ para $1 \le j \le 3$.

Neste caso as equa\c{c}\~oes desse sistema representam planos no espa\c{c}o tridimensional. Nesse sistema ent\~ao queremos saber quais s\~ao as poss{\'\i}veis interse\c{c}\~oes entre estes tr\^es planos.

Temos ent\~ao as seguintes possibilidades:

\begin{enumerate}
    \item Os tr\^es planos se interceptam em um \'unico ponto. Assim existe um \'unico conjunto de valores $x = \alpha \in \cp{K}$, $y = \lambda \in \cp{K}$ e $z = \gamma \in \cp{K}$  que satisfaz as tr\^es equa\c{c}\~oes simultaneamente.  Neste caso dizemos que o sistema \eqref{sistema_linear_3x3}  \'e \textbf{poss{\'\i}vel e determinado}.
        \begin{figure}[h]
            \centering
            \includegraphics[scale=0.6]{3-planos-solucao-unica.pdf}
                \caption{Solu\c{c}\~ao \'unica}
        \end{figure}
    \item Os tr\^es planos se interceptam em infinitos pontos. Existem infinitos valores de $x$, $y$, $z \in \cp{K}$ que satisfazem as tr\^es equa\c{c}\~oes simultanemente. Neste caso dizemos que o sistema \eqref{sistema_linear_3x3} \'e \textbf{poss{\'\i}vel e indeterminado}.
        \begin{figure}[h]
            \centering
            \begin{subfigure}{.32\textwidth}
                \centering
                \includegraphics[width=\linewidth]{2-planos-coincidentes-infinitas-solucoes-intersecao-reta.pdf}
                \caption{Infinitas solu\c{c}\~oes}
            \end{subfigure}
            \begin{subfigure}{.32\textwidth}
                \centering
                \includegraphics[width=\linewidth]{3-planos-coincidentes-infinitas-solucoes.pdf}
                \caption{Infinitas solu\c{c}\~oes}
            \end{subfigure}
            \begin{subfigure}{.32\textwidth}
                \centering
                \includegraphics[width=\linewidth]{2-planos-coincidentes-terceiro-fora-infinitas-solucoes-intersecao-reta.pdf}
                \caption{Infinitas solu\c{c}\~oes}
            \end{subfigure}
        \end{figure}
    \item Os tr\^es planos n\~ao possuem uma interse\c{c}\~ao em comum. Neste caso n\~ao existem valores de $x$, $y$, $z \in \cp{K}$  satisfazendo as tr\^es equa\c{c}\~oes simultaneamente. Assim dizemos que o sistema \eqref{sistema_linear_3x3}  \'e \textbf{imposs{\'\i}vel}.
        \begin{figure}[h]
            \centering
            \begin{subfigure}{.32\textwidth}
                \centering
                \includegraphics[width=\linewidth]{3-planos-paralelos-nenhuma-solucao.pdf}
                \caption{Nenhuma solu\c{c}\~ao}
            \end{subfigure}
            \begin{subfigure}{.32\textwidth}
                \centering
                \includegraphics[width=\linewidth]{2-planos-paralelos-nenhuma-solucao.pdf}
                 \caption{Nenhuma solu\c{c}\~ao}
              \end{subfigure}
            \begin{subfigure}{.32\textwidth}
                \centering  
           \includegraphics[width=\linewidth]{3-planos-sem-nenhuma-intersecao-dois-a-dois-retas-nenhuma-solucao.pdf}
            \caption{Nenhuma solu\c{c}\~ao}
           \end{subfigure}
        \end{figure}
\end{enumerate}

Agora, de modo geral queremos trabalhar com uma quantidade qualquer de equa\c{c}\~oes. Para isso come\c{c}amos escolhendo escalares $b_1$, \dots, $b_m$  e $a_{ij}$,  $1 \le i \le m$, $1 \le j \le n$ todos em $\cp{K}$.

Queremos saber se \'e poss{\'\i}vel encontrar valores para  $x_1$, $x_2$, \dots, $x_n$  de modo que o seguinte conjunto de equa\c{c}\~oes sejam v\'alidas: 
\begin{equation}\label{sistema_linear_geral}
\begin{cases}
        a_{11}x_1 + a_{12}x_2 + \cdots + a_{1n}x_n = b_1\\
        a_{21}x_1 + a_{22}x_2 + \cdots + a_{2n}x_n = b_2\\
        \qquad \vdots\\
        a_{m1}x_1 + a_{m2}x_2 + \cdots + a_{mn}x_n = b_m
    \end{cases}
\end{equation}

O conjunto de equa\c{c}\~oes em \eqref{sistema_linear_geral} \'e chamado de um  \textbf{sistema de $m$ equa\c{c}\~oes lineares  a $n$ inc\'ognitas}\index{Sistema Linear} $x_1$, $x_2$, \dots, $x_n$ , ou simplesmente de um \textbf{sistema linear}, nas inc\'ognitas  $x_1$, $x_2$, \dots, $x_n$.

Uma solu\c{c}\~ao de um sistema linear do tipo \eqref{sistema_linear_geral}
\[
    x_1 = \alpha_1,  x_2 = \alpha_2,  \dots, x_n = \alpha_n
\]
onde $\alpha_1$, $\alpha_2$, \dots, $\alpha_n \in \cp{K}$,  pode ser escrita como
\[
    (\alpha_1, \alpha_2, \dots, \alpha_n)
\]
e \'e chamada de uma \textbf{\^enupla ordenada}\index{Sistemas linears!\^enupla ordenada} ou uma \textbf{n-upla ordenada}.


Se $b_1 = b_2 = \cdots = b_m = 0_\cp{K} \in K$,  dizemos que o sistema
\begin{equation}\label{sistemalinearhomogeneo}\index{Sistema Linear}
    \begin{cases}
        a_{11}x_1 + a_{12}x_2 + \cdots + a_{1n}x_n = 0_\cp{K}\\
        a_{21}x_1 + a_{22}x_2 + \cdots + a_{2n}x_n = 0_\cp{K}\\
        \qquad \vdots\\
        a_{m1}x_1 + a_{m2}x_2 + \cdots + a_{mn}x_n = 0_\cp{K}
    \end{cases}
\end{equation}
\'e um \textbf{sistema linear homog\^eneo}.\index{Sistema linear!homog\^eneo} 

Observe que tal sistema sempre possui solu\c{c}\~ao,  a saber, $x_1 = x_2 = \cdots = x_n = 0_\cp{K}$.

Para o caso de sistemas lineares temos o seguinte resultado:

\begin{teorema}
    Todo sistema linear do tipo \eqref{sistema_linear_geral} tem zero,  uma  ou uma infinidade de solu\c{c}\~oes.  N\~ao existem outras possibilidades.
\end{teorema}

No caso de um sistema linear da forma \eqref{sistema_linear_geral},  o processo para encontrar suas solu\c{c}\~oes ser\'a feito mediante o uso de 3 tipos de opera\c{c}\~oes.  S\~ao elas:
\begin{itemize}
\item[$e_1$)] Troca da posi\c{c}\~ao de duas equa\c{c}\~oes.
\item[$e_2$)] Multiplica\c{c}\~ao de uma equa\c{c}\~ao por um escalar n\~ao nulo.
\item[$e_3$)] Substitui\c{c}\~ao de uma equa\c{c}\~ao pela soma desta equa\c{c}\~ao com alguma outra.
\end{itemize}

Estas tr\^es opera\c{c}\~oes s\~ao chamadas de  \textbf{opera\c{c}\~oes elementares}.

\begin{exemplos}
    Utilizando somente opera\c{c}\~oes elementares encontre as solu\c{c}\~oes dos seguintes sistemas lineares:
    \begin{enumerate}[label={\arabic*})]
        \item $\begin{cases}x + y = 4\\3x + 3y = 6\end{cases}$

        \item $\begin{cases}x + y + 2z = 9\\ 2x + 4y - 3z = 1\\ 3x + 6y - 3z = 0\end{cases}$
    \end{enumerate}

    \begin{solucao}
        \begin{enumerate}[label={\arabic*})]
            \item Neste caso podemos somar a segunda equa\c{c}\~ao do sistema com ela mesma mais -3 vezes a primeira equa\c{c}\~ao:
                \[
                    \begin{cases}
                        x + y = 4\\
                        3x + 3y + (-3x) + (-3y) = 4 + (-3)\cdot 4
                    \end{cases}
                \]
            que resulta em
            \[
                \begin{cases}
                    x + y = 4\\
                    0 = -6
                \end{cases}
            \]
            e a segunda equa\c{c}\~ao desse sistema \'e uma contradi\c{c}\~ao. Logo n\~ao existem valores para $x$ e $y$ em $\real$ que satisfazem as duas equa\c{c}\~oes desse sistema simultaneamente. Portanto tal sistema \'e um \textbf{sistema imposs{\'\i}vel}.
                    \item Denote as linhas desse sistema por $L_1$, $L_2$ e $L_3$. Come\c{c}amos trocando a linha 2, $L_2$, por ela mesma somanda com -2 vezes a linha 1, $L_1$. Denotemos essa opera\c{c}\~ao por $L_2 \to L_2 - 2L_1$. Com essa opera\c{c}\~ao obtemos o sistema
                        \[
                            \begin{cases}
                                x + y + 2z = 9\\
                                2y - 7z = -17\\
                                3x + 6y - 3z = 0
                            \end{cases}
                        \]
                        Agora trocamos a linha 3 por ela mesma menos 3 vezes a linha 1 ($L_3 \to L_3 - 3L_1$). Ap\'os essa opera\c{c}\~ao o sistema anterior torna-se
                        \[
                            \begin{cases}
                                x + y + 2z = 9\\
                                2y - 7z = -17\\
                                3y - 9z = -27
                            \end{cases}
                        \]
                        Multiplicando a segunda linha por 1/2 ($L_2 \to \frac{1}{2}L_2$) obtemos:
                        \[
                            \begin{cases}
                                x + y + 2z = 9\\
                                y - (7/2)z = -17/2\\
                                3y - 9z = -27
                            \end{cases}
                        \]
                        Agora vamos multiplicar a segunda linha por -3 e somar com a terceira linha ($L_3 \to L_3 - 3L_2$).
                        \[
                            \begin{cases}
                                x + y + 2z = 9\\
                                y - (7/2)z = -17/2\\
                                3/2z = -3/2
                            \end{cases}
                        \]
                        Podemos multiplicar a terceira linha por 2/3 ($L_3 \to \dfrac{2}{3}L_3$) para obter:
                        \[
                            \begin{cases}
                                x + y + 2z = 9\\
                                y - (7/2)z = -17/2\\
                                z = -1
                            \end{cases}
                        \]
                        Troque a primeira linha por ela mesma menos a segunda linha ($L_1 \to L_1 - L_2$):
                        \[
                            \begin{cases}
                                x + (11/2)z = 35/2\\
                                y - (7/2)z = -17/2\\
                                z = -1
                            \end{cases}
                        \]
                        Agora trocamos a segunda linha por ela mesma mais 7/2 vezes a terceira linha ($L_2 \to L_2 + (7/2)L_3$):
                        \[
                            \begin{cases}
                                x + (11/2)z = 35/2\\
                                y  = -12\\
                                z = -1
                            \end{cases}
                        \]
                        Finalmente, trocando a primeira linha por ela mesma menos (11/2) vezes a terceira linha ($L_1 \to L_1 - (11/2)L_3$) obtemos
                        \[
                            \begin{cases}
                                x = 23\\
                                y = -12\\
                                z = -1
                            \end{cases}
                        \]
                        Assim vemos que o sistema tem uma \'unica solu\c{c}\~ao que \'e $x = 23$, $y = -12$ e $z = -1$.
        \end{enumerate}
    \end{solucao}
\end{exemplos}

Observe nos exemplos anteriores que a aplica\c{c}\~ao das opera\c{c}\~oes elementares afeta somente os coeficientes das vari\'aveis e os termos independentes. As vari\'aveis do sistema permanecem inalteradas durante todo o processo. Assim n\~ao precisamos ficar repetindo as vari\'aveis o tempo todo. Podemos tratar somente com seus coeficientes e os termos independetes do sistema.

A partir dessa observa\c{c}\~ao, podemos associar ao sistema \eqref{sistema_linear_geral} uma matriz contendo os coeficientes de cada equa\c{c}\~ao junto com seus respectivos termos independentes. Obtemos assim a matriz
\[
\begin{bmatrix}
        a_{11} & a_{12} & \cdots & a_{1n} & b_1\\
a_{21} & a_{22} & \cdots & a_{2n} & b_2\\
\vdots & \vdots & \vdots & \vdots & \vdots\\
a_{m1} & a_{m2} & \cdots & a_{mn} & b_m\\
    \end{bmatrix}
\]

que \'e chamada de \textbf{matriz ampliada do sistema}  ou \textbf{matriz aumentada do sistema}.

Para destacar que a \'ultima coluna dessa matriz cont\'em os termos independentes do sistema podemos tamb\'em escrever:
\[
    \begin{amatrix}{4}
        a_{11} & a_{12} & \cdots & a_{1n} & b_1\\
    a_{21} & a_{22} & \cdots & a_{2n} & b_2\\
    \vdots & \vdots & \vdots & \vdots & \vdots\\
    a_{m1} & a_{m2} & \cdots & a_{mn} & b_m\\
    \end{amatrix}
\]

\begin{exemplos}
    Escreva a matriz ampliada dos seguintes sistemas lineares:
    \begin{enumerate}[label={\arabic*})]
        \item \[\begin{cases} x_1 + x_2 - 3x_3 + x_4 = -1\\ 2x_1 - 3x_2 + x_4 = 0\\ 3x_1 + 2x_3 = -3\\ x_1 + x_4 = -2\end{cases}\]
        \item \[\begin{cases} x_1 + x_2 - 3x_3 + x_4 = -2\\ 2x_1 - 3x_2 + x_4 = 0\end{cases}\]
        \item \[\begin{cases} 2x_1 - 5x_2 + 4x_3 - 7x_4 + 8x_5 = 0\\ 3x_1 - 7x_2 + 2x_4 - 9x_5 = 0\\ 5x_3 + 8x_4 = 0\\ x_3 - x_4 = 0\end{cases}\]
    \end{enumerate}
    \begin{solucao}
        \begin{enumerate}[label={\arabic*})]
            \item Neste caso a matriz ampliada \'e:
                \[
                    \begin{amatrix}{4}
                        1 & 1 & -3 & 1 & -1\\
                        2 & -3 & 0 & 1 & 0\\
                        3 & 0 & 2 & 0 & -3\\
                        1 & 0 & 0 & 1 & -2
                    \end{amatrix}
                \]
            \item Neste caso a matriz ampliada \'e:
                \[
                    \begin{amatrix}{4}
                        1 & 1 & -3 & 1 & -2\\
                        2 & -3 & 0 & 1 & 0\\
                    \end{amatrix}
                \]
            \item Neste caso a matriz ampliada \'e:
                \[
                    \begin{amatrix}{5}
                        2 & -5 & 4 & -7 & 8 & 0\\
                        3 & -7 & 0 & 2 & -9 & 0\\
                        0 & 0 & 5 & 8 & 0 & 0\\
                        0 & 0 & 1 & -1 & 0 & 0
                    \end{amatrix}
                \]
                Como a \'ultima coluna dessa matriz \'e composta somente de zeros podemos omit{\'\i}-la e escrever:
                \[
                    \begin{bmatrix}
                        2 & -5 & 4 & -7 & 8\\
                        3 & -7 & 0 & 2 & -9\\
                        0 & 0 & 5 & 8 & 0\\
                        0 & 0 & 1 & -1 & 0 
                    \end{bmatrix}
                \]

        \end{enumerate}
    \end{solucao}
\end{exemplos}
Na forma matricial as opera\c{c}\~oes elementares s\~ao descritas como:

\vspace{.3cm}

\begin{itemize}
    \item[$e_1$)] Trocar a $i$-\'esima linha de $A$ pela $j$-\'esima linha de $A$: $L_i \leftrightarrow L_j$;

    \item[$e_2$)] Multiplica\c{c}\~ao da $i$-\'esima linha de $A$ por um escalar $\alpha \in \cp{K}$ n\~ao nulo: $L_i \rightarrow \alpha L_i$;

   \item[$e_3$)] Substitui\c{c}\~ao da $i$-\'esima linha de $A$ pela $i$-\'esima linha mais $\alpha$ vezes a $j$-\'esima linha: $L_i \rightarrow L_i + \alpha L_j$.
\end{itemize}

O que iremos fazer para resolver um sistema linear do tipo \eqref{sistema_linear_geral} \'e aplicar opera\c{c}\~oes opera\c{c}\~oes elementares na linhas da matriz ampliada at\'e que ela esteja num formato especial. Esse formato que buscamos \'e o seguinte:

\begin{definicao}\label{linhareduzida}
    Uma matriz $A$ $m \times n$ \'e  dita estar na \textbf{forma escalonada reduzida por linhas} se:
    \begin{enumerate}[label={\roman*})]
        \item O primeiro elemento n\~ao nulo em cada linha n\~ao nula de $A$ \'e $1$. Dizemos que esse n\'umero 1 \'e um \textbf{piv\^o}.

        \item Toda linha de $A$ cujos elementos s\~ao todos nulos ocorre abaixo de todas as linhas que possuem um elemento n\~ao-nulo. 

        \item Se as linhas 1, 2, \dots, $r$ s\~ao as linhas n\~ao-nulas de $A$ e se o \textbf{piv\^o} da linha $i$ ocorre na coluna $k_i$, $i = 1$, \dots, $r$, ent\~ao $k_1 < k_2 < \cdots < k_r$.

        \item Cada coluna de $A$ que cont\'em um \textbf{piv\^o} tem todos os seus outros elementos nulos.
    \end{enumerate}
\end{definicao}

\begin{observacao}
    Uma matriz que satisfaz as tr\^es primeiras propriedades da defini\c{c}\~ao anterior \'e dita estar na \textbf{forma escalonada por linhas}, ou simplesmente, em \textbf{forma escalonada}.
\end{observacao}

\begin{exemplos}
    \begin{enumerate}[label={\arabic*})]
        \item As seguintes matrizes est\~ao na \textbf{forma escalonada reduzida por linhas}:
            \begin{align*}
                \begin{bmatrix}
                    1 & 0 & 0 & 4\\
                    0 & 1 & 0 & 7\\
                    0 & 0 & 1 & 1
                \end{bmatrix};
                \begin{bmatrix}
                    0 & 1 & i & 3 & 3\\
                    0 & 0 & 0 & 1 & -2 + 4i\\
                    0 & 0 & 0 & 0 & 0
                \end{bmatrix};
                \begin{bmatrix}
                    0 & 0\\
                    0 & 0
                \end{bmatrix};
                \begin{bmatrix}
                    1 & 0 & 0\\
                    0 & 1 & 0\\
                    0 & 0 & 1
                \end{bmatrix}.
            \end{align*}

        \item J\'a as seguintes matrizes est\~ao na \textbf{forma escalonada}:
            \begin{align*}
                \begin{bmatrix}
                    1 & 4 & -3 & 7\\
                    0 & 1 & 6 & 2\\
                    0 & 0 & 1 & 5
                \end{bmatrix};
                \begin{bmatrix}
                    1 & 1 & 0\\
                    0 & 1 & 0\\
                    0 & 0 & 0
                \end{bmatrix};
                \begin{bmatrix}
                    0 & 1 & 2 & 3 & 0\\
                    0 & 0 & 0 & 1 & 1\\
                    0 & 0 & 0 & 0 & 1
                \end{bmatrix}.
            \end{align*}
        \item Usando a matriz ampliada e as opera\c{c}\~oes elementares sobre as linhas da matriz resolva o seguinte sistema:
            \[
                \begin{cases}
                    2x_1 - 4x_2 - x_3 = 1\\
                    x_1 - 3x_2 + x_3 = 1\\
                    3x_1 - 5x_2 - 3x_3 = 1
                \end{cases}
            \]
            \begin{solucao}
                Come\c{c}amos montando a matriz ampliada do sistema:
                \[
                    \begin{amatrix}{3}
                    2 & -4 & -1 & 1\\
                1 & -3 & 1 & 1\\
                3 & -5 & -3 & 1
                    \end{amatrix}.
                \]
                Vamos aplicar as opera\c{c}\~oes elementares nessa matriz:
                \begin{align*}
                    &\begin{bmatrix}
                        2 & -4 & -1 & 1\\
                1 & -3 & 1 & 1\\
                3 & -5 & -3 & 1
                    \end{bmatrix}
                    \begin{array}{l}
                        L_1 \leftrightarrow L_2\\\phantom{x} \\\phantom{x} 
                    \end{array} \sim
                    \begin{bmatrix}
                1 & -3 & 1 & 1\\
                        2 & -4 & -1 & 1\\
                3 & -5 & -3 & 1
                    \end{bmatrix}
                    \begin{array}{l}
                        \phantom{x}\\  L_2 \to L_2 - 2L_1\\\phantom{x}
                    \end{array} \sim\\
                    &\begin{bmatrix}
                1 & -3 & 1 & 1\\
                        0 & 2 & -3 & -1\\
                3 & -5 & -3 & 1
                    \end{bmatrix}
                    \begin{array}{l}
                        \phantom{x}\\ \phantom{x}\\ L_3 \to L_3 - 3L_1
                    \end{array} \sim
                    \begin{bmatrix}
                1 & -3 & 1 & 1\\
                        0 & 2 & -3 & -1\\
                0 & 4 & -6 & -2
                    \end{bmatrix}
                    \begin{array}{l}
                        \phantom{x}\\ \phantom{x}\\ L_3 \to L_3 - 2L_1
                    \end{array} \sim\\
                    &\begin{bmatrix}
                1 & -3 & 1 & 1\\
                        0 & 2 & -3 & -1\\
                0 & 0 & 0 & 0
                    \end{bmatrix}
                    \begin{array}{l}
                        \phantom{x}\\ L_2 \to \dfrac{1}{2}L_2\\\phantom{x}
                    \end{array} \sim
                    \begin{bmatrix}
                1 & -3 & 1 & 1\\
                        0 & 1 & -3/2 & -1/2\\
                0 & 0 & 0 & 0
                    \end{bmatrix}
                    \begin{array}{l}
                        L_1 \to L_1 + 3L_2\\\phantom{x}\\\phantom{x}
                    \end{array} \sim\\
                    &\begin{bmatrix}
                1 & 0 & -7/2 & -1/2\\
                        0 & 1 & -3/2 & -1/2\\
                0 & 0 & 0 & 0
                    \end{bmatrix}
                \end{align*}
                Essa \'ultima matriz est\'a na forma escalonada reduzida por linhas. Assim obtemos o sistema
                \[
                    \begin{cases}
                        x_1 - (7/2)x_3 = -1/2\\
                        x_2 - (3/2)x_3 = -1/2
                    \end{cases}.
                \]

                Desse sistema obtemos
                \[
                    x_2 = -\dfrac{1}{2} + \dfrac{3}{2}x_3
                \]
                e substituindo essa equa\c{c}\~ao na primeira encontramos
                \[
                    x_1 = -\dfrac{1}{2} + \dfrac{7}{2}x_3.
                \]
                Portanto fazendo $x_3 = t \in \real$ a solu\c{c}\~ao desse sistema pode ser escrita como
                \[
                    x_1 = -\dfrac{1}{2} + \dfrac{7}{2}t, x_2 = -\dfrac{1}{2} + \dfrac{3}{2}t, x_3 = t, t \in \real.
                \]
                Logo o sistema \'e poss{\'\i}vel e indeterminado.
            \end{solucao}
    \end{enumerate}
\end{exemplos}
    
\begin{definicao}
    Se $A$ e $B$ s\~ao matrizes $m \times n$, dizemos que $B$ \'e \textbf{linha-equivalente}\index{Matrizes!linha-equivalente} a $A$, se $B$ for obtida de $A$ atrav\'es de uma quantidade finita de opera\c{c}\~oes elementares sobre as linhas de $A$.
\end{definicao}

\begin{notacao}
    $A \rightarrow B$ ou $A \sim B$.
\end{notacao}

\begin{exemplos}
    \begin{enumerate}[label={\arabic*})]
        \item As matrizes
            \[
                A = \begin{pmatrix}
                        1 & 2 & -1 & 0\\
                        0 & 1 & 3 & 4\\
                        2 & 0 & 1 & 3
                    \end{pmatrix}
                    \mbox{ e }
                B = \begin{pmatrix}
                        0 & 1 & 3 & 4\\
                        1 & 2 & -1 & 0\\
                        2 & 0 & 1 & 3
                    \end{pmatrix}
            \]
            s\~ao matrizes linha-equivalentes pois podemos obter a matriz $B$ de $A$ trocando a primeira linha de $A$ com a segunda linha. Assim $B \sim A$.
            
            Observe que tamb\'em podemos obter a matriz $A$ a partir de $B$ simplesmente trocando a primeira linha de $B$ com a segunda linha. Logo $A \sim B$.
        \item As matrizes
            \[
                C = \begin{pmatrix}
                        1 & 0 & 3 & 0\\
                        2 & 1 & 0 & 1\\
                        -1 & 3 & 2 & 1
                    \end{pmatrix}
                    \mbox{ e }
                D = \begin{pmatrix}
                        1 & 0 & 3 & 0\\
                        4 & 1 & 6 & 1\\
                        0 & 3 & 5 & 1
                    \end{pmatrix}
            \]
            tamb\'em s\~ao linha-equivalentes. De fato, partindo de $C$ podemos efetuar as seguintes opera\c{c}\~oes elementares:
            \begin{align*}
                C &= \begin{pmatrix}
                    1 & 0 & 3 & 0\\
                    2 & 1 & 0 & 1\\
                    -1 & 3 & 2 & 1
                \end{pmatrix}
                \begin{array}{l}
                    \phantom{x}\\ L_2 \to L_2 + 2L_1\\\phantom{x}
                \end{array} \sim
                \begin{pmatrix}
                    1 & 0 & 3 & 0\\
                    4 & 1 & 6 & 1\\
                    -1 & 3 & 2 & 1
                \end{pmatrix}
                \begin{array}{l}
                    \phantom{x}\\ \phantom{x}\\ L_3 \to L_2 + L_1
                \end{array}\\ \sim
                  &\begin{pmatrix}
                    1 & 0 & 3 & 0\\
                    4 & 1 & 6 & 1\\
                    0 & 3 & 5 & 1
                \end{pmatrix} = D
            \end{align*}
            Assim $D \sim C$.

            Poder{\'\i}amos fazer o contr\'ario, come\c{c}ar de $D$ e obter a matriz $C$:
            \begin{align*}
                D &= \begin{pmatrix}
                    1 & 0 & 3 & 0\\
                    4 & 1 & 6 & 1\\
                    0 & 3 & 5 & 1
                \end{pmatrix}
                \begin{array}{l}
                    \phantom{x}\\ L_2 \to L_2 - 2L_1\\\phantom{x}
                \end{array} \sim
                \begin{pmatrix}
                    1 & 0 & 3 & 0\\
                    2 & 1 & 0 & 1\\
                    0 & 3 & 5 & 1
                \end{pmatrix}
                \begin{array}{l}
                    \phantom{x}\\ \phantom{x}\\ L_3 \to L_3 - L_1
                \end{array}\\ \sim
                  &\begin{pmatrix}
                    1 & 0 & 3 & 0\\
                    2 & 1 & 0 & 1\\
                    -1 & 3 & 2 & 1
                \end{pmatrix} = C
            \end{align*}

            Assim $C \sim D$.
    \end{enumerate}
\end{exemplos}
\begin{teorema}
    Duas matrizes $A$ e $B$ s\~ao equivalentes por linha se, e somente, se elas puderem ser reduzidas \`a mesma forma escalonada por linhas.
\end{teorema}

\begin{teorema}
    Se as matrizes ampliadas de dois sistemas lineares s\~ao equivalentes por linhas, ent\~ao os dois sistemas possuem as mesmas solu\c{c}\~oes.
\end{teorema}

O \textbf{m\'etodo de elimina\c{c}\~ao de Gauss}\index{Sistemas lineares!m\'etodo de elimina\c{c}\~ao de Gauss} ou \textbf{m\'etodo de elemina\c{c}\~ao gaussiana}\index{Sistemas lineares!elimina\c{c}\~ao gaussiana} consiste em substituir um dado sistema de equa\c{c}\~oes lineares  por outro \textbf{equivalente}, que seja mais simples de ser solucionado e que tenha a mesma solu\c{c}\~ao do sistema original.

\begin{definicao}[M\'etodo de elimina\c{c}ao de Gauss]
    Seja $A$ a matriz ampliada de um sistema linear. O m\'etodo de \textbf{elimina\c{c}\~ao de Gauss} consiste em:
    \begin{enumerate}[label={\roman*})]
        \item Escreva a matriz ampliada do sistema de equa\c{c}\~oes lineares.

        \item Use opera\c{c}\~oes elementares nas linhas de $A$ para reduzir a matriz ampliada \`a \textbf{forma escalonada por linhas}.
        
        \item Quando a matriz ampliada estiver na forma escalonada, usando substitui\c{c}\~ao de tr\'as para a frente, resolva o sistema equivalente que corresponde a matriz escalonada reduzida por linhas.
    \end{enumerate}
\end{definicao}

\begin{observacao}
    No momento de aplicar o passo 2 do m\'etodo de elimina\c{c}\~ao de Guass, temos v\'arias escolhas que podemos fazer. Algumas dicas para a escolha da opera\c{c}\~ao s\~ao as seguintes:
    \begin{enumerate}[label=({\alph*})]
        \item Localize a coluna mais \`a esquerda que n\~ao \'e toda formada por zeros.
        
        \item Crie um \textbf{piv\^o} no topo desta coluna.
         
        \item Use o \textbf{piv\^o} para criar zeros abaixo dele.

        \item Fa\c{c}a a linha contendo este \textbf{piv\^o} ir para a parte de cima e volte ao passo (a) para repetir o procedimento com o restante da submatriz. Pare quando toda a matriz estiver na forma escalonada por linhas.
    \end{enumerate}
\end{observacao}

\begin{exemplos}
    \begin{enumerate}[label={\arabic*})]
        \item Encontre a solu\c{c}\~ao do seguinte sistema em $\rac$:
        \[
            \begin{cases}
                6x_3 + 19x_5 + 11x_6 = -27\\
                3x_1 + 12x_2 + 9x_3 - 6x_4 + 26x_5 + 31x_6 = -63\\
                x_1 + 4x_2 + 3x_3 - 2x_4 + 10x_5 + 9x_6 = -17\\
                -x_1 - 4x_2 - 4x_3 + 2x_4 - 13x_5 - 11x_6 = 22
            \end{cases}
        \]
        \item Considere o seguinte sistema em $\real$:
        \[
            \begin{cases}
                2x_1 + 4x_5 = 16\\
                5x_1 - 2x_2 = 4\\
                10x_1 - 4x_2 = 3
            \end{cases}
        \]
        \item Estude o seguinte sistema linear:
        \[
            \begin{cases}
                x_1 - x_2 + kx_3 = 1\\
                2x_1 + x_2 + x_3 = 0\\
                x_1 + 2x_2 + (1 - k)x_3 = k
            \end{cases}
        \]
        onde $k \in \real$. Isto \'e, decida para quais valores de $k \in \real$ esse sistema admite solu\c{c}\~ao \'unica, infinitas solu\c{c}\~oes e nenhuma solu\c{c}\~ao.
    \end{enumerate}
    \begin{solucao}
        \begin{enumerate}
            \item Primeiro montamos a matriz ampliada do sistema:
            \[
                \begin{amatrix}{6}
                    0 & 0 & 6 & 0 & 19 & 11 & -27\\
                    3 & 12 & 9 & -6 & 26 & 31 & -63\\
                    1 & 4 & 3 & -2 & 10 & 9 & -17\\
                    -1 & -4 & -4 & 2 & -13 & -11 & 22
                \end{amatrix}
            \]
            O primeiro passo e colocar um piv\^o na primeira linha. Para isso podemos trocar as linhas 1 e 3 de posi\c{c}\~ao: $L_1 \leftrightarrow L_3$:
            \[
               \begin{amatrix}{6}
                    1 & 4 & 3 & -2 & 10 & 9 & -17\\
                    3 & 12 & 9 & -6 & 26 & 31 & -63\\
                    0 & 0 & 6 & 0 & 19 & 11 & -27\\
                    -1 & -4 & -4 & 2 & -13 & -11 & 22
               \end{amatrix}
            \]
            Agora vamos zerar todos os elementos na primeira coluna abaixo do 1:
            \begin{align*}
                &\begin{amatrix}{6}
                    1 & 4 & 3 & -2 & 10 & 9 & -17\\
                    3 & 12 & 9 & -6 & 26 & 31 & -63\\
                    0 & 0 & 6 & 0 & 19 & 11 & -27\\
                    -1 & -4 & -4 & 2 & -13 & -11 & 22
                \end{amatrix}
                \begin{array}{l}
                    \phantom{x}\\L_2 \to L_2 - 3L_1\\\phantom{x}\\L_4 \to L_4 + L_1
                \end{array}\sim\\
                &\begin{amatrix}{6}
                    1 & 4 & 3 & -2 & 10 & 9 & -17\\
                    0 & 0 & 0 & 0 & -4 & 4 & -12\\
                    0 & 0 & 6 & 0 & 19 & 11 & -27\\
                    0 & 0 & -1 & 0 & -3 & -2 & 5
                \end{amatrix}
            \end{align*}
            Na quarta linha temos um -1 na terceira coluna. Por isso, podemos trocar a segunda e a quarta linhas de posi\c{c}\~ao: $L_2 \leftrightarrow L_4$
            \[
                \begin{amatrix}{6}
                    1 & 4 & 3 & -2 & 10 & 9 & -17\\
                    0 & 0 & -1 & 0 & -3 & -2 & 5\\
                    0 & 0 & 6 & 0 & 19 & 11 & -27\\
                    0 & 0 & 0 & 0 & -4 & 4 & -12
                \end{amatrix}
            \]
            Multiplicamos a segunda linha por -1 para obter um piv\^o na terceira coluna:
            \[
                \begin{amatrix}{6}
                    1 & 4 & 3 & -2 & 10 & 9 & -17\\
                    0 & 0 & 1 & 0 & 3 & 2 & -5\\
                    0 & 0 & 6 & 0 & 19 & 11 & -27\\
                    0 & 0 & 0 & 0 & -4 & 4 & -12
                \end{amatrix}
            \]
            Agora vamos zerar todos os coeficientes que est\~ao na terceira coluna e abaixo do piv\^o:
            \begin{align*}
                &\begin{amatrix}{6}
                    1 & 4 & 3 & -2 & 10 & 9 & -17\\
                    0 & 0 & 1 & 0 & 3 & 2 & -5\\
                    0 & 0 & 6 & 0 & 19 & 11 & -27\\
                    0 & 0 & 0 & 0 & -4 & 4 & -12
                \end{amatrix}
                \begin{array}{l}
                    \phantom{x}\\ \phantom{x}\\ L_3 \to L_3 - 6L_2\\ \phantom{x} 
                \end{array}\sim\\
                &\begin{amatrix}{6}
                    1 & 4 & 3 & -2 & 10 & 9 & -17\\
                    0 & 0 & 1 & 0 & 3 & 2 & -5\\
                    0 & 0 & 0 & 0 & 1 & -1 & 3\\
                    0 & 0 & 0 & 0 & -4 & 4 & -12
                \end{amatrix}
            \end{align*}
            Como na terceira linha j\'a temos um piv\^o na quinta coluna, vamos somente zerar os coeficientes abaixo desse piv\^o.
            \begin{align*}
                &\begin{amatrix}{6}
                    1 & 4 & 3 & -2 & 10 & 9 & -17\\
                    0 & 0 & 1 & 0 & 3 & 2 & -5\\
                    0 & 0 & 0 & 0 & 1 & -1 & 3\\
                    0 & 0 & 0 & 0 & -4 & 4 & -12
                \end{amatrix}
                \begin{array}{l}
                    \phantom{x}\\ \phantom{x}\\ \phantom{x}\\ L_4 \to L_4 - 4L_3 
                \end{array}\sim\\
                &\begin{amatrix}{6}
                    1 & 4 & 3 & -2 & 10 & 9 & -17\\
                    0 & 0 & 1 & 0 & 3 & 2 & -5\\
                    0 & 0 & 0 & 0 & 1 & -1 & 3\\
                    0 & 0 & 0 & 0 & 0 & 0 & 0
                \end{amatrix}
            \end{align*}
            Como essa matriz est\'a na forma escalonada temos o seguinte sistema:
            \[
                \begin{cases}
                    x_1 + 4x_2 + 3x_3 - 2x_4 + 10x_5 + 9x_6 = -17\\
                    x_3 + 3x_5 + 2x_6 = -5\\
                    x_5 - x_6 = 3
                \end{cases}
            \]
            As vari\'aveis $x_2$, $x_4$ e $x_6$ s\~ao chamadas de \textbf{vari\'aveis livres} do sistema linear pois n\~ao est\~ao associadas a nenhum \textbf{piv\^o} na matriz escalonada desse sistema.
            Da terceira equa\c{c}\~ao temos
            \[
                x_5 = 3 + x_6.
            \]
            Substituindo esse valor na segunda equa\c{c}\~ao:
            \begin{align*}
                x_3 + 3(3 + x_6) + 2x_6 = -5\\
                x_3 + 9 + 3x_6 + 2x_6 = -5\\
                x_3 = -14 - 5x_6
            \end{align*}
            Finalmente, substituindo os valores encontrados para $x_3$ e $x_5$ na primeira equa\c{c}\~ao:
            \begin{align*}
                x_1 + 4x_2 + 3(-14 - 5x_6) - 2x_4 + 10(3 + x_6) + 9x_6 = -17\\
                x_1 + 4x_2 - 42 - 15x_6 - 2x_4 + 30 + 10x_6 + 9x_6 = -17\\
                x_1 = -5 - 4x_2 + 2x_4 - 4x_6
            \end{align*}

            Fazendo $x_2 = s$, $x_4 = t$ e $x_6 = z$ com $s$, $t$ e $z \in \rac$ podemos escrever
            \begin{align*}
                x_1 &= -5 - 4s + 2t - 4z\\
                x_2 &= s\\
                x_3 &= -14 - 5z\\
                x_4 &= t\\
                x_5 &= 3 + z\\
                x_6 &= z
            \end{align*}
            Nesse caso temos um sistema \textbf{poss{\'\i}vel e indeterminado}. O sistema admite infinitas solu\c{c}\~oes. Podemos escrever as solu\c{c}\~oes desse sistema como o conjunto
            \[
                S = \{(-5 - 4s + 2t - 4z, s, -14 - 5z, t, 3 + z, z) \mid s, t, z \in \rac\}
            \]
            que \'e chamado de \textbf{conjunto-solu\c{c}\~ao} do sistema linear.

            \item A matriz ampliada do sistema \'e
                \[
                    \begin{amatrix}{2}
                        2 & 4 & 16\\
                        5 & 2 & 4\\
                        10 & 4 & 3
                    \end{amatrix}
                \]
                Primeiro multiplicamos a linha 1 por 1/2 para termos um piv\^o na primeira coluna:
                \begin{align*}
                    \begin{amatrix}{2}
                        2 & 4 & 16\\
                        5 & 2 & 4\\
                        10 & 4 & 3
                    \end{amatrix}
                    \begin{array}{l}
                        L_1 \to \dfrac{1}{2}L_1\\\phantom{x}\\\phantom{x}
                    \end{array}\sim
                    \begin{amatrix}{2}
                        1 & 2 & 8\\
                        5 & 2 & 4\\
                        10 & 4 & 3
                    \end{amatrix}
                \end{align*}
                Agora vamos zerar os demais coeficientes na primeira coluna:
                \begin{align*}
                    \begin{amatrix}{2}
                        1 & 2 & 8\\
                        5 & 2 & 4\\
                        10 & 4 & 3
                    \end{amatrix}
                    \begin{array}{l}
                        \phantom{x}\\L_2 \to L_2 - 5L_1\\L_3 \to L_3 - 10L_1
                    \end{array}\sim
                    \begin{amatrix}{2}
                        1 & 2 & 8\\
                        0 & -8 & -36\\
                        0 & -16 & -77
                    \end{amatrix}
                \end{align*}
                Agora vamos criar um piv\^o na segunda linha, segunda coluna. Para isso fazemos:
                \begin{align*}
                    \begin{amatrix}{2}
                        1 & 2 & 8\\
                        0 & -8 & -36\\
                        0 & -16 & -77
                    \end{amatrix}
                    \begin{array}{l}
                        \phantom{x}\\L_2 \to -\dfrac{1}{8}\\\phantom{x}
                    \end{array}\sim
                    \begin{amatrix}{2}
                        1 & 2 & 8\\
                        0 & 1 & 9/4\\
                        0 & -16 & -77
                    \end{amatrix}
                \end{align*}
                Por \'ultimo podemos zerar o coeficiente abaixo do piv\^o da segunda linha:
                \begin{align*}
                    \begin{amatrix}{2}
                        1 & 2 & 8\\
                        0 & 1 & 9/4\\
                        0 & -16 & -77
                    \end{amatrix}
                    \begin{array}{l}
                        \phantom{x}\\\phantom{x}\\L_3 \to L_3 + 16L_2!
                    \end{array}\sim
                    \begin{amatrix}{2}
                        1 & 2 & 8\\
                        0 & 1 & 9/4\\
                        0 & 0 & -5
                    \end{amatrix}
                \end{align*}
                Dessa \'ultima matriz, obtemos o sistema
                \[
                    \begin{cases}
                        x_1 - 2x_2 = 8\\
                        x_2 = 3\\
                        0 = -5
                    \end{cases}
                \]
                A \'ultima equa\c{c}\~ao desse sistema n\~ao admite solu\c{c}\~ao. Logo tal sistema \'e imposs{\'\i}vel.
                \item A matriz ampliada desse sistema \'e:
                \[
                    \begin{amatrix}{3}
                        1 & -1 & k & 1\\
                        2 & 1 & 1 & 0\\
                        1 & 2 & 1 - k & k\\
                    \end{amatrix}    
                \]
                Apliquemos a elimina\c{c}\~ao gaussiana
                \begin{align*}
                    &\begin{amatrix}{3}
                        1 & -1 & k & 1\\
                        2 & 1 & 1 & 0\\
                        1 & 2 & 1 - k & k\\
                    \end{amatrix}
                    \begin{array}{l}
                        \phantom{x}\\L_2 \to L_2 - 2L_1\\L_3 \to L_3 - L_1
                    \end{array}\sim
                    \begin{amatrix}{3}
                        1 & -1 & k & 1\\
                        0 & 3 & 1 - 2k & -2\\
                        0 & 3 & 1 - 2k & k - 1\\
                    \end{amatrix}
                    \begin{array}{l}
                        \phantom{x}\\\phantom{x}\\L_3 \to L_3 - L_2
                    \end{array}\\ &\sim
                    \begin{amatrix}{3}
                        1 & -1 & k & 1\\
                        0 & 3 & 1 - 2k & -2\\
                        0 & 0 & 0 & k + 1\\
                    \end{amatrix}
                    \begin{array}{l}
                        \phantom{x}\\L_2 \to \dfrac{1}{3}L_2\\\phantom{x}
                    \end{array}\sim
                    \begin{amatrix}{3}
                        1 & -1 & k & 1\\
                        0 & 1 & (1 - 2k)/3 & -2/3\\
                        0 & 0 & 0 & k + 1\\
                    \end{amatrix}
                \end{align*}
                Da \'ultima linha dessa matriz temos que se $k + 1 \ne 0$, isto \'e, $k \ne -1$ ent\~ao o sistema \'e imposs{\'\i}vel. Ent\~ao para qualquer $K \in \real$ com $k \ne -1$ o sistema \'e imposs{\'\i}vel.

                Se $k = -1$, a \'ultima matriz torna-se
                \[
                    \begin{amatrix}{3}
                        1 & -1 & -1 & 1\\
                        0 & 1 & 1 & -2/3\\
                        0 & 0 & 0 & 0\\
                    \end{amatrix}
                \]
                Com isso obtemos as equa\c{c}\~oes
                \[
                    \begin{cases}
                        x_1 = 1/3\\
                        x_2 + x_3 = -2/3
                    \end{cases}
                \]
                Da segunda equa\c{c}\~ao desse sistema encontramos
                \[
                    x_2 = -\dfrac{2}{3} - x_3
                \]
                Com isso, $x_3$ \'e uma vari\'avel livre. Da{\'\i}, quando $k = -1$ o sistema \'e poss{\'\i}vel e indeterminado. O conjunto-solu\c{c}\~ao \'e
                \[
                    S = \{(1/3, -2/3 - x_3, x_3) \mid x_3 \in \real\}.
                \]
        \end{enumerate}
    \end{solucao}
\end{exemplos}

Considere o sistema linear:
\begin{equation*}
    \begin{cases}
        a_{11}x_1 + a_{12}x_2 + \cdots + a_{1n}x_n = b_1\\
        a_{21}x_1 + a_{22}x_2 + \cdots + a_{2n}x_n = b_2\\
        \qquad \vdots\\
        a_{m1}x_1 + a_{m2}x_2 + \cdots + a_{mn}x_n = b_m
    \end{cases}
\end{equation*}
A matriz
\[
    A = \begin{bmatrix}
        a_{11} & a_{12} & \cdots & a_{1n}\\
        a_{21} & a_{22} & \cdots & a_{2n}\\
        \vdots & \vdots & \cdots & \vdots\\
        a_{m1} & a_{m2} & \cdots & a_{mn}
    \end{bmatrix}
\]
\'e chamada de \textbf{matriz dos coeficientes} do sistema linear.

\begin{definicao}
    Seja $A$ a matriz ampliada de um sistema linear. Se $A$ est\'a na forma escalonada reduzida por linhas, ent\~ao as vari\'aveis desse 
    sistema que n\~ao correspondem aos piv\^os s\~ao chamadas de \textbf{vari\'aveis livres}.\index{Sistemas lineares!vari\'aveis livres}
\end{definicao}

\begin{definicao}
    O \textbf{posto}\index{Matriz!posto}, denotado $\p{A}$ ou $p(A)$, de uma matriz $A$ \'e n\'umero de linhas n\~ao nulas de qualquer 
    uma de suas formas escalonadas por linhas.
\end{definicao}

\begin{teorema}[Teorema do Posto]
    Seja $A$ a matriz dos coeficientes de um sistema linear com $n$ vari\'aveis. Ent\~ao
    \[
        \mbox{n\'umero de vari\'aveis livres} =  n - \p{A}.
    \]
\end{teorema}

Durante a aplica\c{c}\~ao do \textbf{m\'etodo de elimina\c{c}\~ao de Gauss}, paramos quando a matriz ampliada do sistema est\'a na forma escalonada.
Todavia podemos continuar aplicando opera\c{c}\~oes elementares at\'e que a matriz atinja a forma escalonada reduzida por linhas. Esse \'e o caso do 
\textbf{m\'etodo de Gaus-Jordan}.

\begin{definicao}[M\'etodo de elimina\c{c}ao de Gauss-Jordan]
    O m\'etodo de \textbf{elimina\c{c}\~ao de Gauss-Jordan}, para solu\c{c}\~ao de um sistema linear, consiste em:
    \begin{enumerate}[label={\roman*})]
        \item Escreva a matriz ampliada do sistema de equa\c{c}\~oes lineares.

        \item Use opera\c{c}\~oes elementares nas linhas de $A$ para reduzir a matriz ampliada \`a \textbf{forma escalonada reduzida por linhas}.
        
        \item Se o sistema resultante for poss{\'\i}vel, resolva-o para as vari\'aveis dependentes em termos de quaisquer vari\'aveis livres que 
            tenham sobrado.
    \end{enumerate}
\end{definicao}
\begin{exemplo}
    Resolva os seguintes sistemas usando o m\'etodo de Guass-Jordan. Encontre o posto e o n\'umero de vari\'aveis livres:
    \begin{enumerate}
        \item $\begin{cases}
                x_1 + 2x_2 - x_3 = 9\\
                2x_1 - x_2 + x_3 = 0\\
                4x_1 - x_2 + x_3 = 4
            \end{cases}$

        \item $\begin{cases}
                w + x + 2y + z = 0\\
                w - x - y + z = 0\\
                x + y = 0\\
                w + x + z = 0
        \end{cases}$
    \end{enumerate}
    \begin{solucao}
        \begin{enumerate}
            \item A matriz ampliada do sistema \'e
                \[
                    \begin{amatrix}{3}
                        1 & 2 & -1 & 9\\
                        2 & -1 & 1 & 0\\
                        4 & -1 & 1 & 4
                    \end{amatrix}
                \]
                Aplicando om\'etodo de Guass-Jordan \`a essa matriz:
                \begin{align*}
                &\begin{amatrix}{3}
                    1 & 2 & -1 & 9\\
                    2 & -1 & 1 & 0\\
                    4 & -1 & 1 & 4
                \end{amatrix}
                \begin{array}{l}
                    \phantom{x}\\L_2 \to L_2 - 2L_1\\L_3 \to L_3 - 4L_1
                \end{array}\sim
                \begin{amatrix}{3}
                    1 & 2 & -1 & 9\\
                    0 & -5 & 3 & -18\\
                    0 & -9 & 5 & -32
                \end{amatrix}
                \begin{array}{l}
                    \phantom{x}\\L_2 \to -\dfrac{1}{5}L_2 \\\phantom{x}
                \end{array}\sim\\
                &\begin{amatrix}{3}
                    1 & 2 & -1 & 9\\
                    0 & 1 & -3/5 & 18/5\\
                    0 & -9 & 5 & -32
                \end{amatrix}
                \begin{array}{l}
                    \phantom{x}\\\phantom{x}\\L_3 \to L_3 + 9L_2
                \end{array}\sim
                \begin{amatrix}{3}
                    1 & 2 & -1 & 9\\
                    0 & 1 & -3/5 & 18/5\\
                    0 & 0 & -2/5 & 2/5
                \end{amatrix}
                \begin{array}{l}
                    \phantom{x}\\\phantom{x}\\L_3 \to -\dfrac{5}{2}L_3
                \end{array}\sim\\
                &\begin{amatrix}{3}
                    1 & 2 & -1 & 9\\
                    0 & 1 & -3/5 & 18/5\\
                    0 & 0 & 1 & -1
                \end{amatrix}
                \begin{array}{l}
                    \phantom{x}\\L_2 \to L_2 +\dfrac{3}{5}L_3\\\phantom{x}\\\phantom{x}
                \end{array}\sim
                \begin{amatrix}{3}
                    1 & 2 & -1 & 9\\
                    0 & 1 & 0 & 3\\
                    0 & 0 & 1 & -1
                \end{amatrix}
                \begin{array}{l}
                    L_1 \to L_1 - 2L_2\\\phantom{x}\\\phantom{x}
                \end{array}\sim\\
                &\begin{amatrix}{3}
                    1 & 0 & -1 & 3\\
                    0 & 1 & 0 & 3\\
                    0 & 0 & 1 & -1
                \end{amatrix}
                \begin{array}{l}
                    L_1 \to L_1 + L_3\\\phantom{x}\\\phantom{x}
                \end{array}\sim
                \begin{amatrix}{3}
                    1 & 0 & 0 & 2\\
                    0 & 1 & 0 & 3\\
                    0 & 0 & 1 & -1
                \end{amatrix}
            \end{align*}
            Assim o posto da matriz \'e 3 e o n\'umero de vari\'aveis livres \'e 0.
            O sistema admite solu\c{c}\~ao \'unica $x_1 = 2$, $x_2 = 3$ e $x_3 = -1$, que pode ser denotada como $S = \{(2,3,-1)\}$.

            \item A matriz ampliada do sistema \'e
                \[
                    \begin{amatrix}{4}
                        1 & 1 & 2 & 1 & 0\\
                        1 & -1 & -1 & 1 & 0\\
                        0 & 1 & 1 & 0 & 0\\
                        1 & 1 & 0 & 1 & 0
                    \end{amatrix}.
                \]
            Como a \'ultima coluna dessa matriz \'e nula podemos omit{\'\i}-la na hora de aplicar o m\'etodo de Gauss-Jordan.
            \begin{align*}
                &\begin{pmatrix}
                    1 &\phantom{-} 1 &\phantom{-} 2 & 1\\
                    1 & -1 & -1 & 1\\
                    0 &\phantom{-} 1 &\phantom{-} 1 & 0\\
                    1 &\phantom{-} 1 &\phantom{-} 0 & 1
                \end{pmatrix}
                \begin{array}{l}
                    \phantom{x}\\L_2 \to L_2 - L_1\\\phantom{x}\\L_4 \to L_4 - L_1
                \end{array}\sim
                \begin{pmatrix}
                    1 &\phantom{-} 1 &\phantom{-} 2 & 1\\
                    0 & -2 & -3 & 1\\
                    0 &\phantom{-} 1 &\phantom{-} 1 & 0\\
                    0 &\phantom{-} 0 & -2 & 0
                \end{pmatrix}
                \begin{array}{l}
                    \phantom{x}\\L_2 \leftrightarrow L_3\\\phantom{x}\\\phantom{x}
                \end{array}\sim\\
                &\begin{pmatrix}
                    1 &\phantom{-} 1 &\phantom{-} 2 & 1\\
                    0 &\phantom{-} 1 &\phantom{-} 1 & 0\\
                    0 & -2 & -3 & 0\\
                    0 &\phantom{-} 0 & -2 & 0
                \end{pmatrix}
                \begin{array}{l}
                    L_1 \to L_1 - L_2\\\phantom{x}\\L_3 \to L_3 + 2L_2\\\phantom{x}
                \end{array}\sim
                \begin{pmatrix}
                    1 & 0 &\phantom{-} 1 & 1\\
                    0 & 1 &\phantom{-} 1 & 0\\
                    0 & 0 & -1 & 0\\
                    0 & 0 & -2 & 0
                \end{pmatrix}
                \begin{array}{l}
                    \phantom{x}\\\phantom{x}\\L_3 \to -L_3\\\phantom{x}
                \end{array}\sim\\
                &\begin{pmatrix}
                    1 & 0 &\phantom{-} 1 & 1\\
                    0 & 1 &\phantom{-} 1 & 0\\
                    0 & 0 &\phantom{-} 1 & 0\\
                    0 & 0 & -2 & 0
                \end{pmatrix}
                \begin{array}{l}
                    L_1 \to L_1 - L_3\\L_2 \to L_2 - L_3\\\phantom{x}\\L_4 \to L_4 + 2L_3
                \end{array}\sim
                \begin{pmatrix}
                    1 & 0 & 0 & 1\\
                    0 & 1 & 0 & 0\\
                    0 & 0 & 1 & 0\\
                    0 & 0 & 0 & 0
                \end{pmatrix}
            \end{align*}
            Neste caso o posto \'e 3 e o n\'umero de vari\'aveis livres \'e 1. A solu\c{c}\~ao desse sistema \'e
            \[
                x_2 = x_3 = 0, x_1 = -x_4
            \]
            que pode ser dada pelo conjunto
            \[
                S = \{(-x_4, 0, 0, x_4) \mid x_4 \in \real\}.
            \]

    \end{enumerate}
\end{solucao}
\end{exemplo}
\begin{teorema}
    Um sistema linear homog\^eneo com mais inc\'ognitas que equa\c{c}\~oes tem uma infinidade de solu\c{c}\~oes.
\end{teorema}

\section{Matriz Inversa}

\begin{definicao}
    Seja $A$ uma matriz quadrada de ordem $n$ e $A \ne 0$. Se for poss{\'\i}vel encontrar uma matriz quadrada $B$ tamb\'em de 
    ordem $n$ tal que
    \[
        AB = I_n = BA   
    \]
    onde $I_n$ \'e a matriz identidade de ordem $n$, ent\~ao diremos que $A$ \'e \textbf{invert{\'\i}vel}, ou \textbf{n\~ao singular},
    e que $B$ \'e a \textbf{inversa} de $A$. Se n\~ao pudermos encontrar tal matriz $B$, ent\~ao diremos que $A$ \'e \textbf{n\~ao invert{\'\i}vel}  
    ou \textbf{singular}.
\end{definicao}

\begin{exemplos}
\begin{enumerate}
    \item A matriz
    \[
       A = \begin{pmatrix}
            -1 & -1 & 0\\
            0 & -1 & -1\\
            1 & -1 & -3
        \end{pmatrix}
    \]
    é invertível e sua inversa é a matriz
    \[
       B = \begin{pmatrix}
            -2 & 3 & 1\\
            1 & -3 & 1\\
            -1 & 2 & -1
        \end{pmatrix}.
    \]
    \item A matriz
    \[
        C = \begin{pmatrix}1 & -1 & 0\\2 & 5 & 0\\3 & 6 & 0\end{pmatrix}
    \]
    é singular.
\end{enumerate}
    
    \begin{solucao}
    \begin{enumerate}
        \item De fato,
        \begin{align*}
            AB &= \begin{pmatrix} -1 & -1 & 0\\ 0 & -1 & -1\\ 1 & -1 & -3\end{pmatrix}
            \begin{pmatrix}-2 & 3 & 1\\ 1 & -3 & 1\\ -1 & 2 & -1\end{pmatrix} = 
            \begin{pmatrix} 1 & 0 & 0\\ 0 & 1 & 0\\ 0 & 0 & 1\end{pmatrix} = I_3\\
            BA &= \begin{pmatrix}-2 & 3 & 1\\ 1 & -3 & 1\\ -1 & 2 & -1\end{pmatrix}
            \begin{pmatrix} -1 & -1 & 0\\ 0 & -1 & -1\\ 1 & -1 & -3\end{pmatrix} = \begin{pmatrix} 1 & 0 & 0\\ 0 & 1 & 0\\ 0 & 0 & 1\end{pmatrix} = I_3.
        \end{align*}
        Logo, $AB = I_3 = BA$ e com isso $A$ é invertível e sua inversa é $B$.

        \item De fato, para toda matriz
        \[
            D = \begin{pmatrix}d_{11} & d_{12} & d_{13}\\ d_{21} & d_{22} & d_{23}\\d_{31} & d_{32} & d_{33}\end{pmatrix}
        \]
        temos
        \begin{align*}
            DC &= \begin{pmatrix}d_{11} & d_{12} & d_{13}\\ d_{21} & d_{22} & d_{23}\\d_{31} & d_{32} & d_{33}\end{pmatrix}
            \begin{pmatrix}1 & -1 & 0\\2 & 5 & 0\\3 & 6 & 0\end{pmatrix}\\ &= \begin{pmatrix}
                d_{11}\cdot 1 + d_{12}\cdot 2 + d_{13}\cdot 3 & d_{11}\cdot(-1) + d_{12}\cdot 5 + d_{13}\cdot 6 & 0\\
                d_{21}\cdot 1 + d_{22}\cdot 2 + d_{23}\cdot 3 & d_{21}\cdot(-1) + d_{22}\cdot 5 + d_{23}\cdot 6 & 0\\
                d_{31}\cdot 1 + d_{32}\cdot 2 + d_{33}\cdot 3 & d_{31}\cdot(-1) + d_{32}\cdot 5 + d_{33}\cdot 6 & 0
            \end{pmatrix} \ne \begin{pmatrix}1 & 0 & 0\\0 & 1 & 0\\0 & 0 & 1\end{pmatrix}
        \end{align*}
        Logo $C$ é uma matriz não invertível ou singular.
    \end{enumerate}
    \end{solucao}
\end{exemplos}

\begin{teorema}
    Se $B$ e $C$ s\~ao ambas inversas da matriz $A$, ent\~ao $B = C$.
\end{teorema}

\begin{notacao}
    Se $A$ \'e uma matriz invert{\'\i}vel e $B$ \'e a sua inversa, vamos escrever $B = A^{-1}$.
\end{notacao}

\begin{proposicao}
    Se $A$ e $B$ s\~ao matrizes invert{\'\i}veis de mesmo tamanho, ent\~ao $AB$ \'e invert{\'\i}vel e
    \[
        (AB)^{-1} = B^{-1} A^{-1}.
    \]
\end{proposicao}

\begin{proposicao}
    Seja $A$ uma matriz invert{\'\i}vel e $n$ um n\'umero inteiro n\~ao negativo. Ent\~ao:
    \begin{enumerate}[label={\roman*})]
        \item $A^{-1}$ \'e invert{\'\i}vel e $(A^{-1})^{-1} = A$.

        \item $A^n$ \'e invert{\'\i}vel e $(A^n)^{-1} = A^{-n} = (A^{1})^n$.

        \item $kA$ \'e invert{\'\i}vel para todo escalar $k$ n\~ao nulo e $(kA)^{-1} = k^{-1}A^{-1}$.
    \end{enumerate}
\end{proposicao}

\begin{enumerate}[label={\arabic*})]
    \item Seja $A$ uma matriz quadrada de ordem $n$. Como podemos encontrar a inversa de $A$, se existir.

    \item Dada uma matriz $A$, como podemos decidir se $A$ \'e invert{\'\i}vel?
\end{enumerate}

Considere o sistema linear: 
\begin{equation}
\begin{cases}
        a_{11}x_1 + a_{12}x_2 + \cdots + a_{1n}x_n = b_1\\
        a_{21}x_1 + a_{22}x_2 + \cdots + a_{2n}x_n = b_2\\
        \qquad \vdots\\
        a_{m1}x_1 + a_{m2}x_2 + \cdots + a_{mn}x_n = b_m
    \end{cases}
\end{equation}
A esse sistema podemos associar algumas matrizes. A saber:
\[
    A = \begin{pmatrix}
        a_{11} & a_{12} & \cdots & a_{1n}\\
        a_{21} & a_{22} & \cdots & a_{2n}\\
        \vdots & \vdots & \cdots & \vdots\\
        a_{m1} & a_{m2} & \cdots & a_{mn}
    \end{pmatrix}, \quad
    X = \begin{pmatrix}
        x_1\\
        x_2\\
        \vdots\\
        x_n
    \end{pmatrix},\quad 
    B = \begin{pmatrix}
        b_1\\
        b_2\\
        \vdots\\
        b_m
    \end{pmatrix}
\]

A partir dessas matrizes o sistema linear anterior pode ser escrito como a equa\c{c}\~ao matricial:
\[
    AX = B
\]

No caso em que
\[
    B = \begin{pmatrix}
        0\\0\\\vdots\\0
    \end{pmatrix}
\]
Vamos escrever simplesmente
\[
    AX = 0.
\]

\begin{teorema}
    Seja $A$ uma matriz invert{\'\i}vel de ordem $n$. Para cada matriz $B$ de ordem $n\times 1$ o sistema linear de forma matricial
    $AX = B$ admite solu\c{c}\~ao \'unica que \'e dada por
    \[
        X = \begin{pmatrix}
            x_1 \\ x_2 \\ \vdots \\ x_n
        \end{pmatrix} = A^{-1}B.
    \]
\end{teorema}

\begin{definicao}
    Uma matriz $n\times n$ que pode ser obtida da matriz identidade $I_n$, de tamanho $n\times n$, efetuando uma 
    \'unica opera\c{c}\~ao elementar sobre linhas \'e chamada de \textbf{matriz elementar}.
\end{definicao}

\begin{teorema}
    Se a matriz elementar $E$ \'e o resultado de efetuar uma certa opera\c{c}\~ao com as linhas de $I_m$ e se $A$ \'e uma matriz 
    $m \times n$, ent\~ao o produto $EA$ \'e a matriz que resulta quando essa mesma opera\c{c}\~ao com linhas \'e efetuada em $A$.
\end{teorema}

\begin{teorema}
    Qualquer matriz elementar \'e invert{\'\i}vel, e a inversa tamb\'em \'e uma matriz elementar.
\end{teorema}

\begin{teorema}
    Se $A$ \'e uma matriz $n \times n$, ent\~ao as seguintes afirma\c{c}\~oes s\~ao equivalentes, ou seja, s\~ao todas verdadeiras  
    ou todas falsas.
    \begin{enumerate}[label={\roman*})]
        \item $A$ \'e invert{\'\i}vel.
        
        \item $AX = 0$ tem somente a solu\c{c}\~ao trivial.

        \item A forma escalonada reduzida por linhas de $A$ \'e $I_n$.

        \item $A$ pode ser expressa como um produto de matrizes elementares.
    \end{enumerate}
\end{teorema}

\section{Aplicações de sistema lineares}

\subsection{Análise de Redes}

Redes aparecem em várias situações práticas: redes de transporte, redes de comunicação e redes econômicas, entre outros. Para nós, uma \textbf{rede} consiste em um número finito de \textbf{nós}, também chamdados \textbf{junções} ou \textbf{vértices}, conectados por uma série de segmentos dirigidos, conhecidos como \textbf{ramos} ou \textbf{arcos}. Cada ramo é rotulado com um \textbf{fluxo} que representa a quantidade de alguma mercadoria  que pode fluir ao longo ou através daquele ramo na direção indicada.

A regra fundamental que governa o fluxo através de uma rede é a \textbf{conservação de fluxo}:

\begin{definicao}
  \textit{Em cada nó, o fluxo de entrada é igual ao fluxo de saída.}
\end{definicao}

Por exemplo, para uma rede de um único nó:
\begin{figure}[!h]
    \centering
    \ifx\du\undefined
  \newlength{\du}
\fi
\setlength{\du}{15\unitlength}
\begin{tikzpicture}
\pgftransformxscale{1.000000}
\pgftransformyscale{-1.000000}
\definecolor{dialinecolor}{rgb}{0.000000, 0.000000, 0.000000}
\pgfsetstrokecolor{dialinecolor}
\definecolor{dialinecolor}{rgb}{1.000000, 1.000000, 1.000000}
\pgfsetfillcolor{dialinecolor}
\pgfsetlinewidth{0.100000\du}
\pgfsetdash{}{0pt}
\pgfsetdash{}{0pt}
\pgfsetbuttcap
{
\definecolor{dialinecolor}{rgb}{0.000000, 0.000000, 0.000000}
\pgfsetfillcolor{dialinecolor}
% was here!!!
\definecolor{dialinecolor}{rgb}{0.000000, 0.000000, 0.000000}
\pgfsetstrokecolor{dialinecolor}
\draw (22.050000\du,11.400000\du)--(22.000000\du,24.300000\du);
}
\pgfsetlinewidth{0.100000\du}
\pgfsetdash{}{0pt}
\pgfsetdash{}{0pt}
\pgfsetbuttcap
{
\definecolor{dialinecolor}{rgb}{0.000000, 0.000000, 0.000000}
\pgfsetfillcolor{dialinecolor}
% was here!!!
\definecolor{dialinecolor}{rgb}{0.000000, 0.000000, 0.000000}
\pgfsetstrokecolor{dialinecolor}
\draw (15.350000\du,18.300000\du)--(28.850000\du,18.300000\du);
}
\pgfsetlinewidth{0.150000\du}
\pgfsetdash{}{0pt}
\pgfsetdash{}{0pt}
\pgfsetbuttcap
\pgfsetmiterjoin
\pgfsetlinewidth{0.150000\du}
\pgfsetbuttcap
\pgfsetmiterjoin
\pgfsetdash{}{0pt}
\definecolor{dialinecolor}{rgb}{0.000000, 0.000000, 0.000000}
\pgfsetfillcolor{dialinecolor}
\pgfpathellipse{\pgfpoint{22.037500\du}{18.287500\du}}{\pgfpoint{0.312500\du}{0\du}}{\pgfpoint{0\du}{0.312500\du}}
\pgfusepath{fill}
\definecolor{dialinecolor}{rgb}{0.000000, 0.000000, 0.000000}
\pgfsetstrokecolor{dialinecolor}
\pgfpathellipse{\pgfpoint{22.037500\du}{18.287500\du}}{\pgfpoint{0.312500\du}{0\du}}{\pgfpoint{0\du}{0.312500\du}}
\pgfusepath{stroke}
\pgfsetbuttcap
\pgfsetmiterjoin
\pgfsetdash{}{0pt}
\definecolor{dialinecolor}{rgb}{0.000000, 0.000000, 0.000000}
\pgfsetstrokecolor{dialinecolor}
\pgfpathellipse{\pgfpoint{22.037500\du}{18.287500\du}}{\pgfpoint{0.312500\du}{0\du}}{\pgfpoint{0\du}{0.312500\du}}
\pgfusepath{stroke}
% setfont left to latex
\definecolor{dialinecolor}{rgb}{0.000000, 0.000000, 0.000000}
\pgfsetstrokecolor{dialinecolor}
\node[anchor=west] at (18\du,16.3\du){$f_1$};
\pgfsetlinewidth{0.150000\du}
\pgfsetdash{}{0pt}
\pgfsetdash{}{0pt}
\pgfsetbuttcap
{
\definecolor{dialinecolor}{rgb}{1.000000, 0.000000, 0.000000}
\pgfsetfillcolor{dialinecolor}
% was here!!!
\pgfsetarrowsend{latex}
\definecolor{dialinecolor}{rgb}{1.000000, 0.000000, 0.000000}
\pgfsetstrokecolor{dialinecolor}
\draw (17.450000\du,16.900000\du)--(19.800000\du,16.900000\du);
}
\pgfsetlinewidth{0.150000\du}
\pgfsetdash{}{0pt}
\pgfsetdash{}{0pt}
\pgfsetbuttcap
{
\definecolor{dialinecolor}{rgb}{1.000000, 0.000000, 0.000000}
\pgfsetfillcolor{dialinecolor}
% was here!!!
\pgfsetarrowsend{latex}
\definecolor{dialinecolor}{rgb}{1.000000, 0.000000, 0.000000}
\pgfsetstrokecolor{dialinecolor}
\draw (27.15\du,17\du)--(24.8\du,17\du);
}
% setfont left to latex
\definecolor{dialinecolor}{rgb}{0.000000, 0.000000, 0.000000}
\pgfsetstrokecolor{dialinecolor}
\node[anchor=west] at (25.4\du,16.2\du){$f_2$};
\pgfsetlinewidth{0.150000\du}
\pgfsetdash{}{0pt}
\pgfsetdash{}{0pt}
\pgfsetbuttcap
{
\definecolor{dialinecolor}{rgb}{0.000000, 0.000000, 1.000000}
\pgfsetfillcolor{dialinecolor}
% was here!!!
\pgfsetarrowsend{latex}
\definecolor{dialinecolor}{rgb}{0.000000, 0.000000, 1.000000}
\pgfsetstrokecolor{dialinecolor}
\draw (23.050000\du,16.850000\du)--(23.032700\du,14.467700\du);
}
\pgfsetlinewidth{0.150000\du}
\pgfsetdash{}{0pt}
\pgfsetdash{}{0pt}
\pgfsetbuttcap
{
\definecolor{dialinecolor}{rgb}{0.000000, 0.000000, 1.000000}
\pgfsetfillcolor{dialinecolor}
% was here!!!
\pgfsetarrowsend{latex}
\definecolor{dialinecolor}{rgb}{0.000000, 0.000000, 1.000000}
\pgfsetstrokecolor{dialinecolor}
\draw (23.050000\du,20.100000\du)--(23.047700\du,22.417700\du);
}
% setfont left to latex
\definecolor{dialinecolor}{rgb}{0.000000, 0.000000, 0.000000}
\pgfsetstrokecolor{dialinecolor}
\node[anchor=west] at (23.2\du,15.8\du){$20$};
% setfont left to latex
\definecolor{dialinecolor}{rgb}{0.000000, 0.000000, 0.000000}
\pgfsetstrokecolor{dialinecolor}
\node[anchor=west] at (23.2\du,21.2\du){$30$};
\end{tikzpicture}

\end{figure}
devemos ter:
\[
    f_1 + f_2 = 50.
\]

\begin{figure}[!h]
    \centering
    % Graphic for TeX using PGF
% Title: C:\Users\josea\GitHub\IAL\diagramas_aplicacoes\Diagrama_exemplo.dia
% Creator: Dia v0.97.2
% CreationDate: Thu Apr 13 14:51:42 2023
% For: josea
% \usepackage{tikz}
% The following commands are not supported in PSTricks at present
% We define them conditionally, so when they are implemented,
% this pgf file will use them.
\ifx\du\undefined
  \newlength{\du}
\fi
\setlength{\du}{15\unitlength}
\begin{tikzpicture}
\pgftransformxscale{1.000000}
\pgftransformyscale{-1.000000}
\definecolor{dialinecolor}{rgb}{0.000000, 0.000000, 0.000000}
\pgfsetstrokecolor{dialinecolor}
\definecolor{dialinecolor}{rgb}{1.000000, 1.000000, 1.000000}
\pgfsetfillcolor{dialinecolor}
\pgfsetlinewidth{0.100000\du}
\pgfsetdash{}{0pt}
\pgfsetdash{}{0pt}
\pgfsetbuttcap
{
\definecolor{dialinecolor}{rgb}{0.000000, 0.000000, 0.000000}
\pgfsetfillcolor{dialinecolor}
% was here!!!
\definecolor{dialinecolor}{rgb}{0.000000, 0.000000, 0.000000}
\pgfsetstrokecolor{dialinecolor}
\draw(22.050000\du,7.800000\du)--(22.000000\du,19.900000\du);%vertical AD
}
\pgfsetlinewidth{0.100000\du}
\pgfsetdash{}{0pt}
\pgfsetdash{}{0pt}
\pgfsetbuttcap
{
\definecolor{dialinecolor}{rgb}{0.000000, 0.000000, 0.000000}
\pgfsetfillcolor{dialinecolor}
% was here!!!
\definecolor{dialinecolor}{rgb}{0.000000, 0.000000, 0.000000}
\pgfsetstrokecolor{dialinecolor}
\draw (15.350000\du,13.000000\du)--(32.950000\du,12.950000\du);%horizontal AB
}
\pgfsetlinewidth{0.150000\du}
\pgfsetdash{}{0pt}
\pgfsetdash{}{0pt}
\pgfsetbuttcap
\pgfsetmiterjoin
\pgfsetlinewidth{0.150000\du}
\pgfsetbuttcap
\pgfsetmiterjoin
\pgfsetdash{}{0pt}
\definecolor{dialinecolor}{rgb}{0.000000, 0.000000, 0.000000}
\pgfsetfillcolor{dialinecolor}
\pgfpathellipse{\pgfpoint{22.052500\du}{13\du}}{\pgfpoint{0.312500\du}{0\du}}{\pgfpoint{0\du}{0.312500\du}}%círculo no nó A
\pgfusepath{fill}
\definecolor{dialinecolor}{rgb}{0.000000, 0.000000, 0.000000}
\pgfsetstrokecolor{dialinecolor}

\pgfusepath{stroke}
\pgfsetbuttcap
\pgfsetmiterjoin
\pgfsetdash{}{0pt}
\definecolor{dialinecolor}{rgb}{0.000000, 0.000000, 0.000000}
\pgfsetstrokecolor{dialinecolor}

\pgfusepath{stroke}
% setfont left to latex
\definecolor{dialinecolor}{rgb}{0.000000, 0.000000, 0.000000}
\pgfsetstrokecolor{dialinecolor}
\node[anchor=west] at (20\du,16.550000\du){$f_4$};
% setfont left to latex
\definecolor{dialinecolor}{rgb}{0.000000, 0.000000, 0.000000}
\pgfsetstrokecolor{dialinecolor}
\node[anchor=west] at (23.850000\du,18.3\du){$f_3$};
\pgfsetlinewidth{0.100000\du}
\pgfsetdash{}{0pt}
\pgfsetdash{}{0pt}
\pgfsetbuttcap
{
\definecolor{dialinecolor}{rgb}{0.000000, 0.000000, 1.000000}
\pgfsetfillcolor{dialinecolor}
% was here!!!
\pgfsetarrowsend{latex}
\definecolor{dialinecolor}{rgb}{0.000000, 0.000000, 1.000000}
\pgfsetstrokecolor{dialinecolor}
\draw (27.118000\du,21.150000\du)--(27.118000\du,23.150000\du);%30 saindo de C
}
% setfont left to latex
\definecolor{dialinecolor}{rgb}{0.000000, 0.000000, 0.000000}
\pgfsetstrokecolor{dialinecolor}
\node[anchor=west] at (20\du,9.150000\du){5};
% setfont left to latex
\definecolor{dialinecolor}{rgb}{0.000000, 0.000000, 0.000000}
\pgfsetstrokecolor{dialinecolor}
\node[anchor=west] at (25.815000\du,22\du){30};
\pgfsetlinewidth{0.100000\du}
\pgfsetdash{}{0pt}
\pgfsetdash{}{0pt}
\pgfsetbuttcap
{
\definecolor{dialinecolor}{rgb}{0.000000, 0.000000, 0.000000}
\pgfsetfillcolor{dialinecolor}
% was here!!!
\definecolor{dialinecolor}{rgb}{0.000000, 0.000000, 0.000000}
\pgfsetstrokecolor{dialinecolor}
\draw (15.300000\du,19.950000\du)--(33.100000\du,19.950000\du);%horizontal DC
}
\pgfsetlinewidth{0.100000\du}
\pgfsetdash{}{0pt}
\pgfsetdash{}{0pt}
\pgfsetbuttcap
{
\definecolor{dialinecolor}{rgb}{0.000000, 0.000000, 0.000000}
\pgfsetfillcolor{dialinecolor}
% was here!!!
\definecolor{dialinecolor}{rgb}{0.000000, 0.000000, 0.000000}
\pgfsetstrokecolor{dialinecolor}
\draw (27.950000\du,12.950000\du)--(27.950000\du,24.050000\du);%vertical CB
}
\pgfsetlinewidth{0.150000\du}
\pgfsetdash{}{0pt}
\pgfsetdash{}{0pt}
\pgfsetbuttcap
\pgfsetmiterjoin
\pgfsetlinewidth{0.150000\du}
\pgfsetbuttcap
\pgfsetmiterjoin
\pgfsetdash{}{0pt}
\definecolor{dialinecolor}{rgb}{0.000000, 0.000000, 0.000000}
\pgfsetfillcolor{dialinecolor}
\pgfpathellipse{\pgfpoint{22.052500\du}{19.987500\du}}{\pgfpoint{0.312500\du}{0\du}}{\pgfpoint{0\du}{0.312500\du}}%círculo no nó D
\pgfusepath{fill}
\definecolor{dialinecolor}{rgb}{0.000000, 0.000000, 0.000000}
\pgfsetstrokecolor{dialinecolor}
\pgfusepath{stroke}
\pgfsetbuttcap
\pgfsetmiterjoin
\pgfsetdash{}{0pt}
\definecolor{dialinecolor}{rgb}{0.000000, 0.000000, 0.000000}
\pgfsetstrokecolor{dialinecolor}
\pgfusepath{stroke}
\pgfsetlinewidth{0.150000\du}
\pgfsetdash{}{0pt}
\pgfsetdash{}{0pt}
\pgfsetbuttcap
\pgfsetmiterjoin
\pgfsetlinewidth{0.150000\du}
\pgfsetbuttcap
\pgfsetmiterjoin
\pgfsetdash{}{0pt}
\definecolor{dialinecolor}{rgb}{0.000000, 0.000000, 0.000000}
\pgfsetfillcolor{dialinecolor}
\pgfpathellipse{\pgfpoint{27.952500\du}{19.987500\du}}{\pgfpoint{0.312500\du}{0\du}}{\pgfpoint{0\du}{0.312500\du}}%círculo no nó C
\pgfusepath{fill}
\definecolor{dialinecolor}{rgb}{0.000000, 0.000000, 0.000000}
\pgfsetstrokecolor{dialinecolor}
\pgfusepath{stroke}
\pgfsetlinewidth{0.150000\du}
\pgfsetdash{}{0pt}
\pgfsetdash{}{0pt}
\pgfsetbuttcap
\pgfsetmiterjoin
\pgfsetlinewidth{0.150000\du}
\pgfsetbuttcap
\pgfsetmiterjoin
\pgfsetdash{}{0pt}
\definecolor{dialinecolor}{rgb}{0.000000, 0.000000, 0.000000}
\pgfsetfillcolor{dialinecolor}
\pgfpathellipse{\pgfpoint{27.952500\du}{13\du}}{\pgfpoint{0.312500\du}{0\du}}{\pgfpoint{0\du}{0.312500\du}}%círculo no nó B
\pgfusepath{fill}
\definecolor{dialinecolor}{rgb}{0.000000, 0.000000, 0.000000}
\pgfsetstrokecolor{dialinecolor}
\pgfusepath{stroke}
\pgfsetbuttcap
\pgfsetmiterjoin
\pgfsetdash{}{0pt}
\definecolor{dialinecolor}{rgb}{0.000000, 0.000000, 0.000000}
\pgfsetstrokecolor{dialinecolor}
\pgfusepath{stroke}
\pgfsetlinewidth{0.100000\du}
\pgfsetdash{}{0pt}
\pgfsetdash{}{0pt}
\pgfsetbuttcap
{
\definecolor{dialinecolor}{rgb}{1.000000, 0.000000, 0.000000}
\pgfsetfillcolor{dialinecolor}
% was here!!!
\pgfsetarrowsend{latex}
\definecolor{dialinecolor}{rgb}{1.000000, 0.000000, 0.000000}
\pgfsetstrokecolor{dialinecolor}
\draw (23.415000\du,19.015500\du)--(25.415000\du,19.015500\du);%f_3
}
\pgfsetlinewidth{0.100000\du}
\pgfsetdash{}{0pt}
\pgfsetdash{}{0pt}
\pgfsetbuttcap
{
\definecolor{dialinecolor}{rgb}{0.000000, 0.000000, 1.000000}
\pgfsetfillcolor{dialinecolor}
% was here!!!
\pgfsetarrowsend{latex}
\definecolor{dialinecolor}{rgb}{0.000000, 0.000000, 1.000000}
\pgfsetstrokecolor{dialinecolor}
\draw (21.247700\du,8.4\du)--(21.245400\du,10.4\du);%5 chegando em A
}
\pgfsetlinewidth{0.100000\du}
\pgfsetdash{}{0pt}
\pgfsetdash{}{0pt}
\pgfsetbuttcap
{
\definecolor{dialinecolor}{rgb}{0.000000, 0.000000, 1.000000}
\pgfsetfillcolor{dialinecolor}
% was here!!!
\pgfsetarrowsend{latex}
\definecolor{dialinecolor}{rgb}{0.000000, 0.000000, 1.000000}
\pgfsetstrokecolor{dialinecolor}
\draw (18\du,12.165500\du)--(20\du,12.165500\du);%10 chegando em A
}
% setfont left to latex
\definecolor{dialinecolor}{rgb}{0.000000, 0.000000, 0.000000}
\pgfsetstrokecolor{dialinecolor}
\node[anchor=west] at (18\du,11.5\du){10};
\pgfsetlinewidth{0.100000\du}
\pgfsetdash{}{0pt}
\pgfsetdash{}{0pt}
\pgfsetbuttcap
{
\definecolor{dialinecolor}{rgb}{0.000000, 0.000000, 1.000000}
\pgfsetfillcolor{dialinecolor}
% was here!!!
\pgfsetarrowsend{latex}
\definecolor{dialinecolor}{rgb}{0.000000, 0.000000, 1.000000}
\pgfsetstrokecolor{dialinecolor}
\draw (29.431400\du,12.165500\du)--(31.431400\du,12.165500\du);%10 saindo de B
}
\pgfsetlinewidth{0.100000\du}
\pgfsetdash{}{0pt}
\pgfsetdash{}{0pt}
\pgfsetbuttcap
{
\definecolor{dialinecolor}{rgb}{0.000000, 0.000000, 1.000000}
\pgfsetfillcolor{dialinecolor}
% was here!!!
\pgfsetarrowsstart{latex}
\definecolor{dialinecolor}{rgb}{0.000000, 0.000000, 1.000000}
\pgfsetstrokecolor{dialinecolor}
\draw (29\du,19.015500\du)--(31\du,19.015500\du);%5 entrando em C
}
% setfont left to latex
\definecolor{dialinecolor}{rgb}{0.000000, 0.000000, 0.000000}
\pgfsetstrokecolor{dialinecolor}
\node[anchor=west] at (23.850000\du,11.5\du){$f_1$};
% setfont left to latex
\definecolor{dialinecolor}{rgb}{0.000000, 0.000000, 0.000000}
\pgfsetstrokecolor{dialinecolor}
\node[anchor=west] at (29.5\du,11.5\du){10};
\pgfsetlinewidth{0.100000\du}
\pgfsetdash{}{0pt}
\pgfsetdash{}{0pt}
\pgfsetbuttcap
{
\definecolor{dialinecolor}{rgb}{1.000000, 0.000000, 0.000000}
\pgfsetfillcolor{dialinecolor}
% was here!!!
\pgfsetarrowsend{latex}
\definecolor{dialinecolor}{rgb}{1.000000, 0.000000, 0.000000}
\pgfsetstrokecolor{dialinecolor}
\draw (23.726800\du,12.165500\du)--(25.726800\du,12.165500\du);%f_1
}
\pgfsetlinewidth{0.100000\du}
\pgfsetdash{}{0pt}
\pgfsetdash{}{0pt}
\pgfsetbuttcap
{
\definecolor{dialinecolor}{rgb}{0.000000, 0.000000, 1.000000}
\pgfsetfillcolor{dialinecolor}
% was here!!!
\pgfsetarrowsend{latex}
\definecolor{dialinecolor}{rgb}{0.000000, 0.000000, 1.000000}
\pgfsetstrokecolor{dialinecolor}
\draw (17.766400\du,19.015500\du)--(19.766400\du,19.015500\du);%20 chegando em D
}
% setfont left to latex
\definecolor{dialinecolor}{rgb}{0.000000, 0.000000, 0.000000}
\pgfsetstrokecolor{dialinecolor}
\node[anchor=west] at (18\du,18.3\du){20};
\pgfsetlinewidth{0.100000\du}
\pgfsetdash{}{0pt}
\pgfsetdash{}{0pt}
\pgfsetbuttcap
{
\definecolor{dialinecolor}{rgb}{1.000000, 0.000000, 0.000000}
\pgfsetfillcolor{dialinecolor}
% was here!!!
\pgfsetarrowsend{latex}
\definecolor{dialinecolor}{rgb}{1.000000, 0.000000, 0.000000}
\pgfsetstrokecolor{dialinecolor}
\draw (21.318000\du,15.452400\du)--(21.318000\du,17.452400\du);%f_4
}
\pgfsetlinewidth{0.100000\du}
\pgfsetdash{}{0pt}
\pgfsetdash{}{0pt}
\pgfsetbuttcap
{
\definecolor{dialinecolor}{rgb}{1.000000, 0.000000, 0.000000}
\pgfsetfillcolor{dialinecolor}
% was here!!!
\pgfsetarrowsend{latex}
\definecolor{dialinecolor}{rgb}{1.000000, 0.000000, 0.000000}
\pgfsetstrokecolor{dialinecolor}
\draw (27.118000\du,15.602400\du)--(27.118000\du,17.602400\du);%f_2
}
% setfont left to latex
\definecolor{dialinecolor}{rgb}{0.000000, 0.000000, 0.000000}
\pgfsetstrokecolor{dialinecolor}
\node[anchor=west] at (25.815000\du,16.540000\du){$f_2$};
% setfont left to latex
\definecolor{dialinecolor}{rgb}{0.000000, 0.000000, 0.000000}
\pgfsetstrokecolor{dialinecolor}
\node[anchor=west] at (29.5\du,18.3\du){5};
% setfont left to latex
\definecolor{dialinecolor}{rgb}{0.000000, 0.000000, 0.000000}
\pgfsetstrokecolor{dialinecolor}
\node[anchor=west] at (20.8\du,12.500000\du){A};
% setfont left to latex
\definecolor{dialinecolor}{rgb}{0.000000, 0.000000, 0.000000}
\pgfsetstrokecolor{dialinecolor}
\node[anchor=west] at (27.265000\du,12.190000\du){B};
% setfont left to latex
\definecolor{dialinecolor}{rgb}{0.000000, 0.000000, 0.000000}
\pgfsetstrokecolor{dialinecolor}
\node[anchor=west] at (28\du,20.8\du){C};
% setfont left to latex
\definecolor{dialinecolor}{rgb}{0.000000, 0.000000, 0.000000}
\pgfsetstrokecolor{dialinecolor}
\node[anchor=west] at (21.5\du,20.8\du){D};
\end{tikzpicture}

\end{figure}

\begin{enumerate}[label={\roman*})]
    \item para o \textbf{nó} A temos a equação: $15 = f_1 + f_4$
    \item para o \textbf{nó} B temos a equação: $f_1 = f_2 + 10$
    \item para o \textbf{nó} C temos a equação: $f_2 + f_3 + 5 = 30$
    \item para o \textbf{nó} D temos a equação: $f_4 + 20 = f_3$
\end{enumerate}

Que produz o seguinte sistema:
\[
    \begin{cases}
        f_1 + f_4 = 15\\
        f_1 - f_2 = 10\\
        f_2 + f_3 = 25\\
        f_3 - f_4 = 20
    \end{cases}
\]

Nesse caso podemos usar o método de eliminação de Gauss-Jordan para resolver tal sistema linear e encontrar:
\begin{align*}
    f_1 &= 15 - t\\
    f_2 &= 5 - t\\
    f_3 &= 20 + t\\
    f_4 &= t
\end{align*}

\subsection{Circuitos Élétricos}

Circuitos elétricos formam um tipo especializado de rede com informações sobre fontes de energia, tais como baterias, e dispositivos alimentados por essas fontes, tais como lâmpadas ou motores.

A lei fundamental da eletricidade é a \textrm{lei de Ohm}, que estabelece exatamente quanta força elétrica $E$ é necessária para fazer uma corrente $I$ atravessar um resistor $R$.

\begin{definicao}[Lei de Ohm]
    \[
        E = RI,
    \] 
    onde $E$ é medida em \textrm{volts}, a resistência $R$ em \textrm{ohms} e a corrente $I$ em \textrm{amperes}.
\end{definicao}

A corrente sai pelo terminal positivo de uma bateria e entra pelo terminal negativo, viajando por um ou mais circuitos fechados no processo. Em um diagrama de um circuito elétrico, baterias são representadas por
\begin{center}
    \begin{tikzpicture}[circuit ee IEC]
        \draw (-1, 0) to [battery] (0, 0) {};
    \end{tikzpicture}
\end{center}
em que o terminal positivo é indicado pela barra vertical mais longa.

Resistores são representados por
\begin{center}
    \begin{tikzpicture}[circuit ee IEC]
        \begin{scope}[set resistor graphic=var resistor IEC graphic]
            \draw (0,1) to [resistor] (3,1);
        \end{scope}
    \end{tikzpicture}
\end{center}

As duas leis a seguir, descobertas por Kirchhoff, regem os circuitos elétricos. A primeira é uma lei de ``converservação de fluxo'' em cada nó:

\begin{definicao}[Lei da Corrente (nós)]
    A soma das correntes que entram em qualquer nó é igual à soma das correntes que saem dele.
\end{definicao}

A segunda é uma lei de ``balanceamento da voltagem'' ao longo de cada circuito:

\begin{definicao}[Lei da Voltagem (circuitos)]
    A soma das quedas de voltagem ao longo de qualquer circuito  é igual à voltagem total em torno do circuito (fornecida pelas baterias).
\end{definicao}

\begin{figure}[!h]
    \centering
    % Graphic for TeX using PGF
% Title: /home/jfreitas/GitHub/IAL/diagramas_aplicacoes/diagrama_corrente_simples.dia
% Creator: Dia vDIA_0_97_0-2473-g1a1b882bd+
% CreationDate: Tue Apr 18 10:33:58 2023
% For: jfreitas
% \usepackage{tikz}
% The following commands are not supported in PSTricks at present
% We define them conditionally, so when they are implemented,
% this pgf file will use them.
\ifx\du\undefined
  \newlength{\du}
\fi
\setlength{\du}{15\unitlength}
\begin{tikzpicture}[even odd rule]
\pgftransformxscale{1.000000}
\pgftransformyscale{-1.000000}
\definecolor{dialinecolor}{rgb}{0.000000, 0.000000, 0.000000}
\pgfsetstrokecolor{dialinecolor}
\pgfsetstrokeopacity{1.000000}
\definecolor{diafillcolor}{rgb}{1.000000, 1.000000, 1.000000}
\pgfsetfillcolor{diafillcolor}
\pgfsetfillopacity{1.000000}
\pgfsetlinewidth{0.100000\du}
\pgfsetdash{}{0pt}
\pgfsetbuttcap
{
\definecolor{diafillcolor}{rgb}{0.000000, 0.000000, 0.000000}
\pgfsetfillcolor{diafillcolor}
\pgfsetfillopacity{1.000000}
% was here!!!
\pgfsetarrowsstart{latex}
\definecolor{dialinecolor}{rgb}{0.000000, 0.000000, 0.000000}
\pgfsetstrokecolor{dialinecolor}
\pgfsetstrokeopacity{1.000000}
\draw (26.082700\du,14.067700\du)--(23.700000\du,14.050000\du);
}
\pgfsetlinewidth{0.100000\du}
\pgfsetdash{}{0pt}
\pgfsetbuttcap
{
\definecolor{diafillcolor}{rgb}{0.000000, 0.000000, 0.000000}
\pgfsetfillcolor{diafillcolor}
\pgfsetfillopacity{1.000000}
% was here!!!
\pgfsetarrowsend{latex}
\definecolor{dialinecolor}{rgb}{0.000000, 0.000000, 0.000000}
\pgfsetstrokecolor{dialinecolor}
\pgfsetstrokeopacity{1.000000}
\draw (28.820300\du,12.947300\du)--(30.708200\du,11.057700\du);
}
% setfont left to latex
% setfont left to latex
\definecolor{dialinecolor}{rgb}{0.000000, 0.000000, 0.000000}
\pgfsetstrokecolor{dialinecolor}
\pgfsetstrokeopacity{1.000000}
\definecolor{diafillcolor}{rgb}{0.000000, 0.000000, 0.000000}
\pgfsetfillcolor{diafillcolor}
\pgfsetfillopacity{1.000000}
\node[anchor=base west,inner sep=0pt,outer sep=0pt,color=dialinecolor] at (24.400000\du,13.600000\du){$I_1$};
% setfont left to latex
% setfont left to latex
\definecolor{dialinecolor}{rgb}{0.000000, 0.000000, 0.000000}
\pgfsetstrokecolor{dialinecolor}
\pgfsetstrokeopacity{1.000000}
\definecolor{diafillcolor}{rgb}{0.000000, 0.000000, 0.000000}
\pgfsetfillcolor{diafillcolor}
\pgfsetfillopacity{1.000000}
\node[anchor=base west,inner sep=0pt,outer sep=0pt,color=dialinecolor] at (28.812825\du,11.990000\du){$I_2$};
\pgfsetlinewidth{0.150000\du}
\pgfsetdash{}{0pt}
\pgfsetbuttcap
{
\definecolor{diafillcolor}{rgb}{0.000000, 0.000000, 0.000000}
\pgfsetfillcolor{diafillcolor}
\pgfsetfillopacity{1.000000}
% was here!!!
\definecolor{dialinecolor}{rgb}{0.000000, 0.000000, 0.000000}
\pgfsetstrokecolor{dialinecolor}
\pgfsetstrokeopacity{1.000000}
\draw (28.050000\du,15.000000\du)--(32.475200\du,19.422300\du);
}
\pgfsetlinewidth{0.150000\du}
\pgfsetdash{}{0pt}
\pgfsetbuttcap
{
\definecolor{diafillcolor}{rgb}{0.000000, 0.000000, 0.000000}
\pgfsetfillcolor{diafillcolor}
\pgfsetfillopacity{1.000000}
% was here!!!
\definecolor{dialinecolor}{rgb}{0.000000, 0.000000, 0.000000}
\pgfsetstrokecolor{dialinecolor}
\pgfsetstrokeopacity{1.000000}
\draw (28.100000\du,14.950000\du)--(32.621000\du,10.409300\du);
}
\pgfsetlinewidth{0.150000\du}
\pgfsetdash{}{0pt}
\pgfsetbuttcap
{
\definecolor{diafillcolor}{rgb}{0.000000, 0.000000, 0.000000}
\pgfsetfillcolor{diafillcolor}
\pgfsetfillopacity{1.000000}
% was here!!!
\definecolor{dialinecolor}{rgb}{0.000000, 0.000000, 0.000000}
\pgfsetstrokecolor{dialinecolor}
\pgfsetstrokeopacity{1.000000}
\draw (22.800000\du,15.050000\du)--(28.000000\du,15.050000\du);
}
\pgfsetlinewidth{0.150000\du}
\pgfsetdash{}{0pt}
\pgfsetbuttcap
\pgfsetmiterjoin
\pgfsetlinewidth{0.150000\du}
\pgfsetbuttcap
\pgfsetmiterjoin
\pgfsetdash{}{0pt}
\definecolor{diafillcolor}{rgb}{0.000000, 0.000000, 0.000000}
\pgfsetfillcolor{diafillcolor}
\pgfsetfillopacity{1.000000}
\pgfpathellipse{\pgfpoint{28.144900\du}{15.052300\du}}{\pgfpoint{0.312500\du}{0\du}}{\pgfpoint{0\du}{0.312500\du}}
\pgfusepath{fill}
\definecolor{dialinecolor}{rgb}{0.000000, 0.000000, 0.000000}
\pgfsetstrokecolor{dialinecolor}
\pgfsetstrokeopacity{1.000000}
\pgfpathellipse{\pgfpoint{28.144900\du}{15.052300\du}}{\pgfpoint{0.312500\du}{0\du}}{\pgfpoint{0\du}{0.312500\du}}
\pgfusepath{stroke}
\pgfsetlinewidth{0.015000\du}
\pgfsetbuttcap
\pgfsetmiterjoin
\pgfsetdash{}{0pt}
\definecolor{dialinecolor}{rgb}{0.000000, 0.000000, 0.000000}
\pgfsetstrokecolor{dialinecolor}
\pgfsetstrokeopacity{1.000000}
\pgfpathellipse{\pgfpoint{28.144900\du}{15.052300\du}}{\pgfpoint{0.312500\du}{0\du}}{\pgfpoint{0\du}{0.312500\du}}
\pgfusepath{stroke}
\pgfsetlinewidth{0.100000\du}
\pgfsetdash{}{0pt}
\pgfsetbuttcap
{
\definecolor{diafillcolor}{rgb}{0.000000, 0.000000, 0.000000}
\pgfsetfillcolor{diafillcolor}
\pgfsetfillopacity{1.000000}
% was here!!!
\pgfsetarrowsend{latex}
\definecolor{dialinecolor}{rgb}{0.000000, 0.000000, 0.000000}
\pgfsetstrokecolor{dialinecolor}
\pgfsetstrokeopacity{1.000000}
\draw (28.935700\du,17.264300\du)--(30.700000\du,19.000000\du);
}
% setfont left to latex
% setfont left to latex
\definecolor{dialinecolor}{rgb}{0.000000, 0.000000, 0.000000}
\pgfsetstrokecolor{dialinecolor}
\pgfsetstrokeopacity{1.000000}
\definecolor{diafillcolor}{rgb}{0.000000, 0.000000, 0.000000}
\pgfsetfillcolor{diafillcolor}
\pgfsetfillopacity{1.000000}
\node[anchor=base west,inner sep=0pt,outer sep=0pt,color=dialinecolor] at (28.805512\du,18.590000\du){$I_3$};
\end{tikzpicture}

    \caption{$I_1 = I_2 + I_3$}
\end{figure}

\begin{figure}[!h]
    \centering
    % Graphic for TeX using PGF
% Title: /home/jfreitas/GitHub/IAL/diagramas_aplicacoes/diagrama_circuito_um_resistor.dia
% Creator: Dia vDIA_0_97_0-2473-g1a1b882bd+
% CreationDate: Tue Apr 18 10:39:18 2023
% For: jfreitas
% \usepackage{tikz}
% The following commands are not supported in PSTricks at present
% We define them conditionally, so when they are implemented,
% this pgf file will use them.
\ifx\du\undefined
  \newlength{\du}
\fi
\setlength{\du}{15\unitlength}
\begin{tikzpicture}[even odd rule]
\pgftransformxscale{1.000000}
\pgftransformyscale{-1.000000}
\definecolor{dialinecolor}{rgb}{0.000000, 0.000000, 0.000000}
\pgfsetstrokecolor{dialinecolor}
\pgfsetstrokeopacity{1.000000}
\definecolor{diafillcolor}{rgb}{1.000000, 1.000000, 1.000000}
\pgfsetfillcolor{diafillcolor}
\pgfsetfillopacity{1.000000}
\pgfsetlinewidth{0.150000\du}
\pgfsetdash{}{0pt}
\pgfsetbuttcap
\pgfsetmiterjoin
\pgfsetlinewidth{0.150000\du}
\pgfsetbuttcap
\pgfsetmiterjoin
\pgfsetdash{}{0pt}
\definecolor{dialinecolor}{rgb}{0.000000, 0.000000, 0.000000}
\pgfsetstrokecolor{dialinecolor}
\pgfsetstrokeopacity{1.000000}
\draw (24.500000\du,10.900000\du)--(23.000000\du,10.900000\du);
\pgfsetbuttcap
\pgfsetmiterjoin
\pgfsetdash{}{0pt}
\definecolor{dialinecolor}{rgb}{0.000000, 0.000000, 0.000000}
\pgfsetstrokecolor{dialinecolor}
\pgfsetstrokeopacity{1.000000}
\draw (23.000000\du,10.425000\du)--(23.000000\du,11.375000\du);
\pgfsetbuttcap
\pgfsetmiterjoin
\pgfsetdash{}{0pt}
\definecolor{dialinecolor}{rgb}{0.000000, 0.000000, 0.000000}
\pgfsetstrokecolor{dialinecolor}
\pgfsetstrokeopacity{1.000000}
\draw (22.400000\du,9.950000\du)--(22.400000\du,11.850000\du);
\pgfsetbuttcap
\pgfsetmiterjoin
\pgfsetdash{}{0pt}
\definecolor{dialinecolor}{rgb}{0.000000, 0.000000, 0.000000}
\pgfsetstrokecolor{dialinecolor}
\pgfsetstrokeopacity{1.000000}
\pgfsetbuttcap
\pgfsetmiterjoin
\pgfsetdash{}{0pt}
\definecolor{dialinecolor}{rgb}{0.000000, 0.000000, 0.000000}
\pgfsetstrokecolor{dialinecolor}
\pgfsetstrokeopacity{1.000000}
\pgfsetbuttcap
\pgfsetmiterjoin
\pgfsetdash{}{0pt}
\definecolor{dialinecolor}{rgb}{0.000000, 0.000000, 0.000000}
\pgfsetstrokecolor{dialinecolor}
\pgfsetstrokeopacity{1.000000}
\draw (22.400000\du,10.900000\du)--(20.900000\du,10.900000\du);
\pgfsetlinewidth{0.150000\du}
\pgfsetdash{}{0pt}
\pgfsetbuttcap
\pgfsetmiterjoin
\pgfsetlinewidth{0.150000\du}
\pgfsetbuttcap
\pgfsetmiterjoin
\pgfsetdash{}{0pt}
\definecolor{dialinecolor}{rgb}{0.000000, 0.000000, 0.000000}
\pgfsetstrokecolor{dialinecolor}
\pgfsetstrokeopacity{1.000000}
\draw (20.900000\du,17.550000\du)--(21.845000\du,17.550000\du)--(21.950000\du,17.050000\du)--(22.160000\du,18.050000\du)--(22.370000\du,17.050000\du)--(22.580000\du,18.050000\du)--(22.790000\du,17.050000\du)--(23.000000\du,18.050000\du)--(23.105000\du,17.550000\du)--(24.050000\du,17.550000\du);
\pgfsetlinewidth{0.150000\du}
\pgfsetdash{}{0pt}
\pgfsetmiterjoin
\pgfsetbuttcap
{
\definecolor{diafillcolor}{rgb}{0.000000, 0.000000, 0.000000}
\pgfsetfillcolor{diafillcolor}
\pgfsetfillopacity{1.000000}
% was here!!!
{\pgfsetcornersarced{\pgfpoint{0.000000\du}{0.000000\du}}\definecolor{dialinecolor}{rgb}{0.000000, 0.000000, 0.000000}
\pgfsetstrokecolor{dialinecolor}
\pgfsetstrokeopacity{1.000000}
\draw (24.499792\du,10.900000\du)--(27.400000\du,10.900000\du)--(27.400000\du,17.550000\du)--(24.048932\du,17.550000\du);
}}
\pgfsetlinewidth{0.150000\du}
\pgfsetdash{}{0pt}
\pgfsetmiterjoin
\pgfsetbuttcap
{
\definecolor{diafillcolor}{rgb}{0.000000, 0.000000, 0.000000}
\pgfsetfillcolor{diafillcolor}
\pgfsetfillopacity{1.000000}
% was here!!!
{\pgfsetcornersarced{\pgfpoint{0.000000\du}{0.000000\du}}\definecolor{dialinecolor}{rgb}{0.000000, 0.000000, 0.000000}
\pgfsetstrokecolor{dialinecolor}
\pgfsetstrokeopacity{1.000000}
\draw (20.899796\du,10.900000\du)--(17.350000\du,10.900000\du)--(17.350000\du,17.550000\du)--(20.900000\du,17.550000\du);
}}
\pgfsetlinewidth{0.100000\du}
\pgfsetdash{}{0pt}
\pgfsetbuttcap
{
\definecolor{diafillcolor}{rgb}{0.000000, 0.000000, 0.000000}
\pgfsetfillcolor{diafillcolor}
\pgfsetfillopacity{1.000000}
% was here!!!
\pgfsetarrowsend{stealth}
\definecolor{dialinecolor}{rgb}{0.000000, 0.000000, 0.000000}
\pgfsetstrokecolor{dialinecolor}
\pgfsetstrokeopacity{1.000000}
\draw (20.782700\du,10.243758\du)--(18.900000\du,10.226058\du);
}
% setfont left to latex
% setfont left to latex
\definecolor{dialinecolor}{rgb}{0.000000, 0.000000, 0.000000}
\pgfsetstrokecolor{dialinecolor}
\pgfsetstrokeopacity{1.000000}
\definecolor{diafillcolor}{rgb}{0.000000, 0.000000, 0.000000}
\pgfsetfillcolor{diafillcolor}
\pgfsetfillopacity{1.000000}
\node[anchor=base west,inner sep=0pt,outer sep=0pt,color=dialinecolor] at (19.933726\du,9.973023\du){$I$};
% setfont left to latex
% setfont left to latex
\definecolor{dialinecolor}{rgb}{0.000000, 0.000000, 0.000000}
\pgfsetstrokecolor{dialinecolor}
\pgfsetstrokeopacity{1.000000}
\definecolor{diafillcolor}{rgb}{0.000000, 0.000000, 0.000000}
\pgfsetfillcolor{diafillcolor}
\pgfsetfillopacity{1.000000}
\node[anchor=base west,inner sep=0pt,outer sep=0pt,color=dialinecolor] at (24.968004\du,10.011578\du){$I$};
\pgfsetlinewidth{0.100000\du}
\pgfsetdash{}{0pt}
\pgfsetbuttcap
{
\definecolor{diafillcolor}{rgb}{0.000000, 0.000000, 0.000000}
\pgfsetfillcolor{diafillcolor}
\pgfsetfillopacity{1.000000}
% was here!!!
\pgfsetarrowsend{stealth}
\definecolor{dialinecolor}{rgb}{0.000000, 0.000000, 0.000000}
\pgfsetstrokecolor{dialinecolor}
\pgfsetstrokeopacity{1.000000}
\draw (25.905551\du,10.243758\du)--(24.022851\du,10.226058\du);
}
% setfont left to latex
% setfont left to latex
\definecolor{dialinecolor}{rgb}{0.000000, 0.000000, 0.000000}
\pgfsetstrokecolor{dialinecolor}
\pgfsetstrokeopacity{1.000000}
\definecolor{diafillcolor}{rgb}{0.000000, 0.000000, 0.000000}
\pgfsetfillcolor{diafillcolor}
\pgfsetfillopacity{1.000000}
\node[anchor=base west,inner sep=0pt,outer sep=0pt,color=dialinecolor] at (21.811366\du,19.236633\du){4 ohms};
% setfont left to latex
% setfont left to latex
\definecolor{dialinecolor}{rgb}{0.000000, 0.000000, 0.000000}
\pgfsetstrokecolor{dialinecolor}
\pgfsetstrokeopacity{1.000000}
\definecolor{diafillcolor}{rgb}{0.000000, 0.000000, 0.000000}
\pgfsetfillcolor{diafillcolor}
\pgfsetfillopacity{1.000000}
\node[anchor=base west,inner sep=0pt,outer sep=0pt,color=dialinecolor] at (21.697628\du,12.700250\du){10 volts};
\end{tikzpicture}

    \caption{$4I = 10$}
\end{figure}

\begin{figure}[!h]
    \centering
    % Graphic for TeX using PGF
% Title: /home/jfreitas/GitHub/IAL/diagramas_aplicacoes/diagrama_circuito_quatro_resistores.dia
% Creator: Dia vDIA_0_97_0-2473-g1a1b882bd+
% CreationDate: Tue Apr 18 10:46:41 2023
% For: jfreitas
% \usepackage{tikz}
% The following commands are not supported in PSTricks at present
% We define them conditionally, so when they are implemented,
% this pgf file will use them.
\ifx\du\undefined
  \newlength{\du}
\fi
\setlength{\du}{15\unitlength}
\begin{tikzpicture}[even odd rule]
\pgftransformxscale{1.000000}
\pgftransformyscale{-1.000000}
\definecolor{dialinecolor}{rgb}{0.000000, 0.000000, 0.000000}
\pgfsetstrokecolor{dialinecolor}
\pgfsetstrokeopacity{1.000000}
\definecolor{diafillcolor}{rgb}{1.000000, 1.000000, 1.000000}
\pgfsetfillcolor{diafillcolor}
\pgfsetfillopacity{1.000000}
\pgfsetlinewidth{0.150000\du}
\pgfsetdash{}{0pt}
\pgfsetbuttcap
\pgfsetmiterjoin
\pgfsetlinewidth{0.150000\du}
\pgfsetbuttcap
\pgfsetmiterjoin
\pgfsetdash{}{0pt}
\definecolor{dialinecolor}{rgb}{0.000000, 0.000000, 0.000000}
\pgfsetstrokecolor{dialinecolor}
\pgfsetstrokeopacity{1.000000}
\draw (21.839525\du,9.725000\du)--(20.589525\du,9.725000\du);
\pgfsetbuttcap
\pgfsetmiterjoin
\pgfsetdash{}{0pt}
\definecolor{dialinecolor}{rgb}{0.000000, 0.000000, 0.000000}
\pgfsetstrokecolor{dialinecolor}
\pgfsetstrokeopacity{1.000000}
\draw (20.589525\du,9.337500\du)--(20.589525\du,10.112500\du);
\pgfsetbuttcap
\pgfsetmiterjoin
\pgfsetdash{}{0pt}
\definecolor{dialinecolor}{rgb}{0.000000, 0.000000, 0.000000}
\pgfsetstrokecolor{dialinecolor}
\pgfsetstrokeopacity{1.000000}
\draw (20.089525\du,8.950000\du)--(20.089525\du,10.500000\du);
\pgfsetbuttcap
\pgfsetmiterjoin
\pgfsetdash{}{0pt}
\definecolor{dialinecolor}{rgb}{0.000000, 0.000000, 0.000000}
\pgfsetstrokecolor{dialinecolor}
\pgfsetstrokeopacity{1.000000}
\draw (19.839525\du,9.337500\du)--(19.339525\du,9.337500\du);
\pgfsetbuttcap
\pgfsetmiterjoin
\pgfsetdash{}{0pt}
\definecolor{dialinecolor}{rgb}{0.000000, 0.000000, 0.000000}
\pgfsetstrokecolor{dialinecolor}
\pgfsetstrokeopacity{1.000000}
\draw (19.589525\du,9.531250\du)--(19.589525\du,9.143750\du);
\pgfsetbuttcap
\pgfsetmiterjoin
\pgfsetdash{}{0pt}
\definecolor{dialinecolor}{rgb}{0.000000, 0.000000, 0.000000}
\pgfsetstrokecolor{dialinecolor}
\pgfsetstrokeopacity{1.000000}
\draw (20.089525\du,9.725000\du)--(18.839525\du,9.725000\du);
\pgfsetlinewidth{0.150000\du}
\pgfsetdash{}{0pt}
\pgfsetbuttcap
\pgfsetmiterjoin
\pgfsetlinewidth{0.150000\du}
\pgfsetbuttcap
\pgfsetmiterjoin
\pgfsetdash{}{0pt}
\definecolor{dialinecolor}{rgb}{0.000000, 0.000000, 0.000000}
\pgfsetstrokecolor{dialinecolor}
\pgfsetstrokeopacity{1.000000}
\draw (22.098900\du,21.876000\du)--(20.848900\du,21.876000\du);
\pgfsetbuttcap
\pgfsetmiterjoin
\pgfsetdash{}{0pt}
\definecolor{dialinecolor}{rgb}{0.000000, 0.000000, 0.000000}
\pgfsetstrokecolor{dialinecolor}
\pgfsetstrokeopacity{1.000000}
\draw (20.848900\du,21.488500\du)--(20.848900\du,22.263500\du);
\pgfsetbuttcap
\pgfsetmiterjoin
\pgfsetdash{}{0pt}
\definecolor{dialinecolor}{rgb}{0.000000, 0.000000, 0.000000}
\pgfsetstrokecolor{dialinecolor}
\pgfsetstrokeopacity{1.000000}
\draw (20.348900\du,21.101000\du)--(20.348900\du,22.651000\du);
\pgfsetbuttcap
\pgfsetmiterjoin
\pgfsetdash{}{0pt}
\definecolor{dialinecolor}{rgb}{0.000000, 0.000000, 0.000000}
\pgfsetstrokecolor{dialinecolor}
\pgfsetstrokeopacity{1.000000}
\draw (20.098900\du,21.488500\du)--(19.598900\du,21.488500\du);
\pgfsetbuttcap
\pgfsetmiterjoin
\pgfsetdash{}{0pt}
\definecolor{dialinecolor}{rgb}{0.000000, 0.000000, 0.000000}
\pgfsetstrokecolor{dialinecolor}
\pgfsetstrokeopacity{1.000000}
\draw (19.848900\du,21.682250\du)--(19.848900\du,21.294750\du);
\pgfsetbuttcap
\pgfsetmiterjoin
\pgfsetdash{}{0pt}
\definecolor{dialinecolor}{rgb}{0.000000, 0.000000, 0.000000}
\pgfsetstrokecolor{dialinecolor}
\pgfsetstrokeopacity{1.000000}
\draw (20.348900\du,21.876000\du)--(19.098900\du,21.876000\du);
\pgfsetlinewidth{0.150000\du}
\pgfsetdash{}{0pt}
\pgfsetbuttcap
\pgfsetmiterjoin
\pgfsetlinewidth{0.150000\du}
\pgfsetbuttcap
\pgfsetmiterjoin
\pgfsetdash{}{0pt}
\definecolor{dialinecolor}{rgb}{0.000000, 0.000000, 0.000000}
\pgfsetstrokecolor{dialinecolor}
\pgfsetstrokeopacity{1.000000}
\draw (27.606900\du,11.243200\du)--(27.606900\du,12.143200\du)--(27.106900\du,12.243200\du)--(28.106900\du,12.443200\du)--(27.106900\du,12.643200\du)--(28.106900\du,12.843200\du)--(27.106900\du,13.043200\du)--(28.106900\du,13.243200\du)--(27.606900\du,13.343200\du)--(27.606900\du,14.243200\du);
\pgfsetlinewidth{0.150000\du}
\pgfsetdash{}{0pt}
\pgfsetbuttcap
\pgfsetmiterjoin
\pgfsetlinewidth{0.150000\du}
\pgfsetbuttcap
\pgfsetmiterjoin
\pgfsetdash{}{0pt}
\definecolor{dialinecolor}{rgb}{0.000000, 0.000000, 0.000000}
\pgfsetstrokecolor{dialinecolor}
\pgfsetstrokeopacity{1.000000}
\draw (13.996800\du,11.209000\du)--(13.996800\du,12.109000\du)--(13.496800\du,12.209000\du)--(14.496800\du,12.409000\du)--(13.496800\du,12.609000\du)--(14.496800\du,12.809000\du)--(13.496800\du,13.009000\du)--(14.496800\du,13.209000\du)--(13.996800\du,13.309000\du)--(13.996800\du,14.209000\du);
\pgfsetlinewidth{0.150000\du}
\pgfsetdash{}{0pt}
\pgfsetbuttcap
\pgfsetmiterjoin
\pgfsetlinewidth{0.150000\du}
\pgfsetbuttcap
\pgfsetmiterjoin
\pgfsetdash{}{0pt}
\definecolor{dialinecolor}{rgb}{0.000000, 0.000000, 0.000000}
\pgfsetstrokecolor{dialinecolor}
\pgfsetstrokeopacity{1.000000}
\draw (19.168900\du,15.845900\du)--(20.034047\du,15.845900\du)--(20.130174\du,15.345900\du)--(20.322429\du,16.345900\du)--(20.514684\du,15.345900\du)--(20.706939\du,16.345900\du)--(20.899194\du,15.345900\du)--(21.091449\du,16.345900\du)--(21.187576\du,15.845900\du)--(22.052723\du,15.845900\du);
\pgfsetlinewidth{0.150000\du}
\pgfsetdash{}{0pt}
\pgfsetbuttcap
\pgfsetmiterjoin
\pgfsetlinewidth{0.150000\du}
\pgfsetbuttcap
\pgfsetmiterjoin
\pgfsetdash{}{0pt}
\definecolor{dialinecolor}{rgb}{0.000000, 0.000000, 0.000000}
\pgfsetstrokecolor{dialinecolor}
\pgfsetstrokeopacity{1.000000}
\draw (14.004200\du,17.816500\du)--(14.004200\du,18.716500\du)--(13.504200\du,18.816500\du)--(14.504200\du,19.016500\du)--(13.504200\du,19.216500\du)--(14.504200\du,19.416500\du)--(13.504200\du,19.616500\du)--(14.504200\du,19.816500\du)--(14.004200\du,19.916500\du)--(14.004200\du,20.816500\du);
\pgfsetlinewidth{0.150000\du}
\pgfsetdash{}{0pt}
\pgfsetmiterjoin
\pgfsetbuttcap
{
\definecolor{diafillcolor}{rgb}{0.000000, 0.000000, 0.000000}
\pgfsetfillcolor{diafillcolor}
\pgfsetfillopacity{1.000000}
% was here!!!
{\pgfsetcornersarced{\pgfpoint{0.000000\du}{0.000000\du}}\definecolor{dialinecolor}{rgb}{0.000000, 0.000000, 0.000000}
\pgfsetstrokecolor{dialinecolor}
\pgfsetstrokeopacity{1.000000}
\draw (21.839525\du,9.725000\du)--(22.064300\du,9.725000\du)--(22.064300\du,9.726550\du)--(27.606900\du,9.726550\du)--(27.606900\du,11.243200\du);
}}
\pgfsetlinewidth{0.150000\du}
\pgfsetdash{}{0pt}
\pgfsetmiterjoin
\pgfsetbuttcap
{
\definecolor{diafillcolor}{rgb}{0.000000, 0.000000, 0.000000}
\pgfsetfillcolor{diafillcolor}
\pgfsetfillopacity{1.000000}
% was here!!!
{\pgfsetcornersarced{\pgfpoint{0.000000\du}{0.000000\du}}\definecolor{dialinecolor}{rgb}{0.000000, 0.000000, 0.000000}
\pgfsetstrokecolor{dialinecolor}
\pgfsetstrokeopacity{1.000000}
\draw (27.606900\du,14.243200\du)--(27.606900\du,20.066800\du)--(27.607400\du,20.066800\du)--(27.607400\du,21.876000\du)--(22.098900\du,21.876000\du);
}}
\pgfsetlinewidth{0.150000\du}
\pgfsetdash{}{0pt}
\pgfsetmiterjoin
\pgfsetbuttcap
{
\definecolor{diafillcolor}{rgb}{0.000000, 0.000000, 0.000000}
\pgfsetfillcolor{diafillcolor}
\pgfsetfillopacity{1.000000}
% was here!!!
{\pgfsetcornersarced{\pgfpoint{0.000000\du}{0.000000\du}}\definecolor{dialinecolor}{rgb}{0.000000, 0.000000, 0.000000}
\pgfsetstrokecolor{dialinecolor}
\pgfsetstrokeopacity{1.000000}
\draw (19.098900\du,21.876000\du)--(19.134700\du,21.876000\du)--(19.134700\du,21.865000\du)--(14.004200\du,21.865000\du)--(14.004200\du,20.816500\du);
}}
\pgfsetlinewidth{0.150000\du}
\pgfsetdash{}{0pt}
\pgfsetmiterjoin
\pgfsetbuttcap
{
\definecolor{diafillcolor}{rgb}{0.000000, 0.000000, 0.000000}
\pgfsetfillcolor{diafillcolor}
\pgfsetfillopacity{1.000000}
% was here!!!
{\pgfsetcornersarced{\pgfpoint{0.000000\du}{0.000000\du}}\definecolor{dialinecolor}{rgb}{0.000000, 0.000000, 0.000000}
\pgfsetstrokecolor{dialinecolor}
\pgfsetstrokeopacity{1.000000}
\draw (13.996800\du,13.968100\du)--(13.996800\du,16.012700\du)--(14.004200\du,16.012700\du)--(14.004200\du,18.057400\du);
}}
\pgfsetlinewidth{0.150000\du}
\pgfsetdash{}{0pt}
\pgfsetmiterjoin
\pgfsetbuttcap
{
\definecolor{diafillcolor}{rgb}{0.000000, 0.000000, 0.000000}
\pgfsetfillcolor{diafillcolor}
\pgfsetfillopacity{1.000000}
% was here!!!
{\pgfsetcornersarced{\pgfpoint{0.000000\du}{0.000000\du}}\definecolor{dialinecolor}{rgb}{0.000000, 0.000000, 0.000000}
\pgfsetstrokecolor{dialinecolor}
\pgfsetstrokeopacity{1.000000}
\draw (19.785800\du,15.845900\du)--(19.785800\du,15.847900\du)--(18.837400\du,15.847900\du)--(18.837400\du,15.847900\du)--(13.981200\du,15.847900\du)--(13.981200\du,15.813900\du);
}}
\pgfsetlinewidth{0.150000\du}
\pgfsetdash{}{0pt}
\pgfsetmiterjoin
\pgfsetbuttcap
{
\definecolor{diafillcolor}{rgb}{0.000000, 0.000000, 0.000000}
\pgfsetfillcolor{diafillcolor}
\pgfsetfillopacity{1.000000}
% was here!!!
{\pgfsetcornersarced{\pgfpoint{0.000000\du}{0.000000\du}}\definecolor{dialinecolor}{rgb}{0.000000, 0.000000, 0.000000}
\pgfsetstrokecolor{dialinecolor}
\pgfsetstrokeopacity{1.000000}
\draw (13.996800\du,11.209000\du)--(13.996800\du,9.722110\du)--(19.031200\du,9.722110\du)--(19.031200\du,9.718750\du);
}}
\pgfsetlinewidth{0.150000\du}
\pgfsetdash{}{0pt}
\pgfsetmiterjoin
\pgfsetbuttcap
{
\definecolor{diafillcolor}{rgb}{0.000000, 0.000000, 0.000000}
\pgfsetfillcolor{diafillcolor}
\pgfsetfillopacity{1.000000}
% was here!!!
{\pgfsetcornersarced{\pgfpoint{0.000000\du}{0.000000\du}}\definecolor{dialinecolor}{rgb}{0.000000, 0.000000, 0.000000}
\pgfsetstrokecolor{dialinecolor}
\pgfsetstrokeopacity{1.000000}
\draw (27.598200\du,15.843800\du)--(27.598200\du,15.845800\du)--(26.649800\du,15.845800\du)--(26.649800\du,15.845800\du)--(21.770400\du,15.845800\du)--(21.770400\du,15.828700\du);
}}
\pgfsetlinewidth{0.150000\du}
\pgfsetdash{}{0pt}
\pgfsetbuttcap
\pgfsetmiterjoin
\pgfsetlinewidth{0.150000\du}
\pgfsetbuttcap
\pgfsetmiterjoin
\pgfsetdash{}{0pt}
\definecolor{diafillcolor}{rgb}{0.000000, 0.000000, 0.000000}
\pgfsetfillcolor{diafillcolor}
\pgfsetfillopacity{1.000000}
\pgfpathellipse{\pgfpoint{14.019283\du}{15.827583\du}}{\pgfpoint{0.222183\du}{0\du}}{\pgfpoint{0\du}{0.222183\du}}
\pgfusepath{fill}
\definecolor{dialinecolor}{rgb}{0.000000, 0.000000, 0.000000}
\pgfsetstrokecolor{dialinecolor}
\pgfsetstrokeopacity{1.000000}
\pgfpathellipse{\pgfpoint{14.019283\du}{15.827583\du}}{\pgfpoint{0.222183\du}{0\du}}{\pgfpoint{0\du}{0.222183\du}}
\pgfusepath{stroke}
\pgfsetlinewidth{0.015000\du}
\pgfsetbuttcap
\pgfsetmiterjoin
\pgfsetdash{}{0pt}
\definecolor{dialinecolor}{rgb}{0.000000, 0.000000, 0.000000}
\pgfsetstrokecolor{dialinecolor}
\pgfsetstrokeopacity{1.000000}
\pgfpathellipse{\pgfpoint{14.019283\du}{15.827583\du}}{\pgfpoint{0.222183\du}{0\du}}{\pgfpoint{0\du}{0.222183\du}}
\pgfusepath{stroke}
\pgfsetlinewidth{0.150000\du}
\pgfsetdash{}{0pt}
\pgfsetbuttcap
\pgfsetmiterjoin
\pgfsetlinewidth{0.150000\du}
\pgfsetbuttcap
\pgfsetmiterjoin
\pgfsetdash{}{0pt}
\definecolor{diafillcolor}{rgb}{0.000000, 0.000000, 0.000000}
\pgfsetfillcolor{diafillcolor}
\pgfsetfillopacity{1.000000}
\pgfpathellipse{\pgfpoint{27.575783\du}{15.829983\du}}{\pgfpoint{0.222183\du}{0\du}}{\pgfpoint{0\du}{0.222183\du}}
\pgfusepath{fill}
\definecolor{dialinecolor}{rgb}{0.000000, 0.000000, 0.000000}
\pgfsetstrokecolor{dialinecolor}
\pgfsetstrokeopacity{1.000000}
\pgfpathellipse{\pgfpoint{27.575783\du}{15.829983\du}}{\pgfpoint{0.222183\du}{0\du}}{\pgfpoint{0\du}{0.222183\du}}
\pgfusepath{stroke}
\pgfsetlinewidth{0.015000\du}
\pgfsetbuttcap
\pgfsetmiterjoin
\pgfsetdash{}{0pt}
\definecolor{dialinecolor}{rgb}{0.000000, 0.000000, 0.000000}
\pgfsetstrokecolor{dialinecolor}
\pgfsetstrokeopacity{1.000000}
\pgfpathellipse{\pgfpoint{27.575783\du}{15.829983\du}}{\pgfpoint{0.222183\du}{0\du}}{\pgfpoint{0\du}{0.222183\du}}
\pgfusepath{stroke}
% setfont left to latex
% setfont left to latex
\definecolor{dialinecolor}{rgb}{0.000000, 0.000000, 0.000000}
\pgfsetstrokecolor{dialinecolor}
\pgfsetstrokeopacity{1.000000}
\definecolor{diafillcolor}{rgb}{0.000000, 0.000000, 0.000000}
\pgfsetfillcolor{diafillcolor}
\pgfsetfillopacity{1.000000}
\node[anchor=base west,inner sep=0pt,outer sep=0pt,color=dialinecolor] at (19.556400\du,11.084300\du){8 volts};
% setfont left to latex
% setfont left to latex
\definecolor{dialinecolor}{rgb}{0.000000, 0.000000, 0.000000}
\pgfsetstrokecolor{dialinecolor}
\pgfsetstrokeopacity{1.000000}
\definecolor{diafillcolor}{rgb}{0.000000, 0.000000, 0.000000}
\pgfsetfillcolor{diafillcolor}
\pgfsetfillopacity{1.000000}
\node[anchor=base west,inner sep=0pt,outer sep=0pt,color=dialinecolor] at (20.288900\du,8.612010\du){C};
% setfont left to latex
% setfont left to latex
\definecolor{dialinecolor}{rgb}{0.000000, 0.000000, 0.000000}
\pgfsetstrokecolor{dialinecolor}
\pgfsetstrokeopacity{1.000000}
\definecolor{diafillcolor}{rgb}{0.000000, 0.000000, 0.000000}
\pgfsetfillcolor{diafillcolor}
\pgfsetfillopacity{1.000000}
\node[anchor=base west,inner sep=0pt,outer sep=0pt,color=dialinecolor] at (10.511089\du,12.765470\du){2 ohms};
% setfont left to latex
% setfont left to latex
\definecolor{dialinecolor}{rgb}{0.000000, 0.000000, 0.000000}
\pgfsetstrokecolor{dialinecolor}
\pgfsetstrokeopacity{1.000000}
\definecolor{diafillcolor}{rgb}{0.000000, 0.000000, 0.000000}
\pgfsetfillcolor{diafillcolor}
\pgfsetfillopacity{1.000000}
\node[anchor=base west,inner sep=0pt,outer sep=0pt,color=dialinecolor] at (10.610219\du,19.513390\du){4 ohms};
% setfont left to latex
% setfont left to latex
\definecolor{dialinecolor}{rgb}{0.000000, 0.000000, 0.000000}
\pgfsetstrokecolor{dialinecolor}
\pgfsetstrokeopacity{1.000000}
\definecolor{diafillcolor}{rgb}{0.000000, 0.000000, 0.000000}
\pgfsetfillcolor{diafillcolor}
\pgfsetfillopacity{1.000000}
\node[anchor=base west,inner sep=0pt,outer sep=0pt,color=dialinecolor] at (19.672012\du,23.547100\du){16 volts};
% setfont left to latex
% setfont left to latex
\definecolor{dialinecolor}{rgb}{0.000000, 0.000000, 0.000000}
\pgfsetstrokecolor{dialinecolor}
\pgfsetstrokeopacity{1.000000}
\definecolor{diafillcolor}{rgb}{0.000000, 0.000000, 0.000000}
\pgfsetfillcolor{diafillcolor}
\pgfsetfillopacity{1.000000}
\node[anchor=base west,inner sep=0pt,outer sep=0pt,color=dialinecolor] at (20.366000\du,20.636300\du){D};
% setfont left to latex
% setfont left to latex
\definecolor{dialinecolor}{rgb}{0.000000, 0.000000, 0.000000}
\pgfsetstrokecolor{dialinecolor}
\pgfsetstrokeopacity{1.000000}
\definecolor{diafillcolor}{rgb}{0.000000, 0.000000, 0.000000}
\pgfsetfillcolor{diafillcolor}
\pgfsetfillopacity{1.000000}
\node[anchor=base west,inner sep=0pt,outer sep=0pt,color=dialinecolor] at (19.556314\du,17.320500\du){1 ohm};
% setfont left to latex
% setfont left to latex
\definecolor{dialinecolor}{rgb}{0.000000, 0.000000, 0.000000}
\pgfsetstrokecolor{dialinecolor}
\pgfsetstrokeopacity{1.000000}
\definecolor{diafillcolor}{rgb}{0.000000, 0.000000, 0.000000}
\pgfsetfillcolor{diafillcolor}
\pgfsetfillopacity{1.000000}
\node[anchor=base west,inner sep=0pt,outer sep=0pt,color=dialinecolor] at (13.098400\du,15.932600\du){A};
% setfont left to latex
% setfont left to latex
\definecolor{dialinecolor}{rgb}{0.000000, 0.000000, 0.000000}
\pgfsetstrokecolor{dialinecolor}
\pgfsetstrokeopacity{1.000000}
\definecolor{diafillcolor}{rgb}{0.000000, 0.000000, 0.000000}
\pgfsetfillcolor{diafillcolor}
\pgfsetfillopacity{1.000000}
\node[anchor=base west,inner sep=0pt,outer sep=0pt,color=dialinecolor] at (28.049900\du,15.880500\du){B};
% setfont left to latex
% setfont left to latex
\definecolor{dialinecolor}{rgb}{0.000000, 0.000000, 0.000000}
\pgfsetstrokecolor{dialinecolor}
\pgfsetstrokeopacity{1.000000}
\definecolor{diafillcolor}{rgb}{0.000000, 0.000000, 0.000000}
\pgfsetfillcolor{diafillcolor}
\pgfsetfillopacity{1.000000}
\node[anchor=base west,inner sep=0pt,outer sep=0pt,color=dialinecolor] at (28.501000\du,15.682000\du){};
% setfont left to latex
% setfont left to latex
\definecolor{dialinecolor}{rgb}{0.000000, 0.000000, 0.000000}
\pgfsetstrokecolor{dialinecolor}
\pgfsetstrokeopacity{1.000000}
\definecolor{diafillcolor}{rgb}{0.000000, 0.000000, 0.000000}
\pgfsetfillcolor{diafillcolor}
\pgfsetfillopacity{1.000000}
\node[anchor=base west,inner sep=0pt,outer sep=0pt,color=dialinecolor] at (28.757650\du,12.861940\du){2 ohms};
\pgfsetlinewidth{0.150000\du}
\pgfsetdash{}{0pt}
\pgfsetbuttcap
{
\definecolor{diafillcolor}{rgb}{0.000000, 0.000000, 0.000000}
\pgfsetfillcolor{diafillcolor}
\pgfsetfillopacity{1.000000}
% was here!!!
\pgfsetarrowsstart{stealth}
\definecolor{dialinecolor}{rgb}{0.000000, 0.000000, 0.000000}
\pgfsetstrokecolor{dialinecolor}
\pgfsetstrokeopacity{1.000000}
\draw (16.063275\du,8.923363\du)--(17.913875\du,8.923363\du);
}
% setfont left to latex
% setfont left to latex
\definecolor{dialinecolor}{rgb}{0.000000, 0.000000, 0.000000}
\pgfsetstrokecolor{dialinecolor}
\pgfsetstrokeopacity{1.000000}
\definecolor{diafillcolor}{rgb}{0.000000, 0.000000, 0.000000}
\pgfsetfillcolor{diafillcolor}
\pgfsetfillopacity{1.000000}
\node[anchor=base west,inner sep=0pt,outer sep=0pt,color=dialinecolor] at (17.106246\du,20.653753\du){$I_3$};
% setfont left to latex
% setfont left to latex
\definecolor{dialinecolor}{rgb}{0.000000, 0.000000, 0.000000}
\pgfsetstrokecolor{dialinecolor}
\pgfsetstrokeopacity{1.000000}
\definecolor{diafillcolor}{rgb}{0.000000, 0.000000, 0.000000}
\pgfsetfillcolor{diafillcolor}
\pgfsetfillopacity{1.000000}
\node[anchor=base west,inner sep=0pt,outer sep=0pt,color=dialinecolor] at (24.062700\du,20.712100\du){$I_3$};
% setfont left to latex
% setfont left to latex
\definecolor{dialinecolor}{rgb}{0.000000, 0.000000, 0.000000}
\pgfsetstrokecolor{dialinecolor}
\pgfsetstrokeopacity{1.000000}
\definecolor{diafillcolor}{rgb}{0.000000, 0.000000, 0.000000}
\pgfsetfillcolor{diafillcolor}
\pgfsetfillopacity{1.000000}
\node[anchor=base west,inner sep=0pt,outer sep=0pt,color=dialinecolor] at (23.234302\du,14.682688\du){$I_2$};
% setfont left to latex
% setfont left to latex
\definecolor{dialinecolor}{rgb}{0.000000, 0.000000, 0.000000}
\pgfsetstrokecolor{dialinecolor}
\pgfsetstrokeopacity{1.000000}
\definecolor{diafillcolor}{rgb}{0.000000, 0.000000, 0.000000}
\pgfsetfillcolor{diafillcolor}
\pgfsetfillopacity{1.000000}
\node[anchor=base west,inner sep=0pt,outer sep=0pt,color=dialinecolor] at (16.504218\du,14.666363\du){$I_2$};
% setfont left to latex
% setfont left to latex
\definecolor{dialinecolor}{rgb}{0.000000, 0.000000, 0.000000}
\pgfsetstrokecolor{dialinecolor}
\pgfsetstrokeopacity{1.000000}
\definecolor{diafillcolor}{rgb}{0.000000, 0.000000, 0.000000}
\pgfsetfillcolor{diafillcolor}
\pgfsetfillopacity{1.000000}
\node[anchor=base west,inner sep=0pt,outer sep=0pt,color=dialinecolor] at (17.035403\du,8.550975\du){$I_1$};
% setfont left to latex
% setfont left to latex
\definecolor{dialinecolor}{rgb}{0.000000, 0.000000, 0.000000}
\pgfsetstrokecolor{dialinecolor}
\pgfsetstrokeopacity{1.000000}
\definecolor{diafillcolor}{rgb}{0.000000, 0.000000, 0.000000}
\pgfsetfillcolor{diafillcolor}
\pgfsetfillopacity{1.000000}
\node[anchor=base west,inner sep=0pt,outer sep=0pt,color=dialinecolor] at (23.723553\du,8.599471\du){$I_1$};
\pgfsetlinewidth{0.150000\du}
\pgfsetdash{}{0pt}
\pgfsetbuttcap
{
\definecolor{diafillcolor}{rgb}{0.000000, 0.000000, 0.000000}
\pgfsetfillcolor{diafillcolor}
\pgfsetfillopacity{1.000000}
% was here!!!
\pgfsetarrowsstart{stealth}
\definecolor{dialinecolor}{rgb}{0.000000, 0.000000, 0.000000}
\pgfsetstrokecolor{dialinecolor}
\pgfsetstrokeopacity{1.000000}
\draw (22.629468\du,8.971989\du)--(24.480068\du,8.971989\du);
}
\pgfsetlinewidth{0.150000\du}
\pgfsetdash{}{0pt}
\pgfsetbuttcap
{
\definecolor{diafillcolor}{rgb}{0.000000, 0.000000, 0.000000}
\pgfsetfillcolor{diafillcolor}
\pgfsetfillopacity{1.000000}
% was here!!!
\pgfsetarrowsend{stealth}
\definecolor{dialinecolor}{rgb}{0.000000, 0.000000, 0.000000}
\pgfsetstrokecolor{dialinecolor}
\pgfsetstrokeopacity{1.000000}
\draw (16.093821\du,14.941152\du)--(17.944421\du,14.941152\du);
}
\pgfsetlinewidth{0.150000\du}
\pgfsetdash{}{0pt}
\pgfsetbuttcap
{
\definecolor{diafillcolor}{rgb}{0.000000, 0.000000, 0.000000}
\pgfsetfillcolor{diafillcolor}
\pgfsetfillopacity{1.000000}
% was here!!!
\pgfsetarrowsend{stealth}
\definecolor{dialinecolor}{rgb}{0.000000, 0.000000, 0.000000}
\pgfsetstrokecolor{dialinecolor}
\pgfsetstrokeopacity{1.000000}
\draw (23.012551\du,14.992976\du)--(24.863151\du,14.992976\du);
}
\pgfsetlinewidth{0.150000\du}
\pgfsetdash{}{0pt}
\pgfsetbuttcap
{
\definecolor{diafillcolor}{rgb}{0.000000, 0.000000, 0.000000}
\pgfsetfillcolor{diafillcolor}
\pgfsetfillopacity{1.000000}
% was here!!!
\pgfsetarrowsstart{stealth}
\definecolor{dialinecolor}{rgb}{0.000000, 0.000000, 0.000000}
\pgfsetstrokecolor{dialinecolor}
\pgfsetstrokeopacity{1.000000}
\draw (16.057985\du,20.973837\du)--(17.908585\du,20.973837\du);
}
\pgfsetlinewidth{0.150000\du}
\pgfsetdash{}{0pt}
\pgfsetbuttcap
{
\definecolor{diafillcolor}{rgb}{0.000000, 0.000000, 0.000000}
\pgfsetfillcolor{diafillcolor}
\pgfsetfillopacity{1.000000}
% was here!!!
\pgfsetarrowsstart{stealth}
\definecolor{dialinecolor}{rgb}{0.000000, 0.000000, 0.000000}
\pgfsetstrokecolor{dialinecolor}
\pgfsetstrokeopacity{1.000000}
\draw (23.091589\du,21.010464\du)--(24.942189\du,21.010464\du);
}
\end{tikzpicture}

\end{figure}

Neste caso a corrente $I_1$ flui pelo ramo superior $BCA$. A corrente $I_2$ pelo ramo do meio, $AB$. A corrente $I_3$ pelo ramo inferior, $BDA$. No nó $A$ temos: $I_1 + I_3 = I_2$. (Que é a mesma equação no nó $B$.)  Essa equação pode ser escrita como

\[
    I_1 - I_2 + I_3 = 0.
\]


Agora vamos aplicar a lei da voltagem para cada circuito. Para o circuito $CABC$, as quedas de voltagem nos resistores são $2I_1$, $I_2$ e $2I_1$. Assim temos a equação
\[
    4I_1 + I_2 = 8.
\]

Para o circuito $DABD$, as quedas de voltagem nos resistores são $4I_3$, $I_2$. Assim temos a equação
\[
    4I_3 + I_2 = 16.
\]

Com isso temos o seguinte sistema linear.
\[
    \begin{cases}
        I_1 - I_2 + I_3 = 0\\
        4I_1 + I_2 = 8\\
        I_2 + 4I_3 = 16
    \end{cases}
\]
Aplicando o método de eliminação de Gauss-Jordan obtemos a solução
\begin{center}
    $I_1 = 1 \mbox{ ampere}$, $I_2 = 4 \mbox{ amperes}$ $I_1 = 3 \mbox{ amperes}$.
\end{center}

\begin{figure}[!h]
    \centering
    % Graphic for TeX using PGF
% Title: C:\Users\josea\GitHub\IAL\diagramas_aplicacoes\diagrama_circuito_uma_unica_fonte_energia.dia
% Creator: Dia v0.97.2
% CreationDate: Tue Apr 18 14:09:16 2023
% For: josea
% \usepackage{tikz}
% The following commands are not supported in PSTricks at present
% We define them conditionally, so when they are implemented,
% this pgf file will use them.
\ifx\du\undefined
  \newlength{\du}
\fi
\setlength{\du}{15\unitlength}
\begin{tikzpicture}
\pgftransformxscale{1.000000}
\pgftransformyscale{-1.000000}
\definecolor{dialinecolor}{rgb}{0.000000, 0.000000, 0.000000}
\pgfsetstrokecolor{dialinecolor}
\definecolor{dialinecolor}{rgb}{1.000000, 1.000000, 1.000000}
\pgfsetfillcolor{dialinecolor}
\pgfsetlinewidth{0.100000\du}
\pgfsetdash{}{0pt}
\pgfsetdash{}{0pt}
\pgfsetbuttcap
\pgfsetmiterjoin
\pgfsetlinewidth{0.100000\du}
\pgfsetbuttcap
\pgfsetmiterjoin
\pgfsetdash{}{0pt}
\definecolor{dialinecolor}{rgb}{0.000000, 0.000000, 0.000000}
\pgfsetstrokecolor{dialinecolor}
\draw (18.525300\du,19.929400\du)--(17.275300\du,19.929400\du);
\pgfsetbuttcap
\pgfsetmiterjoin
\pgfsetdash{}{0pt}
\definecolor{dialinecolor}{rgb}{0.000000, 0.000000, 0.000000}
\pgfsetstrokecolor{dialinecolor}
\draw (17.275300\du,19.679400\du)--(17.275300\du,20.179400\du);
\pgfsetbuttcap
\pgfsetmiterjoin
\pgfsetdash{}{0pt}
\definecolor{dialinecolor}{rgb}{0.000000, 0.000000, 0.000000}
\pgfsetstrokecolor{dialinecolor}
\draw (16.775300\du,19.429400\du)--(16.775300\du,20.429400\du);
\pgfsetbuttcap
\pgfsetmiterjoin
\pgfsetdash{}{0pt}
\definecolor{dialinecolor}{rgb}{0.000000, 0.000000, 0.000000}
\pgfsetstrokecolor{dialinecolor}
\pgfsetbuttcap
\pgfsetmiterjoin
\pgfsetdash{}{0pt}
\definecolor{dialinecolor}{rgb}{0.000000, 0.000000, 0.000000}
\pgfsetstrokecolor{dialinecolor}
\pgfsetbuttcap
\pgfsetmiterjoin
\pgfsetdash{}{0pt}
\definecolor{dialinecolor}{rgb}{0.000000, 0.000000, 0.000000}
\pgfsetstrokecolor{dialinecolor}
\draw (16.775300\du,19.929400\du)--(15.525300\du,19.929400\du);
\pgfsetlinewidth{0.100000\du}
\pgfsetdash{}{0pt}
\pgfsetdash{}{0pt}
\pgfsetbuttcap
\pgfsetmiterjoin
\pgfsetlinewidth{0.100000\du}
\pgfsetbuttcap
\pgfsetmiterjoin
\pgfsetdash{}{0pt}
\definecolor{dialinecolor}{rgb}{0.000000, 0.000000, 0.000000}
\pgfsetstrokecolor{dialinecolor}
\draw (10.414421\du,15.035526\du)--(11.314421\du,15.035526\du)--(11.414421\du,14.535526\du)--(11.614421\du,15.535526\du)--(11.814421\du,14.535526\du)--(12.014421\du,15.535526\du)--(12.214421\du,14.535526\du)--(12.414421\du,15.535526\du)--(12.514421\du,15.035526\du)--(13.414421\du,15.035526\du);
\pgfsetlinewidth{0.100000\du}
\pgfsetdash{}{0pt}
\pgfsetdash{}{0pt}
\pgfsetbuttcap
\pgfsetmiterjoin
\pgfsetlinewidth{0.100000\du}
\pgfsetbuttcap
\pgfsetmiterjoin
\pgfsetdash{}{0pt}
\definecolor{dialinecolor}{rgb}{0.000000, 0.000000, 0.000000}
\pgfsetstrokecolor{dialinecolor}
\draw (20.010392\du,15.053421\du)--(20.910392\du,15.053421\du)--(21.010392\du,14.553421\du)--(21.210392\du,15.553421\du)--(21.410392\du,14.553421\du)--(21.610392\du,15.553421\du)--(21.810392\du,14.553421\du)--(22.010392\du,15.553421\du)--(22.110392\du,15.053421\du)--(23.010392\du,15.053421\du);
\pgfsetlinewidth{0.100000\du}
\pgfsetdash{}{0pt}
\pgfsetdash{}{0pt}
\pgfsetbuttcap
\pgfsetmiterjoin
\pgfsetlinewidth{0.100000\du}
\pgfsetbuttcap
\pgfsetmiterjoin
\pgfsetdash{}{0pt}
\definecolor{dialinecolor}{rgb}{0.000000, 0.000000, 0.000000}
\pgfsetfillcolor{dialinecolor}
\pgfpathellipse{\pgfpoint{17.031514\du}{15.036714\du}}{\pgfpoint{0.188214\du}{0\du}}{\pgfpoint{0\du}{0.188214\du}}
\pgfusepath{fill}
\definecolor{dialinecolor}{rgb}{0.000000, 0.000000, 0.000000}
\pgfsetstrokecolor{dialinecolor}
\pgfpathellipse{\pgfpoint{17.031514\du}{15.036714\du}}{\pgfpoint{0.188214\du}{0\du}}{\pgfpoint{0\du}{0.188214\du}}
\pgfusepath{stroke}
\pgfsetlinewidth{0.010000\du}
\pgfsetbuttcap
\pgfsetmiterjoin
\pgfsetdash{}{0pt}
\definecolor{dialinecolor}{rgb}{0.000000, 0.000000, 0.000000}
\pgfsetstrokecolor{dialinecolor}
\pgfpathellipse{\pgfpoint{17.031514\du}{15.036714\du}}{\pgfpoint{0.188214\du}{0\du}}{\pgfpoint{0\du}{0.188214\du}}
\pgfusepath{stroke}
\pgfsetlinewidth{0.100000\du}
\pgfsetdash{}{0pt}
\pgfsetdash{}{0pt}
\pgfsetbuttcap
\pgfsetmiterjoin
\pgfsetlinewidth{0.100000\du}
\pgfsetbuttcap
\pgfsetmiterjoin
\pgfsetdash{}{0pt}
\definecolor{dialinecolor}{rgb}{0.000000, 0.000000, 0.000000}
\pgfsetfillcolor{dialinecolor}
\pgfpathellipse{\pgfpoint{17.027814\du}{6.002254\du}}{\pgfpoint{0.188214\du}{0\du}}{\pgfpoint{0\du}{0.188214\du}}
\pgfusepath{fill}
\definecolor{dialinecolor}{rgb}{0.000000, 0.000000, 0.000000}
\pgfsetstrokecolor{dialinecolor}
\pgfpathellipse{\pgfpoint{17.027814\du}{6.002254\du}}{\pgfpoint{0.188214\du}{0\du}}{\pgfpoint{0\du}{0.188214\du}}
\pgfusepath{stroke}
\pgfsetlinewidth{0.010000\du}
\pgfsetbuttcap
\pgfsetmiterjoin
\pgfsetdash{}{0pt}
\definecolor{dialinecolor}{rgb}{0.000000, 0.000000, 0.000000}
\pgfsetstrokecolor{dialinecolor}
\pgfpathellipse{\pgfpoint{17.027814\du}{6.002254\du}}{\pgfpoint{0.188214\du}{0\du}}{\pgfpoint{0\du}{0.188214\du}}
\pgfusepath{stroke}
\pgfsetlinewidth{0.100000\du}
\pgfsetdash{}{0pt}
\pgfsetdash{}{0pt}
\pgfsetbuttcap
\pgfsetmiterjoin
\pgfsetlinewidth{0.100000\du}
\pgfsetbuttcap
\pgfsetmiterjoin
\pgfsetdash{}{0pt}
\definecolor{dialinecolor}{rgb}{0.000000, 0.000000, 0.000000}
\pgfsetfillcolor{dialinecolor}
\pgfpathellipse{\pgfpoint{7.683757\du}{15.045160\du}}{\pgfpoint{0.188214\du}{0\du}}{\pgfpoint{0\du}{0.188214\du}}
\pgfusepath{fill}
\definecolor{dialinecolor}{rgb}{0.000000, 0.000000, 0.000000}
\pgfsetstrokecolor{dialinecolor}
\pgfpathellipse{\pgfpoint{7.683757\du}{15.045160\du}}{\pgfpoint{0.188214\du}{0\du}}{\pgfpoint{0\du}{0.188214\du}}
\pgfusepath{stroke}
\pgfsetlinewidth{0.010000\du}
\pgfsetbuttcap
\pgfsetmiterjoin
\pgfsetdash{}{0pt}
\definecolor{dialinecolor}{rgb}{0.000000, 0.000000, 0.000000}
\pgfsetstrokecolor{dialinecolor}
\pgfpathellipse{\pgfpoint{7.683757\du}{15.045160\du}}{\pgfpoint{0.188214\du}{0\du}}{\pgfpoint{0\du}{0.188214\du}}
\pgfusepath{stroke}
\pgfsetlinewidth{0.100000\du}
\pgfsetdash{}{0pt}
\pgfsetdash{}{0pt}
\pgfsetbuttcap
\pgfsetmiterjoin
\pgfsetlinewidth{0.100000\du}
\pgfsetbuttcap
\pgfsetmiterjoin
\pgfsetdash{}{0pt}
\definecolor{dialinecolor}{rgb}{0.000000, 0.000000, 0.000000}
\pgfsetfillcolor{dialinecolor}
\pgfpathellipse{\pgfpoint{25.219128\du}{15.027435\du}}{\pgfpoint{0.188214\du}{0\du}}{\pgfpoint{0\du}{0.188214\du}}
\pgfusepath{fill}
\definecolor{dialinecolor}{rgb}{0.000000, 0.000000, 0.000000}
\pgfsetstrokecolor{dialinecolor}
\pgfpathellipse{\pgfpoint{25.219128\du}{15.027435\du}}{\pgfpoint{0.188214\du}{0\du}}{\pgfpoint{0\du}{0.188214\du}}
\pgfusepath{stroke}
\pgfsetlinewidth{0.010000\du}
\pgfsetbuttcap
\pgfsetmiterjoin
\pgfsetdash{}{0pt}
\definecolor{dialinecolor}{rgb}{0.000000, 0.000000, 0.000000}
\pgfsetstrokecolor{dialinecolor}
\pgfpathellipse{\pgfpoint{25.219128\du}{15.027435\du}}{\pgfpoint{0.188214\du}{0\du}}{\pgfpoint{0\du}{0.188214\du}}
\pgfusepath{stroke}
\pgfsetlinewidth{0.100000\du}
\pgfsetdash{}{0pt}
\pgfsetdash{}{0pt}
\pgfsetbuttcap
\pgfsetmiterjoin
\pgfsetlinewidth{0.100000\du}
\pgfsetbuttcap
\pgfsetmiterjoin
\pgfsetdash{}{0pt}
\definecolor{dialinecolor}{rgb}{0.000000, 0.000000, 0.000000}
\pgfsetstrokecolor{dialinecolor}
\draw (9.942110\du,6.007360\du)--(10.842110\du,6.007360\du)--(10.942110\du,5.507360\du)--(11.142110\du,6.507360\du)--(11.342110\du,5.507360\du)--(11.542110\du,6.507360\du)--(11.742110\du,5.507360\du)--(11.942110\du,6.507360\du)--(12.042110\du,6.007360\du)--(12.942110\du,6.007360\du);
\pgfsetlinewidth{0.100000\du}
\pgfsetdash{}{0pt}
\pgfsetdash{}{0pt}
\pgfsetbuttcap
\pgfsetmiterjoin
\pgfsetlinewidth{0.100000\du}
\pgfsetbuttcap
\pgfsetmiterjoin
\pgfsetdash{}{0pt}
\definecolor{dialinecolor}{rgb}{0.000000, 0.000000, 0.000000}
\pgfsetstrokecolor{dialinecolor}
\draw (18.796644\du,6.015700\du)--(19.696644\du,6.015700\du)--(19.796644\du,5.515700\du)--(19.996644\du,6.515700\du)--(20.196644\du,5.515700\du)--(20.396644\du,6.515700\du)--(20.596644\du,5.515700\du)--(20.796644\du,6.515700\du)--(20.896644\du,6.015700\du)--(21.796644\du,6.015700\du);
\pgfsetlinewidth{0.100000\du}
\pgfsetdash{}{0pt}
\pgfsetdash{}{0pt}
\pgfsetbuttcap
\pgfsetmiterjoin
\pgfsetlinewidth{0.100000\du}
\pgfsetbuttcap
\pgfsetmiterjoin
\pgfsetdash{}{0pt}
\definecolor{dialinecolor}{rgb}{0.000000, 0.000000, 0.000000}
\pgfsetstrokecolor{dialinecolor}
\draw (17.026521\du,9.477980\du)--(17.026521\du,10.377980\du)--(16.526521\du,10.477980\du)--(17.526521\du,10.677980\du)--(16.526521\du,10.877980\du)--(17.526521\du,11.077980\du)--(16.526521\du,11.277980\du)--(17.526521\du,11.477980\du)--(17.026521\du,11.577980\du)--(17.026521\du,12.477980\du);
\pgfsetlinewidth{0.100000\du}
\pgfsetdash{}{0pt}
\pgfsetdash{}{0pt}
\pgfsetmiterjoin
\pgfsetbuttcap
{
\definecolor{dialinecolor}{rgb}{0.000000, 0.000000, 0.000000}
\pgfsetfillcolor{dialinecolor}
% was here!!!
{\pgfsetcornersarced{\pgfpoint{0.000000\du}{0.000000\du}}\definecolor{dialinecolor}{rgb}{0.000000, 0.000000, 0.000000}
\pgfsetstrokecolor{dialinecolor}
\draw (7.683757\du,15.279017\du)--(7.683757\du,19.932300\du)--(15.525300\du,19.932300\du)--(15.525300\du,19.929400\du);
}}
\pgfsetlinewidth{0.100000\du}
\pgfsetdash{}{0pt}
\pgfsetdash{}{0pt}
\pgfsetmiterjoin
\pgfsetbuttcap
{
\definecolor{dialinecolor}{rgb}{0.000000, 0.000000, 0.000000}
\pgfsetfillcolor{dialinecolor}
% was here!!!
{\pgfsetcornersarced{\pgfpoint{0.000000\du}{0.000000\du}}\definecolor{dialinecolor}{rgb}{0.000000, 0.000000, 0.000000}
\pgfsetstrokecolor{dialinecolor}
\draw (18.525300\du,19.929400\du)--(18.525300\du,19.930569\du)--(25.242091\du,19.930569\du)--(25.242091\du,15.056182\du);
}}
\pgfsetlinewidth{0.100000\du}
\pgfsetdash{}{0pt}
\pgfsetdash{}{0pt}
\pgfsetbuttcap
{
\definecolor{dialinecolor}{rgb}{0.000000, 0.000000, 0.000000}
\pgfsetfillcolor{dialinecolor}
% was here!!!
\definecolor{dialinecolor}{rgb}{0.000000, 0.000000, 0.000000}
\pgfsetstrokecolor{dialinecolor}
\draw (17.269622\du,15.038118\du)--(20.108135\du,15.038118\du);
}
\pgfsetlinewidth{0.100000\du}
\pgfsetdash{}{0pt}
\pgfsetdash{}{0pt}
\pgfsetbuttcap
{
\definecolor{dialinecolor}{rgb}{0.000000, 0.000000, 0.000000}
\pgfsetfillcolor{dialinecolor}
% was here!!!
\definecolor{dialinecolor}{rgb}{0.000000, 0.000000, 0.000000}
\pgfsetstrokecolor{dialinecolor}
\draw (13.414194\du,15.035874\du)--(16.793524\du,15.036659\du);
}
\pgfsetlinewidth{0.100000\du}
\pgfsetdash{}{0pt}
\pgfsetdash{}{0pt}
\pgfsetmiterjoin
\pgfsetbuttcap
{
\definecolor{dialinecolor}{rgb}{0.000000, 0.000000, 0.000000}
\pgfsetfillcolor{dialinecolor}
% was here!!!
{\pgfsetcornersarced{\pgfpoint{0.000000\du}{0.000000\du}}\definecolor{dialinecolor}{rgb}{0.000000, 0.000000, 0.000000}
\pgfsetstrokecolor{dialinecolor}
\draw (7.683757\du,14.856946\du)--(7.690563\du,14.856946\du)--(7.690563\du,6.007360\du)--(9.942110\du,6.007360\du);
}}
\pgfsetlinewidth{0.100000\du}
\pgfsetdash{}{0pt}
\pgfsetdash{}{0pt}
\pgfsetbuttcap
{
\definecolor{dialinecolor}{rgb}{0.000000, 0.000000, 0.000000}
\pgfsetfillcolor{dialinecolor}
% was here!!!
\definecolor{dialinecolor}{rgb}{0.000000, 0.000000, 0.000000}
\pgfsetstrokecolor{dialinecolor}
\draw (12.942110\du,6.007360\du)--(16.789913\du,6.002551\du);
}
\pgfsetlinewidth{0.100000\du}
\pgfsetdash{}{0pt}
\pgfsetdash{}{0pt}
\pgfsetbuttcap
{
\definecolor{dialinecolor}{rgb}{0.000000, 0.000000, 0.000000}
\pgfsetfillcolor{dialinecolor}
% was here!!!
\definecolor{dialinecolor}{rgb}{0.000000, 0.000000, 0.000000}
\pgfsetstrokecolor{dialinecolor}
\draw (17.216100\du,6.002250\du)--(18.796644\du,6.015700\du);
}
\pgfsetlinewidth{0.100000\du}
\pgfsetdash{}{0pt}
\pgfsetdash{}{0pt}
\pgfsetbuttcap
{
\definecolor{dialinecolor}{rgb}{0.000000, 0.000000, 0.000000}
\pgfsetfillcolor{dialinecolor}
% was here!!!
\definecolor{dialinecolor}{rgb}{0.000000, 0.000000, 0.000000}
\pgfsetstrokecolor{dialinecolor}
\draw (17.027725\du,6.240277\du)--(17.026521\du,9.477980\du);
}
\pgfsetlinewidth{0.100000\du}
\pgfsetdash{}{0pt}
\pgfsetdash{}{0pt}
\pgfsetbuttcap
{
\definecolor{dialinecolor}{rgb}{0.000000, 0.000000, 0.000000}
\pgfsetfillcolor{dialinecolor}
% was here!!!
\definecolor{dialinecolor}{rgb}{0.000000, 0.000000, 0.000000}
\pgfsetstrokecolor{dialinecolor}
\draw (17.028401\du,12.477980\du)--(17.028401\du,14.961973\du);
}
% setfont left to latex
\definecolor{dialinecolor}{rgb}{0.000000, 0.000000, 0.000000}
\pgfsetstrokecolor{dialinecolor}
\node[anchor=west] at (16.5\du,15.8\du){E};
% setfont left to latex
\definecolor{dialinecolor}{rgb}{0.000000, 0.000000, 0.000000}
\pgfsetstrokecolor{dialinecolor}
\node[anchor=west] at (16.5\du,5.3\du){C};
% setfont left to latex
\definecolor{dialinecolor}{rgb}{0.000000, 0.000000, 0.000000}
\pgfsetstrokecolor{dialinecolor}
\node[anchor=west] at (6.5\du,15\du){B};
% setfont left to latex
\definecolor{dialinecolor}{rgb}{0.000000, 0.000000, 0.000000}
\pgfsetstrokecolor{dialinecolor}
\node[anchor=west] at (25.5\du,15\du){D};
\pgfsetlinewidth{0.100000\du}
\pgfsetdash{}{0pt}
\pgfsetdash{}{0pt}
\pgfsetbuttcap
{
\definecolor{dialinecolor}{rgb}{0.000000, 0.000000, 0.000000}
\pgfsetfillcolor{dialinecolor}
% was here!!!
\definecolor{dialinecolor}{rgb}{0.000000, 0.000000, 0.000000}
\pgfsetstrokecolor{dialinecolor}
\draw (22.962755\du,15.055107\du)--(25.291977\du,15.055107\du);
}
\pgfsetlinewidth{0.100000\du}
\pgfsetdash{}{0pt}
\pgfsetdash{}{0pt}
\pgfsetbuttcap
{
\definecolor{dialinecolor}{rgb}{0.000000, 0.000000, 0.000000}
\pgfsetfillcolor{dialinecolor}
% was here!!!
\definecolor{dialinecolor}{rgb}{0.000000, 0.000000, 0.000000}
\pgfsetstrokecolor{dialinecolor}
\draw (10.414421\du,15.039101\du)--(7.871971\du,15.045160\du);
}
\pgfsetlinewidth{0.100000\du}
\pgfsetdash{}{0pt}
\pgfsetdash{}{0pt}
\pgfsetmiterjoin
\pgfsetbuttcap
{
\definecolor{dialinecolor}{rgb}{0.000000, 0.000000, 0.000000}
\pgfsetfillcolor{dialinecolor}
% was here!!!
{\pgfsetcornersarced{\pgfpoint{0.000000\du}{0.000000\du}}\definecolor{dialinecolor}{rgb}{0.000000, 0.000000, 0.000000}
\pgfsetstrokecolor{dialinecolor}
\draw (21.796644\du,6.015700\du)--(25.229591\du,6.015700\du)--(25.229591\du,15.027435\du)--(25.219128\du,15.027435\du);
}}
% setfont left to latex
\definecolor{dialinecolor}{rgb}{0.000000, 0.000000, 0.000000}
\pgfsetstrokecolor{dialinecolor}
\node[anchor=west] at (16.5\du,18.9\du){A};
% setfont left to latex
\definecolor{dialinecolor}{rgb}{0.000000, 0.000000, 0.000000}
\pgfsetstrokecolor{dialinecolor}
\node[anchor=west] at (10.116283\du,7.348232\du){2 ohms};
% setfont left to latex
\definecolor{dialinecolor}{rgb}{0.000000, 0.000000, 0.000000}
\pgfsetstrokecolor{dialinecolor}
\node[anchor=west] at (19.167532\du,7.239482\du){1 ohm};
% setfont left to latex
\definecolor{dialinecolor}{rgb}{0.000000, 0.000000, 0.000000}
\pgfsetstrokecolor{dialinecolor}
\node[anchor=west] at (10.859146\du,16.5\du){1 ohm};
% setfont left to latex
\definecolor{dialinecolor}{rgb}{0.000000, 0.000000, 0.000000}
\pgfsetstrokecolor{dialinecolor}
\node[anchor=west] at (20\du,16.5\du){2 ohms};
% setfont left to latex
\definecolor{dialinecolor}{rgb}{0.000000, 0.000000, 0.000000}
\pgfsetstrokecolor{dialinecolor}
\node[anchor=west] at (18.077579\du,11.300768\du){2 ohms};
% setfont left to latex
\definecolor{dialinecolor}{rgb}{0.000000, 0.000000, 0.000000}
\pgfsetstrokecolor{dialinecolor}
\node[anchor=west] at (16.154051\du,21.090037\du){10 volts};
\pgfsetlinewidth{0.100000\du}
\pgfsetdash{}{0pt}
\pgfsetdash{}{0pt}
\pgfsetbuttcap
{
\definecolor{dialinecolor}{rgb}{0.000000, 0.000000, 0.000000}
\pgfsetfillcolor{dialinecolor}
% was here!!!
\pgfsetarrowsend{stealth}
\definecolor{dialinecolor}{rgb}{0.000000, 0.000000, 0.000000}
\pgfsetstrokecolor{dialinecolor}
\draw (8.723575\du,14.549447\du)--(10.834402\du,14.549447\du);
}
\pgfsetlinewidth{0.100000\du}
\pgfsetdash{}{0pt}
\pgfsetdash{}{0pt}
\pgfsetbuttcap
{
\definecolor{dialinecolor}{rgb}{0.000000, 0.000000, 0.000000}
\pgfsetfillcolor{dialinecolor}
% was here!!!
\pgfsetarrowsend{stealth}
\definecolor{dialinecolor}{rgb}{0.000000, 0.000000, 0.000000}
\pgfsetstrokecolor{dialinecolor}
\draw (18.105164\du,14.506923\du)--(20.215991\du,14.506923\du);
}
\pgfsetlinewidth{0.100000\du}
\pgfsetdash{}{0pt}
\pgfsetdash{}{0pt}
\pgfsetbuttcap
{
\definecolor{dialinecolor}{rgb}{0.000000, 0.000000, 0.000000}
\pgfsetfillcolor{dialinecolor}
% was here!!!
\pgfsetarrowsstart{stealth}
\definecolor{dialinecolor}{rgb}{0.000000, 0.000000, 0.000000}
\pgfsetstrokecolor{dialinecolor}
\draw (19.133968\du,19.121305\du)--(21.244795\du,19.121305\du);
}
\pgfsetlinewidth{0.100000\du}
\pgfsetdash{}{0pt}
\pgfsetdash{}{0pt}
\pgfsetbuttcap
{
\definecolor{dialinecolor}{rgb}{0.000000, 0.000000, 0.000000}
\pgfsetfillcolor{dialinecolor}
% was here!!!
\pgfsetarrowsstart{stealth}
\definecolor{dialinecolor}{rgb}{0.000000, 0.000000, 0.000000}
\pgfsetstrokecolor{dialinecolor}
\draw (10.859622\du,19.269509\du)--(12.970449\du,19.269509\du);
}
\pgfsetlinewidth{0.100000\du}
\pgfsetdash{}{0pt}
\pgfsetdash{}{0pt}
\pgfsetbuttcap
{
\definecolor{dialinecolor}{rgb}{0.000000, 0.000000, 0.000000}
\pgfsetfillcolor{dialinecolor}
% was here!!!
\pgfsetarrowsend{stealth}
\definecolor{dialinecolor}{rgb}{0.000000, 0.000000, 0.000000}
\pgfsetstrokecolor{dialinecolor}
\draw (6.938362\du,10.892662\du)--(6.938362\du,9.079134\du);
}
\pgfsetlinewidth{0.100000\du}
\pgfsetdash{}{0pt}
\pgfsetdash{}{0pt}
\pgfsetbuttcap
{
\definecolor{dialinecolor}{rgb}{0.000000, 0.000000, 0.000000}
\pgfsetfillcolor{dialinecolor}
% was here!!!
\pgfsetarrowsstart{stealth}
\definecolor{dialinecolor}{rgb}{0.000000, 0.000000, 0.000000}
\pgfsetstrokecolor{dialinecolor}
\draw (16.393962\du,13.929893\du)--(16.393962\du,12.116366\du);
}
\pgfsetlinewidth{0.100000\du}
\pgfsetdash{}{0pt}
\pgfsetdash{}{0pt}
\pgfsetbuttcap
{
\definecolor{dialinecolor}{rgb}{0.000000, 0.000000, 0.000000}
\pgfsetfillcolor{dialinecolor}
% was here!!!
\pgfsetarrowsstart{stealth}
\definecolor{dialinecolor}{rgb}{0.000000, 0.000000, 0.000000}
\pgfsetstrokecolor{dialinecolor}
\draw (25.823174\du,11.395818\du)--(25.823174\du,9.582290\du);
}
% setfont left to latex
\definecolor{dialinecolor}{rgb}{0.000000, 0.000000, 0.000000}
\pgfsetstrokecolor{dialinecolor}
\node[anchor=west] at (5.9\du,10\du){$I_1$};
% setfont left to latex
\definecolor{dialinecolor}{rgb}{0.000000, 0.000000, 0.000000}
\pgfsetstrokecolor{dialinecolor}
\node[anchor=west] at (9.006750\du,14.041870\du){$I_4$};
% setfont left to latex
\definecolor{dialinecolor}{rgb}{0.000000, 0.000000, 0.000000}
\pgfsetstrokecolor{dialinecolor}
\node[anchor=west] at (15.137287\du,12.953754\du){$I_3$};
% setfont left to latex
\definecolor{dialinecolor}{rgb}{0.000000, 0.000000, 0.000000}
\pgfsetstrokecolor{dialinecolor}
\node[anchor=west] at (18.377121\du,13.982410\du){$I_5$};
% setfont left to latex
\definecolor{dialinecolor}{rgb}{0.000000, 0.000000, 0.000000}
\pgfsetstrokecolor{dialinecolor}
\node[anchor=west] at (26\du,10\du){$I_2$};
% setfont left to latex
\definecolor{dialinecolor}{rgb}{0.000000, 0.000000, 0.000000}
\pgfsetstrokecolor{dialinecolor}
\node[anchor=west] at (11.8\du,18.7\du){$I$};
% setfont left to latex
\definecolor{dialinecolor}{rgb}{0.000000, 0.000000, 0.000000}
\pgfsetstrokecolor{dialinecolor}
\node[anchor=west] at (20\du,18.7\du){$I$};
\end{tikzpicture}

\end{figure}

A lei da corrente de Kirchhoff fornece as seguintes equações nos quatro nós:
\begin{center}
    \begin{tabular}{lc}
        \textbf{Nó B:} & $I - I_1 - I_4 = 0$\\
        \textbf{Nó C:} & $I_1 - I_2 - I_3 = 0$\\
        \textbf{Nó D:} & $I - I_2 - I_5 = 0$\\
        \textbf{Nó B:}& $I_3 + I_4 - I_5 = 0$
    \end{tabular}
\end{center}

Para os circuitos obtemos
\begin{center}
    \begin{tabular}{lc}
        \textbf{Circuito $ABEDA$:} & $I_4 + 2I_5 = 10$\\
        \textbf{Circuito $BCED$:} & $2I_1 + 2I_3 - I_4 = 0$\\
        \textbf{Circuito $CDEC$:} & $I_2 - 2I_5 - 2I_3 = 0$
    \end{tabular}
\end{center}

Essas equações produzem o sistema linear:
\[
    \begin{cases}
        I - I_1 - I_4 = 0\\
        I_1 - I_2 - I_3 = 0\\
        I - I_2 - I_5 = 0\\
        I_3 + I_4 - I_5 = 0\\
        I_4 + 2I_5 = 10\\
        2I_1 + 2I_3 - I_4 = 0\\
        I_2 - 2I_5 - 2I_3 = 0
    \end{cases}
\]
Cuja solução é: $I = 7$, $I_1 = I_5 = 3$, $I_2 = I_4 = 4$ e $I_3 = -1$.

\begin{observacao}
    Há uma única fonte de energia neste caso. Assim a bateria de 10 volts fornece uma corrente de 7 amperes ao circuito. Usando a lei de Ohm, $E = RI$, obtemos $R = 10/7$. Assim, todo o circuito comporta-se como se houvesse um único resistor de $10/7$ ohms. Esse valor é chamado de \textrm{resistência efetiva}, $R_{ef}$, do circuito.
\end{observacao}