%!TEX root = IAL.tex

\chapter{Sistemas Lineares}

Ao longo de todo esse capítulo o conjunto $\cp{K}$ será considerado como sendo um dos conjuntos: $\rac$, $\real$ ou $\complex$.

\section{Sistemas Lineares}\label{ssub:sistemas_lineares}
Vamos trabalhar com \textbf{equações lineares}\index{Sistemas Lineares!Equação linear} em $n$ variáveis $x_1$, $x_2$, \dots, $x_n$ que são equações que podem ser escritas na forma
\begin{equation}\label{equacao_linear}
    a_1x_1  + a_2x_2 +  \cdots + a_qx_q  = b,
\end{equation}
onde $a_1$, $a_2$, \dots, $a_q$, $b$ são escalares no corpo $\cp{K}$, sendo que nem todos os coeficientes $a_i$ são nulos.

No caso em que $b = 0$,  a equação \eqref{equacao_linear} tem a forma
\begin{equation}
    a_1x_1 + a_2x_2 + \cdots + a_qx_q = 0
\end{equation}
e é chamada de uma \textbf{equação linear homogênea}\index{Sistemas Lineares!Equação linear homogênea}.

\begin{exemplos}
	\begin{enumerate}
		\item Equações da forma
			\begin{enumerate}
				\item $x_1 + 2x_2 - 3x_3 = 10$, em $\cp{K} = \rac$
				\item $2x_1 + 0x_2 + \sqrt{2}x_3 - 4x_4 = 0$, em $\cp{K} = \real$, que será escrita como $2x_1 + 0x_2 + \sqrt{2}x_3 - 4x_4 = 0$
				\item $ix_1 + (2 +i)x_2 = 3 + 2i$, em $\cp{K} = \complex$
			\end{enumerate}
		são exemplos de \textbf{equações lineares}.

		\item Já equações na forma
			\begin{enumerate}
				\item $x_1^2 + 2x_2 - 3x_3 = 0$ em $\cp{K} = \real$
				\item $x_1 + 2\sin(x_2) + 3x_3 + x_ 4 = 1$ em $\cp{K} = \real$
				\item $2x_1 + 3x_2 - 4x_3^{-1} = 2$ em $\cp{K} = \rac$
			\end{enumerate}
		não são equações lineares e portanto não serão abordadas nesse curso.
	\end{enumerate}
\end{exemplos}

Considerenado $\cp{K} = \real$, queremos analisar um conjuto de equações lineares do tipo:

\begin{align}
    \begin{cases}\label{sistema_linear_2x2}
        ax + by = b_1\\
        cx + dy = b_2
    \end{cases}
\end{align}
onde $a$, $b$, $c$, $d$, $b_1$, $b_2 \in \cp{K}$.

Queremos saber se é possível encontrar valores para $x$, $y$ no conjunto $\real$ de modo que essas duas equações sejam simultaneamente verdadeiras? Caso existam, gostaríamos de determinar todos os possíveis valores que $x$ e $y$ podem assumir.

Observe que cada equação linear em \eqref{sistema_linear_2x2} representa uma reta no plano cartesiano. Assim queremos determinar se essas restas possuem alguma interseção.

Mas sabemos que duas retas no plano podem ser coincidentes, paralelas ou se interceptam em um único ponto.

Assim se as retas do sistema \eqref{sistema_linear_2x2} são \textbf{coincidentes}, existem infinitos valores de $x$, $y \in \cp{K}$ que satisfazem as duas equações simultanemente. Neste caso dizemos que o sistema \eqref{sistema_linear_2x2} é \textbf{possível e indeterminado}.

\definecolor{qqqqff}{rgb}{0,0,1}
\definecolor{ffqqqq}{rgb}{1,0,0}
\begin{center}
    \begin{tikzpicture}[line cap=round,line join=round,>=triangle 45,x=1cm,y=1cm]
        \begin{axis}[
            x=1cm,y=1cm,
            axis lines=middle,
            xticklabels={,,},
            yticklabels={,,},
            xlabel=$x$,
            ylabel=$y$,
            xmin=-2,
            xmax=5,
            ymin=-1.3,
            ymax=3,]
            \draw [line width=5.2pt,dash pattern=on 1pt off 1pt,color=ffqqqq,domain=-1.348257839721257:5.807994044102577] plot(\x,{(--4-1*\x)/2});
            \draw [line width=2pt,color=qqqqff,domain=-1.348257839721257:5.807994044102577] plot(\x,{(--8-2*\x)/4});
            \begin{scriptsize}
                \draw[color=ffqqqq] (1.8,2) node {$a_{11}x + a_{12}y = b_1$};
                \draw[color=qqqqff] (3,1.5) node {$a_{21}x + a_{22}y = b_2$};
            \end{scriptsize}
        \end{axis}
    \end{tikzpicture}
\end{center}

Agora se as retas em \eqref{sistema_linear_2x2} são \textbf{paralelas}, não existem valores de $x$, $y \in \cp{K}$ satisfazendo as duas equações simultaneamente. Nesse caso dizemos que o sistema \eqref{sistema_linear_2x2} é \textbf{impossível}.

\definecolor{ffqqqq}{rgb}{1,0,0}
\definecolor{qqqqff}{rgb}{0,0,1}
\begin{center}
    \begin{tikzpicture}[line cap=round,line join=round,>=triangle 45,x=1cm,y=1cm]
        \begin{axis}[
            x=1cm,y=1cm,
            axis lines=middle,
            xticklabels={,,},
            yticklabels={,,},
            xlabel=$x$,
            ylabel=$y$,
            xmin=-5.3,
            xmax=6.3,
            ymin=-1,
            ymax=4.3,]
            \draw [line width=2pt,color=qqqqff,domain=-5:6] plot(\x,{(--4-1*\x)/5});
            \draw [line width=2pt,color=ffqqqq,domain=-5:6] plot(\x,{(--16-1*\x)/5});
            \begin{scriptsize}
                \draw[color=qqqqff] (-3.8,2.2) node {$a_{21}x + a_{22}y = b_2$};
                \draw[color=ffqqqq] (-3.8,3.4) node {$a_{11}x + a_{12}y = b_1$};
            \end{scriptsize}
        \end{axis}
    \end{tikzpicture}
\end{center}

Por fim, as retas em \eqref{sistema_linear_2x2} podem se \textbf{interceptar}. Nesse caso existe um único conjunto de valores $x = \alpha \in \cp{K}$ e $y = \lambda \in \cp{K}$ que satisfaz as duas equações simultaneamente. Neste caso dizemos que o sistema \eqref{sistema_linear_2x2} é \textbf{possível e determinado}.

\definecolor{qqqqff}{rgb}{0,0,1}
\definecolor{ffqqqq}{rgb}{1,0,0}
\begin{center}
    \begin{tikzpicture}[line cap=round,line join=round,>=triangle 45,x=1cm,y=1cm]
        \begin{axis}[
            x=1cm,y=1cm,
            axis lines=middle,
            xticklabels={,,},
            yticklabels={,,},
            xlabel=$x$,
            ylabel=$y$,
            xmin=-2,
            xmax=4.3,
            ymin=-2,
            ymax=3.5,]
            \draw [line width=2pt,color=ffqqqq,domain=-4:5] plot(\x,{(--4.94--7.92*\x)/7.58});
            \draw [line width=2pt,color=qqqqff,domain=-4:5] plot(\x,{(--4.824-8.66*\x)/8.24});
            \begin{scriptsize}
                \draw[color=ffqqqq] (2.9,2) node {$a_{11}x + a_{12}y = b_1$};
                \draw[color=qqqqff] (2.7,-0.5) node {$a_{21}x + a_{22}y = b_2$};
            \end{scriptsize}
        \end{axis}
    \end{tikzpicture}
\end{center}

Esse mesmo tipo de análise pode ser aplicada num sistema da forma
\begin{align}
    \begin{cases}\label{sistema_linear_3x3}
        a_1x + a_2y + a_3z = b_1\\
        a_4x + a_5y + a_6z = b_2\\
        a_7x + a_8y + a_9z = b_3
    \end{cases}
\end{align}
onde $a_i \in \cp{K}$ para $1 \le i \le 9$ e $b_j \in \cp{K}$ para $1 \le j \le 3$.

\vspace{.3cm}

Existem valores de $x$, $y$ e $z \in \cp{K}$ que satisfaçam as três equações simultaneamente?

Como no caso anterior, temos as seguintes possibilidades:
\begin{enumerate}[label={\roman*})]
    \item Existe um único conjunto de valores $x = \alpha \in \cp{K}$, $y = \lambda \in \cp{K}$ e $z = \gamma \in \cp{K}$  que satisfaz as três equações simultaneamente.  Neste caso dizemos que o sistema \eqref{sistema_linear_3x3}  é \textbf{possível e determinado}.

    \item Existem infinitos valores de $x$, $y$, $z \in \cp{K}$  que satisfazem as três equações simultanemente.  Neste caso dizemos que o sistema \eqref{sistema_linear_3x3}  é \textbf{possível e indeterminado}.

    \item Não existem valores de $x$, $y$, $z \in \cp{K}$  satisfazendo as três equações simultaneamente.  Nesse caso dizemos que o sistema \eqref{sistema_linear_3x3}  é \textbf{impossível}.
\end{enumerate}


Agora, de modo geral queremos trabalhar com uma quantidade qualquer de equações. Para isso começamos escolhendo escalares $b_1$, \dots, $b_m$  e $a_{ij}$,  $1 \le i \le m$, $1 \le j \le n$ todos em $\cp{K}$.

Queremos saber se é possível encontrar valores para  $x_1$, $x_2$, \dots, $x_n$  de modo que o seguinte conjunto de equações sejam válidas: 
\begin{equation}\label{sistema_linear_geral}
\begin{cases}
        a_{11}x_1 + a_{12}x_2 + \cdots + a_{1n}x_n = b_1\\
        a_{21}x_1 + a_{22}x_2 + \cdots + a_{2n}x_n = b_2\\
        \qquad \vdots\\
        a_{m1}x_1 + a_{m2}x_2 + \cdots + a_{mn}x_n = b_m
    \end{cases}
\end{equation}

O conjunto de equações em \eqref{sistema_linear_geral} é chamado de um  \textbf{sistema de $m$ equa\c{c}\~oes lineares  a $n$ inc\'ognitas}\index{Sistema Linear} $x_1$, $x_2$, \dots, $x_n$ , ou simplesmente de um \textbf{sistema linear}, nas incógnitas  $x_1$, $x_2$, \dots, $x_n$.

Uma solução de um sistema linear do tipo \eqref{sistema_linear_geral}
\[
    x_1 = \alpha_1,  x_2 = \alpha_2,  \dots, x_n = \alpha_n
\]
onde $\alpha_1$, $\alpha_2$, \dots, $\alpha_n \in \cp{K}$,  pode ser escrita como
\[
    (\alpha_1, \alpha_2, \dots, \alpha_n)
\]
e é chamada de uma \textbf{ênupla ordenada}\index{Sistemas linears!ênupla ordenada} ou uma \textbf{n-upla ordenada}.


Se $b_1 = b_2 = \cdots = b_m = 0_\cp{K} \in K$,  dizemos que o sistema
\begin{equation}\label{sistemalinearhomogeneo}\index{Sistema Linear}
    \begin{cases}
        a_{11}x_1 + a_{12}x_2 + \cdots + a_{1n}x_n = 0_\cp{K}\\
        a_{21}x_1 + a_{22}x_2 + \cdots + a_{2n}x_n = 0_\cp{K}\\
        \qquad \vdots\\
        a_{m1}x_1 + a_{m2}x_2 + \cdots + a_{mn}x_n = 0_\cp{K}
    \end{cases}
\end{equation}
\'e um \textbf{sistema linear homog\^eneo}.\index{Sistema linear!homogêneo} 

Observe que tal sistema sempre possui solu\c{c}\~ao,  a saber, $x_1 = x_2 = \cdots = x_n = 0_\cp{K}$.

Para o caso de sistemas lineares temos o seguinte resultado:

\begin{teorema}
    Todo sistema linear do tipo \eqref{sistema_linear_geral} tem zero,  uma  ou uma infinidade de soluções.  Não existem outras possibilidades.
\end{teorema}

No caso de um sistema linear da forma \eqref{sistema_linear_geral},  o processo para encontrar suas soluções ser\'a feito mediante o uso de 3 tipos de opera\c{c}\~oes.  S\~ao elas:
\begin{itemize}
\item[$e_1$)] Troca da posi\c{c}\~ao de duas equa\c{c}\~oes.
\item[$e_2$)] Multiplica\c{c}\~ao de uma equa\c{c}\~ao por um escalar n\~ao nulo.
\item[$e_3$)] Substitui\c{c}\~ao de uma equa\c{c}\~ao pela soma desta equa\c{c}\~ao com alguma outra.
\end{itemize}

Estas tr\^es opera\c{c}\~oes s\~ao chamadas de  \textbf{opera\c{c}\~oes elementares}.



Uma outra matriz que podemos associar ao sistema \eqref{sistema_linear_geral} \'e
\[
\begin{bmatrix}
        a_{11} & a_{12} & \cdots & a_{1n} & b_1\\
a_{21} & a_{22} & \cdots & a_{2n} & b_2\\
\vdots & \vdots & \vdots & \vdots & \vdots\\
a_{m1} & a_{m2} & \cdots & a_{mn} & b_m\\
    \end{bmatrix}
\]

que \'e chamada de \textbf{matriz ampliada do sistema}  ou \textbf{matriz aumentada do sistema}.



Na forma matricial  as opera\c{c}\~oes elementares s\~ao descritas como:

\vspace{.3cm}

\begin{itemize}
    \item[$e_1$)] Trocar a $i$-\'esima linha de $A$  pela $j$-\'esima linha de $A$:  $L_i \leftrightarrow L_j$;

    \vspace{.3cm}

    \item[$e_2$)] Multiplica\c{c}\~ao da $i$-\'esima linha de $A$  por um escalar $\alpha \in \cp{K}$ n\~ao nulo:  $L_i \rightarrow \alpha L_i$;

    \vspace{.3cm}

   \item[$e_3$)] Substitui\c{c}\~ao da $i$-\'esima linha de $A$  pela $i$-\'esima linha mais $\alpha$ vezes a $j$-\'esima linha:  $L_i \rightarrow L_i + \alpha L_j$.
\end{itemize}



\begin{definicao}\label{linhareduzida}
    Uma matriz $A$ $m \times n$ \'e  dita estar na \textbf{forma escalonada reduzida por linhas} se:
    \begin{enumerate}[label={\roman*})]
        \item O primeiro elemento n\~ao nulo  em cada linha n\~ao nula de $A$  \'e $1$.  Dizemos que esse número 1 é um \textbf{pivô}.

        \vspace{.3cm}

        \item Toda linha de $A$ cujos elementos s\~ao todos nulos  ocorre abaixo de todas as linhas que possuem um elemento n\~ao-nulo. 

        \vspace{.3cm}

        \item Se as linhas 1, 2, \dots, $r$ s\~ao as linhas n\~ao-nulas de $A$  e se o \textbf{pivô} da linha $i$ ocorre na coluna $k_i$,  $i = 1$, \dots, $r$,  ent\~ao $k_1 < k_2 < \cdots < k_r$.

        \vspace{.3cm}

        \item Cada coluna de $A$ que cont\'em um \textbf{pivô}  tem todos os seus outros elementos nulos.
    \end{enumerate}
\end{definicao}

\begin{observacao}
    Uma matriz que satisfaz as três primeiras propriedades da definição anterior  é dita estar na \textbf{forma escalonada por linhas},  ou simplesmente, em \textbf{forma escalonada}.
\end{observacao}

Seja $\cp{K}$ um corpo. Consideremos o problema de determinar $n$ escalares, ou seja, $n$ elementos $x_1$, $x_2$, \dots, $x_n$ em $\cp{K}$ que satisfa\c{c}am simultaneamente as equa\c{c}\~oes
\begin{equation}\label{sistemalinear}\index{Sistema Linear}
	\begin{cases}
		a_{11}x_1 + a_{12}x_2 + \cdots + a_{1n}x_n = b_1\\
		a_{21}x_1 + a_{22}x_2 + \cdots + a_{2n}x_n = b_2\\
		\qquad \vdots\\
		a_{m1}x_1 + a_{m2}x_2 + \cdots + a_{mn}x_n = b_m\\
	\end{cases}
\end{equation}
onde $b_1$, \dots, $b_m$ e $a_{ij}$, $1 \le i \le m$, $1 \le j \le n$ s\~ao elementos de $\cp{K}$ previamente conhecidos. Chamamos \eqref{sistemalinear} de um \textbf{sistema de $m$ equa\c{c}\~oes lineares a $n$ inc\'ognitas} $x_1$, $x_2$, \dots, $x_n$. Toda $n$-upla $(\alpha_1, \alpha_2, \dots, \alpha_n)$ onde $\alpha_i \in \cp{K}$ para $1 \le i \le n$, que satisfazem a cada uma das equa\c{c}\~oes de \eqref{sistemalinear} \'e chamada de uma \textbf{solu\c{c}\~ao} do sistema.

Se $b_1 = b_2 = \cdots = b_m = 0_\cp{K} \in K$, dizemos que o sistema
\begin{equation}\label{sistemalinearhomogeneo}\index{Sistema Linear}
	\begin{cases}
		a_{11}x_1 + a_{12}x_2 + \cdots + a_{1n}x_n = 0_\cp{K}\\
		a_{21}x_1 + a_{22}x_2 + \cdots + a_{2n}x_n = 0_\cp{K}\\
		\qquad \vdots\\
		a_{m1}x_1 + a_{m2}x_2 + \cdots + a_{mn}x_n = 0_\cp{K}\\
	\end{cases}
\end{equation}
\'e um \textbf{sistema linear homog\^eneo}, ou que cada uma de suas equa\c{c}\~oes \'e homog\^enea. Observe que tal sistema sempre possui solu\c{c}\~ao, a saber, $x_1 = x_2 = \cdots = x_n = 0_\cp{K}$.

O m\'etodo mais importante para determinar as solu\c{c}\~oes de um sistema de equa\c{c}\~oes lineares \'e o m\'etodo do \textbf{escalonamento}. Por exemplo, considere o sistema
\begin{equation}\label{exemploplo1}
	\begin{cases}
		2x_1 - x_2 + x_3 = 0\\
		x_1 + 3x_2 + 4x_ 3 = 0
	\end{cases}
\end{equation}
onde o corpo considerado \'e $\real$.

Observe que multiplicando a segunda equa\c{c}\~ao de \eqref{exemploplo1} por $-2$ e somando o resultado \`a primeira equa\c{c}\~ao obtemos
\[
	-7x_2 - 7x_3 = 0
\]
o que resulta em $x_2 = -x_3$. Agora se multiplicarmos a primeira equa\c{c}\~ao de \eqref{exemploplo1} por $3$ e somarmos com a segunda, obtemos
\[
	7x_1 + 7x_3 = 0
\]
e da{\'\i} $x_1 = -x_3$.

Assim para que uma terna $(x_1, x_2, x_3)$ de n\'umeros reais seja solu\c{c}\~ao de \eqref{exemploplo1} deve satisfazer
\[
	x_1 = x_2 = -x_3.
\]
Por outro lado, qualquer terna da forma $(a, a, -a)$ \'e solu\c{c}\~ao de \eqref{exemploplo1}. Portanto a solu\c{c}\~ao de \eqref{exemploplo1} \'e da forma
\[
	(a, a, -a)
\]
onde $a \in \real$.

No caso de um sistema linear da forma \eqref{sistemalinear}, o processo de elemina\c{c}\~ao de vari\'aveis ser\'a feito mediante o uso de 3 tipos de opera\c{c}\~oes. S\~ao elas:
\begin{itemize}
	\item[$e_1$)] Troca da posi\c{c}\~ao de duas equa\c{c}\~oes.
	\item[$e_2$)] Multiplica\c{c}\~ao de uma equa\c{c}\~ao por um escalar n\~ao nulo.
	\item[$e_3$)] Substitui\c{c}\~ao de uma equa\c{c}\~ao pela soma desta equa\c{c}\~ao com alguma outra.
\end{itemize}

Estas tr\^es opera\c{c}\~oes s\~ao chamadas de \textbf{opera\c{c}\~oes elementares}.\index{Opera\c{c}\~oes Elementares}

\begin{exemplo}
	Considere o seguinte sistema sobre o corpo $\real$:
	\[
		\begin{cases}
			x_1 + 4x_2 + 3x_3 = 1\\
			2x_1 + 5x_2 + 4x_3 = 4\\
			x_1 - 3x_2 - 2x_3 = 5
		\end{cases}
	\]
	Efetuando opera\c{c}\~oes elementares podemos escrever:
	\begin{align*}
		&\begin{cases}
			x_1 + 4x_2 + 3x_3 = 1\\
			2x_1 + 5x_2 + 4x_3 = 4 & L_2 \rightarrow L_2 - 2L_1\\
			x_1 - 3x_2 - 2x_3 = 5
		\end{cases} \sim
		\begin{cases}
			x_1 + 4x_2 + 3x_3 = 1\\
			\phantom{0x_1} -3x_2 - 2x_3 = 2\\
			x_1 - 3x_2 - 2x_3 = 5 & L_3 \rightarrow L_2 - L_1
		\end{cases}\\ \\ & \sim
		\begin{cases}
			x_1 + 4x_2 + 3x_3 = 1\\
			\phantom{0x_1} - 3x_2 - 2x_3 = 2 & L_2 \rightarrow (-1/3)L_2\\
			\phantom{0x_1} - 7x_2 - 5x_3 = 4
		\end{cases} \sim
		\begin{cases}
			x_1 + 4x_2 + 3x_3 = 1\\
			\phantom{0x_1} x_2 + (2/3)x_3 = (-2/3)\\
			\phantom{0x_1} - 7x_2 - 5x_3 = 4 & L_3 \rightarrow L_3 + 7L_2
		\end{cases}\\ \\ & \sim
		\begin{cases}
			x_1 + 4x_2 + 3x_3 = 1\\
			\phantom{0x_1} x_2 + (2/3)x_3 = (-2/3)\\
			\phantom{0x_1} \phantom{0x_2}  -(1/3)x_3 = -(2/3)
		\end{cases}
	\end{align*}
	Assim encontramos
	\[
		x_1 = 3, \quad x_2 = -2, \quad x_3 = 2.
	\]
\end{exemplo}

\begin{definicao}\index{Sistemas Equivalentes}
	Dois sistemas de equa\c{c}\~oes lineares s\~ao chamados de \textbf{equivalentes} se, e somente, se toda solu\c{c}\~ao de qualquer um dos sistemas \'e solu\c{c}\~ao do outro.
\end{definicao}

Dado um sistema linear
\begin{equation}
	\begin{cases}
		a_{11}x_1 + a_{12}x_2 + \cdots + a_{1n}x_n = b_1\\
		a_{21}x_1 + a_{22}x_2 + \cdots + a_{2n}x_n = b_2\\
		\qquad \vdots\\
		a_{m1}x_1 + a_{m2}x_2 + \cdots + a_{mn}x_n = b_m\\
	\end{cases}
\end{equation}
com o objetivo de simplificar sua nota\c{c}\~ao vamos escrev\^e-lo na forma
\begin{equation}\label{formamatricial}
	AX = B
\end{equation}
onde
\begin{enumerate}
	\item
	\[
		A = \begin{bmatrix}
				a_{11} & a_{12} & \cdots & a_{1n}\\
				\vdots & & & \vdots\\
				a_{m1} & a_{m2} & \cdots & a_{mn}
			\end{bmatrix}_{m\times n}; \quad a_{ij} \in \cp{K},\ 1 \le i \le m,\ 1 \le j \le n
	\]
	\'e chamada \textbf{matriz dos coeficientes do sistema};
	\item
	\[
		X = \begin{bmatrix}
			x_1\\
			x_2\\
			\vdots\\
			x_n
		\end{bmatrix}_{n \times 1}
	\]
	\item
	\[
		B = \begin{bmatrix}
			b_1\\
			b_2\\
			\vdots\\
			b_m
		\end{bmatrix}_{m \times 1}; \quad b_1, b_2, \dots, b_m \in \cp{K}.
	\]
\end{enumerate}

Uma outra matriz que podemos associar ao sistema \eqref{sistemalinear} \'e
\[
	\begin{amatrix}{4}
		a_{11} & a_{12} & \cdots & a_{1n} & b_1\\
		a_{21} & a_{22} & \cdots & a_{2n} & b_2\\
		\vdots & \vdots & \vdots & \vdots & \vdots\\
		a_{m1} & a_{m2} & \cdots & a_{mn} & b_m\\
	\end{amatrix}
\]
que \'e chamada de \textbf{matriz ampliada do sistema} ou \textbf{matriz aumentada do sistema}.

Na forma matricial as opera\c{c}\~oes elementares s\~ao descritas como:\index{Opera\c{c}\~oes Elementares!Sobre Matrizes}
\begin{itemize}
	\item[$e_1$)] Trocar a $i$-\'esima linha de $A$ pela $j$-\'esima linha de $A$: $L_i \leftrightarrow L_j$;
	\item[$e_2$)] Multiplica\c{c}\~ao da $i$-\'esima linha de $A$ por um escalar $\alpha \in \cp{K}$ n\~ao nulo: $L_i \rightarrow \alpha L_i$;
	\item[$e_3$)] Substitui\c{c}\~ao da $i$-\'esima linha de $A$ pela $i$-\'esima linha mais $\alpha$ vezes a $j$-\'esima linha: $L_i \rightarrow L_i + \alpha L_j$.
\end{itemize}

\begin{observacao}
	Denotaremos a matriz
	\[
		\begin{bmatrix}
			0_{\cp{K}} & 0_{\cp{K}} \cdots & 0_{\cp{K}}\\
			0_{\cp{K}} & 0_{\cp{K}} \cdots & 0_{\cp{K}}\\
			\vdots & & \vdots\\
			0_{\cp{K}} & 0_{\cp{K}} \cdots & 0_{\cp{K}}
		\end{bmatrix},
	\]
	onde $0_{\cp{K}}$ \'e o elemento neutro da soma no corpo $\cp{K}$, simplesmente por $0$.
\end{observacao}

Uma raz\~ao para nos restringirmos a estes tr\^es tipos simples de opera\c{c}\~oes sobre linhas \'e que, tendo efetuado uma tal opera\c{c}\~ao $e$ sobre uma matriz $A$, podemos desfazer essa opera\c{c}\~ao efetuando uma opera\c{c}\~ao de mesmo tipo sobre $e(A)$.

\begin{teorema}
	A cada opera\c{c}\~ao elementar sobre linhas $e$, corresponde uma opera\c{c}\~ao elementar sobre linhas $e'$, do mesmo tipo que $e$, tal que $e'(e(A)) = A$ para qualquer matriz $A$. Em outras palavras, a opera\c{c}\~ao inversa de uma opera\c{c}\~ao elementar sobre linhas existe e \'e uma opera\c{c}\~ao elementar sobre linhas do mesmo tipo.
\end{teorema}
\begin{prova}
	Vamos verificar que cada uma das opera\c{c}\~oes elementares possui uma opera\c{c}\~ao inversa. Seja $A$ uma matriz $m \times n$ sobre o corpo $\cp{K}$
	\[
		A =
		\begin{bmatrix}
			a_{11} & a_{12} & \cdots & a_{1n}\\
			a_{21} & a_{22} & \cdots & a_{2n}\\
			\vdots & & & \vdots\\
			a_{m1} & a_{m2} & \cdots & a_{mn}
		\end{bmatrix}.
	\]
	\begin{enumerate}
		\item [e1)] Suponha que $e$ seja a opera\c{c}\~ao que troca a linha $i$ pela linha $j$ de $A$. Temos
		\[
			e(A) =
				\begin{bmatrix}
					a_{11} & a_{12} & \cdots & a_{1n}\\
					a_{21} & a_{22} & \cdots & a_{2n}\\
					\vdots\\
					a_{j1} & a_{j2} & \cdots & a_{jn}\\
					\vdots\\
					a_{i1} & a_{i2} & \cdots & a_{in}\\
					\vdots\\
					a_{m1} & a_{m2} & \cdots & a_{mn}
				\end{bmatrix}.
		\]
		Ent\~ao, seja $e'$ a opera\c{c}\~ao que troca a linha $i$ pela linha $j$ de $e(A)$. Assim
		\[
			e'(e(A)) = A
		\]
		como quer{\'\i}amos.

		\item [e2)] Suponha que $e$ seja a opera\c{c}\~ao que multiplica a $i$-\'esima de $A$ por $\alpha \in \cp{K}$, onde $\alpha \ne 0_\cp{K}$. Temos
		\[
			e(A) =
				\begin{bmatrix}
					a_{11} & a_{12} & \cdots & a_{1n}\\
					a_{21} & a_{22} & \cdots & a_{2n}\\
					\vdots\\
					\alpha a_{i1} & \alpha a_{i2} & \cdots & \alpha a_{in}\\
					\vdots\\
					a_{m1} & a_{m2} & \cdots & a_{mn}
				\end{bmatrix}.
		\]
		Seja $e'$ a opera\c{c}\~ao que multiplica a linha $i$ de $e(A)$ por $\alpha^{-1} \in \cp{K}$. Ent\~ao
		\[
			e'(e(A)) = A.
		\]
		\item [e3)] Suponha que $e$ seja a opera\c{c}\~ao que substitui a linha $i$ de $A$ pela linha $i$ mais $\alpha$ vezes a linha $j$. Temos
		\[
			e(A) =
				\begin{bmatrix}
					a_{11} & a_{12} & \cdots & a_{1n}\\
					a_{21} & a_{22} & \cdots & a_{2n}\\
					\vdots\\
					a_{i1} + \alpha a_{j1} & a_{i2} + \alpha a_{j2} & \cdots & a_{in} + \alpha a_{jn}\\
					\vdots\\
					a_{m1} & a_{m2} & \cdots & a_{mn}
				\end{bmatrix}.
		\]
		Seja $e'$ a opera\c{c}\~ao que substitui a linha $i$ de $e(A)$ pela linha $i$ mais $(-\alpha)$ vezes a linha $j$. Ent\~ao
		\[
			e'(e(A)) =
					\begin{bmatrix}
						a_{11} & a_{12} & \cdots & a_{1n}\\
						a_{21} & a_{22} & \cdots & a_{2n}\\
						\vdots\\
						a_{i1} + \alpha a_{j1} + (-\alpha)a_{j1} & a_{i2} + \alpha a_{j2} + (-\alpha)a_{j2} & \cdots & a_{in} + \alpha a_{jn} + (-\alpha)a_{jn}\\
						\vdots\\
						a_{m1} & a_{m2} & \cdots & a_{mn}
					\end{bmatrix}.
		\]
		e assim
		\[
			e'(e(A)) = A.
		\]
	\end{enumerate}
	Portanto cada opera\c{c}\~ao elementar sobre linhas possui uma opera\c{c}\~ao inversa.
\end{prova}

\begin{definicao}\index{Matriz!Linha Equivalente}
	Se $A$ e $B$ s\~ao matrizes $m \times n$, dizemos que $B$ \'e \textbf{linha-equivalente} a $A$, se $B$ for obtida de $A$ atrav\'es de uma quantidade finita de opera\c{c}\~oes elementares sobre as linhas de $A$.
\end{definicao}

\begin{notacao}
	$A \rightarrow B$ ou $A \sim B$.
\end{notacao}

\begin{exemplo}
	A matriz
	\[
		B =
			\begin{bmatrix}
				1 & 0\\
				0 & 1\\
				0 & 0
			\end{bmatrix}
	\]
	\'e linha equivalente \`a matriz
	\[
		A =
			\begin{bmatrix}
				\phantom{-}1 & \phantom{-}0\\
				\phantom{-}4 & -1\\
				-3 & \phantom{-}4
			\end{bmatrix}
	\]
	pois
	\begin{align*}
		A &=
			\left[
				\begin{array}{cc}
					\phantom{-}1 & \phantom{-}0\\
					\phantom{-}4 & -1\\
					-3 & \phantom{-}4
				\end{array}
			\right]
			\begin{array}{l}
				\phantom{x}\\
				L_2 \to L_2 - 4L_1\\
				\phantom{x}
			\end{array} \sim
			\left[
				\begin{array}{cc}
					\phantom{-}1 & \phantom{-}0\\
					\phantom{-}0 & -1\\
					-3 & \phantom{-}4
				\end{array}
			\right]
			\begin{array}{l}
				\phantom{x}\\
				\phantom{x}\\
				L_3 \to L_3 + 3L_1
			\end{array}\\ \\ &\sim
			\left[
				\begin{array}{cc}
					1 & \phantom{-}0\\
					0 & -1\\
					0 & \phantom{-}4
				\end{array}
			\right]
			\begin{array}{l}
				\phantom{x}\\
				L_2 \to (-1)L_2\\
				\phantom{x}
			\end{array} \sim
			\left[
				\begin{array}{cc}
					1 & 0\\
					0 & 1\\
					0 & 4
				\end{array}
			\right]
			\begin{array}{l}
				\phantom{x}\\
				\phantom{x}\\
				L_3 \to L_3 - 4L_2
			\end{array}\\ \\ &\sim
			\left[
				\begin{array}{cc}
					1 & 0\\
					0 & 1\\
					0 & 0
				\end{array}
			\right] = B.
	\end{align*}
\end{exemplo}

\begin{teorema}
	Se $X_1$ e $X_2$ s\~ao duas solu\c{c}\~oes de
	\[
	AX = 0,
	\]
	ent\~ao $\alpha X_1 + \beta X_2$ tamb\'em \'e solu\c{c}\~ao de $AX = 0$, para quaisquer $\alpha$, $\beta \in \cp{K}$.
\end{teorema}

\begin{teorema}
	Se $A$ e $B$ s\~ao matrizes $m \times n$ que s\~ao linha-equivalentes, ent\~ao os sistemas homog\^eneos de equa\c{c}\~oes lineares $AX = 0$ e $BX = 0$ t\^em exatamente as mesmas solu\c{c}\~oes.
\end{teorema}
\begin{prova}
	Suponha que podemos obter a matriz $B$ \`a partir da matriz $A$ por meio de uma sequ\^encia finita de opera\c{c}\~oes elementares sobre linhas:
	\[
	A = A_0 \sim A_1 \sim A_2 \sim \cdots \sim A_r = B.
	\]
	Nesta situa\c{c}\~ao, para provar que $AX = 0$ e $BX = 0$ tem as mesmas solu\c{c}\~oes basta provar que $A_iX = 0$ e $A_{i + 1}X = 0$ tem as mesmas solu\c{c}\~oes, isto \'e, que uma opera\c{c}\~ao elementar sobre linhas n\~ao altera o conjunto das solu\c{c}\~oes.

	Assim podemos supor que $B$ \'e obtida de $A$ por meio de uma \'unica opera\c{c}\~ao elementar. Qualquer que seja a opera\c{c}\~ao elementar, $e_1$ ou $e_2$ ou $e_3$, cada equa\c{c}\~ao do sistema $BX = 0$ ser\'a uma combina\c{c}\~ao das equa\c{c}\~oes do sistema $AX = 0$. Como a inversa de uma opera\c{c}\~ao elementar sobre linhas \'e ainda uma opera\c{c}\~ao elementar sobre linhas, cada equa\c{c}\~ao de $AX = 0$ tamb\'em ser\'a uma combina\c{c}\~ao das equa\c{c}\~oes em $BX = 0$. Logo toda solu\c{c}\~ao de $AX = 0$ tamb\'em \'e solu\c{c}\~ao de $BX = 0$ e toda solu\c{c}\~ao de $BX = 0$ tamb\'em \'e solu\c{c}\~ao de $AX = 0$, como quer{\'\i}amos.
\end{prova}

\begin{exemplo}
	Considere o sistema homog\^eneo $AX = 0$, onde:
	\begin{enumerate}[label={\arabic*})]
		\item $A = \begin{bmatrix}
						2 & -1 & \phantom{-}3 & \phantom{-}2\\
						1 & \phantom{-}4 & \phantom{-}0 & -1\\
						2 & \phantom{-}6 & -1 & \phantom{-}5
					\end{bmatrix}.$
		Para encontrar a solu\c{c}\~ao deste sistema s\'o precisamos encontrar uma matriz $B$ que seja linha equivalente \`a $A$ e que seja mais f\'acil de determinar a solu\c{c}\~ao do sistema resultante. Assim, vamos executar as opera\c{c}\~oes elementares em $A$ de modo a simplific\'a-la:
		\begin{align*}
			A &=
				\left[
					\begin{array}{cccc}
						2 & -1 & \phantom{-}3 & \phantom{-}2\\
						1 & \phantom{-}4 & \phantom{-}0 & -1\\
						2 & \phantom{-}6 & -1 & \phantom{-}5
					\end{array}
				\right]
				\begin{array}{l}
					L_1 \leftrightarrow L_2\\
					\phantom{x}\\
					\phantom{x}
				\end{array} \sim
				\left[
					\begin{array}{cccc}
						1 & \phantom{-}4 & \phantom{-}0 & -1\\
						2 & -1 & \phantom{-}3 & \phantom{-}2\\
						2 & \phantom{-}6 & -1 & \phantom{-}5
					\end{array}
				\right]
				\begin{array}{l}
					\phantom{x}\\
					L_2 \to L_2 - 2L_1\\
					L_3 \to L_3 - 2L_1
				\end{array}\\ \\ &\sim
				\left[
					\begin{array}{cccc}
						1 & \phantom{-}4 & \phantom{-}0 & -1\\
						0 & -9 & \phantom{-}3 & \phantom{-}4\\
						0 & -2 & -1 & \phantom{-}7
					\end{array}
				\right]
				\begin{array}{l}
					\phantom{x}\\
					L_2 \leftrightarrow L_3\\
					\phantom{x}
				\end{array} \sim
				\left[
					\begin{array}{cccc}
						1 & \phantom{-}4 & 0 & -1\\
						0 & -2 & -1 & \phantom{-}7\\
						0 & -9 & \phantom{-}3 & \phantom{-}4
					\end{array}
				\right]
				\begin{array}{l}
					\phantom{x}\\
					L_2 \to (-1/2)L_2\\
					\phantom{x}
				\end{array}\\ \\ &\sim
				\left[
					\begin{array}{cccc}
						1 & \phantom{-}4 & 0 & -1\\
						0 & \phantom{-}1 & 1/2 & -7/2\\
						0 & -9 & 3 & \phantom{-}4
					\end{array}
				\right]
				\begin{array}{l}
					\phantom{x}\\
					\phantom{x}\\
					L_3 \to L_3 + 9L_2
				\end{array} \sim
				\left[
					\begin{array}{cccc}
						1 & 4 & 0 & -1\\
						0 & 1 & 1/2 & -7/2\\
						0 & 0 & 15/2 & -55/2
					\end{array}
				\right]
		\end{align*}
		assim obtemos o sistema
		\[
			\begin{cases}
				x_1 + 4x_2 - x_4 = 0\\
				x_2 + (1/2)x_3 - (7/2)x_4 = 0\\
				(15/2)/x_3 - (55/2)x_4 = 0
			\end{cases}.
		\]
		Isolando $x_3$ na \'ultima equa\c{c}\~ao temos a solu\c{c}\~ao dada por
		\[
			S = \left\{\left(\dfrac{-17}{3}x_4, \dfrac{5}{3}x_4, \dfrac{11}{3}x_4, x_4\right) \mid x_4 \in \real\right\}.
		\]

		\item $A = \begin{bmatrix}
		-1 & i\\
		-i & 1\\
		\phantom{-}1 & 2
		\end{bmatrix}.$ Temos:
		\begin{align*}
			A &=
				\left[
					\begin{array}{cc}
						-1 & i\\
						-i & 1\\
						\phantom{-}1 & 2
					\end{array}
				\right]
				\begin{array}{l}
					L_1 \leftrightarrow L_3\\
					\phantom{x}\\
					\phantom{x}
				\end{array} \sim
				\left[
					\begin{array}{cc}
						\phantom{-}1 & 2\\
						-i & 1\\
						-1 & i
					\end{array}
				\right]
				\begin{array}{l}
					\phantom{x}\\
					L_2 \to L_2 + iL_1\\
					L_3 \to L_3 + L_1
				\end{array}\\ \\ &\sim
				\left[
					\begin{array}{cc}
						1 & 2\\
						0 & 1 + 2i\\
						0 & 2 + i
					\end{array}
				\right]
				\begin{array}{l}
					\phantom{x}\\
					L_2 \to \dfrac{1 - 2i}{5}L_2\\
					\phantom{x}
				\end{array} \sim
				\left[
					\begin{array}{cc}
						1 & 2\\
						0 & 1\\
						0 & 2 + i
					\end{array}
				\right]
				\begin{array}{l}
					\phantom{x}\\
					\phantom{x}\\
					L_3 \to L_3 - (2 + i)L_2
				\end{array}\\ \\ &\sim
				\left[
					\begin{array}{cc}
						1 & 2\\
						0 & 1\\
						0 & 0
					\end{array}
				\right]
		\end{align*}
		Assim obtemos o sistema
		\[
			\begin{cases}
				x_1 + 2x_2 = 0\\
				x_2 = 0
			\end{cases}
		\]
		cuja solu\c{c}\~ao \'e $x_1 = x_2 = 0$.
	\end{enumerate}
\end{exemplo}

\begin{definicao}\label{linhareduzida}\index{Matriz!Linha-reduzida}
	Uma matriz R $m \times n$ \'e chamada de \textbf{linha-reduzida} se:
	\begin{enumerate}[label={\roman*})]
		\item o primeiro elemento n\~ao nulo em cada linha n\~ao nula de $R$ \'e $1_\cp{K}$.
		\item cada coluna de $R$ que cont\'em o primeiro elemento n\~ao nulo de alguma linha tem todos os seus outros elementos nulos.
	\end{enumerate}
\end{definicao}

\begin{exemplo}
	\begin{enumerate}[label={\arabic*})]
		\item Um exemplo de uma matriz linha-reduzida \'e a matriz identidade $n \times n$. Tal matriz pode ser definida por
		\[
			I = (a_{ij})_{1 \le i,j \le n}
		\]
		onde
		\[
			a_{ij} = \delta_{ij} =
			\begin{cases}
				1, & \mbox{ se } i = j\\
				0, & \mbox{ se } i \ne j
			\end{cases}.
		\]
		O s{\'\i}mbolo $\delta_{ij}$ \'e chamada \textbf{s{\'\i}mbolo de Kronecher} \'e ser\'a utilizado com certa frequ\^encia.
		\item As matrizes
		\[
			A =
			\begin{bmatrix}
				1 & 0 & \phantom{-}0 & 0\\
				0 & 1 & -1 & 0\\
				0 & 0 & \phantom{-}1 & 0
			\end{bmatrix};\quad
			B =
			\begin{bmatrix}
				0 & 2 & \phantom{-}1\\
				1 & 0 & -3\\
				0 & 0 & \phantom{-}0
			\end{bmatrix}
		\]
		n\~ao s\~ao linha-reduzidas.
	\end{enumerate}
\end{exemplo}

\begin{teorema}
	Toda matriz $m \times n$ sobre um corpo $\cp{K}$ \'e linha-equivalente a uma matriz linha-reduzida.
\end{teorema}
\begin{prova}
	Seja $A$ uma matriz $m \times n$ sobre um corpo $\cp{K}$. Se todo elemento na primeira linha de $A$ \'e $0_\cp{K}$, ent\~ao a condi\c{c}\~ao (a) de \eqref{linhareduzida} est\'a satisfeita no que diz respeito a linha 1. Se a linha 1 tem um elemento n\~ao nulo, seja $r$ o menor inteiro positivo $j$ tal que $a_{1r} \ne 0$. Multiplique a linha 1 por $a_{1r}^{-1}$ e condi\c{c}\~ao (a) de \eqref{linhareduzida} est\'a satisfeita em rela\c{c}\~ao a linha 1. Agora, para cada $i \ge 2$, somemos $-a_{ir}$ vezes a linha 1 \`a linha i. Assim o primeiro elemento n\~ao nulo da linha 1 ocorre na coluna $r$, este elemento \'e $1_\cp{K}$, e todos os outros elementos da coluna $r$ s\~ao nulos.

	Considere agora a matriz que resultou das opera\c{c}\~oes acima. Se todo elemento na linha 2 \'e nulo, nada h\'a a fazer. Se algum elemento na linha 2 \'e n\~ao nulo, multiplicamos a linha 2 por um escalar de modo que o primeiro elemento n\~ao nulo da linha 2 seja $1_\cp{K}$. Caso o primeiro elemento n\~ao nulo da linha 1 ocorra na coluna $r$, o primeiro elemento n\~ao nulo da linha 2 n\~ao pode ocorrer na coluna $r$. Digamos ent\~ao que ele ocorra na coluna $r'$. Somando m\'ultiplos adequados da linha 2 \`as diversas linhas, podemos fazer com que todos os elementos da coluna $r'$ seja nulos, com exce\c{c}\~ao do elemento $1_\cp{K}$ da linha 2. O importante a ser observado \'e: ao efetuarmos estas \'ultimas opera\c{c}\~oes, n\~ao alteramos os elementos da linha 1 na colunas 1, 2, \dots, $r$; al\'em disso, n\~ao alteramos nenhum elemento da coluna $r$. \'E claro que, se a linha 1 fosse identicamente nula, as opera\c{c}\~oes com a linha 2 n\~ao afetariam a linha 1.

	Operando com uma linha de cada vez da maneira acima, \'e evidente que, com uma quantidade finita de passos, chegamos a uma matriz linha-reduzida.
\end{prova}

\begin{definicao}\index{Matriz!Na forma escada}
	Uma matriz $R$ $m \times n$ \'e chamada uma \textbf{matriz linha-reduzida \`a forma em escada} se:
	\begin{enumerate}[label={\roman*})]
		\item $R$ \'e linha-reduzida;
		\item toda linha de $R$ cujos elementos s\~ao todos nulos ocorre abaixo de todas as linhas que possuem um elemento n\~ao-nulo;
		\item se as linhas 1, 2, \dots, $r$ s\~ao as linhas n\~ao-nulas de $R$ e se o primeiro elemento n\~ao-nulo da linha $i$ ocorre na coluna $k_i$, $i = 1$, \dots, $r$, ent\~ao $k_1 < k_2 < \cdots < k_r$.
	\end{enumerate}
\end{definicao}

\begin{exemplo}
	\begin{enumerate}[label={\arabic*})]
		\item  A matriz identidade e a matriz nula s\~ao linha-reduzidas \`a forma escada;
		\item $\begin{bmatrix}
		1 & 0 & \phantom{-}0 & 0\\
		0 & 1 & -1 & 0\\
		0 & 0 & \phantom{-}1 & 0
		\end{bmatrix}$ N\~ao \'e linha-reduzida \`a forma escada.
		\item $\begin{bmatrix}
		0 & 2 & \phantom{-}1\\
		1 & 0 & -3\\
		0 & 0 & \phantom{-}0
		\end{bmatrix}$ N\~ao \'e linha-reduzida \`a forma escada.
		\item $\begin{bmatrix}
		0 & 1 & -3 & 0 & 2\\
		0 & 0 & \phantom{-}0 & 1 & 2\\
		0 & 0 & \phantom{-}0 & 0 & 0
		\end{bmatrix}$ \'E linha-reduzida \`a forma escada.
	\end{enumerate}
\end{exemplo}

\begin{teorema}
	Toda matriz $A$ $m \times n$ \'e linha-equivalente a uma matriz linha-reduzida \`a forma em escada.
\end{teorema}
\begin{prova}
	Sabemos que $A$ \'e linha-equivalente a uma matriz linha-reduzida. Portanto, basta notar que, efetuando uma quantidade finita de permuta\c{c}\~oes das linhas de uma matriz linha-reduzida, podemos transform\'a-la numa matriz linha-reduzida \`a forma em escada.
\end{prova}

\begin{definicao}\index{Posto!de uma matriz}\index{Nulidade!de uma matriz}
	Dada uma matriz $A$ $m \times n$, seja $B$ a matriz $m \times n$ linha-reduzida \`a forma em escada linha-equivalente a $A$. O \textbf{posto} de $A$, denotado por $p$, \'e o n\'umero de linhas n\~ao-nulas de $B$. A \textbf{nulidade} de $A$ \'e o n\'umero $n - p$.
\end{definicao}

\begin{exemplo}
	Qual o posto e a nulidade da matriz $A$, onde
	\[
		A =
		\begin{bmatrix}
			\phantom{-}1 & \phantom{-}2 & 1 & 0\\
			-1 & \phantom{-}0 & 3 & 5\\
			\phantom{-}1 & -2 & 1 & 1
		\end{bmatrix}?
	\]
	Precisamos primeiro reduzir $A$ a sua forma escada:
	\begin{align*}
		A &=
			\left[
			\begin{array}{cccc}
				\phantom{-}1 & \phantom{-}2 & 1 & 0\\
				-1 & \phantom{-}0 & 3 & 5\\
				\phantom{-}1 & -2 & 1 & 1
			\end{array}
			\right]
			\begin{array}{l}
				\\
				L_2 \to L_2 + L_1\\
				L_3 \to L_3 - L_1
			\end{array} \sim
			\left[
				\begin{array}{cccc}
					1 & \phantom{-}2 & 1 & 0\\
					0 & \phantom{-}2 & 4 & 5\\
					0 & -4 & 0 & 1
				\end{array}
			\right]
			\begin{array}{l}
				\\
				L_2 \to (1/2)L_2\\
				\phantom{x}
			\end{array}\\ \\ &\sim
			\left[
				\begin{array}{cccc}
					1 & \phantom{-}2 & 1 & 0\\
					0 & \phantom{-}1 & 2 & 5/2\\
					0 & -4 & 0 & 1
				\end{array}
			\right]
			\begin{array}{l}
				L_1 \to L_1 - 2L_2\\
				\phantom{x}\\
				L_3 \to L_3 + 4L_2
			\end{array} \sim
			\left[
				\begin{array}{cccc}
					1 & 0 & -3 & -5\\
					0 & 1 & \phantom{-}2 & \phantom{-}5/2\\
					0 & 0 & \phantom{-}8 & \phantom{-}11
				\end{array}
			\right]
			\begin{array}{l}
				\phantom{x}\\
				\phantom{x}\\
				L_3 \to (1/8)L_3
			\end{array}\\ \\ &\sim
			\left[
				\begin{array}{cccc}
					1 & 0 & -3 & -5\\
					0 & 1 & \phantom{-}2 & \phantom{-}5/2\\
					0 & 0 & \phantom{-}1 & \phantom{-}11/8
				\end{array}
			\right]
			\begin{array}{l}
				L_1 \to L_1 + 3L_3\\
				L_2 \to L_2 - 2L_3\\
				\phantom{x}
			\end{array} \sim
			\left[
				\begin{array}{cccc}
					1 & 0 & 0 & -7/8\\
					0 & 1 & 0 & -1/4\\
					0 & 0 & 1 & \phantom{-}11/8
				\end{array}
			\right]
	\end{align*}
	Logo o posto de $A$ \'e $p = 3$ e a nulidade \'e $n - p = 4 - 3 = 1$.
\end{exemplo}

Considere o sistema
\begin{equation}\label{equacaolinear}
	AX = B
\end{equation}
onde $A$ \'e uma matriz $m \times n$ e $B$ \'e uma matriz $m \times 1$, ambas com entradas no corpo $\cp{K}$ e $X$ \'e uma matriz $n \times 1$. Observe que, enquanto uma sistema homog\^eneo $AX = 0$ sempre admite a solu\c{c}\~ao
\[
x_1 = x_2 = \cdots = x_n = 0_\cp{K},
\]
um sistema n\~ao homog\^eneo pode ter:
\begin{enumerate}
	\item Uma \'unica solu\c{c}\~ao $x_1 = \alpha_1$, $x_2 = \alpha_2$, \dots, $x_n = \alpha_n$, onde $\alpha_i \in \cp{K}$, para $i = 1$, 2, \dots, $n$. Neste caso dizemos que o sistema \'e \textbf{poss{\'\i}vel e determinado}.
	\item Mais de uma solu\c{c}\~ao. Neste caso dizemos que o sistema \'e \textbf{poss{\'\i}vel e indeterminado}. Caso o corpo $\cp{K}$ tenha infinitos elementos, o sistema ter\'a infinitas solu\c{c}\~oes.
	\item Nenhuma solu\c{c}\~ao. Neste caso dizemos o que sistema \'e \textbf{imposs{\'\i}vel}.
\end{enumerate}

Com o objetivo de resolver o sistema \eqref{equacaolinear} vamos come\c{c}ar formando a matriz ampliada
\[
	P = [A|B] =
	\begin{amatrix}{4}
		a_{11} & a_{12} & \dots & a_{1n} & b_1\\
		a_{21} & a_{22} & \dots & a_{2n} & b_2\\
		\vdots & \vdots & \vdots & \vdots & \vdots\\
		a_{m1} & a_{m2} & \dots & a_{mn} & b_m\\
	\end{amatrix}_{m \times (n + 1)}.
\]

Sabemos que $P$ \'e linha-equivalente a uma matriz linha-reduzida \`a forma em escada $R$. A \'ultima coluna de $R$ cont\'em elementos $z_1$, $z_2$, \dots, $z_m$ que s\~ao resultados das opera\c{c}\~oes elementares aplicadas \`a matriz $P$. Seja
\[
	Z =
	\begin{bmatrix}
		z_1\\
		z_2\\
		\vdots\\
		z_m
	\end{bmatrix}.
\]
Ent\~ao $R$ pode ser escrita como $R = [R' \mid Z]$. Como no caso homog\^eneo, \'e poss{\'\i}vel mostrar que os sistemas
\[
	AX = B \mbox{ e } R'X = Z
\]
possuem exatamente as mesmas solu\c{c}\~oes.

As possibilidades para as solu\c{c}\~oes de tal sistema s\~ao descritas no seguinte teorema:

\begin{teorema}
	Considere o sistema
	\[
		AX = B
	\]
	onde $A$ \'e uma matriz $m \times n$ e $B$ \'e uma matriz $m \times 1$, ambas com entradas no corpo $\cp{K}$ e $X$ \'e uma matriz $n \times 1$. Ent\~ao:
	\begin{enumerate}[label={\roman*})]
		\item O sistema tem solu\c{c}\~ao se, e somente se, o posto da matriz ampliada \'e igual ao posto da matriz dos coeficientes.

		\item Se a matriz ampliada e a matriz dos coeficientes t\^em o mesmo posto $p$ e $p = n$, ent\~ao a solu\c{c}\~ao \'e \'unica.

		\item Se a matriz ampliada e a matriz dos coeficientes t\^em o mesmo posto $p$ e $p < n$, ent\~ao podemos escolher $n - p$ vari\'aveis, e as outras $p$ vari\'aveis ser\~ao dadas em fun\c{c}\~ao destas $n - p$ vari\'aveis escolhidas.
	\end{enumerate}
	O n\'umero $n - p$ \'e chamado de \textbf{grau de liberdade} e as $n - p$ vari\'aveis s\~ao chamadas de \textbf{vari\'aveis livres}.
\end{teorema}
\begin{prova}
	\textit{$1^a$ Parte: Se existe solu\c{c}\~ao para o sistema, ent\~ao a matriz ampliada e a matriz dos coeficientes t\^em o mesmo posto:} Para mostrar isso, vamos provar que se a matriz ampliada e a matriz dos coeficientes tiverem postos diferentes, ent\~ao o sistema n\~ao ter\'a solu\c{c}\~ao. Observe primeiro que o posto da matriz ampliada n\~ao pode ser menor que o posto da matriz dos coeficientes uma vez que a matriz ampliada \'e formada a partir da matriz dos coeficientes. Assim o \'unico caso poss{\'\i}vel \'e o posto da matriz ampliada ser maior que o posto da matriz dos coeficientes. Ent\~ao esta matriz, quando reduzida \`a forma em escada deve conter uma linha da forma
	\[
		\begin{bmatrix}
			0_\cp{K} & 0_\cp{K} & \cdots & 0_\cp{K} & \mid & 1_\cp{K}
		\end{bmatrix}.
	\]
	Logo o sistema associado a essa matriz tem uma equa\c{c}\~ao do tipo
	\[
		0_\cp{K}x_1 + 0_\cp{K}x_2 + \cdots + 0_\cp{K}x_n = 1_\cp{K}
	\]
	o que \'e imposs{\'\i}vel. Logo n\~ao existe solu\c{c}\~ao.

	\textit{$2^a$ Parte: Se o posto \'e igual, ent\~ao existe solu\c{c}\~ao:} Nesta situa\c{c}\~ao podem ocorrer dois casos:
	\begin{enumerate}
		\item Se $p = n$, ent\~ao a matriz linha-reduzida \`a forma em escada tem a forma
		\[
			\begin{amatrix}{5}
				1_\cp{K} & 0_\cp{K} & 0_\cp{K} & \cdots & 0_\cp{K} & z_1\\
				0_\cp{K} & 1_\cp{K} & 0_\cp{K} & \cdots & 0_\cp{K} & z_2\\
				\vdots & \vdots & \vdots & \vdots & \vdots & \vdots\\
				0_\cp{K} & 0_\cp{K} & 0_\cp{K} & \cdots & 1_\cp{K} & z_n\\
				0_\cp{K} & 0_\cp{K} & 0_\cp{K} & 0_\cp{K} & 0_\cp{K} & 0_\cp{K}\\
				\vdots & \vdots & \vdots & \vdots & \vdots & \vdots\\
				0_\cp{K} & 0_\cp{K} & 0_\cp{K} & 0_\cp{K} & 0_\cp{K} & 0_\cp{K}
			\end{amatrix}
		\]
		e a solu\c{c}\~ao do sistema ser\'a
		\[
			x_1 = z_1, x_2 = z_2, \cdots, x_n = z_n.
		\]
		\item Se $p \ne n$, ent\~ao devemos ter $p < n$. Caso $p > n$, como a matriz est\'a na forma escada o elemento $1_\cp{K}$ deve ocorrer em duas linhas diferentes, mas na mesma coluna. Mas neste caso, podemos anular uma destas linhas repetidas. Logo, $p < n$. Neste caso a matriz na forma escada pode ter a forma:
		\begin{enumerate}
			\item
			\[
				\begin{amatrix}{9}
					1_\cp{K} & 0_\cp{K} & 0_\cp{K} & \cdots & 0_\cp{K} & a_{1p+1} & a_{1p+2} & \cdots & a_{1n} & z_1\\
					0_\cp{K} & 1_\cp{K} & 0_\cp{K} & \cdots & 0_\cp{K} & a_{2p+1} & a_{2p+2} & \cdots & a_{2n} & z_2\\
					\vdots & \vdots & \vdots & \vdots & \vdots & \vdots & \vdots & \vdots & \vdots & \vdots\\
					0_\cp{K} & 0_\cp{K} & 0_\cp{K} & \cdots & 1_\cp{K} & a_{pp+1} & a_{pp+2} & \cdots & a_{pn} & z_p\\
					0_\cp{K} & 0_\cp{K} & 0_\cp{K} & 0_\cp{K} & 0_\cp{K} & 0_\cp{K} & 0_\cp{K} & \cdots & 0_\cp{K} & 0_\cp{K}\\
					\vdots & \vdots & \vdots & \vdots & \vdots & \vdots & \vdots & \vdots & \vdots & \vdots\\
					0_\cp{K} & 0_\cp{K} & 0_\cp{K} & 0_\cp{K} & 0_\cp{K} & 0_\cp{K} & 0_\cp{K} & \cdots & 0_\cp{K}  & 0_\cp{K}
				\end{amatrix}.
			\]
			Neste caso teremos
			\[
				\begin{cases}
					x_1 = z_1 + (-a_{1 p + 1})x_{p + 1} + (-a_{1 p + 2})x_{p + 2} + \cdots + (-a_{1n})x_{n}\\
					x_2 = z_2 + (-a_{2 p + 1})x_{p + 1} + (-a_{2 p + 2})x_{p + 2} + \cdots + (-a_{2n})x_{n}\\
					\qquad \vdots\\
					x_p = z_p + (-a_{p p + 1})x_{p + 1} + (-a_{p p + 2})x_{p + 2} + \cdots + (-a_{pn})x_{n}\\
				\end{cases}
			\]
			e o sistema ter\'a mais de uma solu\c{c}\~ao, sendo $x_{p + 1}$ , $x_{p + 2}$, \dots, $x_n$ as vari\'aveis livres.
			\item Uma segunda forma a ser considerada para a matriz reduzida \'e
			\[
				\begin{amatrix}{9}
					0_\cp{K} & 1_\cp{K} & 0_\cp{K} & \cdots & 0_\cp{K} & a_{1p+2} & a_{1p+3} & \cdots & a_{1n} & z_1\\
					0_\cp{K} & 0_\cp{K} & 1_\cp{K} & \cdots & 0_\cp{K} & a_{2p+2} & a_{2p+3} & \cdots & a_{2n} & z_2\\
					\vdots & \vdots & \vdots & \vdots & \vdots & \vdots & \vdots & \vdots & \vdots & \vdots\\
					0_\cp{K} & 0_\cp{K} & 0_\cp{K} & \cdots & 1_\cp{K} & a_{pp+2} & a_{pp+3} & \cdots & a_{pn} & z_p\\
					0_\cp{K} & 0_\cp{K} & 0_\cp{K} & 0_\cp{K} & 0_\cp{K} & 0_\cp{K} & 0_\cp{K} & \cdots & 0_\cp{K} & 0_\cp{K}\\
					\vdots & \vdots & \vdots & \vdots & \vdots & \vdots & \vdots & \vdots & \vdots & \vdots\\
					0_\cp{K} & 0_\cp{K} & 0_\cp{K} & 0_\cp{K} & 0_\cp{K} & 0_\cp{K} & 0_\cp{K} & \cdots & 0_\cp{K}  & 0_\cp{K}
				\end{amatrix}.
			\]
			Neste caso teremos
			\[
				\begin{cases}
					x_2 = z_1 + (-a_{1 p + 2})x_{p + 2} + (-a_{1 p + 3})x_{p + 3} + \cdots + (-a_{1n})x_{n}\\
					x_3 = z_2 + (-a_{2 p + 3})x_{p + 3} + (-a_{2 p + 3})x_{p + 3} + \cdots + (-a_{2n})x_{n}\\
					\qquad \vdots\\
					x_{p+1} = z_p + (-a_{p p + 2})x_{p + 2} + (-a_{p p + 3})x_{p + 3} + \cdots + (-a_{pn})x_{n}\\
				\end{cases}
			\]
			e o sistema ter\'a mais de uma solu\c{c}\~ao, sendo $x_1$, $x_{p + 1}$ , $x_{p + 2}$, \dots, $x_n$ as vari\'aveis livres.
		\end{enumerate}
		Prosseguindo com esse racioc{\'\i}nio, vemos que para qualquer posto $p < n$ teremos um sistema com mais de uma solu\c{c}\~ao e $n - p$ vari\'aveis livres.
	\end{enumerate}
	Portanto a condi\c{c}\~ao (i) do teorema est\'a provada.

	Observe que os itens (ii) e (iii) foram automaticamente demonstrados nos itens (a) e (b) anteriores.

	Logo o teorema est\'a provado.
\end{prova}

\begin{exemplo}
	Encontre a solu\c{c}\~ao dos seguintes sistemas lineares:
	\begin{enumerate}[label={\arabic*})]
		\item $\begin{cases}
		x + 3y + z = 0\\
		2x + 6y + 2z = 0\\
		-x - 3y - z = 0
		\end{cases}$ em $\real$.
		\begin{solucao}
		A matriz dos coeficentes deste sistema \'e
		\[
			\begin{bmatrix}
				\phantom{-}1 & \phantom{-}3 & 1\\
				\phantom{-}2 & \phantom{-}6 & 2\\
				-1 & -3 & -1
			\end{bmatrix}.
		\]
		Aplicando as opera\c{c}\~oes elementares para reduzir $A$ \`a forma em escada:
		\begin{align*}
			A &=
				\left[
					\begin{array}{ccc}
						\phantom{-}1 & \phantom{-}3 & \phantom{-}1 \\
						\phantom{-}2 & \phantom{-}6 & \phantom{-}2 \\
						-1 & -3 & -1
					\end{array}
				\right]
				\begin{array}{l}
					\\
					L_2 \to L_2 - 2L_1\\
					L_3 \to L_3 + L_1
				\end{array} \sim
				\left[
				\begin{array}{ccc}
					1 & 3 & 1 \\
					0 & 0 & 0 \\
					0 & 0 & 0
				\end{array}
				\right]
		\end{align*}
		Assim o posto de $A$ \'e $p = 1$ e a nulidade \'e 2, ou seja, temos duas vari\'aveis livres, a saber $y$ e $z$. Logo a solu\c{c}\~ao \'e dada por
		\[
			x = -3y - z;\quad y,\ z \in \real.
		\]
		Que pode ser escrita como
		\[
			S = \{(x, y ,z) \mid x, y, z \in \real \} = \{(-3y - z, y, z) \mid y, z \in \real\}.
		\]
		\end{solucao}
		\item $\begin{cases}
		\overline{1}x + \overline{4}y + \overline{2}z = \overline{6}\\
		\overline{1}x + \overline{5}y + \overline{2}z = \overline{2}\\
		\overline{2}x + \overline{3}y + \overline{4}z = \overline{4}\\
		\overline{4}x + \overline{5}y + \overline{1}z = \overline{5}
		\end{cases}$ em $\z_7$.
		\begin{solucao}
		A matriz ampliada do sistema \'e
		\[
			A =
			\begin{amatrix}{3}
				\overline{1} & \overline{4} & \overline{2} & \overline{6}\\
				\overline{1} & \overline{5} & \overline{2} & \overline{2}\\
				\overline{2} & \overline{3} & \overline{4} & \overline{4}\\
				\overline{4} & \overline{5} & \overline{1} & \overline{5}
			\end{amatrix}.
		\]
		Aplicando as opera\c{c}\~oes elementares para reduzir $A$ a forma em escada:
		\begin{align*}
			A &=
				\left[
					\begin{array}{ccc|c}
						\overline{1} & \overline{4} & \overline{2} & \overline{6}\\
						\overline{1} & \overline{5} & \overline{2} & \overline{2}\\
						\overline{2} & \overline{3} & \overline{4} & \overline{4}\\
						\overline{4} & \overline{5} & \overline{1} & \overline{5}
					\end{array}
				\right]
				\begin{array}{l}
					\\
					L_2 \to L_2 + \overline{6}L_1\\
					L_3 \to L_3 + \overline{5}L_1\\
					L_4 \to L_4 + \overline{4}L_1
				\end{array} \sim
				\left[
					\begin{array}{ccc|c}
						\overline{1} & \overline{4} & \overline{2} & \overline{6}\\
						\overline{0} & \overline{1} & \overline{0} & \overline{3}\\
						\overline{0} & \overline{2} & \overline{0} & \overline{6}\\
						\overline{0} & \overline{3} & \overline{0} & \overline{2}
					\end{array}
				\right]
				\begin{array}{l}
					L_1 \to L_1 + \overline{3}L_2\\
					\\
					L_3 \to L_3 + \overline{5}L_2\\
					L_4 \to L_4 + \overline{4}L_2
				\end{array}\\ \\ &\sim
				\left[
					\begin{array}{ccc|c}
						\overline{1} & \overline{0} & \overline{2} & \overline{1}\\
						\overline{0} & \overline{1} & \overline{0} & \overline{3}\\
						\overline{0} & \overline{0} & \overline{0} & \overline{0}\\
						\overline{0} & \overline{0} & \overline{0} & \overline{0}
					\end{array}
				\right]
		\end{align*}
		Assim o posto de $A$ \'e $p = 2$ e a nulidade \'e 1. Logo temos uma \'unica vari\'avel livre que \'e $z$. A solu\c{c}\~ao ent\~ao \'e dada por
		\[
			x = \overline{1} + \overline{5}z,\quad y = \overline{3},\quad z \in \z_7.
		\]
		O conjunto solu\c{c}\~ao \'e
		\[
			S = \{(x, y, z) \mid x, y , z \in \z_7\} = \{(\overline{1} + \overline{5}z, \overline{3}, z) \mid z \in \z_7\}.
		\]
		Tal conjunto cont\'em exatamente 7 solu\c{c}\~oes distintas.
		\end{solucao}
		\item $
		\begin{cases}
			\overline{2}x_1 + \overline{1}x_2 + \overline{2}x_3 + \overline{2}x_4 = \overline{7}\\
			\overline{3}x_1 + \overline{1}x_2 + \overline{2}x_3 + \overline{1}x_4 = \overline{9}\\
			\overline{1}x_1 + \overline{4}x_3 + \overline{3}x_4 = \overline{6}\\
			\overline{5}x_1 + \overline{1}x_3 + \overline{1}x_4 = \overline{9}
		\end{cases}$
		em $\z_{11}$.
		\begin{solucao}
		A matriz ampliada do sistema \'e
		\[
			A =
			\begin{amatrix}{4}
				\overline{2} & \overline{1} & \overline{2} & \overline{2} & \overline{7}\\
				\overline{3} & \overline{1} & \overline{2} & \overline{1} & \overline{9}\\
				\overline{1} & \overline{0} & \overline{4} & \overline{3} & \overline{6}\\
				\overline{5} & \overline{0} & \overline{1} & \overline{1} & \overline{9}
			\end{amatrix}.
		\]
		Aplicando as opera\c{c}\~oes elementares para reduzir $A$ \`a forma em escada:
		\begin{align*}
			A &=
				\left[
					\begin{array}{cccc|c}
						\overline{2} & \overline{1} & \overline{2} & \overline{2} & \overline{7}\\
						\overline{3} & \overline{1} & \overline{2} & \overline{1} & \overline{9}\\
						\overline{1} & \overline{0} & \overline{4} & \overline{3} & \overline{6}\\
						\overline{5} & \overline{0} & \overline{1} & \overline{1} & \overline{9}
					\end{array}
				\right]
				\begin{array}{l}
					L_1 \leftrightarrow L_3
				\end{array} \sim
				\left[
					\begin{array}{cccc|c}
						\overline{1} & \overline{0} & \overline{4} & \overline{3} & \overline{6}\\
						\overline{3} & \overline{1} & \overline{2} & \overline{1} & \overline{9}\\
						\overline{2} & \overline{1} & \overline{2} & \overline{2} & \overline{7}\\
						\overline{5} & \overline{0} & \overline{1} & \overline{1} & \overline{9}
					\end{array}
				\right]
				\begin{array}{l}
					\\
					L_2 \to L_2 + \overline{8}L_1\\
					L_3 \to L_3 + \overline{9}L_1\\
					L_4 \to L_4 + \overline{6}L_1
				\end{array}\\ \\ &\sim
				\left[
					\begin{array}{cccc|c}
						\overline{1} & \overline{0} & \overline{4} & \overline{3} & \overline{6}\\
						\overline{0} & \overline{1} & \overline{1} & \overline{3} & \overline{2}\\
						\overline{0} & \overline{1} & \overline{5} & \overline{7} & \overline{6}\\
						\overline{0} & \overline{0} & \overline{3} & \overline{8} & \overline{1}
					\end{array}
				\right]
				\begin{array}{l}
					\\
					\\
					L_3 \to L_3 + \overline{10}L_2\\
					\phantom{x}
				\end{array} \sim
				\left[
					\begin{array}{cccc|c}
						\overline{1} & \overline{0} & \overline{4} & \overline{3} & \overline{6}\\
						\overline{0} & \overline{1} & \overline{1} & \overline{3} & \overline{2}\\
						\overline{0} & \overline{0} & \overline{4} & \overline{4} & \overline{4}\\
						\overline{0} & \overline{0} & \overline{3} & \overline{8} & \overline{1}
					\end{array}
				\right]
				\begin{array}{l}
					\\
					\\
					L_3 \to \overline{3}L_3\\
					\phantom{x}
				\end{array}\\ \\ &\sim
				\left[
					\begin{array}{cccc|c}
						\overline{1} & \overline{0} & \overline{4} & \overline{3} & \overline{6}\\
						\overline{0} & \overline{1} & \overline{1} & \overline{3} & \overline{2}\\
						\overline{0} & \overline{0} & \overline{1} & \overline{1} & \overline{1}\\
						\overline{0} & \overline{0} & \overline{3} & \overline{8} & \overline{1}
					\end{array}
				\right]
				\begin{array}{l}
					L_1 \to L_1 + \overline{7}L_3\\
					L_2 \to L_2 + \overline{10}L_3\\
					\\
					L_4 \to L_4 + \overline{8}L_3
				\end{array} \sim
				\left[
					\begin{array}{cccc|c}
						\overline{1} & \overline{0} & \overline{0} & \overline{10} & \overline{2}\\
						\overline{0} & \overline{1} & \overline{0} & \overline{2} & \overline{1}\\
						\overline{0} & \overline{0} & \overline{1} & \overline{1} & \overline{1}\\
						\overline{0} & \overline{0} & \overline{0} & \overline{5} & \overline{9}
					\end{array}
				\right]
				\begin{array}{l}
					\\
					\\
					\\
					L_4 \to \overline{9}L_4
				\end{array}\\ \\ &\sim
				\left[
					\begin{array}{cccc|c}
						\overline{1} & \overline{0} & \overline{0} & \overline{10} & \overline{2}\\
						\overline{0} & \overline{1} & \overline{0} & \overline{2} & \overline{1}\\
						\overline{0} & \overline{0} & \overline{1} & \overline{1} & \overline{1}\\
						\overline{0} & \overline{0} & \overline{0} & \overline{1} & \overline{4}
					\end{array}
				\right]
				\begin{array}{l}
					L_1 \to L_1 + L_4\\
					L_2 \to L_2 + \overline{9}L_4\\
					L_3 \to L_3 + \overline{10}L_4\\
					\phantom{x}
				\end{array} \sim
				\left[
					\begin{array}{cccc|c}
						\overline{1} & \overline{0} & \overline{0} & \overline{0} & \overline{6}\\
						\overline{0} & \overline{1} & \overline{0} & \overline{0} & \overline{4}\\
						\overline{0} & \overline{0} & \overline{1} & \overline{0} & \overline{8}\\
						\overline{0} & \overline{0} & \overline{0} & \overline{1} & \overline{4}
					\end{array}
				\right]
		\end{align*}
		Assim o posto de $A$ \'e $p = 4$ e a nulidade \'e 0. Logo o sistema tem uma \'unica solu\c{c}\~ao dada por
		\[
			x_1 = \overline{6}, x_2 = \overline{4}, x_3 = \overline{8}, x_4 = \overline{4}.
		\]
		\end{solucao}
		\item $
		\begin{cases}
			x_1 - x_2 + 2x_3 = 4\\
			x_1 + x_3 = 6\\
			2x_1 - 3x_2 + 5x_3 = 4
		\end{cases}
		$  em $\rac$.
		\begin{solucao}
		A matriz dos coeficentes deste sistema \'e
		\[
			\begin{amatrix}{3}
				1 & -1 & 2 & 4 \\
				1 & \phantom{-}0 & 1 & 6 \\
				2 & -3 & 5 & 4
			\end{amatrix}.
		\]
		Aplicando as opera\c{c}\~oes elementares para reduzir $A$ \`a forma em escada:
		\begin{align*}
			A &=
				\left[
					\begin{array}{ccc|c}
						1 & -1 & 2 & 4 \\
						1 & \phantom{-}0 & 1 & 6 \\
						2 & -3 & 5 & 4
					\end{array}
				\right]
				\begin{array}{l}
					\\
					L_2 \to L_2 - L_1\\
					L_3 \to L_3 - 3L_1\\
				\end{array} \sim
				\left[
					\begin{array}{ccc|c}
						1 & -1 & \phantom{-}2 & \phantom{-}4 \\
						0 & \phantom{-}1 & -1 & \phantom{-}2 \\
						0 & -1 & \phantom{-}1 & -4
					\end{array}
				\right]
				\begin{array}{l}
					L_1 \to L_1 + L_2\\
					\\
					L_3 \to L_3 + L_2
				\end{array}\\ \\ &\sim
				\left[
					\begin{array}{ccc|c}
						1 & 0 & \phantom{-}1 & \phantom{-}6 \\
						0 & 1 & -1 & \phantom{-}2 \\
						0 & 0 & \phantom{-}0 & -2
					\end{array}
				\right]
		\end{align*}
		Assim o sistema n\~ao tem solu\c{c}\~ao. Note que o posto da matriz ampliada \'e $p = 3$ e a posto da matriz dos coeficientes \'e 2.
		\end{solucao}
	\end{enumerate}
\end{exemplo}
