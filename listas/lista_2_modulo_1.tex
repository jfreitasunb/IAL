%!TEX program = xelatex
\def\numeromodulo{1}
\def\numerolista{1}
\documentclass[12pt]{exam}

\def\ano{2023}
\def\semestre{1}
\def\disciplina{Introdução à \'Algebra Linear}
\def\turma{3}

\usepackage{caption}
\usepackage{amssymb}
\usepackage{amsmath,amsfonts,amsthm,amstext}
\usepackage[brazil]{babel}
\usepackage{graphicx}
\graphicspath{{../Pictures/}}
\usepackage{enumitem}
\usepackage{multicol}
\usepackage{answers}
\usepackage[svgnames]{xcolor}
\usepackage{tikz}
\usepackage{ifthen}
\usetikzlibrary{lindenmayersystems}
\usetikzlibrary[shadings]

\Newassociation{solucao}{Solution}{ans}
\newtheorem{exercicio}{}

\setlength{\topmargin}{-1.0in}
\setlength{\oddsidemargin}{0in}
\setlength{\textheight}{10.1in}
\setlength{\textwidth}{6.5in}
\setlength{\baselineskip}{12mm}

\extraheadheight{0.7in}
\firstpageheadrule
\runningheadrule
\lhead{
        \begin{minipage}[c]{1.7cm}
        \includegraphics[width=1.7cm]{unb.pdf}
        \end{minipage}%
        \hspace{0pt}
        \begin{minipage}[c]{4in}
          {Universidade de Bras{\'\i}lia} --
          {Departamento de Matem{\'a}tica}
\end{minipage}
\vspace*{-0.8cm}
}
% \chead{Universidade de Bras{\'\i}lia - Departamento de Matem{\'a}tica}
% \rhead{}
% \vspace*{-2cm}

\extrafootheight{.5in}
\footrule
\lfoot{\disciplina\ - \semestre$^o$/\ano\ - Módulo \numeromodulo}
\cfoot{}
\rfoot{P\'agina \thepage\ de \numpages}

\newcounter{exercicios}
\renewcommand{\theexercicios}{\arabic{exercicios}}

\newenvironment{questao}[1]{
\refstepcounter{exercicios}
\ifx&#1&
\else
   \label{#1}
\fi
\noindent\textbf{Exerc{\'\i}cio {\theexercicios}:}
}

\newcommand{\resp}[1]{
\noindent{\bf Exerc{\'\i}cio #1: }}

\def\ano{2023}
\def\semestre{2}
\def\disciplina{Introdução à Álgebra Linear}
\def\nomeabreviado{IAL}
\def\turma{11}

\newcommand{\im}{{\rm Im\,}}
\newcommand{\dlim}[2]{\displaystyle\lim_{#1\rightarrow #2}}
\newcommand{\minf}{+\infty}
\newcommand{\ninf}{-\infty}
\newcommand{\cp}[1]{\mathbb{#1}}
\newcommand{\sub}{\subseteq}
\newcommand{\n}{\mathbb{N}}
\newcommand{\z}{\mathbb{Z}}
\newcommand{\rac}{\mathbb{Q}}
\newcommand{\real}{\mathbb{R}}
\newcommand{\complex}{\mathbb{C}}

\newcommand{\vesp}[1]{\vspace{ #1  cm}}

\newcommand{\compcent}[1]{\vcenter{\hbox{$#1\circ$}}}
\newcommand{\comp}{\mathbin{\mathchoice
        {\compcent\scriptstyle}{\compcent\scriptstyle}
        {\compcent\scriptscriptstyle}{\compcent\scriptscriptstyle}}}
\renewcommand{\sin}{{\rm sen\,}}
\renewcommand{\tan}{{\rm tg\,}}
\renewcommand{\csc}{{\rm cossec\,}}
\renewcommand{\cot}{{\rm cotg\,}}
\renewcommand{\sinh}{{\rm senh\,}}
\newcommand{\integer}{\mathbb{Z}}
\begin{document}

  \Opensolutionfile{ans}[ans1]
  \begin{center}
    {\Large\bf \disciplina\ - Turma \turma\ -- \semestre$^{o}$/\ano} \\ \vspace{9pt} {\large\bf
        $\numerolista^a$ Lista de Exerc{\'\i}cios -- Módulo \numeromodulo}\\ \vspace{9pt} Prof. Jos{\'e} Ant{\^o}nio O. Freitas
  \end{center}
  \hrule

\vesp{.6}
\begin{exercicio}
  Descrever explicitamente todas as poss{\'\i}veis formas que uma matriz $2 \times 2$ linha-reduzidas \`a forma em escada pode assumir.
\end{exercicio}

\begin{exercicio}
  Descrever explicitamente todas as poss{\'\i}veis formas que uma matriz $3 \times 3$ linha-reduzidas \`a forma em escada pode assumir.
\end{exercicio}

\begin{exercicio}
  Descrever explicitamente todas as poss{\'\i}veis formas que uma matriz $4 \times 4$ linha-reduzidas \`a forma em escada pode assumir.
\end{exercicio}

\begin{exercicio}
  Seja
  \[
    A = \begin{bmatrix}
      x & y\\
      z & t
    \end{bmatrix}
  \]
  uma matriz $2 \times 2$ com elementos complexos. Suponhamos que $A$ seja linha-reduzida e tamb\'em que $x + y + z + t = 0$. Mostre que existem exatamente tr\^es destas matrizes.
\end{exercicio}

\begin{exercicio}
  Seja $\cp{K}$ um corpo. Dados $a$, $b$, $c$, $d$, $e$, $f \in \cp{K}$, mostre que as duas matrizes seguintes s\~ao linha-equivalentes, se supormos que $ad - bc \ne 0_{\cp{K}}$:
  \[
    \begin{amatrix}{2}
      a & b & e\\
      c & d & f
    \end{amatrix} \sim \begin{amatrix}{2}
      1_{\cp{K}} & 0_{\cp{K}} & (de - bf)(ad - bc)^{-1}\\
      0_{\cp{K}} & 1_{\cp{K}} & (af - ce)(ad - bc)^{-1}
    \end{amatrix}
  \]
  onde $(ad - bc)^{-1}$ \'e o inverso multiplicativo de $ad - bc$ no corpo $\cp{K}$.
\end{exercicio}

\begin{exercicio}
  Qual o erro na seguinte sequ\^encia de opera\c{c}\~oes elementares?
    \begin{align*}
      \begin{amatrix}{3} 
        1 & 0 & 2 & 3 \\ 
        0 & 1 & 2& 3 \\
        2 & 0 & 1 & 4 \\
      \end{amatrix} 
      \stackrel{L_1 \to L_1+L_2}{
      \stackrel{L_2 \to L_1-L_2}{ 
      \stackrel{L_3 \to  L_1+2L_2}{\sim}}}
      \begin{amatrix}{3} 
      1 & 1 & 4 & 6 \\
      1 & \!\!-1 & 0& 0 \\
      1 & 2 & 6 & 9 
      \end{amatrix}
    \end{align*}
\end{exercicio}

\begin{exercicio}
  Demonstrar que as duas matrizes seguintes \textbf{n\~ao} s\~ao linha-equivalentes:
  \[
    \begin{bmatrix}
      2 & 0 & 0\\
      a & -1 & 0\\
      b & c & 3
    \end{bmatrix} \qquad
    \begin{bmatrix}
      1 & 1 & 2\\
      -2 & 0 & -1\\
      1 & 3 & 5
    \end{bmatrix}.
  \]
\end{exercicio}

\begin{exercicio}
  Seja $A$ uma matriz $m \times n$ com entradas no corpo $\cp{K}$. Mostre que a matriz linha-reduzida \`a forma em escada que \'e linha-equivalente \`a $A$ \'e \'unica.
\end{exercicio}

\begin{exercicio}
  Sejam $A$, $R \in M_{m \times n}(\cp{K})$ e $B$, $C \in M_{m \times 1}(\cp{K})$. Mostre que se $[A | B]$ e $[R | C]$ s\~ao matrizes linha-equivalentes, ent\~ao os sistemas lineares
  \[
    AX = B \quad {e} \quad RX = C,
  \]
  s\~ao equivalentes.
\end{exercicio}

\begin{exercicio}
  Seja
  \[
    A = \begin{bmatrix}
      3 & -1 & 2\\
      2 & 1 & 1\\
      1 & -3 & 0
    \end{bmatrix}.
  \]
  Para que valores
  \[
    Y = \begin{bmatrix}
      y_1\\
      y_2\\
      y_3
    \end{bmatrix}
  \]
  o sistema $AX = Y$ admite solu\c{c}\~ao? Considere $y_i \in \real$, $i=1$, 2, 3.
\end{exercicio}

\begin{exercicio}
  Determine o n\'umero de solu\c{c}\~oes do sistema
  \[
    \begin{cases}
      x + 2y - 3z = 4\\
      4x + y + 2z = 6\\
      x + 2y + (\alpha^2 - 19)z = \alpha,  
    \end{cases}
\]
em fun\c{c}\~ao do par\^ametro:
\begin{enumerate}[label={\alph*})]
   \item $\alpha \in \real$;
   \item $\alpha \in \rac$?
 \end{enumerate} 
\end{exercicio}

\begin{exercicio}
  Considere o sistema linear
  \[
    \begin{cases}
      (k + 2)x + 2ky - z = 1\\
      x - 2y + kz = -k\\
      y + z = k
    \end{cases}.
  \]
  \begin{enumerate}[label={\alph*})]
    \item Para quais valores de $k \in \real$ o sistema admite solu\c{c}\~ao \'unica? Determine a solu\c{c}\~ao, caso exista.
    \item Para quais valores de $k \in \real$ o sistema admite infinitas solu\c{c}\~oes? Determine tais solu\c{c}\~oes, caso existam.
  \end{enumerate}
\end{exercicio}

Nos exerc{\'\i}cios \ref{sistemalinearinicio} \`a \ref{sistemalinearfim}, encontre a solu\c{c}\~ao geral dos seguintes sistemas lineares. Encontre o posto e a nulidade do sistema.
\begin{exercicio}\label{sistemalinearinicio}
  $
    \begin{cases}
      x + y + z = 4\\
      2x + 5y - 2z = 3\\
      x + 7y - 7z = 5
    \end{cases}
  $
  em $\rac$.
  \begin{solucao}
    N\~ao existe solu\c{c}\~ao.
  \end{solucao}
\end{exercicio}

\begin{exercicio}
  $
    \begin{cases}
      x - 2y + 3z = 0\\
      2x + 5y + 6z = 0
    \end{cases}
  $
  em $\real$.
  \begin{solucao}
    $p = 2$, Nulidade = 1, $S = \{(x, y, z) \mid z, y, z \in \real\} = \{(-3z, 0, z) \mid z \in \real\}$
  \end{solucao}
\end{exercicio}

\begin{exercicio}
  $
    \begin{cases}
      x_1 + x_2 + 2x_3 = 8\\
      -x_1 - 2x_2 + 3x_3 = 1\\
      3x_1 - 7x_2 + 4x_3 = 10
    \end{cases}
  $
  em $\real$.
  \begin{solucao}
    $p = 3$, Nulidade = 0, $S = \{(3, 1, 2)\}$
  \end{solucao}
\end{exercicio}

\begin{exercicio}
  $
    \begin{cases}
      2x_1 + 2x_2 + 2x_3 = 0\\
      -2x_1 + 5x_2 + 2x_3 = 1\\
      8x_1 + x_2 + 4x_3 = -1
    \end{cases}
  $
  em $\real$.
  \begin{solucao}
    $p = 2$, Nulidade = 1, $S = \{(x_1, x_2, x_3) \mid x_1, x_2, x_3 \in \real\} = \{(-1/7 - (3/7)x_3, 1/7 - (4/7)x_3, x_3) \mid x_3 \in \real\}$
  \end{solucao}
\end{exercicio}

\begin{exercicio}
  $
    \begin{cases}
      \phantom{2x_1} - 2x_2 + 3x_3 = 1\\
      3x_1 + 6x_2 - 3x_3 = -2\\
      6x_1 + 6x_2 + 3x_3 = 5
    \end{cases}
  $
  em $\real$.
  \begin{solucao}
    O sistema n\~ao tem solu\c{c}\~ao.
  \end{solucao}
\end{exercicio}

\begin{exercicio}
  $
    \begin{cases}
      \overline{1}x + \overline{2}y + \overline{3}z = \overline{0}\\
      \overline{2}x + \overline{1}y + \overline{3}z = \overline{0}\\
      \overline{3}x + \overline{2}y + \overline{1}z = \overline{0}
    \end{cases}
  $
  em $\integer_5$.
  \begin{solucao}
    $p = 3$, Nulidade = 0, $S = \{(\overline{0},\overline{0},\overline{0})\}$
  \end{solucao}
\end{exercicio}

\begin{exercicio}
  $
    \begin{cases}
      (2 + 3\sqrt{2})x_1 - 3x_2 + x_3 = 0\\
      (1 - \sqrt{2})x_1 + \sqrt{2}x_3 = 0
    \end{cases}
  $
  em $\rac[\sqrt{2}]$.
  \begin{solucao}
    $p = 2$, Nulidade = 1, $S = \{(x, y, z) \mid x, y, z \in \rac[\sqrt{2}]\} = $ \\ $\left\{\left((2 + \sqrt{2})z, \dfrac{11 + 8\sqrt{2}}{3}z, z\right) \mid z \in \rac[\sqrt{2}]\right\}$
  \end{solucao}
\end{exercicio}

\begin{exercicio}
  $
    \begin{cases}
      ix + iy = 0\\
      2ix - y = 0\\
    \end{cases}
  $
  em $\complex$.
  \begin{solucao}
    $p = 2$, Nulidade = 0, $S = \{(0, 0)\}$
  \end{solucao}
\end{exercicio}

\begin{exercicio}
  $
    \begin{cases}
      x_1 + x_2 + x_3 + x_4 = 0\\
      x_1 + x_2 + x_3 - x_4 = 4\\
      x_1 + x_2 - x_3 + x_4 = -4\\
      x_1 - x_2 + x_3 + x_4 = 2\\
    \end{cases}
  $
  em $\real$.
  \begin{solucao}
    $p = 4$, Nulidade = 0, $S = \{(1, -1, 2, -2)\}$
  \end{solucao}
\end{exercicio}

\begin{exercicio}
  $
    \begin{cases}
      \overline{1}x + \overline{2}y + \overline{1}z + \overline{3}w = \overline{0}\\
      \overline{1}x + \overline{10}y + \overline{1}w = \overline{0}\\
      \overline{1}y + \overline{10}z + \overline{1}w = \overline{0}
    \end{cases}
  $
  em $\integer_{13}$.
  \begin{solucao}
    $p = 3$, Nulidade = 1, $S = \{(x, y, z, w) \mid x, y, z, w \in \integer_{13}\} = \{(\overline{5}w, \overline{2}w,w, w) \mid w \in \integer_{13}\}$
  \end{solucao}
\end{exercicio}

\begin{exercicio}
  $
    \begin{cases}
      -2x_2 + 3x_3 = 1\\
      3x_1 + 6x_2 - 3x_3 = -1\\
      6x_1 + 6x_2 + 3x_3 = 5
    \end{cases}
  $
  em $\real$.
  \begin{solucao}
    N\~ao existe solu\c{c}\~ao.
  \end{solucao}
\end{exercicio}

\begin{exercicio}
  $
    \begin{cases}
      2x_1 + x_2 + (6 + 6i)x_3 + 8x_4 = 0\\
      x_1 + x_2 + (2 + 5i)x_3 + (5  i)x_4 = 0\\
      (2 + 2i)x_1 + 2x_2 + (2 + 14i)x_3 + (14 + 8i)x_4 =0\\
      (-1 - i)x_1 - x_2 - (3 + 3i)x_3 + (-6 + 2i)x_4 = 0
    \end{cases}
  $
  em $\complex$.
  \begin{solucao}
    $p = 2$, Nulidade = 2, $S = \{(x_1, x_2, x_3, x_4) \mid x_1, x_2, x_3, x_4 \in \complex\} = $\\ $\{(-(2 + i)x_3 - (3 - i)x_4,-3ix_3 - 2x_4, x_3, x_4) \mid x_3, x_4 \in \complex\}$
  \end{solucao}
\end{exercicio}

\begin{exercicio}
  $
    \begin{cases}
      x_1 + 2x_2 - 3x_4 + x_5 = 2\\
      x_1 + 2x_2 + x_3 - 3x_4 + x_5 + 2x_6 = 3\\
      x_1 + 2x_2 - 3x_4 + 2x_5 + x_6 = 4\\
      3x_1 + 6x_2 + x_3 - 9x_4 + 4x_5 + 3x_6 = 9
    \end{cases}
  $
  em $\real$.
  \begin{solucao}
    $p = 3$, Nulidade = 3, $S = \{(x_1, x_2, x_3, x_4, x_5, x_6) \mid x_1, x_2, x_3, x_4, x_5, x_6 \in \real\} = $\\ $\{(x_6 + 3x_4 - 2x_2, x_2, 1 - 2x_6, x_4, 2 - x_6, x_6) \mid x_2, x_4, x_6 \in \real\}$
  \end{solucao}
\end{exercicio}

\begin{exercicio}
  $
    \begin{cases}
      x_1 + 3x_2 - 2x_3 + 2x_5 = 0\\
      2x_1 + 6x_2 - 5x_3 - 2x_4 + 4x_5 - 3x_6 = -1\\
      5x_3 + 10x_4 + 15x_6 = 5\\
      2x_1 + 6x_2 + 8x_4 + 4x_5 + 18x_6 = 6
    \end{cases}
  $
  em $\real$.
  \begin{solucao}
    $p = 3$, Nulidade = 3, $S = \{(x_1, x_2, x_3, x_4, x_5, x_6) \mid x_1, x_2, x_3, x_4, x_5, x_6 \in \real\} = $\\ $\{(-3x_2 - 4x_4 - 2x_5, x_2, -2x_4, x_4, x_5, 1/3) \mid x_2, x_4, x_5 \in \real\}$
  \end{solucao}
\end{exercicio}

\begin{exercicio}
  $
    \begin{cases}
      x + 2z = 1\\
      (2 - \sqrt{2})x + 2y + 6z = 2 + \sqrt{2}\\
      \sqrt{2}x - y + (-1 + \sqrt{2})z = 1 + \sqrt{2}
    \end{cases}
  $
  em $\rac[\sqrt{2}]$.
  \begin{solucao}
    N\~ao existe solu\c{c}\~ao.
  \end{solucao}
\end{exercicio}


\begin{exercicio}\label{sistemalinearfim}
  $
    \begin{cases}
      3x + 2y - 4z = 1\\
      x - y + z = 3\\
      x - y - 3z = -3\\
      3x + 3y - 5z =0\\
      -x + y + z = 1
    \end{cases}
  $
  em $\rac$.
\begin{solucao}
  N\~ao existe solu\c{c}\~ao.
\end{solucao}
\end{exercicio}

\begin{exercicio}
  Seja
  \[
    A =\begin{bmatrix}
      1 & 0 & 5\\
      1 & 1 & 1\\
      0 & 1 & -4
    \end{bmatrix}.
  \]
  \begin{enumerate}[label={\alph*})]
    \item Encontre a solu\c{c}\~ao geral do sistema $(A + 4I_3)X = 0$ em $\real$.
    \item Encontre a solu\c{c}\~ao geral do sistema $(A - 2I_3)X = 0$ em $\real$.
  \end{enumerate}
  \begin{solucao}
    \begin{enumerate}[label={\alph*})]
      \item $S = \{(-\alpha, 0, \alpha) \mid \alpha \in \real\}$.
      \item $S = \{(5\alpha, 6\alpha, \alpha) \mid \alpha \in \real\}$.
    \end{enumerate}
  \end{solucao}
\end{exercicio}

Nos exerc{\'\i}cios \ref{sistemasinicio} \`a \ref{sistemasfim}, encontre o(s) valor(es) de $\lambda$ tal(is) que:
\begin{enumerate}[label={\alph*})]
  \item o sistema tem solu\c{c}\~ao \'unica;
  \item o sistema tem v\'arias solu\c{c}\~oes;
  \item o sistema n\~ao tem solu\c{c}\~ao.
\end{enumerate}

\begin{exercicio}\label{sistemasinicio}
  $
    \begin{cases}
      x + \lambda y = 1\\
      \lambda x + y = 1
    \end{cases}
  $
  em $\real$.
  \begin{solucao}
    \begin{enumerate}[label={\alph*})]
      \item $\lambda ne -1$ e $\lambda \ne 1$.
      \item $\lambda = 1$.
      \item $\lambda = -1$.
    \end{enumerate}
  \end{solucao}
\end{exercicio}

\begin{exercicio}
  $
    \begin{cases}
      (\lambda - 2)x + y = 0\\
      x + (\lambda - 2)y = 0
    \end{cases}
  $
  em $\real$.
  \begin{solucao}
    \begin{enumerate}[label={\alph*})]
      \item $\lambda \ne 1$ e $\lambda \ne 3$
      \item $\lambda = 1$ e $\lambda = 3$.
      \item N\~ao existe tal $\lambda$.
    \end{enumerate}
  \end{solucao}
\end{exercicio}

\begin{exercicio}
  $
    \begin{cases}
      x + 2y - 3z = 4\\
      3x - y + 5z = 2\\
      4x + y + (\lambda^2 - 14)z = \lambda + 2
    \end{cases}
  $
  em $\real$.
  \begin{solucao}
    \begin{enumerate}[label={\alph*})]
      \item $\lambda \ne \pm 4$.
      \item $\lambda = 4$.
      \item $\lambda = -4$.
    \end{enumerate}
  \end{solucao}
\end{exercicio}

\begin{exercicio}
  $
    \begin{cases}
      x + y + z = 2\\
      2x + 3y + 2z = 5\\
      2x + 3y + (\lambda^2 - 1)z = \lambda + 1
    \end{cases}
  $
  em $\rac$.
  \begin{solucao}
    \begin{enumerate}[label={\alph*})]
      \item N\~ao existe tal $\lambda$.
      \item N\~ao existe tal $\lambda$.
      \item N\~ao existe tal $\lambda$.
    \end{enumerate}
  \end{solucao}
\end{exercicio}

\begin{exercicio}
  $
    \begin{cases}
      x_1 + \lambda x_2 + (1 + 4\lambda )x_3 = 1 + 4\lambda \\
      2x_1 + (\lambda  + 1)x_2 + (2 + 7\lambda )x_3 = 1 + 7\lambda \\
      3x_1 + (\lambda  + 2)x_2 + (3 + 9\lambda )x_3 = 1 + 9\lambda 
    \end{cases}  
  $
  \begin{solucao}
    \begin{enumerate}[label={\alph*})]
      \item $\lambda \ne 0$ e $\lambda \ne 1$.
      \item $\lambda = 0$.
      \item $\lambda = 1$.
    \end{enumerate}
  \end{solucao}
\end{exercicio}

\begin{exercicio}
  $
    \begin{cases}
      \overline{1}x_1 + \overline{1}x_2 + \overline{1}x_3= \overline{2}\\
      \overline{1}x_1 + \overline{2}x_2 + \overline{1}x_3 = \overline{4}\\
      \overline{2}x_1 + \overline{2}x_2 + \overline{4}x_3 = \overline{3}\\
      \overline{3}x_1 + \overline{3}x_2 + \overline{3}x_3 + (\lambda^2 + \overline{2})x_4 = \lambda + \overline{4}
    \end{cases}
  $
  em $\integer_5$.
  \begin{solucao}
    \begin{enumerate}[label={\alph*})]
      \item O sistema sempre tem solu\c{c}\~ao \'unica, independente de $\lambda$.
      \item N\~ao existe tal $\lambda$.
      \item N\~ao existe tal $\lambda$.
    \end{enumerate}
  \end{solucao}
\end{exercicio}

\begin{exercicio}
  $
    \begin{cases}
      \overline{1}x_1 + \overline{1}x_2 + \overline{1}x_3 = \overline{3}\\
      \overline{1}x_1 + \overline{2}x_2 + \overline{1}x_3 = \overline{5}\\
      \overline{2}x_1 + \overline{2}x_2 + \overline{4}x_3 = \overline{1}\\
      \overline{3}x_1 + \overline{3}x_2 + \overline{3}x_3 + (\lambda^2 + \overline{2})x_4 = \lambda + \overline{5}
    \end{cases}
  $
  em $\integer_7$.
  \begin{solucao}
    \begin{enumerate}[label={\alph*})]
      \item O sistema sempre tem solu\c{c}\~ao \'unica, independente de $\lambda$.
      \item N\~ao existe tal $\lambda$.
      \item N\~ao existe tal $\lambda$.
    \end{enumerate}
  \end{solucao}
\end{exercicio}

\begin{exercicio}\label{sistemasfim}
  $
    \begin{cases}
      x + y = 2\\
      y + z = 2\\
      x + z = 2\\
      x + y + \lambda z = 0
    \end{cases}
  $
  em $\real$.
  \begin{solucao}
    \begin{enumerate}[label={\alph*})]
      \item $\lambda = -2$
      \item N\~ao existe tal $\lambda$.
      \item $\lambda \ne -2$.
    \end{enumerate}
  \end{solucao}
\end{exercicio}

\begin{exercicio}
  Encontre condi\c{c}\~oes sobre os $b_i \in \real$ para que cada um dos sistemas tenha solu\c{c}\~ao.
  \begin{enumerate}[label={\alph*})]
    \item $\begin{cases}
      x_1 - 2x_2 + 5x_3 = b_1\\
      4x_1 - 5x_2 + 8x_3 = b_2\\
      -3x_1 + 3x_2 - 3x_3 = b_3
    \end{cases}$

    \item $\begin{cases}
      x_1 - 2x_2 - x_3 = b_1\\
      -4x_1 + 5x_2 + 2x_3 = b_2\\
      -4x_1 + 7x_2 + 4x_3 = b_3
    \end{cases}$
  \end{enumerate}
  \begin{solucao}
    \begin{enumerate}[label={\alph*})]
      \item $b_3 - b_1 + b_2 = 0$
      \item O sistema tem solu\c{c}\~ao para todos os valores reais de $b_1$, $b_2$ e $b_3$.
    \end{enumerate}
  \end{solucao}
\end{exercicio}

Nos exerc{\'\i}cios \ref{matrizinicio} \`a \ref{matrizfim}, mostre que $B$ \'e a inversa de $A$.
\begin{exercicio}\label{matrizinicio}
  \[
    A =\begin{bmatrix}
      1 & -1\\
      2 & 3
    \end{bmatrix}, B =\begin{bmatrix}
      3/5 & 1/5\\
      -2/5 & 1/5
    \end{bmatrix}
  \]
\end{exercicio}

\begin{exercicio}
  \[
    A =\begin{bmatrix}
      -2 & 2 & 3\\
      1 & -1 & 0\\
      0 & 1 & 4
    \end{bmatrix}, B =\begin{bmatrix}
      -4/3 & -5/3 & 1\\
      -4/3 & -8/3 & 1\\
      1/3 & 2/3 & 0
    \end{bmatrix}
  \]
\end{exercicio}

\begin{exercicio}\label{matrizfim}
  \[
    A =\begin{bmatrix}
      2 & -17 & 11\\
      -1 & 11 & -7\\
      0 & 3 & -2
    \end{bmatrix}, B =\begin{bmatrix}
      1 & 1 & 2\\
      2 & 4 & -3\\
      3 & 6 & -5
    \end{bmatrix}
  \]
\end{exercicio}

Nos exerc{\'\i}cios \ref{matrizinversainicio} \`a \ref{matrizinversafim}, encontre a inversa da matriz dada, se existir.

\begin{exercicio}\label{matrizinversainicio}
  $
    A =\begin{bmatrix}
        1 & 2\\
        3 & 7
    \end{bmatrix}
  $
  \begin{solucao}
    $A^{-1} =\begin{bmatrix}
      7 & -2\\
      -3 & 1
    \end{bmatrix}$
  \end{solucao}
\end{exercicio}

\begin{exercicio}
  $
    A =\begin{bmatrix}
        -7 & 33\\
        4 & 19
    \end{bmatrix}
  $
  \begin{solucao}
    $A^{-1} =\begin{bmatrix}
      -19 & -33\\
      -4 & -7
    \end{bmatrix}$
  \end{solucao}
\end{exercicio}

\begin{exercicio}
  $
    A = \begin{pmatrix}
        \overline{4} & \overline{0} & \overline{1}\\
        \overline{1} & \overline{1} & \overline{0}\\
        \overline{3} & \overline{0} & \overline{1}
    \end{pmatrix}
  $
  em $\integer_5$.
  \begin{solucao}
    $
      A^{-1} = \begin{pmatrix}
        \overline{1} & \overline{0} & \overline{4}\\
        \overline{4} & \overline{1} & \overline{1}\\
        \overline{2} & \overline{0} & \overline{4}
      \end{pmatrix}
    $
  \end{solucao}
\end{exercicio}

\begin{exercicio}
  $
    A = \begin{bmatrix}
        \overline{1} & \overline{4} & \overline{3} & \overline{3}\\
        \overline{1} & \overline{6} & \overline{0} & \overline{0}\\
        \overline{1} & \overline{6} & \overline{4} & \overline{4}\\
        \overline{1} & \overline{6} & \overline{5} & \overline{5}
      \end{bmatrix}
  $
  em $\integer_7$.
  \begin{solucao}
    N\~ao existe inversa.
  \end{solucao}
\end{exercicio}

\begin{exercicio}
  $
    A =\begin{bmatrix}
        2 & 4\\
        4 & 8
    \end{bmatrix}
  $
  \begin{solucao}
    N\~ao possui inversa.
  \end{solucao}
\end{exercicio}

\begin{exercicio}
  $
    A =\begin{bmatrix}
        1 & 1 & 1\\
        3 & 5 & 4\\
        3 & 6 & 5
      \end{bmatrix}
    $
  \begin{solucao}
    $
      A^{-1} =\begin{bmatrix}
        1 & 1 & -1\\
        -3 & 2 & -1\\
        3 & -3 & 2
      \end{bmatrix}
    $
  \end{solucao}
\end{exercicio}

\begin{exercicio}
  $
    A =\begin{bmatrix}
        1 & 2 & -1\\
        3 & 7 & -10\\
        7 & 16 & -21
    \end{bmatrix}
  $
  \begin{solucao}
    N\~ao tem inversa.
  \end{solucao}
\end{exercicio}

\begin{exercicio}
  $
    A =\begin{bmatrix}
        -8 & 0 & 0 & 0\\
        0 & 1 & 0 & 0\\
        0 & 0 & 0 & 0\\
        0 & 0 & 0 & -5
    \end{bmatrix}
  $
  \begin{solucao}
   N\~ao existe inversa.
  \end{solucao}
\end{exercicio}

\begin{exercicio}\label{matrizinversafim}
  $
    A =\begin{bmatrix}
        1 & 1 & 2\\
        3 & 1 & 0\\
        -2 & 0 & 3
    \end{bmatrix}
  $
  \begin{solucao}
    $
      A^{-1} =\begin{bmatrix}
        -3/2 & 3/2 & 1\\
        9/2 & -7/2 & -3\\
        -1 & 1 & 1
      \end{bmatrix}
    $
  \end{solucao}
\end{exercicio}

\begin{center}
  Exerc{\'\i}cio Extra!
\end{center}
\begin{exercicio}
  Uma rela\c{c}\~ao $\sim$ em um conjunto de objetos $U$ \'e uma \textbf{rela\c{c}\~ao de equival\^encia} se as seguintes propriedades s\~ao satisfeitas:
  \begin{enumerate}
    \item \textit{Reflexiva}: Para todo $x \in U$ temos $x \sim x$.
    \item \textit{Simetria}: Para $x$, $y \in U$ se $x \sim y$, ent\~ao $y \sim x$.
    \item \textit{Transitiva}: Para $x$, $y$ e $z \in U$, se $x \sim y$ e $y \sim z$, ent\~ao $x \sim z$.
  \end{enumerate}
  Mostre que linha-equival\^encia de matrizes \'e uma rela\c{c}\~ao de equival\^encia.
\end{exercicio}

\newpage
\Closesolutionfile{ans}
\hrule
\begin{center}
{\large\bf RESPOSTAS}
\end{center}
\hrule
\input{ans1}

\end{document}