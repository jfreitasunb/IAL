%!TEX program = xelatex
\def\numeromodulo{1}
\def\numerolista{1}
\documentclass[12pt]{exam}

\def\ano{2023}
\def\semestre{1}
\def\disciplina{Introdução à \'Algebra Linear}
\def\turma{3}

\usepackage{caption}
\usepackage{amssymb}
\usepackage{amsmath,amsfonts,amsthm,amstext}
\usepackage[brazil]{babel}
\usepackage{graphicx}
\graphicspath{{../Pictures/}}
\usepackage{enumitem}
\usepackage{multicol}
\usepackage{answers}
\usepackage[svgnames]{xcolor}
\usepackage{tikz}
\usepackage{ifthen}
\usetikzlibrary{lindenmayersystems}
\usetikzlibrary[shadings]

\Newassociation{solucao}{Solution}{ans}
\newtheorem{exercicio}{}

\setlength{\topmargin}{-1.0in}
\setlength{\oddsidemargin}{0in}
\setlength{\textheight}{10.1in}
\setlength{\textwidth}{6.5in}
\setlength{\baselineskip}{12mm}

\extraheadheight{0.7in}
\firstpageheadrule
\runningheadrule
\lhead{
        \begin{minipage}[c]{1.7cm}
        \includegraphics[width=1.7cm]{unb.pdf}
        \end{minipage}%
        \hspace{0pt}
        \begin{minipage}[c]{4in}
          {Universidade de Bras{\'\i}lia} --
          {Departamento de Matem{\'a}tica}
\end{minipage}
\vspace*{-0.8cm}
}
% \chead{Universidade de Bras{\'\i}lia - Departamento de Matem{\'a}tica}
% \rhead{}
% \vspace*{-2cm}

\extrafootheight{.5in}
\footrule
\lfoot{\disciplina\ - \semestre$^o$/\ano\ - Módulo \numeromodulo}
\cfoot{}
\rfoot{P\'agina \thepage\ de \numpages}

\newcounter{exercicios}
\renewcommand{\theexercicios}{\arabic{exercicios}}

\newenvironment{questao}[1]{
\refstepcounter{exercicios}
\ifx&#1&
\else
   \label{#1}
\fi
\noindent\textbf{Exerc{\'\i}cio {\theexercicios}:}
}

\newcommand{\resp}[1]{
\noindent{\bf Exerc{\'\i}cio #1: }}

\def\ano{2023}
\def\semestre{2}
\def\disciplina{Introdução à Álgebra Linear}
\def\nomeabreviado{IAL}
\def\turma{11}

\newcommand{\im}{{\rm Im\,}}
\newcommand{\dlim}[2]{\displaystyle\lim_{#1\rightarrow #2}}
\newcommand{\minf}{+\infty}
\newcommand{\ninf}{-\infty}
\newcommand{\cp}[1]{\mathbb{#1}}
\newcommand{\sub}{\subseteq}
\newcommand{\n}{\mathbb{N}}
\newcommand{\z}{\mathbb{Z}}
\newcommand{\rac}{\mathbb{Q}}
\newcommand{\real}{\mathbb{R}}
\newcommand{\complex}{\mathbb{C}}

\newcommand{\vesp}[1]{\vspace{ #1  cm}}

\newcommand{\compcent}[1]{\vcenter{\hbox{$#1\circ$}}}
\newcommand{\comp}{\mathbin{\mathchoice
        {\compcent\scriptstyle}{\compcent\scriptstyle}
        {\compcent\scriptscriptstyle}{\compcent\scriptscriptstyle}}}
\renewcommand{\sin}{{\rm sen\,}}
\renewcommand{\tan}{{\rm tg\,}}
\renewcommand{\csc}{{\rm cossec\,}}
\renewcommand{\cot}{{\rm cotg\,}}
\renewcommand{\sinh}{{\rm senh\,}}
\newcommand{\integer}{\mathbb{Z}}
\begin{document}

  \Opensolutionfile{ans}[ans1]
  \begin{center}
    {\Large\bf \disciplina\ - Turma \turma\ -- \semestre$^{o}$/\ano} \\ \vspace{9pt} {\large\bf
        $\numerolista^a$ Lista de Exerc{\'\i}cios -- Módulo \numeromodulo}\\ \vspace{9pt} Prof. Jos{\'e} Ant{\^o}nio O. Freitas
  \end{center}
  \hrule

\vesp{.6}

\begin{exercicio}
  Nos itens a seguir determine quais equações lineares nas variáveis $x$, $y$ e $z$. Se alguma não for linear, explique o motivo.
  \begin{multicols}{2}
  \begin{enumerate}
    \item $x - \pi y + \sqrt[3]{5}z = 0$
    \item $x^{-1} + 7y + z = \sin(\dfrac{\pi}{9})$
    \item $x = -7y + 3z$
    \item $3\cos(x) = 4y + z = \sqrt{3}t$
    \item $\pi x - \sqrt{x}y + \dfrac{1}{3}z = 7^{1/3}$
    \item $x^{-2} + y^2 + 8x = 5$
    \item $2x - xy - 5z = 0$
    \item $cos(3)x - 4y + z = \sqrt{3}$
  \end{enumerate}
\end{multicols}
  \begin{solucao}
    \begin{enumerate}
      \item Sim
      \item Não
      \item Sim
      \item Não
      \item Sim
      \item Não
      \item Não
      \item Sim
    \end{enumerate}
  \end{solucao}
\end{exercicio}

\begin{exercicio}
  Nos itens a seguir, determine se as equações formam um sistema linear. Para aqueles que forem um sistema linear encontre sua solução.
  \begin{enumerate}
    \item $\begin{cases} -2x + 4y + z = 2\\ 3x - \dfrac{2}{y} = 0\end{cases}$
    \item $\begin{cases} 4x - 4y + 2z = 0\\ -x + \ln(2)y - 3z = 0\end{cases}$
    \item $\begin{cases} x = 4\\ 2x = 8\end{cases}$
    \item $\begin{cases} -2x_1 - x_4 = 5\\ -x_1 + 5x_2 + 3x_3 - 2x_4 = -1\end{cases}$
    \item $\begin{cases} \sin(2x_1 + x_3) = \sqrt{5}\\ e^{-2x_2 - 2x_4} = 0\\ 4x_4 = 4\end{cases}$
  \end{enumerate}
  \begin{solucao}
    \begin{enumerate}
      \item Não
    \item Sim, $x = \left(-\dfrac{7}{2 + 2\ln(2)} - \dfrac{1}{2}\right)z$, $y = -\dfrac{7}{2 + 2\ln(2)}z$
      \item Sim, $x = 4$
      \item Sim, $x_1 = -\dfrac{5}{2} - \dfrac{7}{2}x_4$, $x_2 = -\dfrac{7}{10} - \dfrac{3}{5}x_3 - \dfrac{3}{10}x_4$
      \item Não
    \end{enumerate}
  \end{solucao}
\end{exercicio}

\begin{exercicio}
  Considere o sistema linear:
  \[
    \begin{cases}
      2x_1 - 4x_2 - x_3 = 1\\
      x_1 - 3x_2 + x_3 = 1\\
      3x_1 - 5x_2 - 3x_3 = 1
    \end{cases}
  \]
  Verifique se as seguintes ternas ordenadas são soluções desse sistema.
  \begin{enumerate}
    \item $(3,1,1)$
    \item $(3,-1,1)$
    \item $(13, 5, 2)$
    \item $\left(\dfrac{13}{2}, \dfrac{5}{2}, 2\right)$
    \item $(17,7,5)$
  \end{enumerate}
  \begin{solucao}
    \begin{enumerate}
      \item Sim
      \item Não
      \item Não
      \item Sim
      \item Sim
    \end{enumerate}
  \end{solucao}
\end{exercicio}

\begin{exercicio}
  Encontre a matriz ampliada dos seguintes sistemas lineares:
  \begin{enumerate}
    \item $\begin{cases}x - y = 0\\2x + y = 3\end{cases}$
    \item $\begin{cases}2x_1 + 3x_2 - x_3 = 1\\x_1 + x_3 = 0\\-x_1 + 2x_2 - 2x_3 = 0\end{cases}$
    \item $\begin{cases}x + 5y = -1\\-x + y = -5\\2x + 4y = 4\end{cases}$
    \item $\begin{cases}a - 2b + d = 2\\-a + b - c - 3d = 1\end{cases}$
  \end{enumerate}
\end{exercicio}

\begin{exercicio}
  Nos itens abaixo encontre um sistema linear que tenha a matriz dada como sua matriz ampliada.
  \begin{multicols}{2}
    \begin{enumerate}
      \item $\begin{amatrix}{3}0 & 1 & 1 & 1\\1 & -1 & 0 & 1\\2 & -1 & 1 & 1\end{amatrix}$
      \item $\begin{amatrix}{5}1 & -1 & 0 & 3 & 1 & 2\\1 & 1 & 2 & 1 & -1 & 4\\0 & 1 & 0 & 2 & 3 & 0\end{amatrix}$
      \item $\begin{pmatrix}7 & 2 & 1 & -3 & 5\\1 & 2 & 4 & 0 & 1\end{pmatrix}$
      \item $\begin{pmatrix}2 & -1 \\ -4 & -6\\ 1 & -1\\3 & 0\end{pmatrix}$
    \end{enumerate}
  \end{multicols}
\end{exercicio}

\begin{exercicio}
  Em cada item, determine se a matriz está em forma escalonada, em forma escalonada reduzida por linhas, ambas ou nenhuma.
  \begin{multicols}{3}
  \begin{enumerate}
    \item $\begin{bmatrix}1 & 0 & 0\\0 & 1 & 0\\0 & 0 & 1\end{bmatrix}$
    \item $\begin{bmatrix}1 & 0 & 0\\0 & 1 & 0\\0 & 0 & 0\end{bmatrix}$
    \item $\begin{bmatrix}7 & 0 & 1 & 4\\0 & 1 & -1 & 4\\0 & 0 & 0 & 0\end{bmatrix}$
    \item $\begin{bmatrix}0 & 1 & 0\\0 & 0 & 1\\0 & 0 & 0\end{bmatrix}$
    \item $\begin{bmatrix}1 & 0 & 3 & 1\\0 & 1 & 2 & 4\end{bmatrix}$
    \item $\begin{bmatrix}1 & 2 & 3\\1 & 0 & 0\\0 & 1 & 1\\0 & 0 & 1\end{bmatrix}$
    \item $\begin{bmatrix}2 & 1 & 3 & 5\\0 & 0 & 1 & -1\\0 & 0 & 0 & 0\\0 & 0 & 0 & 1\end{bmatrix}$
  \end{enumerate}
\end{multicols}

  \begin{solucao}
    \begin{enumerate}
      \item Ambas
      \item Ambas
      \item Nenhuma
      \item Ambas
      \item Ambas
      \item Forma escalonada
      \item Nenhuma
    \end{enumerate}
  \end{solucao}
\end{exercicio}

\begin{exercicio}
  Descrever explicitamente todas as poss{\'\i}veis formas que uma matriz $2 \times 2$ na forma escalonada reduzida por linhas pode assumir.
\end{exercicio}

\begin{exercicio}
  Descrever explicitamente todas as poss{\'\i}veis formas que uma matriz $3 \times 3$ na forma escalonada reduzida por linhas pode assumir.
\end{exercicio}

\begin{exercicio}
  Descrever explicitamente todas as poss{\'\i}veis formas que uma matriz $4 \times 4$ na forma escalonada reduzida por linhas pode assumir.
\end{exercicio}

\begin{exercicio}
  Nos itens a seguir mostre que as matrizes dadas são linha-equivalentes e encontre uma sequência de operações elementares com as 
  linhas que convertem $A$ em $B$.
  \begin{enumerate}
    \item $\begin{bmatrix}1 & 2\\3 & 4\end{bmatrix}$, $B = \begin{bmatrix}3 & 1\\1 & 0\end{bmatrix}$
    \item $\begin{bmatrix}2 & 0 & -1\\1 & 1 & 0\\-1 & 1 & 1\end{bmatrix}$, $B = \begin{bmatrix}3 & 1 & -1\\3 & 5 & 1\\2 & 2 & 0\end{bmatrix}$
  \end{enumerate}
\end{exercicio}

\begin{exercicio}
  Determine o n\'umero de solu\c{c}\~oes do sistema
  \[
    \begin{cases}
      x + 2y - 3z = 4\\
      4x + y + 2z = 6\\
      x + 2y + (\alpha^2 - 19)z = \alpha,  
    \end{cases}
\]
em fun\c{c}\~ao do par\^ametro:
\begin{enumerate}[label={\alph*})]
   \item $\alpha \in \real$;
   \item $\alpha \in \rac$?
 \end{enumerate} 
\end{exercicio}

\begin{exercicio}
  Considere o sistema linear
  \[
    \begin{cases}
      (k + 2)x + 2ky - z = 1\\
      x - 2y + kz = -k\\
      y + z = k
    \end{cases}.
  \]
  \begin{enumerate}[label={\alph*})]
    \item Para quais valores de $k \in \real$ o sistema admite solu\c{c}\~ao \'unica? Determine a solu\c{c}\~ao, caso exista.
    \item Para quais valores de $k \in \real$ o sistema admite infinitas solu\c{c}\~oes? Determine tais solu\c{c}\~oes, caso existam.
  \end{enumerate}
\end{exercicio}

Nos exerc{\'\i}cios \ref{sistemalinearinicio} \`a \ref{sistemalinearfim}, encontre a solu\c{c}\~ao geral dos seguintes sistemas lineares. Encontre o posto e a nulidade do sistema.
\begin{exercicio}\label{sistemalinearinicio}
  $
    \begin{cases}
      x + y + z = 4\\
      2x + 5y - 2z = 3\\
      x + 7y - 7z = 5
    \end{cases}
  $
  em $\rac$.
  \begin{solucao}
    N\~ao existe solu\c{c}\~ao.
  \end{solucao}
\end{exercicio}

\begin{exercicio}
  $
    \begin{cases}
      x - 2y + 3z = 0\\
      2x + 5y + 6z = 0
    \end{cases}
  $
  em $\real$.
  \begin{solucao}
    $p = 2$, Nulidade = 1, $S = \{(x, y, z) \mid z, y, z \in \real\} = \{(-3z, 0, z) \mid z \in \real\}$
  \end{solucao}
\end{exercicio}

\begin{exercicio}
  $
    \begin{cases}
      x_1 + x_2 + 2x_3 = 8\\
      -x_1 - 2x_2 + 3x_3 = 1\\
      3x_1 - 7x_2 + 4x_3 = 10
    \end{cases}
  $
  em $\real$.
  \begin{solucao}
    $p = 3$, Nulidade = 0, $S = \{(3, 1, 2)\}$
  \end{solucao}
\end{exercicio}

\begin{exercicio}
  $
    \begin{cases}
      2x_1 + 2x_2 + 2x_3 = 0\\
      -2x_1 + 5x_2 + 2x_3 = 1\\
      8x_1 + x_2 + 4x_3 = -1
    \end{cases}
  $
  em $\real$.
  \begin{solucao}
    $p = 2$, Nulidade = 1, $S = \{(x_1, x_2, x_3) \mid x_1, x_2, x_3 \in \real\} = \{(-1/7 - (3/7)x_3, 1/7 - (4/7)x_3, x_3) \mid x_3 \in \real\}$
  \end{solucao}
\end{exercicio}

\begin{exercicio}
  $
    \begin{cases}
      \phantom{2x_1} - 2x_2 + 3x_3 = 1\\
      3x_1 + 6x_2 - 3x_3 = -2\\
      6x_1 + 6x_2 + 3x_3 = 5
    \end{cases}
  $
  em $\real$.
  \begin{solucao}
    O sistema n\~ao tem solu\c{c}\~ao.
  \end{solucao}
\end{exercicio}

\begin{exercicio}
  $
    \begin{cases}
      ix + iy = 0\\
      2ix - y = 0\\
    \end{cases}
  $
  em $\complex$.
  \begin{solucao}
    $p = 2$, Nulidade = 0, $S = \{(0, 0)\}$
  \end{solucao}
\end{exercicio}

\begin{exercicio}
  $
    \begin{cases}
      x_1 + x_2 + x_3 + x_4 = 0\\
      x_1 + x_2 + x_3 - x_4 = 4\\
      x_1 + x_2 - x_3 + x_4 = -4\\
      x_1 - x_2 + x_3 + x_4 = 2\\
    \end{cases}
  $
  em $\real$.
  \begin{solucao}
    $p = 4$, Nulidade = 0, $S = \{(1, -1, 2, -2)\}$
  \end{solucao}
\end{exercicio}

\begin{exercicio}
  $
    \begin{cases}
      -2x_2 + 3x_3 = 1\\
      3x_1 + 6x_2 - 3x_3 = -1\\
      6x_1 + 6x_2 + 3x_3 = 5
    \end{cases}
  $
  em $\real$.
  \begin{solucao}
    N\~ao existe solu\c{c}\~ao.
  \end{solucao}
\end{exercicio}

\begin{exercicio}
  $
    \begin{cases}
      2x_1 + x_2 + (6 + 6i)x_3 + 8x_4 = 0\\
      x_1 + x_2 + (2 + 5i)x_3 + (5  i)x_4 = 0\\
      (2 + 2i)x_1 + 2x_2 + (2 + 14i)x_3 + (14 + 8i)x_4 =0\\
      (-1 - i)x_1 - x_2 - (3 + 3i)x_3 + (-6 + 2i)x_4 = 0
    \end{cases}
  $
  em $\complex$.
  \begin{solucao}
    $p = 2$, Nulidade = 2, $S = \{(x_1, x_2, x_3, x_4) \mid x_1, x_2, x_3, x_4 \in \complex\} = $\\ $\{(-(2 + i)x_3 - (3 - i)x_4,-3ix_3 - 2x_4, x_3, x_4) \mid x_3, x_4 \in \complex\}$
  \end{solucao}
\end{exercicio}

\begin{exercicio}
  $
    \begin{cases}
      x_1 + 2x_2 - 3x_4 + x_5 = 2\\
      x_1 + 2x_2 + x_3 - 3x_4 + x_5 + 2x_6 = 3\\
      x_1 + 2x_2 - 3x_4 + 2x_5 + x_6 = 4\\
      3x_1 + 6x_2 + x_3 - 9x_4 + 4x_5 + 3x_6 = 9
    \end{cases}
  $
  em $\real$.
  \begin{solucao}
    $p = 3$, Nulidade = 3, $S = \{(x_1, x_2, x_3, x_4, x_5, x_6) \mid x_1, x_2, x_3, x_4, x_5, x_6 \in \real\} = $\\ $\{(x_6 + 3x_4 - 2x_2, x_2, 1 - 2x_6, x_4, 2 - x_6, x_6) \mid x_2, x_4, x_6 \in \real\}$
  \end{solucao}
\end{exercicio}

\begin{exercicio}
  $
    \begin{cases}
      x_1 + 3x_2 - 2x_3 + 2x_5 = 0\\
      2x_1 + 6x_2 - 5x_3 - 2x_4 + 4x_5 - 3x_6 = -1\\
      5x_3 + 10x_4 + 15x_6 = 5\\
      2x_1 + 6x_2 + 8x_4 + 4x_5 + 18x_6 = 6
    \end{cases}
  $
  em $\real$.
  \begin{solucao}
    $p = 3$, Nulidade = 3, $S = \{(x_1, x_2, x_3, x_4, x_5, x_6) \mid x_1, x_2, x_3, x_4, x_5, x_6 \in \real\} = $\\ $\{(-3x_2 - 4x_4 - 2x_5, x_2, -2x_4, x_4, x_5, 1/3) \mid x_2, x_4, x_5 \in \real\}$
  \end{solucao}
\end{exercicio}

\begin{exercicio}\label{sistemalinearfim}
  $
    \begin{cases}
      3x + 2y - 4z = 1\\
      x - y + z = 3\\
      x - y - 3z = -3\\
      3x + 3y - 5z =0\\
      -x + y + z = 1
    \end{cases}
  $
  em $\rac$.
\begin{solucao}
  N\~ao existe solu\c{c}\~ao.
\end{solucao}
\end{exercicio}

Nos exerc{\'\i}cios \ref{sistemasinicio} \`a \ref{sistemasfim}, encontre o(s) valor(es) de $\lambda$ tal(is) que:
\begin{enumerate}[label={\alph*})]
  \item o sistema tem solu\c{c}\~ao \'unica;
  \item o sistema tem v\'arias solu\c{c}\~oes;
  \item o sistema n\~ao tem solu\c{c}\~ao.
\end{enumerate}

\begin{exercicio}\label{sistemasinicio}
  $
    \begin{cases}
      x + \lambda y = 1\\
      \lambda x + y = 1
    \end{cases}
  $
  em $\real$.
  \begin{solucao}
    \begin{enumerate}[label={\alph*})]
      \item $\lambda ne -1$ e $\lambda \ne 1$.
      \item $\lambda = 1$.
      \item $\lambda = -1$.
    \end{enumerate}
  \end{solucao}
\end{exercicio}

\begin{exercicio}
  $
    \begin{cases}
      (\lambda - 2)x + y = 0\\
      x + (\lambda - 2)y = 0
    \end{cases}
  $
  em $\real$.
  \begin{solucao}
    \begin{enumerate}[label={\alph*})]
      \item $\lambda \ne 1$ e $\lambda \ne 3$
      \item $\lambda = 1$ e $\lambda = 3$.
      \item N\~ao existe tal $\lambda$.
    \end{enumerate}
  \end{solucao}
\end{exercicio}

\begin{exercicio}
  $
    \begin{cases}
      x + 2y - 3z = 4\\
      3x - y + 5z = 2\\
      4x + y + (\lambda^2 - 14)z = \lambda + 2
    \end{cases}
  $
  em $\real$.
  \begin{solucao}
    \begin{enumerate}[label={\alph*})]
      \item $\lambda \ne \pm 4$.
      \item $\lambda = 4$.
      \item $\lambda = -4$.
    \end{enumerate}
  \end{solucao}
\end{exercicio}

\begin{exercicio}
  $
    \begin{cases}
      x + y + z = 2\\
      2x + 3y + 2z = 5\\
      2x + 3y + (\lambda^2 - 1)z = \lambda + 1
    \end{cases}
  $
  em $\rac$.
  \begin{solucao}
    \begin{enumerate}[label={\alph*})]
      \item N\~ao existe tal $\lambda$.
      \item N\~ao existe tal $\lambda$.
      \item N\~ao existe tal $\lambda$.
    \end{enumerate}
  \end{solucao}
\end{exercicio}

\begin{exercicio}
  $
    \begin{cases}
      x_1 + \lambda x_2 + (1 + 4\lambda )x_3 = 1 + 4\lambda \\
      2x_1 + (\lambda  + 1)x_2 + (2 + 7\lambda )x_3 = 1 + 7\lambda \\
      3x_1 + (\lambda  + 2)x_2 + (3 + 9\lambda )x_3 = 1 + 9\lambda 
    \end{cases}  
  $
  \begin{solucao}
    \begin{enumerate}[label={\alph*})]
      \item $\lambda \ne 0$ e $\lambda \ne 1$.
      \item $\lambda = 0$.
      \item $\lambda = 1$.
    \end{enumerate}
  \end{solucao}
\end{exercicio}

\begin{exercicio}\label{sistemasfim}
  $
    \begin{cases}
      x + y = 2\\
      y + z = 2\\
      x + z = 2\\
      x + y + \lambda z = 0
    \end{cases}
  $
  em $\real$.
  \begin{solucao}
    \begin{enumerate}[label={\alph*})]
      \item $\lambda = -2$
      \item N\~ao existe tal $\lambda$.
      \item $\lambda \ne -2$.
    \end{enumerate}
  \end{solucao}
\end{exercicio}

\begin{exercicio}
  Encontre condi\c{c}\~oes sobre os $b_i \in \real$ para que cada um dos sistemas tenha solu\c{c}\~ao.
  \begin{enumerate}[label={\alph*})]
    \item $\begin{cases}
      x_1 - 2x_2 + 5x_3 = b_1\\
      4x_1 - 5x_2 + 8x_3 = b_2\\
      -3x_1 + 3x_2 - 3x_3 = b_3
    \end{cases}$

    \item $\begin{cases}
      x_1 - 2x_2 - x_3 = b_1\\
      -4x_1 + 5x_2 + 2x_3 = b_2\\
      -4x_1 + 7x_2 + 4x_3 = b_3
    \end{cases}$
  \end{enumerate}
  \begin{solucao}
    \begin{enumerate}[label={\alph*})]
      \item $b_3 - b_1 + b_2 = 0$
      \item O sistema tem solu\c{c}\~ao para todos os valores reais de $b_1$, $b_2$ e $b_3$.
    \end{enumerate}
  \end{solucao}
\end{exercicio}

\newpage
\Closesolutionfile{ans}
\hrule
\begin{center}
{\large\bf RESPOSTAS}
\end{center}
\hrule
\input{ans1}

\end{document}
