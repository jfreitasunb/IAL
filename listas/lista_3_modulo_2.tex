%!TEX program = xelatex
\def\numeromodulo{2}

\def\numerolista{3}

\documentclass[12pt]{exam}

\def\ano{2023}
\def\semestre{1}
\def\disciplina{Introdução à \'Algebra Linear}
\def\turma{3}


\usepackage{caption}
\usepackage{amssymb}
\usepackage{amsmath,amsfonts,amsthm,amstext}
\usepackage[brazil]{babel}
\usepackage{graphicx}
\graphicspath{{../Pictures/}}
\usepackage{enumitem}
\usepackage{multicol}
\usepackage{answers}
\usepackage[svgnames]{xcolor}
\usepackage{tikz}
\usepackage{ifthen}
\usetikzlibrary{lindenmayersystems}
\usetikzlibrary[shadings]


\Newassociation{solucao}{Solution}{ans}
\newtheorem{exercicio}{}

\setlength{\topmargin}{-1.0in}
\setlength{\oddsidemargin}{0in}
\setlength{\textheight}{10.1in}
\setlength{\textwidth}{6.5in}
\setlength{\baselineskip}{12mm}

\extraheadheight{0.7in}
\firstpageheadrule
\runningheadrule
\lhead{
        \begin{minipage}[c]{1.7cm}
        \includegraphics[width=1.7cm]{unb.pdf}
        \end{minipage}%
        \hspace{0pt}
        \begin{minipage}[c]{4in}
          {Universidade de Bras{\'\i}lia} --
          {Departamento de Matem{\'a}tica}
\end{minipage}
\vspace*{-0.8cm}
}
% \chead{Universidade de Bras{\'\i}lia - Departamento de Matem{\'a}tica}
% \rhead{}
% \vspace*{-2cm}

\extrafootheight{.5in}
\footrule
\lfoot{\disciplina\ - \semestre$^o$/\ano\ - Módulo \numeromodulo}
\cfoot{}
\rfoot{P\'agina \thepage\ de \numpages}

\newcounter{exercicios}
\renewcommand{\theexercicios}{\arabic{exercicios}}

\newenvironment{questao}[1]{
\refstepcounter{exercicios}
\ifx&#1&
\else
   \label{#1}
\fi
\noindent\textbf{Exerc{\'\i}cio {\theexercicios}:}
}

\newcommand{\resp}[1]{
\noindent{\bf Exerc{\'\i}cio #1: }}


\def\ano{2023}
\def\semestre{2}
\def\disciplina{Introdução à Álgebra Linear}
\def\nomeabreviado{IAL}
\def\turma{11}

\newcommand{\im}{{\rm Im\,}}
\newcommand{\dlim}[2]{\displaystyle\lim_{#1\rightarrow #2}}
\newcommand{\minf}{+\infty}
\newcommand{\ninf}{-\infty}
\newcommand{\cp}[1]{\mathbb{#1}}
\newcommand{\sub}{\subseteq}
\newcommand{\n}{\mathbb{N}}
\newcommand{\z}{\mathbb{Z}}
\newcommand{\rac}{\mathbb{Q}}
\newcommand{\real}{\mathbb{R}}
\newcommand{\complex}{\mathbb{C}}

\newcommand{\vesp}[1]{\vspace{ #1  cm}}

\newcommand{\compcent}[1]{\vcenter{\hbox{$#1\circ$}}}
\newcommand{\comp}{\mathbin{\mathchoice
        {\compcent\scriptstyle}{\compcent\scriptstyle}
        {\compcent\scriptscriptstyle}{\compcent\scriptscriptstyle}}}
\renewcommand{\sin}{{\rm sen\,}}
\renewcommand{\tan}{{\rm tg\,}}
\renewcommand{\csc}{{\rm cossec\,}}
\renewcommand{\cot}{{\rm cotg\,}}
\renewcommand{\sinh}{{\rm senh\,}}

\newcommand{\integer}{\mathbb{Z}}

\begin{document}

    \Opensolutionfile{ans}[ans1]
    
    \begin{center}
        {\Large\bf \disciplina\ - Turma \turma\ -- \semestre$^{o}$/\ano} \\ \vspace{9pt} {\large\bf
            $\numerolista^a$ Lista de Exerc{\'\i}cios -- Módulo \numeromodulo}\\ \vspace{9pt} Prof. Jos{\'e} Ant{\^o}nio O. Freitas
    \end{center}
    
    \hrule

    \vesp{.6}

    \begin{exercicio}
        Sejam $W_1$ e $W_2$  subespa\c{c}os de um $\cp{K}$-espa\c{c}o vetorial $V$. Defina
        \[
            W_1 - W_2 = \{ u_1 - u_2 \mid u_1 \in W_1, u_2 \in W_2\}.
        \]
        O conjunto $W_1 - W_2$ \'e um subespa\c{c}o vetorial de $V$?
    \end{exercicio}

    \begin{exercicio}
        Sejam $W_1$ e $W_2$ subespa\c{c}os de um espa\c{c}o vetorial $V$ sobre $\cp{K}$ tais que $W_1 \cap W_2 = \{0_V\}$,
        \begin{enumerate}[label={\alph*})]
            \item Mostre que se $\mathcal{B}_1$ e $\mathcal{B}_2$ s\~ao conjuntos L.I. em $W_1$ e $W_2$, respectivamente, ent\~ao $\mathcal{B}_1 \cup \mathcal{B}_2$ \'e L.I. em $V$.

            \item Mostre que se $\mathcal{B}_1$ e $\mathcal{B}_2$ s\~ao bases de $W_1$ e $W_2$, respectivamente, ent\~ao $\mathcal{B}_1 \cup \mathcal{B}_2$ \'e base de $W_1 + W_2$.
        \end{enumerate}
    \end{exercicio}

    \begin{exercicio}
        Seja $W = \left\{\begin{pmatrix} a_{11} & a_{12}\\ a_{21} & a_{22}\end{pmatrix} \in \cp{M}_2(\complex) \mid a_{11} + a_{12} = 0\right\}$.
        \begin{enumerate}[label={\alph*})]
            \item Mostre que $W$ \'e um espa\c{c}o vetorial sobre $\real$.
    
            \item Determine uma base de $W$.
    
            \item Seja $W_1 = \{(a_{ij})_{i,j}, \in \cp{M}_2(\complex), i, j = 1, 2 \mid a_{21} = -\overline{a_{12}}\}$, onde $\overline{a + bi} = a - bi$. Prove que $W_1$ \'e um subespa\c{c}o de $\cp{M}_2(\complex)$ sobre $\real$ e ache uma base de $W_1$.
        \end{enumerate}
    \end{exercicio}

    \begin{exercicio}
        Sejam $u = (2, 0 , -1)$, $v = (3, 1 , 0)$ e $w = (1, -1 , c)$ onde $c \in \real$. Para qual(is) valor(es) de $c \in \real$ o conjunto $\{u, v, w\}$ \'e uma base de $\real^3$?
    \end{exercicio}

    \begin{exercicio}
        Sejam $u = (1, -1 , 3)$, $v = (1, 0 , 1)$ e $w = (1, 2 , c)$ onde $c \in \real$. Para qual(is) valor(es) de $c \in \real$ o conjunto $\{u, v, w\}$ \'e uma base de $\real^3$?
    \end{exercicio}

    \begin{exercicio}
        Em $\real^4$ seja $W_1$ o subespa\c{c}o gerado por $(1,1,1,0)$ e $(0,-4,1,5)$ e seja $W_2$ o subespa\c{c}o gerado por $(0,-2,1,2)$ e $(1,-1,1,3)$. Determine $\dim_\real(W_1 + W_2)$ e $\dim_\real(W_1 \cap W_2)$.
    \end{exercicio}

    \begin{exercicio}
        Seja $W$ um subespa\c{c}o vetorial de uma espa\c{c}o vetorial $V$ finitamente gerado sobre $\cp{K}$. Mostre que se $\dim_\cp{K}W = \dim_\cp{K}V$, ent\~ao $W = V$.
    \end{exercicio}

    \begin{exercicio}
        Mostre que 
        \[
            [1, 1 - x, (1 - x)^2, 1 - x^3] = \mathcal{P}_3(\real).
        \]
    \end{exercicio}

    \begin{exercicio}
        Seja $V$ o espa\c{c}o das matrizes $2 \times 2$ sobre $\real$, e seja $W$ o subespa\c{c}o gerado por
        \[
            \begin{bmatrix}
                1 & -5\\
                -4 & 2
            \end{bmatrix},
            \begin{bmatrix}
                1 & 1\\
                -1 & 5
            \end{bmatrix},
            \begin{bmatrix}
                2 & -4\\
                -5 & 7
            \end{bmatrix},
            \begin{bmatrix}
                1 & -7\\
                -5 & 1
            \end{bmatrix}.
        \]
        Encontre uma base e a dimens\~ao de $W$.
    \end{exercicio}

    \begin{exercicio}
        Considere o subespa\c{c}o de $\real^4$ gerado pelos vetores $v_1 = (1, -1, 0, 0)$, $v_2 = (0, 0, 1, 1)$, $v_3 = (2, -2, 1, 1)$ e $v_4 = (1, 0, 0, 0)$.
        \begin{enumerate}[label={\alph*})]
            \item Exiba uma base para $[v_1, v_2, v_3, v_4]$. Qual \'e a dimens\~ao?

            \item $[v_1, v_2, v_3, v_4] = \real^4$? Por qu\^e?
        \end{enumerate}
        \begin{solucao}
            \begin{enumerate}[label={\alph*})]
                \item Sim.

                \item $\dim_\real [v_1, v_2, v_3, v_4] = 3$.

                \item N\~ao.
            \end{enumerate}
        \end{solucao}
    \end{exercicio}

    \begin{exercicio}
        Para quais valores de $\alpha \in \real$ vale
        \[
            [(1, 0, \alpha), (1, 2, -3) , (\alpha, 1, 0)] = \real^3?
        \]
    \end{exercicio}

    \begin{exercicio}
        Encontre uma base de $\real^4$ que contenha os vetores $(1,2,-2,1)$ e $(1,0,-2,2)$.
    \end{exercicio}

    \begin{exercicio}
        Considere o subespa\c{c}o vetorial $W$ de $\mathcal{P}_4(\real)$ gerado pelo conjunto
        \[
            \mathcal{A} = \{1+2x+x^2+3x^3+x^4, 1-2x-2x^2-2x^3-3x^4,2-x^2+x^3-2x^4,x-x^3+x^4,3x^2+6x^3+3x^4\}.
        \]
        Determine uma base de $\mathcal{B}$ de $W$ que esteja contida em $\mathcal{A}$.
    \end{exercicio}

    \begin{exercicio}
        Seja $S \subseteq M_2(\real)$ dado por:
        \[
            S = \left\{
                \begin{bmatrix}
                    a & b\\
                    c & d
                \end{bmatrix}
                \mid c = a+b \mbox{ e } d = a
            \right\}.
        \]

        \begin{enumerate}[label={\alph*})]
            \item Mostre que $S$ \'e um subespa\c{c}o vetorial de $M_2(\real)$.

            \item Qual a dimens\~ao de $S$?

            \item O conjunto
                \[
                    \left\{
                        \begin{bmatrix}
                            1 & -1\\
                            0 & 1
                        \end{bmatrix},
                        \begin{bmatrix}
                            2 & 1\\
                            3 & 4
                        \end{bmatrix}
                    \right\}
                \]
                \'e uma base de $S$? Justifique.
        \end{enumerate}

        \begin{solucao}
            \begin{enumerate}[label={\alph*})]
                \item $\dim_\real S = 2$

                \item N\~ao.
            \end{enumerate}
        \end{solucao}
    \end{exercicio}

    \begin{exercicio}
        Sejam $W_1$, $W_2 \subseteq M_2(\real)$ dados por
        \begin{align*}
            W_1 &= \left\{
                        \begin{bmatrix}
                            u & -u -x\\
                            0 & x
                        \end{bmatrix}
                        \mid u, x \in \real
                    \right\}\\
            W_2 &= \left\{
                        \begin{bmatrix}
                            v & 0\\
                            w & -x
                        \end{bmatrix}
                        \mid v, w, x \in \real
                    \right\}.
        \end{align*}
        
        \begin{enumerate}[label={\alph*})]
        
            \item Mostre que $W_1$ e $W_2$ s\~ao subespa\c{c}os vetoriais de $M_2(\real)$.
                
            \item Encontre uma base para $W_1$.
            
            \item Encontre uma base para $W_2$.
            
            \item Encontre uma base para $W_1 + W_2$.
            
            \item Encontre uma base para $W_1 \cap W_2$.
        \end{enumerate}
    \end{exercicio}

    \begin{exercicio}
        No $\real$-espa\c{c}o vetorial $\real^4$, considere o subespa\c{c}o $V$ dado pelas solu\c{c}\~oes do sistema linear
        \[
            \begin{cases}
                x + 2y + z = 0\\
                -x -y + 3t = 0
            \end{cases}
        \]
        e o subespa\c{c}o $W$ gerado pelos vetores
        \[
            \begin{bmatrix}
                2\\
                0\\
                1\\
                1
            \end{bmatrix} \quad \mbox{e}\quad
            \begin{bmatrix}
                3\\
                -2\\
                -2\\
                0
            \end{bmatrix}.
        \]
        
        Determine:
        \begin{enumerate}[label={\alph*})]
            \item $\dim_\real(V \cap W)$
            
            \item $\dim_\real(V + W)$
        \end{enumerate}
    \end{exercicio}

    \newpage
    
    \Closesolutionfile{ans}

    \begin{center}
        {\large\bf RESPOSTAS}
    \end{center}

    \hrule
    
    \input{ans1}

\end{document}
