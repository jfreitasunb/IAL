%!TEX program = xelatex
\def\numeromodulo{3}
\def\numerolista{1}
\documentclass[12pt]{exam}

\def\ano{2023}
\def\semestre{1}
\def\disciplina{Introdução à \'Algebra Linear}
\def\turma{3}

\usepackage{caption}
\usepackage{amssymb}
\usepackage{amsmath,amsfonts,amsthm,amstext}
\usepackage[brazil]{babel}
\usepackage{graphicx}
\graphicspath{{../Pictures/}}
\usepackage{enumitem}
\usepackage{multicol}
\usepackage{answers}
\usepackage[svgnames]{xcolor}
\usepackage{tikz}
\usepackage{ifthen}
\usetikzlibrary{lindenmayersystems}
\usetikzlibrary[shadings]

\Newassociation{solucao}{Solution}{ans}
\newtheorem{exercicio}{}

\setlength{\topmargin}{-1.0in}
\setlength{\oddsidemargin}{0in}
\setlength{\textheight}{10.1in}
\setlength{\textwidth}{6.5in}
\setlength{\baselineskip}{12mm}

\extraheadheight{0.7in}
\firstpageheadrule
\runningheadrule
\lhead{
        \begin{minipage}[c]{1.7cm}
        \includegraphics[width=1.7cm]{unb.pdf}
        \end{minipage}%
        \hspace{0pt}
        \begin{minipage}[c]{4in}
          {Universidade de Bras{\'\i}lia} --
          {Departamento de Matem{\'a}tica}
\end{minipage}
\vspace*{-0.8cm}
}
% \chead{Universidade de Bras{\'\i}lia - Departamento de Matem{\'a}tica}
% \rhead{}
% \vspace*{-2cm}

\extrafootheight{.5in}
\footrule
\lfoot{\disciplina\ - \semestre$^o$/\ano\ - Módulo \numeromodulo}
\cfoot{}
\rfoot{P\'agina \thepage\ de \numpages}

\newcounter{exercicios}
\renewcommand{\theexercicios}{\arabic{exercicios}}

\newenvironment{questao}[1]{
\refstepcounter{exercicios}
\ifx&#1&
\else
   \label{#1}
\fi
\noindent\textbf{Exerc{\'\i}cio {\theexercicios}:}
}

\newcommand{\resp}[1]{
\noindent{\bf Exerc{\'\i}cio #1: }}

\def\ano{2023}
\def\semestre{2}
\def\disciplina{Introdução à Álgebra Linear}
\def\nomeabreviado{IAL}
\def\turma{11}

\newcommand{\im}{{\rm Im\,}}
\newcommand{\dlim}[2]{\displaystyle\lim_{#1\rightarrow #2}}
\newcommand{\minf}{+\infty}
\newcommand{\ninf}{-\infty}
\newcommand{\cp}[1]{\mathbb{#1}}
\newcommand{\sub}{\subseteq}
\newcommand{\n}{\mathbb{N}}
\newcommand{\z}{\mathbb{Z}}
\newcommand{\rac}{\mathbb{Q}}
\newcommand{\real}{\mathbb{R}}
\newcommand{\complex}{\mathbb{C}}

\newcommand{\vesp}[1]{\vspace{ #1  cm}}

\newcommand{\compcent}[1]{\vcenter{\hbox{$#1\circ$}}}
\newcommand{\comp}{\mathbin{\mathchoice
        {\compcent\scriptstyle}{\compcent\scriptstyle}
        {\compcent\scriptscriptstyle}{\compcent\scriptscriptstyle}}}
\renewcommand{\sin}{{\rm sen\,}}
\renewcommand{\tan}{{\rm tg\,}}
\renewcommand{\csc}{{\rm cossec\,}}
\renewcommand{\cot}{{\rm cotg\,}}
\renewcommand{\sinh}{{\rm senh\,}}
\newcommand{\integer}{\mathbb{Z}}
\begin{document}

  \Opensolutionfile{ans}[ans1]
  \begin{center}
    {\Large\bf \disciplina\ - Turma \turma\ -- \semestre$^{o}$/\ano} \\ \vspace{9pt} {\large\bf
        $\numerolista^a$ Lista de Exerc{\'\i}cios -- Módulo \numeromodulo}\\ \vspace{9pt} Prof. Jos{\'e} Ant{\^o}nio O. Freitas
  \end{center}
  \hrule

\vesp{.6}

\begin{exercicio}
  Prove que cada uma das fun\c{c}\~oes abaixo \'e uma transforma\c{c}\~ao linear.
  \begin{enumerate}[label={\alph*})]
    \item $D : \mathcal{P}(\complex) \to \mathcal{P}(\complex)$ dada por
    \[
        D(a_0 + a_1x + a_2x^2 + \cdots + a_nx^n) = a_1 + 2a_2x + \cdots + na_nx^{n - 1}.
    \]
    Considere $\mathcal{P}(\complex)$ um $\complex$-espa\c{c}o vetorial.

    \item $T : \mathcal{C}([a,b], \real) \to \real$ dada por
    \[
        T(f(x)) = \int_a^bf(x)dx.
    \]
    Considere $\mathcal{C}([a,b], \real)$ e $\real$ como $\real$-espa\c{c}os vetoriais.

    \item $F : \mathcal{P}_2(\real) \to \mathcal{P}_2(\real)$ dada por $F(p(t)) = t^2p''(x)$. Considere $\mathcal{P}_2(\real)$ um $\real$-espa\c{c}o vetorial.

    \item $G : \cp{M}_2(\real) \to \cp{M}_2(\real)$ dada por $G(X) = MX + X$ onde
    \[
        M = \begin{pmatrix}
                1 & 0\\
                0 & 0
            \end{pmatrix}.
    \]
    Considere $\cp{M}_2(\real)$ um $\real$-espa\c{c}o vetorial.

    \item $H : \cp{M}_2(\real) \to \cp{M}_2(\real)$ dada por $H(x) = MX - XM$ onde
    \[
        M = \begin{pmatrix}
                1 & 2\\
                0 & 1
            \end{pmatrix}.
    \]
    Considere $\cp{M}_2(\real)$ um $\real$-espa\c{c}o vetorial.

    \item $T : \complex^3 \to \complex$ dada por $T(x,y,z) = x + 2y + iz$. Considere $\complex^3$ e $\complex^2$ como $\complex$-espa\c{c}os vetoriais.

    \item $F : \mathcal{P}_3(\real) \to \mathcal{P}_4(\real)$ dada por $(Fp)(x) = xp(x + 1)$. Considere $\mathcal{P}_3(\real)$ e $\mathcal{P}_4(\real)$ como $\real$-espa\c{c}os vetoriais.

    \item $G : \real^2 \to \mathcal{P}_2(\real)$ dada por $T(a,b) = ax^2 + bx + (a + b)$. Considere $\real^2$ e $\mathcal{P}_2(\real)$ como $\real$-espa\c{c}os vetoriais.
  \end{enumerate}
\end{exercicio}

\begin{exercicio}
  Seja $V$ um $\cp{K}$-espa\c{c}o vetorial e $T : V \to V$ uma transforma\c{c}\~ao linear. Dado $\lambda \in \cp{K}$, defina
  \[
      V_\lambda = \{ u \in V \mid T(u) = \lambda u\}.
  \]
  Mostre que $V_\lambda$ \'e um $\cp{K}$-subespa\c{c}o vetorial de $V$.
\end{exercicio}

\begin{exercicio}
  Considere a fun\c{c}\~ao $T : \complex \to \cp{M}_2(\real)$ dada por
  \[
      T(x + yi) = \begin{pmatrix}
                        x + 7y & 5y\\
                       -10y & x - 7y
                   \end{pmatrix}.
  \]
  Considerando $\complex$ como um espa\c{c}o vetorial sobre $\real$:
  \begin{enumerate}[label={\alph*})]
    \item Prove que $T$ \'e uma transforma\c{c}\~ao linear.

    \item Prove que $T(z_1\cdot z_2) = T(z_1)\cdot T(z_2)$ para todos $z_1$, $z_2 \in \complex$.
  \end{enumerate}
\end{exercicio}

\begin{exercicio}
  Seja $T : \real^2 \to \real^2$ uma transforma\c{c}\~ao linear tal que $T(1,1) = (1,0)$ e $T(1,-1) = (0,1)$. Encontre $T(1,0)$ e $T(1,2)$.
  \begin{solucao}
    $T(1,0) = (1/2,1/2)$ e $T(0,2) = (1,-1)$.
  \end{solucao}
\end{exercicio}

\begin{exercicio}
  Considere a base $\mathcal{B} = \{v_1, v_2\}$ de $\real^2$, em que $v_1 = (1, 1)$ e $v_2 = (1, 0)$. Considere o operador linear $T \colon \real^2 \to \real^2$ tal que
  \[
    T(v_1) = (1, -2)\quad \mbox{e} \quad T(v_2) = (4, -1).
  \]
  Encontre a expressão para $T(x, y)$ e use essa expressão para obter $T(5, -3)$.
\end{exercicio}

\begin{exercicio}
  Considere a base $\mathcal{B} = \{v_1, v_2\}$ de $\real^2$, em que $v_1 = (-2, 1)$ e $v_2 = (1, 3)$. Considere a transformação linear $T \colon \real^2 \to \real^3$ tal que
  \[
    T(v_1) = (-1, 2, 0)\quad \mbox{e} \quad T(v_2) = (0, -3, 5).
  \]
  Encontre a expressão para $T(x, y)$ e use essa expressão para obter $T(2, -3)$.
\end{exercicio}

\begin{exercicio}
  Considere a base $\mathcal{B} = \{v_1, v_2, v_3\}$ de $\real^3$, em que $v_1 = (1, 1, 1)$, $v_2 = (1, 1, 0)$ e $v_3 = (1, 0, 0)$. Considere o operador linear $T \colon \real^2 \to \real^3$ tal que
  \[
    T(v_1) = (2, -1, 4), T(v_2) = (3, 0, 1), T(v_3) = (-1, 5, 1).
  \]
  Encontre a expressão para $T(x, y, z)$ e use essa expressão para obter $T(2, 4, -1)$.
\end{exercicio}

\begin{exercicio}
  Determine uma transforma\c{c}\~ao linear $T : \real^3 \to \real^3$ tal que:
  \begin{enumerate}[label={\alph*})]
    \item $\dim_\real \ker T = 0$;

    \item $\dim_\real \ker T = 1$;

    \item $\dim_\real \ker T = 2$;

    \item $\dim_\real \ker T = 3$.
  \end{enumerate}
\end{exercicio}

\begin{exercicio}
  Considere $\real^4$ e seus subespa\c{c}os $V = [(1,0,1,1);(0,-1,-1,-1)]$ e $W = \{(x,y,z,t) \in \real^4 \mid x + y = 0,\ t + z = 0\}$. Determine uma transforma\c{c}\~ao linear $T : \real^4 \to \real^4$ tal que $\ker T = V$ e $\im T = W$.
\end{exercicio}

\begin{exercicio}
  Determine o n\'ucleo e a imagem das seguintes transforma\c{c}\~oes lineares:
  \begin{enumerate}[label={\alph*})]
    \item $T : \real^2 \to \real^2$ dada por $T(x,y) = (x - y, x + y)$.

    \item $T : \complex^2 \to \real^2$ dada por $T(x + yi,z + ti) = (x + 2z, -x + 2t)$.
  \end{enumerate}
\end{exercicio}

\begin{exercicio}
  Ache uma transforma\c{c}\~ao linear $T : \real^4 \to \real^4$ tal que
  \begin{align*}
    \ker T &= [(1,0,0,1);(-1,0,0,1)]\\
    \im T &= [(1,-1,0,2);(0,1,-1,0)].
  \end{align*}
\end{exercicio}

Nos exerc{\'\i}cios \eqref{nucleo_imagem_inicio} \`a \eqref{nucleo_imagem_fim}, encontre uma base e a dimens\~ao de:
\begin{enumerate}[label={\alph*})]
  \item $\ker F$

  \item $\im F$
\end{enumerate}

\begin{exercicio}\label{nucleo_imagem_inicio}
  Seja $F : \real^4 \to \real^3$ a transforma\c{c}\~ao linear definida por
  \[
  F(x,y,s,t) = (x - y + s + t, x + 2s - t, x + y + 3s - 3t).
  \]
  \begin{solucao}
    \begin{enumerate}[label={\alph*})]
      \item $\dim_\real\im F = 2$

      \item $\dim_\real\ker F = 2$
    \end{enumerate}
  \end{solucao}
\end{exercicio}

\begin{exercicio}
  Seja $F : \mathcal{P}_3(\real) \to \mathcal{P}_4(\real)$ dada por $(Fp)(x) = xp(x + 1)$. Considere $\mathcal{P}_3(\real)$ e $\mathcal{P}_4(\real)$ como $\real$-espa\c{c}os vetoriais.
\end{exercicio}

\begin{exercicio}
  Seja $G : \real^2 \to \mathcal{P}_2(\real)$ dada por $T(a,b) = ax^2 + bx + (a + b)$. Considere $\real^2$ e $\mathcal{P}_2(\real)$ como $\real$-espa\c{c}os vetoriais.
\end{exercicio}

\begin{exercicio}
  Seja $F : \mathcal{P}_2(\real) \to \mathcal{P}_2(\real)$ dada por $F(p(t)) = t^2p''(t)$. Considere $\mathcal{P}_2(\real)$ um $\real$-espa\c{c}o vetorial.
\end{exercicio}

\begin{exercicio}
  Seja $G : \cp{M}_2(\real) \to \cp{M}_2(\real)$ dada por $G(X) = MX + X$ onde
  \[
      M = \begin{pmatrix}
            1 & 0\\
            0 & 0
          \end{pmatrix}.
  \]
  Considere $\cp{M}_2(\real)$ um $\real$-espa\c{c}o vetorial.
\end{exercicio}

\begin{exercicio}
  Seja $D : \mathcal{P}(\complex) \to \mathcal{P}(\complex)$ dada por
  \[
      D(a_0 + a_1x + a_2x^2 + \cdots + a_nx^n) = a_1 + 2a_2x + \cdots + na_nx^{n - 1}.
  \]
  Considere $\mathcal{P}(\complex)$ um $\complex$-espa\c{c}o vetorial.
\end{exercicio}

\begin{exercicio}
  Seja $F : \real^3 \to \real^3$ a transforma\c{c}\~ao linear definida por
  \[
      F(x,y,z) = (x + 2y - z, y + z, x + y - 2z).
  \]
  \begin{solucao}
    \begin{enumerate}[label={\alph*})]
      \item $\dim_\real\im F = 2$

      \item $\dim_\real\ker F = 1$
    \end{enumerate}
  \end{solucao}
\end{exercicio}

\begin{exercicio}
  Seja $F : \cp{M}_3(\real) \to \cp{M}_3(\real)$ a transforma\c{c}\~ao linear definida por
  \[
      F(A) = AM - MA,
  \]
  onde $M = \begin{bmatrix}
    1 & \phantom{-}2 & 0\\0 & \phantom{-}3 & 1\\0 & -1 & 1
  \end{bmatrix}$ e $\cp{M}_3(\real)$ \'e um $\real$-espa\c{c}o vetorial.
  \begin{solucao}
    \begin{enumerate}[label={\alph*})]
      \item $\dim_\real\im F = 2$

      \item $\dim_\real\ker F = 2$
    \end{enumerate}
  \end{solucao}
\end{exercicio}

\begin{exercicio}\label{nucleo_imagem_fim}
  Seja $F : \cp{M}_2(\complex) \to \cp{M}_2(\complex)$ a transforma\c{c}\~ao linear definida por
  \[
      F(A) = AM - MA,
  \]
  onde $M = \begin{bmatrix}
    1 & 2\\0 & 3
  \end{bmatrix}$ e $\cp{M}_2(\complex)$ \'e um $\complex$-espa\c{c}o vetorial.
  \begin{solucao}
    \begin{enumerate}[label={\alph*})]
      \item $\dim_\real\im F = 2$

      \item $\dim_\real\ker F = 2$
    \end{enumerate}
  \end{solucao}
\end{exercicio}


\begin{exercicio}
  Encontre uma transforma\c{c}\~ao linear $F : \real^3 \to \real^4$, cuja imagem \'e gerada por $w_1 = (1,2,0,-4)$ e $w_2 = (2,0,-1,-3)$.
\end{exercicio}

\begin{exercicio}
  Encontre uma transforma\c{c}\~ao linear $T : \real^3 \to \cp{M}_{3 \times 1}(\real)$, cuja imagem \'e gerada por
  \[
        v_1 = \begin{bmatrix}
                1\\2\\3
              \end{bmatrix},
        v_2 = \begin{bmatrix}
                 4\\5\\6
              \end{bmatrix}.
  \]
\end{exercicio}

\begin{exercicio}
  Encontre uma transforma\c{c}\~ao linear $T : \cp{M}_{1 \times 4}(\real) \to \real^5$ , cujo kernel \'e gerado por
  \[
      v_1 = \begin{bmatrix}
                1 & 2 & 3 & 4
            \end{bmatrix},
      v_2 = \begin{bmatrix}
                0 & 1 & 1 & 1
            \end{bmatrix}.
  \]
\end{exercicio}

\begin{exercicio}
  Nos casos abaixo construa uma transforma\c{c}\~ao linear $T : V \to W$ satisfazendo as condi\c{c}\~oes dadas:
  \begin{enumerate}[label={\alph*})]
    \item $V = \real^3$, $W = \real^2$, $\cp{K} = \real$, $\dim_\cp{K} \ker T = 1$, $\dim_\cp{K} \im T = 2$

    \item $V = \real^3$, $W = \cp{M}_{3\times 1}(\complex)$, $\cp{K} = \real$, $\dim_\cp{K} \ker T = 2$, $\dim_\cp{K} \im T = 1$

    \item $V = \real^3$, $W = \cp{M}_{3\times 1}(\real)$, $\cp{K} = \real$, $\dim_\cp{K} \ker T = 0$, $\dim_\cp{K} \im T = 3$

    \item $V = \real^3$, $W = \cp{M}_{3\times 1}(\real)$, $\cp{K} = \real$, $\dim_\cp{K} \ker T = 3$, $\dim_\cp{K} \im T = 0$

    \item $V = \cp{M}_{3\times 1}(\complex)$, $W = \real^3$, $\cp{K} = \real$, $\dim_\cp{K} \ker T = 3$, $\dim_\cp{K} \im T = 3$

    \item $V = \mathcal{P}_5(\real)$, $W = \real^4$, $\cp{K} = \real$, $\dim_\cp{K} \ker T = 4$, $\dim_\cp{K} \im T = 2$

    \item $V = \mathcal{P}_5(\complex)$, $W = \complex^5$, $\cp{K} = \complex$, $\dim_\cp{K} \ker T = 1$, $\dim_\cp{K} \im T = 5$

    \item $V = \mathcal{P}_5(\complex)$, $W = \cp{M}_{3\times 2}(\complex)$, $\cp{K} = \complex$, $\dim_\cp{K} \ker T = 2$, $\dim_\cp{K} \im T = 4$

    \item $V = \mathcal{P}_4(\real)$, $W = \cp{M}_{2\times 2}(\real)$, $\cp{K} = \real$, $\dim_\cp{K} \ker T = 3$, $\dim_\cp{K} \im T = 2$
  \end{enumerate}
\end{exercicio}

\begin{exercicio}
  Mostre que cada uma das transforma\c{c}\~oes lineares de $\real^3$ em $\real^3$ a seguir \'e invert{\'\i}vel e determine a transforma\c{c}\~ao linear inversa:
  \begin{enumerate}[label={\alph*})]
    \item $T(x,y,z) = (x - 3y - 2z, y - 4z, -z)$

    \item $T(x,y,z) = (x, x - y, 2x + y -z)$
  \end{enumerate}
\end{exercicio}

\begin{exercicio}
  Seja $\cp{K}$ um corpo e $T : \cp{K}^2 \to \cp{K}^2$ a transforma\c{c}\~ao linear dada por $T(x_1,x_2) = (x_1 + x_2, x_1)$ para todo $(x_1,x_2) \in \cp{K}^2$. Prove que $T$ \'e um isomorfismo e exiba $T^{-1}$.
\end{exercicio}

\begin{exercicio}
  Seja $T : \complex^3 \to \complex^3$ a transforma\c{c}\~ao linear definida por $T(1,0,0) = (1,0,i)$, $T(0,1,0) = (0,1,1)$ e $T(0,0,1) = (i,1,0)$. Decida se $T$ \'e invert{\'\i}vel.
\end{exercicio}

\newpage
\Closesolutionfile{ans}
\hrule
\begin{center}
  {\large\bf RESPOSTAS}
\end{center}
\hrule
\input{ans1}


\end{document}
