%!TEX program = xelatex
\def\numeromodulo{2}
\def\numerolista{2}
\documentclass[12pt]{exam}

\def\ano{2023}
\def\semestre{1}
\def\disciplina{Introdução à \'Algebra Linear}
\def\turma{3}

\usepackage{caption}
\usepackage{amssymb}
\usepackage{amsmath,amsfonts,amsthm,amstext}
\usepackage[brazil]{babel}
\usepackage{graphicx}
\graphicspath{{../Pictures/}}
\usepackage{enumitem}
\usepackage{multicol}
\usepackage{answers}
\usepackage[svgnames]{xcolor}
\usepackage{tikz}
\usepackage{ifthen}
\usetikzlibrary{lindenmayersystems}
\usetikzlibrary[shadings]

\Newassociation{solucao}{Solution}{ans}
\newtheorem{exercicio}{}

\setlength{\topmargin}{-1.0in}
\setlength{\oddsidemargin}{0in}
\setlength{\textheight}{10.1in}
\setlength{\textwidth}{6.5in}
\setlength{\baselineskip}{12mm}

\extraheadheight{0.7in}
\firstpageheadrule
\runningheadrule
\lhead{
        \begin{minipage}[c]{1.7cm}
        \includegraphics[width=1.7cm]{unb.pdf}
        \end{minipage}%
        \hspace{0pt}
        \begin{minipage}[c]{4in}
          {Universidade de Bras{\'\i}lia} --
          {Departamento de Matem{\'a}tica}
\end{minipage}
\vspace*{-0.8cm}
}
% \chead{Universidade de Bras{\'\i}lia - Departamento de Matem{\'a}tica}
% \rhead{}
% \vspace*{-2cm}

\extrafootheight{.5in}
\footrule
\lfoot{\disciplina\ - \semestre$^o$/\ano\ - Módulo \numeromodulo}
\cfoot{}
\rfoot{P\'agina \thepage\ de \numpages}

\newcounter{exercicios}
\renewcommand{\theexercicios}{\arabic{exercicios}}

\newenvironment{questao}[1]{
\refstepcounter{exercicios}
\ifx&#1&
\else
   \label{#1}
\fi
\noindent\textbf{Exerc{\'\i}cio {\theexercicios}:}
}

\newcommand{\resp}[1]{
\noindent{\bf Exerc{\'\i}cio #1: }}

\def\ano{2023}
\def\semestre{2}
\def\disciplina{Introdução à Álgebra Linear}
\def\nomeabreviado{IAL}
\def\turma{11}

\newcommand{\im}{{\rm Im\,}}
\newcommand{\dlim}[2]{\displaystyle\lim_{#1\rightarrow #2}}
\newcommand{\minf}{+\infty}
\newcommand{\ninf}{-\infty}
\newcommand{\cp}[1]{\mathbb{#1}}
\newcommand{\sub}{\subseteq}
\newcommand{\n}{\mathbb{N}}
\newcommand{\z}{\mathbb{Z}}
\newcommand{\rac}{\mathbb{Q}}
\newcommand{\real}{\mathbb{R}}
\newcommand{\complex}{\mathbb{C}}

\newcommand{\vesp}[1]{\vspace{ #1  cm}}

\newcommand{\compcent}[1]{\vcenter{\hbox{$#1\circ$}}}
\newcommand{\comp}{\mathbin{\mathchoice
        {\compcent\scriptstyle}{\compcent\scriptstyle}
        {\compcent\scriptscriptstyle}{\compcent\scriptscriptstyle}}}
\renewcommand{\sin}{{\rm sen\,}}
\renewcommand{\tan}{{\rm tg\,}}
\renewcommand{\csc}{{\rm cossec\,}}
\renewcommand{\cot}{{\rm cotg\,}}
\renewcommand{\sinh}{{\rm senh\,}}
\newcommand{\integer}{\mathbb{Z}}
\begin{document}

  \Opensolutionfile{ans}[ans1]
  \begin{center}
    {\Large\bf \disciplina\ - Turma \turma\ -- \semestre$^{o}$/\ano} \\ \vspace{9pt} {\large\bf
        $\numerolista^a$ Lista de Exerc{\'\i}cios -- Módulo \numeromodulo}\\ \vspace{9pt} Prof. Jos{\'e} Ant{\^o}nio O. Freitas
  \end{center}
  \hrule

\vesp{.6}



%\newpage
%\Closesolutionfile{ans}

Nos exerc{\'\i}cios \ref{subespacoinicio} \`a \ref{subespacofim}, verifique se $S$ \'e um subespa\c{c}o vetorial do espa\c{c}o vetorial $V$ sobre o corpo $\cp{K}$ em quest\~ao:
\begin{exercicio}\label{subespacoinicio}
	$V = \real^n$ e $S = \{(a_1, a_2, \dots, a_n) \in \real^n \mid a_1a_2 = 0\}$; $\cp{K} = \real$.
	\begin{solucao}
		N\~ao \'e subespa\c{c}o.
	\end{solucao}
\end{exercicio}

\begin{exercicio}
    $V = \real^3$ e $W = \left\{\begin{bmatrix}a\\0\\a\end{bmatrix} \mid a \in \real\right\}$
\end{exercicio}

\begin{exercicio}
    $V = \real^3$ e $W = \left\{\begin{bmatrix}a\\-a\\2a\end{bmatrix} \mid a \in \real\right\}$
\end{exercicio}

\begin{exercicio}
    $V = \real^3$ e $W = \left\{\begin{bmatrix}a\\b\\a + b + 1\end{bmatrix} \mid a \in \real\right\}$
\end{exercicio}

\begin{exercicio}
    $V = \mathcal{P}_2(\rac)$ e $W = \{a + bx + cx^2 \mid a + b + c = 0\}$
\end{exercicio}

\begin{exercicio}
    $V = \mathcal{P}(\real)$ e $W = \{a + bx + cx^2 + dx^3\mid a, b, c, d \in \real\}$
\end{exercicio}

\begin{exercicio}
	$V = \cp{M}_2(\complex)$ e $S = \left\{\begin{pmatrix} a_{11} & a_{12}\\ a_{21} & a_{22}\end{pmatrix} \in V \mid a_{ij} = \overline{a_{ji}}, i, j = 1, 2\right\}$; $\cp{K} = \complex$, onde $\overline{a + bi} = a - bi$.
	\begin{solucao}
		N\~ao \'e subespa\c{c}o.
	\end{solucao}
\end{exercicio}

\begin{exercicio}
	$V = \real^3$ e $S = \{(x_1, x_2, x_1x_2) \in \real^3 \mid x_1, x_2 \in \real\}$; $\cp{K} = \real$.
	\begin{solucao}
		N\~ao \'e subespa\c{c}o.
	\end{solucao}
\end{exercicio}

\begin{exercicio}
	$V = \real^2$ e $S = \{(x, y) \in \real^2 \mid x + 3y = 0\}$; $\cp{K} = \real$.
	\begin{solucao}
		\'E subespa\c{c}o.
	\end{solucao}
\end{exercicio}

\begin{exercicio}
	$V = \real^2$ e $S = \{(x, y) \in \real^2 \mid y = x + 1\}$; $\cp{K} = \real$.
	\begin{solucao}
		N\~ao \'e subespa\c{c}o.
	\end{solucao}
\end{exercicio}

\begin{exercicio}
	$V = \real^2$ e $S = \{(x, y) \in \real^2 \mid x \ge 0\}$; $\cp{K} = \real$.
	\begin{solucao}
		N\~ao \'e subespa\c{c}o.
	\end{solucao}
\end{exercicio}

\begin{exercicio}
	$V = \real^3$ e $S = \{(x, y, z) \in \real^3 \mid y = x + 2 \mbox{ e } z = 0\}$; $\cp{K} = \real$.
	\begin{solucao}
		N\~ao \'e subespa\c{c}o.
	\end{solucao}
\end{exercicio}

\begin{exercicio}
	$V = \real^3$ e $S = \{(x, y, z) \in \real^3 \mid x + 2y -3z = 4\}$; $\cp{K} = \real$.
	\begin{solucao}
		N\~ao \'e subespa\c{c}o.
	\end{solucao}
\end{exercicio}

\begin{exercicio}
	$V = \real^3$ e $S = \{(x, y, z) \in \real^3 \mid x = -z \mbox{ e } x = z\}$; $\cp{K} = \real$.
	\begin{solucao}
		\'E subespa\c{c}o.
	\end{solucao}
\end{exercicio}

\begin{exercicio}
	$V = \real^3$ e $S = \left\{(x, y, z) \in \real^3 \mid \dfrac{x}{2} = \dfrac{y - 3}{5}\right\}$; $\cp{K} = \real$.
	\begin{solucao}
		N\~ao \'e subespa\c{c}o.
	\end{solucao}
\end{exercicio}

\begin{exercicio}
	$V = \real^4$ e $S = \{(x_1, x_2, x_3, x_4) \in \real^4 \mid x_1 + x_2 = 0, x_3 - x_4 = 0 \in \real\}$; $\cp{K} = \real$.
	\begin{solucao}
		\'E subespa\c{c}o.
	\end{solucao}
\end{exercicio}

\begin{exercicio}
	$V = \real^4$ e $S = \{(x_1, x_2, x_3, x_4) \in \real^4 \mid 2x_1 + x_2 - x_4 = 0, x_3 = 0 \in \real\}$; $\cp{K} = \real$.
	\begin{solucao}
		\'E subespa\c{c}o.
	\end{solucao}
\end{exercicio}

\begin{exercicio}
	$V = \cp{M}_2(\complex)$ e $S = \left\{\begin{bmatrix} a & b\\ c & d\end{bmatrix} \in V \mid b = c\right\}$; $\cp{K} = \complex$.
	\begin{solucao}
		\'E subespa\c{c}o.
	\end{solucao}
\end{exercicio}

\begin{exercicio}
	$V = \cp{M}_2(\real)$ e $S = \left\{\begin{bmatrix} a & b\\ c & d\end{bmatrix} \in V \mid c = a + b \mbox{ e } d = 0\right\}$; $\cp{K} = \real$.
	\begin{solucao}
		\'E subespa\c{c}o.
	\end{solucao}
\end{exercicio}

\begin{exercicio}
	$V = \cp{M}_2(\real)$ e $S = \left\{\begin{bmatrix} a & b\\ 0 & c\end{bmatrix} \in V \mid a, b, c \in \real\right\}$; $\cp{K} = \real$.
	\begin{solucao}
		\'E subespa\c{c}o.
	\end{solucao}
\end{exercicio}

\begin{exercicio}
	$V = \cp{M}_2(\real)$ e $S = \left\{\begin{bmatrix} a & b\\ b & c\end{bmatrix} \in V \mid a, b, c \in \real\right\}$; $\cp{K} = \real$.
	\begin{solucao}
		\'E subespa\c{c}o.
	\end{solucao}
\end{exercicio}

\begin{exercicio}\label{subespacofim}
	$V = \cp{M}_2(\real)$ e $S = \left\{\begin{bmatrix} a & a+b\\ a-b & b\end{bmatrix} \in V \mid a, b \in \real \right\}$; $\cp{K} = \real$.
	\begin{solucao}
		\'E subespa\c{c}o.
	\end{solucao}
\end{exercicio}

\begin{exercicio}
	Sejam $W_1$ e $W_2$  subespa\c{c}os de um $\cp{K}$-espa\c{c}o vetorial $V$. Defina
	\[
	W_1 - W_2 = \{ u_1 - u_2 \mid u_1 \in W_1, u_2 \in W_2\}.
	\]
	O conjunto $W_1 - W_2$ \'e um subespa\c{c}o vetorial de $V$?
\end{exercicio}

\begin{exercicio}
	Seja $W = \left\{\begin{pmatrix} a_{11} & a_{12}\\ a_{21} & a_{22}\end{pmatrix} \in \cp{M}_2(\complex) \mid a_{11} + a_{12} = 0\right\}$.
	\begin{enumerate}[label={\alph*})]
		\item Mostre que $W$ \'e um espa\c{c}o vetorial sobre $\real$.
		\item Determine uma base de $W$.
		\item Seja $W_1 = \{(a_{ij})_{i,j}, \in \cp{M}_2(\complex), i, j = 1, 2 \mid a_{21} = -\overline{a_{12}}\}$, onde $\overline{a + bi} = a - bi$. Prove que $W_1$ \'e um subespa\c{c}o de $\cp{M}_2(\complex)$ sobre $\real$ e ache uma base de $W_1$.
	\end{enumerate}
\end{exercicio}

\begin{exercicio}
	Considere o subespa\c{c}o $S = [(1,1,-2,4),(1,1,-1,2),(1,4,-4,8)]$ de $\real^4$.
	\begin{enumerate}[label={\alph*})]
		\item O vetor $(2/3, 1, -1, 2)$ pertence a $S$?
		\item O vetor $(0, 0, 1, 1)$ pertence a $S$?
	\end{enumerate}
	\begin{solucao}
		\begin{enumerate}[label={\alph*})]
			\item Sim.
			\item N\~ao.
		\end{enumerate}
	\end{solucao}
\end{exercicio}

\begin{exercicio}
	Sejam $u = (2, 0 , -1)$, $v = (3, 1 , 0)$ e $w = (1, -1 , c)$ onde $c \in \real$. Para qual(is) valor(es) de $c \in \real$ o conjunto $\{u, v, w\}$ \'e uma base de $\real^3$?
\end{exercicio}

\begin{exercicio}
	Sejam $u = (1, -1 , 3)$, $v = (1, 0 , 1)$ e $w = (1, 2 , c)$ onde $c \in \real$. Para qual(is) valor(es) de $c \in \real$ o conjunto $\{u, v, w\}$ \'e uma base de $\real^3$?
\end{exercicio}

\begin{exercicio}
	Seja $W$ o subespa\c{c}o de $M_2(\real)$ definido por
	\[
	W = \left\{\begin{bmatrix}2a & a + 2b\\0 & a - b\end{bmatrix} \mid a, b \in \real\right\}.
	\]
	\begin{enumerate}[label={\alph*})]
		\item $\begin{bmatrix}0 & -2\\0 & 1\end{bmatrix} \in W$?
		\item $\begin{bmatrix}0 & 2\\3 & 1\end{bmatrix} \in W$?
	\end{enumerate}
	\begin{solucao}
		\begin{enumerate}[label={\alph*})]
			\item Sim.
			\item N\~ao.
		\end{enumerate}
	\end{solucao}
\end{exercicio}

\begin{exercicio}
	Seja $V$ o espa\c{c}o das matrizes $2 \times 2$ sobre $\real$, e seja $W$ o subespa\c{c}o gerado por
	\[
	\begin{bmatrix}
		1 & -5\\
		-4 & 2
	\end{bmatrix},
	\begin{bmatrix}
		1 & 1\\
		-1 & 5
	\end{bmatrix},
	\begin{bmatrix}
		2 & -4\\
		-5 & 7
	\end{bmatrix},
	\begin{bmatrix}
		1 & -7\\
		-5 & 1
	\end{bmatrix}.
	\]
	Encontre uma base e a dimens\~ao de $W$.
\end{exercicio}

\begin{exercicio}
	Considere o subespa\c{c}o de $\real^4$ gerado pelos vetores $v_1 = (1, -1, 0, 0)$, $v_2 = (0, 0, 1, 1)$, $v_3 = (2, -2, 1, 1)$ e $v_4 = (1, 0, 0, 0)$.
	\begin{enumerate}[label={\alph*})]
		\item O vetor $(2, -3, 2, 2) \in [v_1, v_2, v_3, v_4]$? Justifique.
		\item Exiba uma base para $[v_1, v_2, v_3, v_4]$. Qual \'e a dimens\~ao?
		\item $[v_1, v_2, v_3, v_4] = \real^4$? Por qu\^e?
	\end{enumerate}
	\begin{solucao}
		\begin{enumerate}[label={\alph*})]
			\item Sim.
			\item $\dim_\real [v_1, v_2, v_3, v_4] = 3$.
			\item N\~ao.
		\end{enumerate}
	\end{solucao}
\end{exercicio}

\begin{exercicio}
	Para quais valores de $\alpha \in \real$ vale
	\[
	[(1, 0, \alpha), (1, 2, -3) , (\alpha, 1, 0)] = \real^3?
	\]
\end{exercicio}

\begin{exercicio}
	Encontre uma base de $\real^4$ que contenha os vetores $(1,2,-2,1)$ e $(1,0,-2,2)$.
\end{exercicio}

\begin{exercicio}
	Considere o subespa\c{c}o vetorial $W$ de $\mathcal{P}_4(\real)$ gerado pelo conjunto
	\[
	\mathcal{A} = \{1+2x+x^2+3x^3+x^4, 1-2x-2x^2-2x^3-3x^4,2-x^2+x^3-2x^4,x-x^3+x^4,3x^2+6x^3+3x^4\}.
	\]
	Determine uma base de $\mathcal{B}$ de $W$ que esteja contida em $\mathcal{A}$.
\end{exercicio}

%\hrule
%\begin{center}
%{\large\bf RESPOSTAS}
%\end{center}
%\hrule
%\input{ans1}

\end{document}