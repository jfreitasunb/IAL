%!TEX program = xelatex

\def\numeromodulo{1}
\def\numerolista{1}
\documentclass[12pt]{exam}

\def\ano{2023}
\def\semestre{1}
\def\disciplina{Introdução à \'Algebra Linear}
\def\turma{3}

\usepackage{caption}
\usepackage{amssymb}
\usepackage{amsmath,amsfonts,amsthm,amstext}
\usepackage[brazil]{babel}
\usepackage{graphicx}
\graphicspath{{../Pictures/}}
\usepackage{enumitem}
\usepackage{multicol}
\usepackage{answers}
\usepackage[svgnames]{xcolor}
\usepackage{tikz}
\usepackage{ifthen}
\usetikzlibrary{lindenmayersystems}
\usetikzlibrary[shadings]

\Newassociation{solucao}{Solution}{ans}
\newtheorem{exercicio}{}

\setlength{\topmargin}{-1.0in}
\setlength{\oddsidemargin}{0in}
\setlength{\textheight}{10.1in}
\setlength{\textwidth}{6.5in}
\setlength{\baselineskip}{12mm}

\extraheadheight{0.7in}
\firstpageheadrule
\runningheadrule
\lhead{
        \begin{minipage}[c]{1.7cm}
        \includegraphics[width=1.7cm]{unb.pdf}
        \end{minipage}%
        \hspace{0pt}
        \begin{minipage}[c]{4in}
          {Universidade de Bras{\'\i}lia} --
          {Departamento de Matem{\'a}tica}
\end{minipage}
\vspace*{-0.8cm}
}
% \chead{Universidade de Bras{\'\i}lia - Departamento de Matem{\'a}tica}
% \rhead{}
% \vspace*{-2cm}

\extrafootheight{.5in}
\footrule
\lfoot{\disciplina\ - \semestre$^o$/\ano\ - Módulo \numeromodulo}
\cfoot{}
\rfoot{P\'agina \thepage\ de \numpages}

\newcounter{exercicios}
\renewcommand{\theexercicios}{\arabic{exercicios}}

\newenvironment{questao}[1]{
\refstepcounter{exercicios}
\ifx&#1&
\else
   \label{#1}
\fi
\noindent\textbf{Exerc{\'\i}cio {\theexercicios}:}
}

\newcommand{\resp}[1]{
\noindent{\bf Exerc{\'\i}cio #1: }}

\def\ano{2023}
\def\semestre{2}
\def\disciplina{Introdução à Álgebra Linear}
\def\nomeabreviado{IAL}
\def\turma{11}

\newcommand{\im}{{\rm Im\,}}
\newcommand{\dlim}[2]{\displaystyle\lim_{#1\rightarrow #2}}
\newcommand{\minf}{+\infty}
\newcommand{\ninf}{-\infty}
\newcommand{\cp}[1]{\mathbb{#1}}
\newcommand{\sub}{\subseteq}
\newcommand{\n}{\mathbb{N}}
\newcommand{\z}{\mathbb{Z}}
\newcommand{\rac}{\mathbb{Q}}
\newcommand{\real}{\mathbb{R}}
\newcommand{\complex}{\mathbb{C}}

\newcommand{\vesp}[1]{\vspace{ #1  cm}}

\newcommand{\compcent}[1]{\vcenter{\hbox{$#1\circ$}}}
\newcommand{\comp}{\mathbin{\mathchoice
        {\compcent\scriptstyle}{\compcent\scriptstyle}
        {\compcent\scriptscriptstyle}{\compcent\scriptscriptstyle}}}
\renewcommand{\sin}{{\rm sen\,}}
\renewcommand{\tan}{{\rm tg\,}}
\renewcommand{\csc}{{\rm cossec\,}}
\renewcommand{\cot}{{\rm cotg\,}}
\renewcommand{\sinh}{{\rm senh\,}}

\begin{document}
    \begin{center}
    {\Large\bf \disciplina\ - Turma \turma\ -- \semestre$^{o}$/\ano} \\ \vspace{9pt} {\large\bf
        $\numerolista^a$ Lista de Exerc{\'\i}cios -- Módulo \numeromodulo}\\ \vspace{9pt} Prof. Jos{\'e} Ant{\^o}nio O. Freitas
    \end{center}
    \hrule

    \vspace{.6cm}

    \questao{}\label{operacoes_matrizes} Sejam
    \begin{align*}
        A = \begin{pmatrix}1 & 2 & 3\\2 & 1 & -1\end{pmatrix},
        B = \begin{pmatrix}-2 & 0 & 1\\3 & 0 & 1\end{pmatrix},
        C = \begin{pmatrix}-1\\ 2\\4\end{pmatrix},
        D = \begin{pmatrix}-2\\ 3\\2\end{pmatrix},
        E = \begin{pmatrix}2 & -1\end{pmatrix},
        F = \begin{pmatrix}4 & 3\end{pmatrix},
    \end{align*}.
    Encontre:
    \begin{multicols}{3}
        \begin{enumerate}[label={\arabic*})]
            \item $A + B$
            \item $3B$
            \item $2A - B$
            \item $3(A - \dfrac{1}{2}B) + C$
            \item $A + 2B$
            \item $2C + D$
            \item $\dfrac{2}{3}D - C$
            \item $3E + 4F$
            \item $E - F$
            \item $F - 2E$
            \item $AC$
            \item $BC$
            \item $CD$
            \item $CB$
            \item $DA$
            \item $DB$
            \item $-A$
            \item $-2F$
        \end{enumerate}
    \end{multicols}
    
    \vspace{.3cm}

    \questao{} Nos itens abaixo são dadas duas matrizes $A$ e $B$ e dois números $c$ e $d$. Calcule a matriz $cA + dB$.
    \begin{enumerate}
        \item $A = \begin{pmatrix}3 & -5\\2 & 7\end{pmatrix}$, $B = \begin{pmatrix}-1 & 0\\3 & -4\end{pmatrix}$, $c = 3$, $d = 4$
        \item $A = \begin{pmatrix}2 & 0 & -3\\-1 & 5 & 6\end{pmatrix}$, $B = \begin{pmatrix}-2 & 3 & 1\\7 & 1 & 5\end{pmatrix}$, $c = 5$, $d = -3$
        \item $A = \begin{pmatrix}5 & 0\\0 & 7\\3 & -1\end{pmatrix}$, $B = \begin{pmatrix}-4 & 5\\3 & 2\\7 & 4\end{pmatrix}$, $c = -2$, $d = 4$
        \item $A = \begin{pmatrix}2 & -1 & 0\\4 & 0 & -3\\5 & -2 & 7\end{pmatrix}$, $B = \begin{pmatrix}6 & -3 & -4\\5 & 2 & -1\\0 & 7 & 9\end{pmatrix}$, $c = 7$, $d = 5$
    \end{enumerate}

    \vspace{.3cm}

    \questao{}\label{igualdade_de_matrizes} Nos itens abaixo encontre os valores de $a$, $b$, $c$ e $d$ de modo que as igualdades sejam verdadeiras:
    \begin{enumerate}
        \item $\begin{bmatrix}a & 3\\-1 & a + b\end{bmatrix} = \begin{bmatrix}4 & d - 2c\\d - 2c & -2\end{bmatrix}$
        \item $\begin{bmatrix}a - b & b + a\\3d + c & 2d - c\end{bmatrix} = \begin{bmatrix}8 & 1\\7 & 6\end{bmatrix}$.
    \end{enumerate}

    \vspace{.3cm}

    \questao{} Suponha que $A$, $B$, $C$, $D$ e $E$ sejam matrizes tais que:
    \begin{align*}
        A &= (a_{ij})_{4 \times 5}
        B = (b_{ij})_{4 \times 5}
        C = (c_{ij})_{5 \times 2}\\
        D &= (d_{ij})_{4 \times 2}
        E = (e_{ij})_{5 \times 4}.
    \end{align*}
    Determine quais das operações a seguir estão definidas. Para aquelas que estiverem definidas, determine a ordem da matriz resultante. Quando não estiver definida, explique por que não está definida.
    \begin{multicols}{2}
        \begin{enumerate}[label={\arabic*})]
            \item $AB$

            \item $BC$

            \item $AC + D$

            \item $AE + B$

            \item $AB + B$

            \item $E(A + B)$

            \item $E(AC)$

        \end{enumerate}
    \end{multicols}

    \vspace{.3cm}

    \questao{} No Exercício \ref{operacoes_matrizes} encontre $A^t$, $B^t$, $C^t$, $D^t$, $E^t$ e $F^t$.

    \vspace{.3cm}

    \questao{} Encontre a transposta de todas as matrizes que aparecem no Exercício \ref{igualdade_de_matrizes}.

\end{document}
