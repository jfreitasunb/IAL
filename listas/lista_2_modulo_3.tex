%!TEX program = xelatex
\def\numeromodulo{3}
\def\numerolista{2}
\documentclass[12pt]{exam}

\def\ano{2023}
\def\semestre{1}
\def\disciplina{Introdução à \'Algebra Linear}
\def\turma{3}

\usepackage{caption}
\usepackage{amssymb}
\usepackage{amsmath,amsfonts,amsthm,amstext}
\usepackage[brazil]{babel}
\usepackage{graphicx}
\graphicspath{{../Pictures/}}
\usepackage{enumitem}
\usepackage{multicol}
\usepackage{answers}
\usepackage[svgnames]{xcolor}
\usepackage{tikz}
\usepackage{ifthen}
\usetikzlibrary{lindenmayersystems}
\usetikzlibrary[shadings]

\Newassociation{solucao}{Solution}{ans}
\newtheorem{exercicio}{}

\setlength{\topmargin}{-1.0in}
\setlength{\oddsidemargin}{0in}
\setlength{\textheight}{10.1in}
\setlength{\textwidth}{6.5in}
\setlength{\baselineskip}{12mm}

\extraheadheight{0.7in}
\firstpageheadrule
\runningheadrule
\lhead{
        \begin{minipage}[c]{1.7cm}
        \includegraphics[width=1.7cm]{unb.pdf}
        \end{minipage}%
        \hspace{0pt}
        \begin{minipage}[c]{4in}
          {Universidade de Bras{\'\i}lia} --
          {Departamento de Matem{\'a}tica}
\end{minipage}
\vspace*{-0.8cm}
}
% \chead{Universidade de Bras{\'\i}lia - Departamento de Matem{\'a}tica}
% \rhead{}
% \vspace*{-2cm}

\extrafootheight{.5in}
\footrule
\lfoot{\disciplina\ - \semestre$^o$/\ano\ - Módulo \numeromodulo}
\cfoot{}
\rfoot{P\'agina \thepage\ de \numpages}

\newcounter{exercicios}
\renewcommand{\theexercicios}{\arabic{exercicios}}

\newenvironment{questao}[1]{
\refstepcounter{exercicios}
\ifx&#1&
\else
   \label{#1}
\fi
\noindent\textbf{Exerc{\'\i}cio {\theexercicios}:}
}

\newcommand{\resp}[1]{
\noindent{\bf Exerc{\'\i}cio #1: }}

\def\ano{2023}
\def\semestre{2}
\def\disciplina{Introdução à Álgebra Linear}
\def\nomeabreviado{IAL}
\def\turma{11}

\newcommand{\im}{{\rm Im\,}}
\newcommand{\dlim}[2]{\displaystyle\lim_{#1\rightarrow #2}}
\newcommand{\minf}{+\infty}
\newcommand{\ninf}{-\infty}
\newcommand{\cp}[1]{\mathbb{#1}}
\newcommand{\sub}{\subseteq}
\newcommand{\n}{\mathbb{N}}
\newcommand{\z}{\mathbb{Z}}
\newcommand{\rac}{\mathbb{Q}}
\newcommand{\real}{\mathbb{R}}
\newcommand{\complex}{\mathbb{C}}

\newcommand{\vesp}[1]{\vspace{ #1  cm}}

\newcommand{\compcent}[1]{\vcenter{\hbox{$#1\circ$}}}
\newcommand{\comp}{\mathbin{\mathchoice
        {\compcent\scriptstyle}{\compcent\scriptstyle}
        {\compcent\scriptscriptstyle}{\compcent\scriptscriptstyle}}}
\renewcommand{\sin}{{\rm sen\,}}
\renewcommand{\tan}{{\rm tg\,}}
\renewcommand{\csc}{{\rm cossec\,}}
\renewcommand{\cot}{{\rm cotg\,}}
\renewcommand{\sinh}{{\rm senh\,}}
\newcommand{\integer}{\mathbb{Z}}
\begin{document}

  \Opensolutionfile{ans}[ans1]
  \begin{center}
    {\Large\bf \disciplina\ - Turma \turma\ -- \semestre$^{o}$/\ano} \\ \vspace{9pt} {\large\bf
        $\numerolista^a$ Lista de Exercícios -- Módulo \numeromodulo}\\ \vspace{9pt} Prof. José Antônio O. Freitas
  \end{center}
  \hrule

\vesp{.6}

\begin{exercicio}
    Sejam $\mathcal{B} = \{(1,0);(0,1)\}$, $\mathcal{B}_1 = \{(-1,1);(1,1)\}$, $\mathcal{B}_2 = \{(\sqrt{3},-1);(\sqrt{3},1)\}$ e $\mathcal{B}_3 = \{(2,0);(0,2)\}$ bases ordenadas de $\real^2$.
    \begin{enumerate}[label={\alph*})]
        \item Quais são as coordenadas do vetor $v = (3,-2)$ em relação à base:
        \begin{enumerate}[label=({\roman*})]
            \item $\mathcal{B}$

            \item $\mathcal{B}_1$

            \item $\mathcal{B}_2$

            \item $\mathcal{B}_3$
        \end{enumerate}

        \item Ache a matriz de mudança de base nos seguintes casos:
        \begin{enumerate}[label={\roman*})]
            \item $[I]_{\mathcal{B},\mathcal{B}_1}$

            \item $[I]_{\mathcal{B}_1,\mathcal{B}}$

            \item $[I]_{\mathcal{B}_2,\mathcal{B}}$

            \item $[I]_{\mathcal{B}_3,\mathcal{B}}$
        \end{enumerate}

        \item As coordenadas de um vetor $v \in \real^2$ em relação à base $\mathcal{B}_1$ são dadas por
        \[
        [v]_{\mathcal{B}_1} = \begin{bmatrix}
            4\\0
        \end{bmatrix}.
        \]
        Quais as coordenadas de $v$ em relação à base:
        \begin{enumerate}[label={\roman*})]
            \item $\mathcal{B}$

            \item $\mathcal{B}_2$

            \item $\mathcal{B}_3$
        \end{enumerate}
    \end{enumerate}
\end{exercicio}

\begin{exercicio}
    Sejam $\mathcal{B} = \{(1,0,0);(0,1,0);(0,0,1)\}$, $\mathcal{B}_1 = \{(1,3,-2);(2,1,3);(3,6,0)\}$, \linebreak $\mathcal{B}_2 = \{(1,1,1);(0,1,0);(0,0,1)\}$ e $\mathcal{B}_3 = \{(1,1,1);(0,-1,1);(0,0,1)\}$ bases ordenadas de $\real^2$.
    \begin{enumerate}[label={\alph*})]
        \item Quais são as coordenadas do vetor $v = (3,-2,1)$ em relação à base:
        \begin{enumerate}[label={\roman*})]
            \item $\mathcal{B}$

            \item $\mathcal{B}_1$

            \item $\mathcal{B}_2$

            \item $\mathcal{B}_3$
        \end{enumerate}
        \item Ache a matriz de mudança de base nos seguintes casos:
        \begin{enumerate}[label={\roman*})]
            \item $[I]_{\mathcal{B},\mathcal{B}_1}$

            \item $[I]_{\mathcal{B}_1,\mathcal{B}}$

            \item $[I]_{\mathcal{B}_2,\mathcal{B}}$

            \item $[I]_{\mathcal{B}_3,\mathcal{B}}$
        \end{enumerate}
        \item As coordenadas de um vetor $v \in \real^3$ em relação à base $\mathcal{B}_1$ são dadas por
        \[
        [v]_{\mathcal{B}_1} = \begin{bmatrix}
            \phantom{-}4\\\phantom{-}0\\-1
        \end{bmatrix}.
        \]
        Quais as coordenadas de $v$ em relação à base:
        \begin{enumerate}[label={\roman*})]
            \item $\mathcal{B}$

            \item $\mathcal{B}_2$

            \item $\mathcal{B}_3$
        \end{enumerate}
    \end{enumerate}
\end{exercicio}

\begin{exercicio}
    Se
    \[
    [I]_{\mathcal{B}_1,\mathcal{B}_2} = \begin{bmatrix}
        1 & \phantom{-}1 & 1\\
        0 & -1 & 1\\
        1 & \phantom{-}0 & 1
    \end{bmatrix}
    \]
    encontre:
    \begin{enumerate}[label={\alph*})]
        \item $[v]_{\mathcal{B}_1}$ onde $[v]_{\mathcal{B}_2} = \begin{bmatrix}
            -1\\\phantom{-}2\\\phantom{-}3
        \end{bmatrix}$

        \item $[v]_{\mathcal{B}_2}$ onde $[v]_{\mathcal{B}_1} = \begin{bmatrix}
            -1\\\phantom{-}2\\\phantom{-}3
        \end{bmatrix}$
    \end{enumerate}
\end{exercicio}

    Nos exercícios \eqref{iniciomatriztransformacao} à \eqref{fimmatriztransformacao}, encontre a matriz da transformação linear dada.

\begin{exercicio}\label{iniciomatriztransformacao}
    $F : \mathcal{P}_2(\real) \to \mathcal{P}_2(\real)$ dada por $F(p(t)) = t^2p''(t)$. Considere $\mathcal{P}_2(\real)$ um $\real$-espaço vetorial e $\mathcal{B} = \{u_1 = 1, u_2 = t, u_3 = t^2\}$ uma base ordenada.

    \begin{solucao}
        $[F]_\mathcal{B} = \begin{bmatrix}
                0 & 0 & 0\\
                0 & 0 & 0\\
                0 & 0 & 2
            \end{bmatrix}$
      \end{solucao}
\end{exercicio}

\begin{exercicio}
    $D : \mathcal{P}_2(\complex) \to \mathcal{P}_1(\complex)$ dada por
    \[
        D(a_0 + a_1x + a_2x^2) = a_1 + 2a_2x.
    \]
    Considere $\mathcal{P}_2(\complex)$ e $\mathcal{P}_1(\complex)$ espaços vetoriais sobre $\complex$, $\mathcal{B}_1 = \{1, x, x^2\}$ e $\mathcal{B}_2 = \{1, 2 - x\}$ bases ordenadas de $\mathcal{P}_2(\complex)$ e $\mathcal{P}_1(\complex)$, respectivamente.

    \begin{solucao}
        $[D]_{\mathcal{B}_1, \mathcal{B}_2} = \begin{bmatrix}
            0 & 1 & \phantom{-}4\\
            0 & 0 & -2
        \end{bmatrix}$
    \end{solucao}
\end{exercicio}

\begin{exercicio}
    $D : \mathcal{P}_3(\real) \to \mathcal{P}_5(\real)$ dada por
    \[
    D(a_0 + a_1x + a_2x^2 + a_3x^3) = a_1 + 2a_2x + 3a_3x^2
    \]
    Considere $\mathcal{P}_3(\real)$ e $\mathcal{P}_5(\real)$ espaços vetoriais sobre $\real$, os conjuntos $\mathcal{B}_1 = \{1, x, x^2, x^3\}$ e $\mathcal{B}_2 = \{1, x, x^2, x^3, x^4, x^5\}$ de $\mathcal{P}_3(\real)$ e $\mathcal{P}_5(\real)$, respectivamente.

    \begin{solucao}
        $[D]_{\mathcal{B}_1, \mathcal{B}_2} = \begin{bmatrix}
            0 & 1 & 0 & 0\\
            0 & 0 & 2 & 0\\
            0 & 0 & 0 & 3\\
            0 & 0 & 0 & 0\\
            0 & 0 & 0 & 0\\
            0 & 0 & 0 & 0
        \end{bmatrix}$
    \end{solucao}
\end{exercicio}

\begin{exercicio}
    $G : \cp{M}_2(\real) \to \cp{M}_2(\real)$ dada por $G(X) = MX + X$ onde
    \[
    M = \begin{pmatrix}
        1 & 0\\
        0 & 0
    \end{pmatrix}.
    \]
    Considere $\cp{M}_2(\real)$ como um $\real$-espaço vetorial e
    \[
    \mathcal{B} = \left\{E_{11} = \begin{bmatrix}
        1 & 0\\0 & 0
    \end{bmatrix}, E_{12} = \begin{bmatrix}
        0 & 1\\0 & 0
    \end{bmatrix}, E_{21} = \begin{bmatrix}
        0 & 0\\1 & 0
    \end{bmatrix}, E_{22} = \begin{bmatrix}
        0 & 0\\0 & 1
    \end{bmatrix}\right\}
    \]
    uma base ordenada.

    \begin{solucao}
        $[G]_\mathcal{B} = \begin{bmatrix}
            2 & 0 & 1 & 1\\
            0 & 2 & 0 & 0\\
            0 & 0 & 0 & 0\\
            0 & 0 & 0 & 0
        \end{bmatrix}$
    \end{solucao}
\end{exercicio}

\begin{exercicio}
    $H : \cp{M}_2(\real) \to \cp{M}_2(\real)$ dada por $H(X) = MX - XM$ onde
    \[
    M = \begin{bmatrix}
        1 & 2\\
        0 & 1
    \end{bmatrix}.
    \]
    Considere $\cp{M}_2(\real)$ um $\real$-espaço vetorial e
    \[
    \mathcal{B} = \left\{E_{11} = \begin{bmatrix}
        1 & 0\\0 & 0
    \end{bmatrix}, E_{12} = \begin{bmatrix}
        0 & 1\\0 & 0
    \end{bmatrix}, E_{21} = \begin{bmatrix}
        0 & 0\\1 & 0
    \end{bmatrix}, E_{22} = \begin{bmatrix}
        0 & 0\\0 & 1
    \end{bmatrix}\right\}
    \]
    uma base ordenada.

    \begin{solucao}
        $[H]_\mathcal{B} = \begin{bmatrix}
            \phantom{-}0 & 0 & \phantom{-}2 & 0\\
            -2 & 0 & \phantom{-}0 & 2\\
            \phantom{-}0 & 0 & \phantom{-}0 & 0\\
            \phantom{-}0 & 0 & -2 & 0
        \end{bmatrix}$
    \end{solucao}
\end{exercicio}

\begin{exercicio}
    $T : \complex^3 \to \complex$ dada por $T(x,y,z) = x + 2y + iz$. Considere $\complex^3$ e $\complex$ como $\complex$-espaços vetoriais, $\mathcal{B}_1 = \{(1,0,0); (0,1,0); (0,0,1)\}$ e $\mathcal{B}_2 = \{i\}$ bases ordenadas de $\complex^2$ e $\complex$, respectivamente.

    \begin{solucao}
        $[T]_{\mathcal{B}_1, \mathcal{B}_2} = \begin{bmatrix}
            -i & -2i & 1
        \end{bmatrix}$
    \end{solucao}
\end{exercicio}

\begin{exercicio}
    $F : \mathcal{P}_3(\real) \to \mathcal{P}_4(\real)$ dada por $F(p(x)) = xp(x + 1)$. Considere $\mathcal{P}_3(\real)$ e $\mathcal{P}_4(\real)$ como $\real$-espaços vetoriais, $\mathcal{B}_1 = \{1, x, x^2, x^3\}$ e $\mathcal{B}_2 = \{1, x, x^2, x^3, x^4\}$ bases ordenadas de $\mathcal{P}_3(\real)$ e $\mathcal{P}_4(\real)$, respectivamente.

    \begin{solucao}
        $[F]_{\mathcal{B}_1, \mathcal{B}_2} = \begin{bmatrix}
            0 & 0 & 0 & 0\\
            1 & 1 & 1 & 1\\
            0 & 1 & 2 & 3\\
            0 & 0 & 1 & 3\\
            0 & 0 & 0 & 1
        \end{bmatrix}$
    \end{solucao}
\end{exercicio}

\begin{exercicio}
    $G : \real^2 \to \mathcal{P}_2(\real)$ dada por $G(a,b) = ax^2 + bx + (a + b)$. Considere $\real^2$ e $\mathcal{P}_2(\real)$ como $\real$-espaços vetoriais, $\mathcal{B}_1 = \{(1,3); (2,5)\}$ e $\mathcal{B}_2 = \{2, 1 + x, x^2\}$ bases ordenadas de $\real^2$ e $\mathcal{P}_2(\real)$, respectivamente.

    \begin{solucao}
        $[G]_{\mathcal{B}_1, \mathcal{B}_2} = \begin{bmatrix}
            1/2 & 1\\
            3 & 5\\
            1 & 2
        \end{bmatrix}$
    \end{solucao}
\end{exercicio}

\begin{exercicio}
    $T : \complex \to \cp{M}_2(\real)$ dada por
    \[
    T(x + yi) = \begin{bmatrix}
        x + 7y & 5y\\
        -10y & x - 7y
    \end{bmatrix}.
    \]
    Considere $\cp{M}_2(\real)$ e $\complex$ como espaços vetoriais sobre $\real$,  $\mathcal{B}_1 = \{1, i\}$ e
    \[
    \mathcal{B}_2 = \left\{E_{11} = \begin{bmatrix}
        1 & 0\\0 & 0
    \end{bmatrix}, E_{12} = \begin{bmatrix}
        0 & 1\\0 & 0
    \end{bmatrix}, E_{21} = \begin{bmatrix}
        0 & 0\\1 & 0
    \end{bmatrix}, E_{22} = \begin{bmatrix}
        0 & 0\\0 & 1
    \end{bmatrix}\right\}.
    \]
    bases ordenadas de $\complex$ e $\cp{M}_2(\real)$, respectivamente.

    \begin{solucao}
        $[T]_{\mathcal{B}_1, \mathcal{B}_2} = \begin{bmatrix}
            1 & \phantom{-}7\\
            1 & \phantom{-}5\\
            0 & -10\\
            0 & -7
        \end{bmatrix}$
    \end{solucao}
\end{exercicio}

\begin{exercicio}\label{fimmatriztransformacao}
    $S : \mathcal{P}_3(\real) \to M_2(\real)$ dada por
    \[
    S(a + bx + cx^2 + dx^3) = \begin{bmatrix}
        3a + 7b - 2c - 5d & 8a + 14b - 2c - 11d\\
        -4a - 8b + 2c + 6d & 12a + 22b - 4c - 17d
    \end{bmatrix}.
    \]
    Considere $\mathcal{P}_3(\real)$ e $M_2(\real)$ como $\real$-espaços vetoriais,
    \begin{align*}
        \mathcal{B}_1 &= \{1 + 2x + x^2 - x^3, 1 + 3x + x^2 + x^3, -1 - 2x + 2x^3, 2 + 3x + 2x^2 - 5x^3\}\\
        \mathcal{B}_2 &= \left\{\begin{bmatrix}
            1 & 1\\1 & 2
        \end{bmatrix}, \begin{bmatrix}
            2 & 3\\2 & 5
        \end{bmatrix}, \begin{bmatrix}
            -1 & -1\\\phantom{-}0 & -2
        \end{bmatrix}, \begin{bmatrix}
            -1 & -4\\-2 & -4
        \end{bmatrix}\right\}
    \end{align*}
    bases ordenadas de $\mathcal{P}_3(\real)$ e $M_2(\real)$, respectivamente.

    \begin{solucao}
        $[S]_{\mathcal{B}_1, \mathcal{B}_2} = \begin{bmatrix}
            -172& -72 & \phantom{-}106 & -220\\
            \phantom{-}119 & \phantom{-}29 & -46 & \phantom{-}91\\
            \phantom{-1}1 & -34 & \phantom{-}46 & -96\\
            \phantom{-1}45 & \phantom{-1}3 & -5 & \phantom{-}10
        \end{bmatrix}$
    \end{solucao}
\end{exercicio}

\begin{exercicio}
    Sejam $v_1 = (1,3)$, $v_2 = (-1, 4)$ e
    \[
        A = \begin{bmatrix}
            \phantom{-}1 & 3\\
            -2 & 5
        \end{bmatrix}
    \]
    a matriz de $T : \real^2 \to \real^2$ em relação à base $\mathcal{B} = \{v_1, v_2\}$.
    \begin{enumerate}[label={\alph*})]
        \item Encontre $[T(v_1)]_\mathcal{B}$ e $[T(v_2)]_\mathcal{B}$.

        \item Encontre $T(v_1)$ e $T(v_2)$.

        \item Encontre uma fórmula para $T(x_1, x_2)$.

        \item Use a fórmula do item anterior para calcular $T(1, 1)$.
    \end{enumerate}

    \begin{solucao}
        \begin{enumerate}[label={\alph*})]
            \item $[T(v_1)]_\mathcal{B} = \begin{bmatrix} 1\\2\end{bmatrix}$, $[T(v_2)]_\mathcal{B} = \begin{bmatrix} 3\\5\end{bmatrix}$

            \item $T(v_1) = (3, -5)$, $T(v_2) = (-2, 29)$

            \item $T(x_1, x_2) = \left(\dfrac{18x_1 + x_2}{7}, \dfrac{-107x_1 + 24x_2}{7}\right)$

            \item $T(1, 1) = (19/7, -83/7)$
        \end{enumerate}
    \end{solucao}
\end{exercicio}

\begin{exercicio}
    Seja
    \[
        A = \begin{bmatrix}
            1 & \phantom{-}3 & -1\\
            2 & \phantom{-}0 & \phantom{-}5\\
            6 & -2 & \phantom{-}4
        \end{bmatrix}
    \]
    a matriz de $T : \mathcal{P}_2(\real) \to \mathcal{P}_2(\real)$ em relação às bases $\mathcal{B} = \{v_1, v_2, v_3\}$, em que $v_1 = 3x + 3x^2$, $v_2 = -1 + 3x + 2x^2$, $v_3 = 3 + 7x + 2x^2$.
    \begin{enumerate}[label={\alph*})]
        \item Encontre $[T(v_1)]_\mathcal{B}$, $[T(v_2)]_\mathcal{B}$ e $[T(v_3)]_\mathcal{B}$.

        \item Encontre $T(v_1)$, $T(v_2)$ e $T(v_3)$.

        \item Encontre uma fórmula para $T(a_0 + a_1x + a_2x^2)$.

        \item Use a fórmula do item anterior para calcular $T(1 + x^2)$.
    \end{enumerate}

    \begin{solucao}
        \begin{enumerate}[label={\alph*})]
            \item $[T(v_1)]_\mathcal{B} = \begin{bmatrix} 1\\2\\6\end{bmatrix}$, $[T(v_2)]_\mathcal{B} = \begin{bmatrix}\phantom{-}3\\\phantom{-}0\\-2\end{bmatrix}$, $[T(v_3)]_\mathcal{B} = \begin{bmatrix} -1\\\phantom{-}5\\\phantom{-}4\end{bmatrix}$

            \item $T(v_1) = 16 + 51x + 19x^2$, $T(v_2) = -6 - 5x + 5x^2$, $T(v_3) = 7 + 40x + 15x^2$

            \item $T(a_0 + a_1x + a_2x^2) = \dfrac{239a_0 - 161a_1 + 289a_2}{24} +\dfrac{201a_0 - 111a_1 + 247a_2}{8}x + \dfrac{61a_0 - 31a_1 + 107a_2}{12}x^2$

            \item $T(1 + x_2) = 22 + 56x + 14x^2$
        \end{enumerate}
    \end{solucao}
\end{exercicio}

\begin{exercicio}
    Seja $T : \real^2 \to \real^2$ uma transformação linear definida por
    \[
    T(x,y) = (-y,x).
    \]
    \begin{enumerate}[label={\alph*})]
        \item Qual é a matriz de $T$ em relação à base ordenada canônica, $\mathcal{B} = \{(1,0); (0,1)\}$, de $\real^2$?

        \item Qual é a matriz de $T$ em relação à base ordenada $\mathcal{B}_1 = \{w_1 = (1,2); w_2 = (1,-1)\}$?

        \item Exiba a matriz $P$ tal que $[T]_{\mathcal{B}} = P^{-1}[T]_{\mathcal{B}_1}P$.
    \end{enumerate}
    \begin{solucao}
        \begin{enumerate}[label={\alph*})]
            \item $[T]_\mathcal{B} = \begin{bmatrix}
                0 & -1\\
                1 & \phantom{-}0
            \end{bmatrix}$

            \item $[T]_{\mathcal{B}_1} = \begin{bmatrix}
                -1/3 & 2/3\\
                -5/3 & 1/3
            \end{bmatrix}$

            \item $P = \begin{bmatrix}
                1/3 & \phantom{-}1/3\\
                2/3 & -1/3
            \end{bmatrix}$
        \end{enumerate}
    \end{solucao}
\end{exercicio}

\begin{exercicio}
    Seja $T : \real^3 \to \real^3$ uma transformação linear cuja matriz com relação à base ordenada canônica, $\mathcal{B} = \{(1,0,0); (0,1,0); (0,0,1)\}$, seja
    \[
    [T]_\mathcal{B} = \begin{pmatrix}
        \phantom{-}1 & \phantom{-}1 & \phantom{-}0\\
        -1 & \phantom{-}0 & \phantom{-}1\\
        \phantom{-}0 & -1 & -1
    \end{pmatrix}.
    \]
    \begin{enumerate}[label={\alph*})]
        \item Determine $T(x,y,z)$.

        \item Qual é a matriz de $T$ com relação à base $\mathcal{B}_1 = \{(-1,1,0);(1,-1,1);(0,1,-1)\}$?

        \item A transformação $T$ é invertível? Justifique.
    \end{enumerate}
    \begin{solucao}
        \begin{enumerate}
            \item[b)] $[T]_{\mathcal{B}_1} = \begin{bmatrix}
                0 & 0 & -1\\
                0 & 0 & \phantom{-}0\\
                1 & 0 & \phantom{-}0
            \end{bmatrix}$

            \item[c)] Não.
        \end{enumerate}
    \end{solucao}
\end{exercicio}

\begin{exercicio}
    Mostre que cada uma das transformaç\~oes lineares a seguir é invertível e determine a transformação linear inversa:
    \begin{enumerate}[label={\alph*})]
        \item $T : \real^3 \to \real^3$ dada por $T(x,y,z) = (x - 3y - 2z, y - 4z, -z)$

        \item $G : \real^3 \to \real^3$ dada por $G(x,y,z) = (x, x - y, 2x + y -z)$

        \item $F : \mathcal{P}_3(\real) \to M_2(\real)$ dada por
        \[
        T(a + bx + cx^2 + dx^3) = \begin{bmatrix}
            a + b & a - 2c\\
            d & b - d
        \end{bmatrix}
        \]

        \item $F : M_2(\real) \to \mathcal{P}_2(\real)$ dada por
        \[
        T\left(\begin{bmatrix}
            a & b\\b & c
        \end{bmatrix}\right) = (a + b + c) + (-a + 2c)x + (2a + 3b + 6c)x^2
        \]
    \end{enumerate}
\end{exercicio}

\begin{exercicio}
    A transformação $T : \complex^3 \to M_2(\complex)$ dada por
    \[
    T(a,b,c) = \begin{bmatrix}
        a - b & 2a + 2b + c\\
        3a + b + c & -2a - 6b -2c
    \end{bmatrix}
    \]
    é inversível? Justifique.
\end{exercicio}


\begin{exercicio}
    Seja $\cp{K}$ um corpo e $T : \cp{K}^2 \to \cp{K}^2$ a transformação linear dada por $T(x_1,x_2) = (x_1 + x_2, x_1)$ para todo $(x_1,x_2) \in \cp{K}^2$. Prove que $T$ é um isomorfismo e exiba $T^{-1}$.
\end{exercicio}

\begin{exercicio}
    Seja $T : \complex^3 \to \complex^3$ a transformação linear definida por $T(1,0,0) = (1,0,i)$, $T(0,1,0) = (0,1,1)$ e $T(0,0,1) = (i,1,0)$. Decida se $T$ é invertível.
\end{exercicio}

\begin{exercicio}
    Seja $T : \real^3 \to \real^3$ uma transformação linear definida por
    \[
    T(x,y,z) = (3x + z,-2x + y,-x+2y + 4z).
    \]
    \begin{enumerate}[label={\alph*})]
        \item Qual é a matriz de $T$ em relação à base ordenada canônica, $\mathcal{B} = \{(1,0,0); (0,1,0); (0,0,1)\}$, de $\real^3$?

        \item Qual é a matriz de $T$ em relação à base ordenada $\mathcal{B}_1 = \{w_1 = (1,0,1); w_2 = (-1,2,1); w_3 = (2,1,1)\}$?

        \item Exiba a matriz $P$ tal que $[T]_{\mathcal{B}} = P^{-1}[T]_{\mathcal{B}_1}P$.

        \item Mostrar que $T$ é invertível e achar uma expressão para $T^{-1}$.
    \end{enumerate}
    \begin{solucao}
        \begin{enumerate}[label={\alph*})]
            \item $[T]_\mathcal{B} = \begin{bmatrix}
                \phantom{-}3 & 0 & 1\\
                -2 & 1 & 0\\
                \phantom{-}1 & 2 & 4
            \end{bmatrix}$

            \item $[T]_{\mathcal{B}_1} = \begin{bmatrix}
                \phantom{-}17/4 & \phantom{-}35/4 & \phantom{-}11/2\\
                -3/4 & \phantom{-}15/4 & -3/2\\
                -1/2 & -7/4 & \phantom{-}0
            \end{bmatrix}$

            \item $P = \begin{bmatrix}
                1 & -1 & 2\\
                0 & \phantom{-}2 & 1\\
                1 & \phantom{-}1 & 1
            \end{bmatrix}$
        \end{enumerate}
    \end{solucao}
\end{exercicio}

\begin{exercicio}
    Seja $T : \real^3 \to \real^3$ uma transformação linear tal que em relação à base ordenada canônica $\mathcal{B} = \{(1,0,0); (0,1,0); (0,0,1)\}$:
    \[
    [T]_\mathcal{B} =\begin{bmatrix}
        \phantom{-}1 & 2 & 1\\
        \phantom{-} 0 & 1 & 1\\
        -1 & 3 & 4
    \end{bmatrix}.
    \]
    Ache uma base de $\im(T)$ e uma base de $\ker(T)$.
\end{exercicio}

\begin{exercicio}
    Seja $T : M_2(\real) \to \mathcal{P}_2(\real)$ uma transformação linear tal que em relação às bases ordenadas
    \begin{align*}
        \mathcal{B}_1 &= \left\{\begin{bmatrix}
            \phantom{-}1 & \phantom{-}2\\-1 & -2
        \end{bmatrix}, \begin{bmatrix}
            \phantom{-}1 & \phantom{-}3\\-1 & -4
        \end{bmatrix}, \begin{bmatrix}
            1 & \phantom{-}2\\0 & -2
        \end{bmatrix}, \begin{bmatrix}
            \phantom{-}2 & \phantom{-}5\\-2 & -4
        \end{bmatrix}\right\}\\
        \mathcal{B}_2 &= \{1 + x + x^2, 2 + 3x, -1 - 2x^2\}
    \end{align*}
    de $M_2(\real)$ e $\mathcal{P}_2(\real)$, respectivamente, seja dada por
    \[
    [T]_\mathcal{B} =\begin{bmatrix}
        \phantom{-}2 & -24 & \phantom{-}5 & -8\\
        \phantom{-}0 & \phantom{-}8 & \phantom{-}0 & \phantom{-}4\\
        -2 & -26 & -5 & -17
    \end{bmatrix}.
    \]
    Ache uma base de $\im(T)$ e uma base de $\ker(T)$.
\end{exercicio}

\begin{exercicio}
    Seja $T : \cp{M}_2(\complex) \to \cp{M}_2(\complex)$ uma transformação linear dada por
    \[
    T \begin{pmatrix}
        x & y\\
        z & w
    \end{pmatrix} = \begin{pmatrix}
        0 & x\\
        z - w & 0
    \end{pmatrix},
    \]
    onde $\cp{M}_2(\complex)$ é um $\complex$-espaço vetorial
    e
    \[
    \mathcal{B}_1 = \left\{u_1 = \begin{bmatrix}
        1 & 0\\0 & 0
    \end{bmatrix}, u_2 = \begin{bmatrix}
        0 & 1\\0 & 0
    \end{bmatrix}, u_3 = \begin{bmatrix}
        0 & 0\\1 & 0
    \end{bmatrix}, u_4 = \begin{bmatrix}
        0 & 0\\0 & 1
    \end{bmatrix}\right\}
    \]
    \begin{enumerate}[label={\alph*})]
        \item Determine a matriz de $T$ com relação à base ordenada canônica $\mathcal{B}_1$.

        \item Determine a matriz de $T$ com relação à base
        \[
        \mathcal{B}_2 = \left\{\begin{pmatrix}
            1 & 0\\
            0 & 1
        \end{pmatrix}; \begin{pmatrix}
            0 & 1\\
            1 & 0
        \end{pmatrix}; \begin{pmatrix}
            1 & 0\\
            1 & 1
        \end{pmatrix}; \begin{pmatrix}
            0 & 1\\
            0 & 1
        \end{pmatrix}\right\}
        \]
        de $\cp{M}_2(\complex)$.

        \item Exiba a matriz $P$ tal que $[T]_{\mathcal{B}_2} = P^{-1}[T]_{\mathcal{B}_1}P$.
    \end{enumerate}
    \begin{solucao}
        \begin{enumerate}[label={\alph*})]
            \item $[T]_{\mathcal{B}_1} = \begin{bmatrix}
                0 & 0 & 0 & \phantom{-}0\\
                1 & 0 & 0 & \phantom{-}0\\
                0 & 0 & 1 & \phantom{-}0\\
                0 & 0 & 0 & -1
            \end{bmatrix}$

            \item $[T]_{\mathcal{B}_2} = \begin{bmatrix}
                \phantom{-}2 & -1 & \phantom{-}1 & \phantom{-}1\\
                \phantom{-}2 & \phantom{-}0 & \phantom{-}2 & \phantom{-}1\\
                -2 & \phantom{-}1 & -1 & -1\\
                -1 & \phantom{-}0 & -1 & -1
            \end{bmatrix}$

            \item $P = \begin{bmatrix}
                1 & 0 & 1 & 0\\
                0 & 1 & 0 & 1\\
                0 & 1 & 1 & 0\\
                1 & 0 & 1 & 1
            \end{bmatrix}$
        \end{enumerate}
    \end{solucao}
\end{exercicio}

Nos exercícios \eqref{inicioautovetor} a \eqref{fimautovetor}, verifique que $v$ é um autovetor de $A$ e determine o autovalor correspondente.

\begin{exercicio}\label{inicioautovetor}
    $A = \begin{bmatrix}0 & 3\\3 & 0\end{bmatrix}$, $v = \begin{bmatrix}1\\1\end{bmatrix}$
    \begin{solucao}
        $Av = \begin{bmatrix}3\\3\end{bmatrix} = 3v$, $\lambda = 3$
    \end{solucao}
\end{exercicio}

\begin{exercicio}
    $A = \begin{bmatrix}-1 & 1\\\phantom{-}6 & 0\end{bmatrix}$, $v = \begin{bmatrix}\phantom{-}1\\-2\end{bmatrix}$
    \begin{solucao}
        $Av = \begin{bmatrix}-3\\\phantom{-}6\end{bmatrix} = -3v$, $\lambda = -3$
    \end{solucao}
\end{exercicio}

\begin{exercicio}
    $A = \begin{bmatrix}3 & 0 & \phantom{-}0\\0 & 1 & -2\\1 & 0 & \phantom{-}1\end{bmatrix}$, $v = \begin{bmatrix}\phantom{-}2\\-1\\\phantom{-}1\end{bmatrix}$
    \begin{solucao}
        $Av = \begin{bmatrix}\phantom{-}6\\-3\\\phantom{-}3\end{bmatrix} = 3v$, $\lambda = 3$
    \end{solucao}
\end{exercicio}

\begin{exercicio}
    $A = \begin{bmatrix}2 & \phantom{-}0\\0 & -2\end{bmatrix}$, $v = \begin{bmatrix}0\\1\end{bmatrix}$
    \begin{solucao}
        $Av = \begin{bmatrix}\phantom{-}0\\-2\end{bmatrix} = -2v$, $\lambda = -2$
    \end{solucao}
\end{exercicio}

\begin{exercicio}
    $A = \begin{bmatrix}2 & \phantom{-}3 & 1\\0 & -1 & 2\\0 & \phantom{-}0 & 3\end{bmatrix}$, $v = \begin{bmatrix}\phantom{-}1\\-1\\\phantom{-}0\end{bmatrix}$
    \begin{solucao}
        $Av = \begin{bmatrix}\phantom{-}1\\-1\\\phantom{-}0\end{bmatrix} = -1v$, $\lambda = -1$
    \end{solucao}
\end{exercicio}

\begin{exercicio}\label{fimautovetor}
    $A = \begin{bmatrix}0 & 1 & 0\\0 & 0 & 1\\1 & 0 & 0\end{bmatrix}$, $v = \begin{bmatrix}1\\1\\1\end{bmatrix}$
    \begin{solucao}
        $Av = \begin{bmatrix}1\\1\\1\end{bmatrix} = 1v$, $\lambda = 1$
    \end{solucao}
\end{exercicio}

Nos exercícios \eqref{inicioautovalor} a \eqref{fimautovalor}, verifique que $\lambda$ é um autovalor de $A$ e encontre um autovetor associado a esse autovalor.

\begin{exercicio}\label{inicioautovalor}
    $A = \begin{bmatrix}2 & \phantom{-}2\\2 & -1\end{bmatrix}$, $\lambda = 3$
    \begin{solucao}
        $v = \begin{bmatrix}2\\1\end{bmatrix}$
    \end{solucao}
\end{exercicio}

\begin{exercicio}
    $A = \begin{bmatrix}\phantom{-}0 & 4\\-1 & 5\end{bmatrix}$, $\lambda = 1$
    \begin{solucao}
        $v = \begin{bmatrix}4\\1\end{bmatrix}$
    \end{solucao}
\end{exercicio}

\begin{exercicio}
    $A = \begin{bmatrix}\phantom{-}1 & 0 & 2\\-1 & 1 & 1\\\phantom{-}2 & 0 & 1\end{bmatrix}$, $\lambda = -1$
    \begin{solucao}
        $v = \begin{bmatrix}\phantom{-}1\\\phantom{-}1\\-1\end{bmatrix}$
    \end{solucao}
\end{exercicio}

\begin{exercicio}
    $A = \begin{bmatrix}1 & 3\\2 & 2\end{bmatrix}$, $\lambda = -1$
    \begin{solucao}
        $v = \begin{bmatrix}-3\\\phantom{-}2\end{bmatrix}$
    \end{solucao}
\end{exercicio}

\begin{exercicio}
    $A = \begin{bmatrix}1 & 3\\2 & 2\end{bmatrix}$, $\lambda = 4$
    \begin{solucao}
        $v = \begin{bmatrix}-2\\-2\end{bmatrix}$
    \end{solucao}
\end{exercicio}

\begin{exercicio}
    $A = \begin{bmatrix}2 & 7 & 2\\0 & -1 & 0\\0 & -2 & 1\end{bmatrix}$, $\lambda = 2$
    \begin{solucao}
        $Av = \begin{bmatrix}1\\0\\0\end{bmatrix}$
    \end{solucao}
\end{exercicio}

\begin{exercicio}
    $A = \begin{bmatrix}\phantom{-}6 & -3 & \phantom{-}1 & 0\\\phantom{-}0 & \phantom{-}3 & \phantom{-}1 & 0\\-6 & \phantom{-}6 & \phantom{-}0 & 0\\-3 & \phantom{-}3 & -2 & 3\end{bmatrix}$, $\lambda = 3$
    \begin{solucao}
        $Av = \begin{bmatrix}-3\\\phantom{-}1\\\phantom{-}1\end{bmatrix}$
    \end{solucao}
\end{exercicio}

\begin{exercicio}\label{fimautovalor}
    $A = \begin{bmatrix}\phantom{-}6 & -3 & \phantom{-}1 & 0\\\phantom{-}0 & \phantom{-}3 & \phantom{-}1 & 0\\-6 & \phantom{-}6 & \phantom{-}0 & 0\\-3 & \phantom{-}3 & -2 & 3\end{bmatrix}$, $\lambda = 0$
    \begin{solucao}
        $Av = \begin{bmatrix}\phantom{-}1\\\phantom{-}1\\-3\\\phantom{-}2\end{bmatrix}$
    \end{solucao}
\end{exercicio}

Nos exercícios \eqref{iniciodiagonalizacao} à \eqref{fimdiagonalizacao}, decida se o operador linear $T : \cp{K}^n \to \cp{K}^n$ dado por sua matriz $[T]_\mathcal{B}$ é diagonalizável. Em caso positivo, calcule uma base de autovetores e a sua forma diagonal.

\begin{exercicio}\label{iniciodiagonalizacao}
    $[T]_\mathcal{B} = \begin{bmatrix} 1 & 0\\ 0 & 0\end{bmatrix}$, $\cp{K} = \complex$, $n = 2$
    \begin{solucao}
         $\mathcal{A} = \{(1,0);(0,1)\}$; $[T]_\mathcal{A} = \begin{bmatrix} 1 & 0\\ 0 & 0\end{bmatrix}$
    \end{solucao}
\end{exercicio}

\begin{exercicio}
    $[T]_\mathcal{B} = \begin{bmatrix} 1 & -2\\ 1 & -1\end{bmatrix}$, $\cp{K} = \complex$, $n = 2$

    \begin{solucao}
        $\mathcal{A} = \{(1 + i,1);(1 - i,1)\}$; $[T]_\mathcal{A} = \begin{bmatrix} i & \phantom{-} 0\\ 0 & -i\end{bmatrix}$
    \end{solucao}
\end{exercicio}

\begin{exercicio}
    $[T]_\mathcal{B} = \begin{bmatrix} 5 & -1\\ 1 & \phantom{-} 3\end{bmatrix}$, $\cp{K} = \real$, $n = 2$

    \begin{solucao}
        $T$ não é diagonalizável.
    \end{solucao}
\end{exercicio}

\begin{exercicio}
    $[T]_\mathcal{B} = \begin{bmatrix} \phantom{-} 1 & 0 & 2\\ -1 & 0 & 1\\ \phantom{-} 1 & 1 & 2\end{bmatrix}$, $\cp{K} = \real$, $n = 3$

    \begin{solucao}
        $\mathcal{A} = \{(-1,1,0);(1,2,-1);(1,0,1)\}$; $[T]_\mathcal{A} = \begin{bmatrix} 1 & \phantom{-} 0 & 0\\ 0 & -1 & 0\\0 & \phantom{-} 0 & 3\end{bmatrix}$
    \end{solucao}
\end{exercicio}

\begin{exercicio}
    $[T]_\mathcal{B} = \begin{bmatrix} -1 & -2 & 0\\ \phantom{-} 0 & -1 & 1\\ \phantom{-} 1 & \phantom{-} 0 & 0\end{bmatrix}$, $\cp{K} = \real$, $n = 3$

    \begin{solucao}
        O operador $T$ não é diagonalizável.
    \end{solucao}
\end{exercicio}

\begin{exercicio}
    $[T]_\mathcal{B} = \begin{bmatrix} -1 & -2 & 0\\ \phantom{-} 0 & -1 & 1\\ \phantom{-} 1 & \phantom{-} 0 & 0\end{bmatrix}$, $\cp{K} = \complex$, $n = 3$

    \begin{solucao}
        $\mathcal{A} = \{(2,1,-1);(-1 + i, 1, 1 + i);(-1 - i, 1, 1 - i)\}$; $[T]_\mathcal{A} = \begin{bmatrix} -2 & 0 & \phantom{-} 0\\ \phantom{-} 0 & i & \phantom{-} 0\\ \phantom{-} 0 & 0 & -i\end{bmatrix}$
    \end{solucao}
\end{exercicio}

\begin{exercicio}
    $[T]_\mathcal{B} = \begin{bmatrix} 1 & \phantom{-} 2\\ 0 & -1\end{bmatrix}$, $\cp{K} = \real$, $n = 2$

    \begin{solucao}
        $\mathcal{A} = \{(1,0);(-1,1)\}$; $[T]_\mathcal{A} = \begin{bmatrix} 1 & 0\\ 0 & -1\end{bmatrix}$
    \end{solucao}
\end{exercicio}

\begin{exercicio}
    $[T]_\mathcal{B} = \begin{bmatrix} \phantom{-} 1 & 2 & \phantom{-} 2\\ \phantom{-} 1 & 2 & -1\\ -1 & 1 & \phantom{-} 4\end{bmatrix}$, $\cp{K} = \real$, $n = 3$

    \begin{solucao}
        $\mathcal{A} = \{(1,1,0);(1,0,1);(2,1,1)\}$; $[T]_\mathcal{A} = \begin{bmatrix} 3 & 0 & 0\\ 0 & 3 & 0\\ 0 & 0 & 1\end{bmatrix}$
    \end{solucao}
\end{exercicio}

\begin{exercicio}
    $[T]_\mathcal{B} = \begin{bmatrix} \phantom{-} 1 & 3 & 3\\ \phantom{-} 0 & 4 & 0\\ -3 & 3 & 1\end{bmatrix}$, $\cp{K} = \complex$, $n = 3$

    \begin{solucao}
        $\mathcal{A} = \{(2,1,1);(-i,0,1);(i,0,1)\}$; $[T]_\mathcal{A} = \begin{bmatrix} 4 & 0 & \phantom{-} 0\\ 0 & 1 + 3i & \phantom{-} 0\\ 0 & 0 & 1 - 3i\end{bmatrix}$
    \end{solucao}
\end{exercicio}

\begin{exercicio}
    $[T]_\mathcal{B} = \begin{bmatrix} 0 & 1 & 0 & 0\\ 0 & 0 & 1 & 0\\ 0 & 0 & 0 & 1 \\ 1 & 0 & 0 & 0\end{bmatrix}$, $\cp{K} = \complex$, $n = 4$

    \begin{solucao}
        $\mathcal{A} = \{(-1,1,-1,1); (i,-1,-i,1); (-i,-1,i,1), (1,1,1,1)\}$; $[T]_\mathcal{A} = \begin{bmatrix} -1 & \phantom{-}0 & 0 & 0\\ \phantom{-} 0 & -i & 0 & 0\\ \phantom{-} 0 & \phantom{-}0 & i & 0\\ \phantom{-}0 & \phantom{-}0 & 0 & 1\end{bmatrix}$
    \end{solucao}
\end{exercicio}

\begin{exercicio}\label{fimdiagonalizacao}
    $[T]_\mathcal{B} = \begin{bmatrix} 2 & 0 & 0 & \phantom{-}0\\ 0 & 1 & 0 & \phantom{-}0\\ 0 & 2 & 0 & \phantom{-}0 \\ 1 & 0 & 0 & -2\end{bmatrix}$, $\cp{K} = \real$, $n = 4$

    \begin{solucao}
        $\mathcal{A} = \{(0,0,1,0); (0,1,2,0); (4,0,0,1), (0,0,0,1)\}$; $[T]_\mathcal{A} = \begin{bmatrix} 0 & 0 & 0 & \phantom{-}0\\ 0 & 1 & 0 & \phantom{-}0\\0 & 0 & 2 & \phantom{-}0\\0 & 0 & 0 & -2\end{bmatrix}$
    \end{solucao}
\end{exercicio}

\begin{exercicio}
  Determine, se existir, uma matriz $P$ com coeficientes em $\real$ e invertível tal que $P^{-1}AP$ seja diagonal para cada uma das seguintes matrizes:
    \begin{enumerate}[label=({\alph*})]
      \item $A = \begin{bmatrix} 0 & 1\\ 1 & 0\end{bmatrix}$
      \item $A = \begin{bmatrix} 2 & 3\\ 1 & 4\end{bmatrix}$
      \item $A = \begin{bmatrix} \phantom{-} 1 & 2 & -2\\ -2 & 5 & -2\\ -6 & 6 & -3\end{bmatrix}$
      \item $A = \begin{bmatrix} 1 & 0 & 0\\ 1 & 2 & 1\\ 1 & 0 & 2\end{bmatrix}$
      \item $A = \begin{bmatrix} 1 & a \\ a & 1\end{bmatrix}$, $a \in \real$.
    \end{enumerate}
    \begin{solucao}
      \begin{enumerate}[label=({\alph*})]
        \item $P = \begin{bmatrix}
          \phantom{-} 1 & 1\\ -1 & 1
        \end{bmatrix}$ A resposta não é única.
        \item $P = \begin{bmatrix}
          -3 & 1\\ \phantom{-} 1 & 1
        \end{bmatrix}$ A resposta não é única.
        \item $P = \begin{bmatrix}
          1 & -1 & 1\\ 1 & \phantom{-} 0 & 1\\ 3 & \phantom{-} 1 & 0
        \end{bmatrix}$ A resposta não é única.
        \item $A$ não é diagonalizável.
        \item $P = \begin{bmatrix} 1/2 & -1/2\\ 1/2 & \phantom{-} 1/2\end{bmatrix}$. A resposta não é única.
      \end{enumerate}
    \end{solucao}

\end{exercicio}

\begin{exercicio}
  Seja $T : \real^2 \to \real^2$ uma transformação linear que tem como autovetores $(3,1)$ e $(-2,1)$ associados aos autovalores $-2$ e $3$, respectivamente. Calcule $T(x,y)$.
  \begin{solucao}
    $T(x,y) = (-6y, y -x)$
  \end{solucao}
\end{exercicio}

\begin{exercicio}
  Seja $T : \cp{M}_2(\real) \to \cp{M}_2(\real)$ um operador linear cuja matriz em relação à base
  \[
    \mathcal{B} = \left\{\begin{bmatrix}1 & 0\\ 1 & 0\end{bmatrix}; \begin{bmatrix}1 & 0\\ 0 & 0\end{bmatrix}; \begin{bmatrix}0 & 1\\ 0 & 1\end{bmatrix}; \begin{bmatrix}0 & 0\\ 0 & 1\end{bmatrix}\right\}
  \]
  é dada por
  \[
    [T]_\mathcal{B} = \begin{bmatrix}
      -1 & -4 & -2 & -2\\
      -4 & -1 & -2 & -2\\
      \phantom{-} 2 & \phantom{-} 2 & \phantom{-} 1 & \phantom{-} 4\\
      \phantom{-} 2 & \phantom{-} 2 & \phantom{-} 4 & \phantom{-} 1
    \end{bmatrix}.
  \]
  Determine uma matriz invertível $M \in \cp{M}_4(\real)$ tal que $M^{-1}[T]_\mathcal{B}M$ seja uma matriz diagonal.
\begin{solucao}
  $M = \begin{bmatrix}
    \phantom{-} 1 & \phantom{-} 1 & \phantom{-} 1 & \phantom{-} 1\\
    \phantom{-} 1 & \phantom{-} 0 & \phantom{-} 1 & \phantom{-} 1\\
    \phantom{-} 1 & -1 & \phantom{-} 0 & -2\\
    -2 & -1 & -1 & -2
  \end{bmatrix}$. A resposta não é única.
\end{solucao}
\end{exercicio}

\newpage
\Closesolutionfile{ans}

\begin{center}
    {\large\bf RESPOSTAS}
\end{center}

\input{ans1}


\end{document}
