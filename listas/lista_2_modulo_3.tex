%!TEX program = xelatex
\def\numeromodulo{3}
\def\numerolista{1}
\documentclass[12pt]{exam}

\def\ano{2023}
\def\semestre{1}
\def\disciplina{Introdução à \'Algebra Linear}
\def\turma{3}

\usepackage{caption}
\usepackage{amssymb}
\usepackage{amsmath,amsfonts,amsthm,amstext}
\usepackage[brazil]{babel}
\usepackage{graphicx}
\graphicspath{{../Pictures/}}
\usepackage{enumitem}
\usepackage{multicol}
\usepackage{answers}
\usepackage[svgnames]{xcolor}
\usepackage{tikz}
\usepackage{ifthen}
\usetikzlibrary{lindenmayersystems}
\usetikzlibrary[shadings]

\Newassociation{solucao}{Solution}{ans}
\newtheorem{exercicio}{}

\setlength{\topmargin}{-1.0in}
\setlength{\oddsidemargin}{0in}
\setlength{\textheight}{10.1in}
\setlength{\textwidth}{6.5in}
\setlength{\baselineskip}{12mm}

\extraheadheight{0.7in}
\firstpageheadrule
\runningheadrule
\lhead{
        \begin{minipage}[c]{1.7cm}
        \includegraphics[width=1.7cm]{unb.pdf}
        \end{minipage}%
        \hspace{0pt}
        \begin{minipage}[c]{4in}
          {Universidade de Bras{\'\i}lia} --
          {Departamento de Matem{\'a}tica}
\end{minipage}
\vspace*{-0.8cm}
}
% \chead{Universidade de Bras{\'\i}lia - Departamento de Matem{\'a}tica}
% \rhead{}
% \vspace*{-2cm}

\extrafootheight{.5in}
\footrule
\lfoot{\disciplina\ - \semestre$^o$/\ano\ - Módulo \numeromodulo}
\cfoot{}
\rfoot{P\'agina \thepage\ de \numpages}

\newcounter{exercicios}
\renewcommand{\theexercicios}{\arabic{exercicios}}

\newenvironment{questao}[1]{
\refstepcounter{exercicios}
\ifx&#1&
\else
   \label{#1}
\fi
\noindent\textbf{Exerc{\'\i}cio {\theexercicios}:}
}

\newcommand{\resp}[1]{
\noindent{\bf Exerc{\'\i}cio #1: }}

\def\ano{2023}
\def\semestre{2}
\def\disciplina{Introdução à Álgebra Linear}
\def\nomeabreviado{IAL}
\def\turma{11}

\newcommand{\im}{{\rm Im\,}}
\newcommand{\dlim}[2]{\displaystyle\lim_{#1\rightarrow #2}}
\newcommand{\minf}{+\infty}
\newcommand{\ninf}{-\infty}
\newcommand{\cp}[1]{\mathbb{#1}}
\newcommand{\sub}{\subseteq}
\newcommand{\n}{\mathbb{N}}
\newcommand{\z}{\mathbb{Z}}
\newcommand{\rac}{\mathbb{Q}}
\newcommand{\real}{\mathbb{R}}
\newcommand{\complex}{\mathbb{C}}

\newcommand{\vesp}[1]{\vspace{ #1  cm}}

\newcommand{\compcent}[1]{\vcenter{\hbox{$#1\circ$}}}
\newcommand{\comp}{\mathbin{\mathchoice
        {\compcent\scriptstyle}{\compcent\scriptstyle}
        {\compcent\scriptscriptstyle}{\compcent\scriptscriptstyle}}}
\renewcommand{\sin}{{\rm sen\,}}
\renewcommand{\tan}{{\rm tg\,}}
\renewcommand{\csc}{{\rm cossec\,}}
\renewcommand{\cot}{{\rm cotg\,}}
\renewcommand{\sinh}{{\rm senh\,}}
\newcommand{\integer}{\mathbb{Z}}
\begin{document}

  \Opensolutionfile{ans}[ans1]
  \begin{center}
    {\Large\bf \disciplina\ - Turma \turma\ -- \semestre$^{o}$/\ano} \\ \vspace{9pt} {\large\bf
        $\numerolista^a$ Lista de Exerc{\'\i}cios -- Módulo \numeromodulo}\\ \vspace{9pt} Prof. Jos{\'e} Ant{\^o}nio O. Freitas
  \end{center}
  \hrule

\vesp{.6}

\begin{exercicio}
    Sejam $\mathcal{B} = \{(1,0);(0,1)\}$, $\mathcal{B}_1 = \{(-1,1);(1,1)\}$, $\mathcal{B}_2 = \{(\sqrt{3},-1);(\sqrt{3},1)\}$ e $\mathcal{B}_3 = \{(2,0);(0,2)\}$ bases ordenadas de $\real^2$.
    \begin{enumerate}[label=({\alph*})]
        \item Quais s\~ao as coordenadas do vetor $v = (3,-2)$ em rela\c{c}\~ao \`a base:
        \begin{enumerate}[label=({\roman*})]
            \item $\mathcal{B}$
            \item $\mathcal{B}_1$
            \item $\mathcal{B}_2$
            \item $\mathcal{B}_3$
        \end{enumerate}
        \item Ache a matriz de mudan\c{c}a de base nos seguintes casos:
        \begin{enumerate}[label=({\roman*})]
            \item $[I]_{\mathcal{B},\mathcal{B}_1}$
            \item $[I]_{\mathcal{B}_1,\mathcal{B}}$
            \item $[I]_{\mathcal{B}_2,\mathcal{B}}$
            \item $[I]_{\mathcal{B}_3,\mathcal{B}}$
        \end{enumerate}
        \item As coordenadas de um vetor $v \in \real^2$ em rela\c{c}\~ao \`a base $\mathcal{B}_1$ s\~ao dadas por
        \[
        [v]_{\mathcal{B}_1} = \begin{bmatrix}
            4\\0
        \end{bmatrix}.
        \]
        Quais as coordenadas de $v$ em rela\c{c}\~ao \`a base:
        \begin{enumerate}[label=({\roman*})]
            \item $\mathcal{B}$
            \item $\mathcal{B}_2$
            \item $\mathcal{B}_3$
        \end{enumerate}
    \end{enumerate}
\end{exercicio}

\begin{exercicio}
    Sejam $\mathcal{B} = \{(1,0,0);(0,1,0);(0,0,1)\}$, $\mathcal{B}_1 = \{(1,3,-2);(2,1,3);(3,6,0)\}$, \linebreak $\mathcal{B}_2 = \{(1,1,1);(0,1,0);(0,0,1)\}$ e $\mathcal{B}_3 = \{(1,1,1);(0,-1,1);(0,0,1)\}$ bases ordenadas de $\real^2$.
    \begin{enumerate}[label=({\alph*})]
        \item Quais s\~ao as coordenadas do vetor $v = (3,-2,1)$ em rela\c{c}\~ao \`a base:
        \begin{enumerate}[label=({\roman*})]
            \item $\mathcal{B}$
            \item $\mathcal{B}_1$
            \item $\mathcal{B}_2$
            \item $\mathcal{B}_3$
        \end{enumerate}
        \item Ache a matriz de mudan\c{c}a de base nos seguintes casos:
        \begin{enumerate}[label=({\roman*})]
            \item $[I]_{\mathcal{B},\mathcal{B}_1}$
            \item $[I]_{\mathcal{B}_1,\mathcal{B}}$
            \item $[I]_{\mathcal{B}_2,\mathcal{B}}$
            \item $[I]_{\mathcal{B}_3,\mathcal{B}}$
        \end{enumerate}
        \item As coordenadas de um vetor $v \in \real^3$ em rela\c{c}\~ao \`a base $\mathcal{B}_1$ s\~ao dadas por
        \[
        [v]_{\mathcal{B}_1} = \begin{bmatrix}
            \phantom{-}4\\\phantom{-}0\\-1
        \end{bmatrix}.
        \]
        Quais as coordenadas de $v$ em rela\c{c}\~ao \`a base:
        \begin{enumerate}[label=({\roman*})]
            \item $\mathcal{B}$
            \item $\mathcal{B}_2$
            \item $\mathcal{B}_3$
        \end{enumerate}
    \end{enumerate}
\end{exercicio}

\begin{exercicio}
    Se
    \[
    [I]_{\mathcal{B}_1,\mathcal{B}_2} = \begin{bmatrix}
        1 & \phantom{-}1 & 1\\
        0 & -1 & 1\\
        1 & \phantom{-}0 & 1
    \end{bmatrix}
    \]
    encontre:
    \begin{enumerate}[label=({\alph*})]
        \item $[v]_{\mathcal{B}_1}$ onde $[v]_{\mathcal{B}_2} = \begin{bmatrix}
            -1\\\phantom{-}2\\\phantom{-}3
        \end{bmatrix}$
        \item $[v]_{\mathcal{B}_2}$ onde $[v]_{\mathcal{B}_1} = \begin{bmatrix}
            -1\\\phantom{-}2\\\phantom{-}3
        \end{bmatrix}$
    \end{enumerate}
\end{exercicio}


\begin{exercicio}
    Encontre a matriz da transforma\c{c}\~ao linear dada:
    \begin{enumerate}[label=({\alph*})]
        \item $F : \mathcal{P}_2(\real) \to \mathcal{P}_2(\real)$ dada por $F(p(t)) = t^2p''(t)$. Considere $\mathcal{P}_2(\real)$ um $\real$-espa\c{c}o vetorial e $\mathcal{B} = \{u_1 = 1, u_2 = t, u_3 = t^2\}$ uma base ordenada.

        \item $D : \mathcal{P}_2(\complex) \to \mathcal{P}_1(\complex)$ dada por
        \[
        D(a_0 + a_1x + a_2x^2) = a_1 + 2a_2x.
        \]
        Considere $\mathcal{P}_2(\complex)$ e $\mathcal{P}_1(\complex)$ sendo espa\c{c}os vetoriais sobre $\complex$ e $\mathcal{B}_1 = \{1, x, x^2\}$ e $\mathcal{B}_2 = \{1, 2 - x\}$ bases ordenadas de $\mathcal{P}_2(\complex)$ e $\mathcal{P}_1(\complex)$, respectivamente.

        \item $D : \mathcal{P}_3(\real) \to \mathcal{P}_5(\real)$ dada por
        \[
        D(a_0 + a_1x + a_2x^2 + a_3x^3) = a_1 + 2a_2x + 3a_3x^2
        \]
        Considere $\mathcal{P}_3(\real)$ e $\mathcal{P}_5(\real)$ sendo espa\c{c}os vetoriais sobre $\real$ e $\mathcal{B}_1 = \{1, x, x^2, x^3\}$ e $\mathcal{B}_2 = \{1, x, x^2, x^3, x^4, x^5\}$ bases ordenadas de $\mathcal{P}_3(\real)$ e $\mathcal{P}_5(\real)$, respectivamente.

        \item $D : \mathcal{P}_n(\real) \to \mathcal{P}_n(\real)$ dada por
        \[
        D(a_0 + a_1x + a_2x^2 + \cdots + a_nx^n) = a_1 + 2a_2x + \cdots + na_nx^{n-1}
        \]
        Considere $\mathcal{P}_n(\real)$ como um espa\c{c}o vetorial sobre $\real$ e $\mathcal{B}_1 = \{1, x, x^2, \dots, x^n\}$ uma base ordenada de $\mathcal{P}_n(\real)$.

        \item $G : \cp{M}_2(\real) \to \cp{M}_2(\real)$ dada por $G(X) = MX + X$ onde
        \[
        M = \begin{pmatrix}
            1 & 0\\
            0 & 0
        \end{pmatrix}.
        \]
        Considere $\cp{M}_2(\real)$ um $\real$-espa\c{c}o vetorial e
        \[
        \mathcal{B} = \left\{E_{11} = \begin{bmatrix}
            1 & 0\\0 & 0
        \end{bmatrix}, E_{12} = \begin{bmatrix}
            0 & 1\\0 & 0
        \end{bmatrix}, E_{21} = \begin{bmatrix}
            0 & 0\\1 & 0
        \end{bmatrix}, E_{22} = \begin{bmatrix}
            0 & 0\\0 & 1
        \end{bmatrix}\right\}
        \]
        uma base ordenada.

        \item $H : \cp{M}_2(\real) \to \cp{M}_2(\real)$ dada por $H(X) = MX - XM$ onde
        \[
        M = \begin{bmatrix}
            1 & 2\\
            0 & 1
        \end{bmatrix}.
        \]
        Considere $\cp{M}_2(\real)$ um $\real$-espa\c{c}o vetorial e
        \[
        \mathcal{B} = \left\{E_{11} = \begin{bmatrix}
            1 & 0\\0 & 0
        \end{bmatrix}, E_{12} = \begin{bmatrix}
            0 & 1\\0 & 0
        \end{bmatrix}, E_{21} = \begin{bmatrix}
            0 & 0\\1 & 0
        \end{bmatrix}, E_{22} = \begin{bmatrix}
            0 & 0\\0 & 1
        \end{bmatrix}\right\}
        \]
        uma base ordenada.

        \item $T : \mathcal{P}_3(\real) \to \real^2$ dada por
        \[
        T(p) = \left(\int_{-1}^0p(x)dx, \int_0^1p(x)dx\right).
        \]
        Considere $\mathcal{P}_3(\real)$ e $\real^2$ como $\real$-espa\c{c}os vetoriais e $\mathcal{B}_1 = \{1, 1-x, 1 - x^2, 1 + x^3\}$ e $\mathcal{B}_2 = \{(1,1), (1,-2)\}$ bases ordenadas de $\mathcal{P}_3(\real)$ e $\real^2$, respectivamente.

        \item $T : \complex^3 \to \complex$ dada por $T(x,y,z) = x + 2y + iz$. Considere $\complex^3$ e $\complex$ como $\complex$-espa\c{c}os vetoriais e $\mathcal{B}_1 = \{(1,0,0); (0,1,0); (0,0,1)\}$ e $\mathcal{B}_2 = \{i\}$ bases ordenadas de $\complex^2$ e $\complex$, respectivamente.

        \item $F : \mathcal{P}_3(\real) \to \mathcal{P}_4(\real)$ dada por $(Fp)(x) = xp(x + 1)$. Considere $\mathcal{P}_3(\real)$ e $\mathcal{P}_4(\real)$ como $\real$-espa\c{c}os vetoriais e $\mathcal{B}_1 = \{1, x, x^2, x^3\}$ e $\mathcal{B}_2 = \{1, x, x^2, x^3, x^4\}$ bases ordenadas de $\mathcal{P}_3(\real)$ e $\mathcal{P}_4(\real)$, respectivamente.

        \item $G : \real^2 \to \mathcal{P}_2(\real)$ dada por $G(a,b) = ax^2 + bx + (a + b)$. Considere $\real^2$ e $\mathcal{P}_2(\real)$ como $\real$-espa\c{c}os vetoriais e $\mathcal{B}_1 = \{(1,3); (2,5)\}$ e $\mathcal{B}_2 = \{2, 1 + x, x^2\}$ bases ordenadas de $\real^2$ e $\mathcal{P}_2(\real)$, respectivamente.

        \item $T : \complex \to \cp{M}_2(\real)$ dada por
        \[
        T(x + yi) = \begin{bmatrix}
            x + 7y & 5y\\
            -10y & x - 7y
        \end{bmatrix}.
        \]
        Considere $\cp{M}_2(\real)$ e $\complex$ como espa\c{c}os vetoriais sobre $\real$ e $\mathcal{B}_1 = \{1, i\}$ e
        \[
        \mathcal{B}_2 = \left\{E_{11} = \begin{bmatrix}
            1 & 0\\0 & 0
        \end{bmatrix}, E_{12} = \begin{bmatrix}
            0 & 1\\0 & 0
        \end{bmatrix}, E_{21} = \begin{bmatrix}
            0 & 0\\1 & 0
        \end{bmatrix}, E_{22} = \begin{bmatrix}
            0 & 0\\0 & 1
        \end{bmatrix}\right\}.
        \]
        bases ordenadas de $\cp{M}_2(\real)$ e $\complex$, respectivamente.

        \item $S : \mathcal{P}_3(\real) \to M_2(\real)$ dada por
        \[
        S(a + bx + cx^2 + dx^3) = \begin{bmatrix}
            3a + 7b - 2c - 5d & 8a + 14b - 2c - 11d\\
            -4a - 8b + 2c + 6d & 12a + 22b - 4c - 17d
        \end{bmatrix}.
        \]
        Considere $\mathcal{P}_3(\real)$ e $M_2(\real)$ como $\real$-espa\c{c}os vetoriais e
        \begin{align*}
            \mathcal{B}_1 &= \{1 + 2x + x^2 - x^3, 1 + 3x + x^2 + x^3, -1 - 2x + 2x^3, 2 + 3x + 2x^2 - 5x^3\}\\
            \mathcal{B}_2 &= \left\{\begin{bmatrix}
                1 & 1\\1 & 2
            \end{bmatrix}, \begin{bmatrix}
                2 & 3\\2 & 5
            \end{bmatrix}, \begin{bmatrix}
                -1 & -1\\\phantom{-}0 & -2
            \end{bmatrix}, \begin{bmatrix}
                -1 & -4\\-2 & -4
            \end{bmatrix}\right\}
        \end{align*}
        bases ordenadas de $\mathcal{P}_3(\real)$ e $M_2(\real)$, respectivamente.
    \end{enumerate}
    \begin{solucao}
        \begin{enumerate}[label=({\alph*})]
            \item $[F]_\mathcal{B} = \begin{bmatrix}
                0 & 0 & 0\\
                0 & 0 & 0\\
                0 & 0 & 2
            \end{bmatrix}$

            \item $[D]_{\mathcal{B}_1, \mathcal{B}_2} = \begin{bmatrix}
                0 & 1 & \phantom{-}2\\
                0 & 0 & -2
            \end{bmatrix}$

            \item $[D]_{\mathcal{B}_1, \mathcal{B}_2} = \begin{bmatrix}
                0 & 1 & 0 & 0\\
                0 & 0 & 2 & 0\\
                0 & 0 & 0 & 3\\
                0 & 0 & 0 & 0\\
                0 & 0 & 0 & 0\\
                0 & 0 & 0 & 0
            \end{bmatrix}$

            \item $[D]_\mathcal{B} = \begin{bmatrix}
                0 & 1 & 0 & 0 & \cdots & 0\\
                0 & 0 & 2 & 0 & \cdots & 0\\
                0 & 0 & 0 & 3 & \cdots & 0\\
                \vdots & \vdots & \vdots & \vdots & \cdots & \vdots\\
                0 & 0 & 0 & 0 & \cdots & n - 1\\
                0 & 0 & 0 & 0 & \cdots & 0
            \end{bmatrix}$

            \item $[G]_\mathcal{B} = \begin{bmatrix}
                2 & 1 & 1 & 1\\
                0 & 1 & 0 & 0\\
                0 & 0 & 0 & 0\\
                0 & 0 & 0 & 0
            \end{bmatrix}$

            \item $[H]_\mathcal{B} = \begin{bmatrix}
                \phantom{-}0 & 0 & 2 & 0\\
                -2 & 0 & 0 & 2\\
                \phantom{-}0 & 0 & 0 & 0\\
                \phantom{-}0 & 0 & 0 & 0
            \end{bmatrix}$

            \item $[T]_{\mathcal{B}_1, \mathcal{B}_2} = \begin{bmatrix}
                1 & 7/6 & -35/0 & \phantom{-}11/12\\
                0 & 1/3 & -10/9 & -1/6
            \end{bmatrix}$

            \item $[T]_{\mathcal{B}_1, \mathcal{B}_2} = \begin{bmatrix}
                1 & 2 & i
            \end{bmatrix}$

            \item $[F]_{\mathcal{B}_1, \mathcal{B}_2} = \begin{bmatrix}
                0 & 0 & 0 & 0\\
                1 & 1 & 2 & 1\\
                0 & 1 & 2 & 3\\
                0 & 0 & 1 & 3\\
                0 & 0 & 0 & 1
            \end{bmatrix}$

            \item $[G]_{\mathcal{B}_1, \mathcal{B}_2} = \begin{bmatrix}
                1/2 & 1\\
                3 & 5\\
                1 & 2
            \end{bmatrix}$

            \item $[T]_{\mathcal{B}_1, \mathcal{B}_2} = \begin{bmatrix}
                1 & \phantom{-}7\\
                1 & \phantom{-}5\\
                0 & -10\\
                0 & -7
            \end{bmatrix}$

            \item $[S]_{\mathcal{B}_1, \mathcal{B}_2} = \begin{bmatrix}
                -90 & \phantom{-}72 & \phantom{-}114 & -220\\
                \phantom{-}37 & \phantom{-}29 & -46 & \phantom{-}91\\
                -40 & -34 & \phantom{-}54 & -96\\
                \phantom{-}4 & \phantom{-}3 & -5 & \phantom{-}10
            \end{bmatrix}$
        \end{enumerate}
    \end{solucao}
\end{exercicio}

\begin{exercicio}
    Sejam $T : V \to W$ e $S : W \to U$ transforma\c{c}\~oes lineares. Mostre que $S \comp T : V \to U$ \'e uma transforma\c{c}\~ao linear.
\end{exercicio}

\begin{exercicio}
    Seja $T : \real^2 \to \real^2$ uma transforma\c{c}\~ao linear definida por
    \[
    T(x,y) = (-y,x).
    \]
    \begin{enumerate}[label=({\alph*})]
        \item Qual \'e a matriz de $T$ em rela\c{c}\~ao \`a base ordenada can\^onica, $\mathcal{B} = \{(1,0); (0,1)\}$, de $\real^2$?
        \item Qual \'e a matriz de $T$ em rela\c{c}\~ao \`a base ordenada $\mathcal{B}_1 = \{w_1 = (1,2); w_2 = (1,-1)\}$?
        \item Exiba a matriz $P$ tal que $[T]_{\mathcal{B}} = P^{-1}[T]_{\mathcal{B}_1}P$.
    \end{enumerate}
    \begin{solucao}
        \begin{enumerate}[label=({\alph*})]
            \item $[T]_\mathcal{B} = \begin{bmatrix}
                0 & -1\\
                1 & \phantom{-}0
            \end{bmatrix}$
            \item $[T]_{\mathcal{B}_1} = \begin{bmatrix}
                -1/3 & 2/3\\
                -5/3 & 1/3
            \end{bmatrix}$
            \item $P = \begin{bmatrix}
                1/3 & \phantom{-}1/3\\
                2/3 & -1/3
            \end{bmatrix}$
        \end{enumerate}
    \end{solucao}
\end{exercicio}

\begin{exercicio}
    Seja $T : V \to W$ um isomorfismo, onde $V$ e $W$ s\~ao $\cp{K}$-espa\c{c}os vetoriais. Seja $G : W \to V$ definida por: $G(w) = v$ se, e somente se, $T(v) = w$, para todo $w \in W$. Mostre que:
    \begin{enumerate}[label=({\alph*})]
        \item $G$ \'e uma transforma\c{c}\~ao linear.
        \item $T\circ G = Id_W$, onde $Id_V : W \to W$ tal que $Id_W(x) = x$, para todo $x \in W$.
        \item $G\circ T = Id_V$, onde $Id_V : V \to V$ tal que $Id_V(y) = y$, para todo $y \in V$.
    \end{enumerate}
    A transforma\c{c}\~ao linear $G$ \'e chamada de \textbf{inversa} de $T$ e ser\'a denotada por $G = T^{-1}$.
\end{exercicio}

\begin{exercicio}
    Sejam $V$ e $W$ dois espa\c{c}os vetoriais sobre $\cp{K}$ tais que $\dim_\cp{K}V = \dim_\cp{K}W = n \ge 1$ e considere $\mathcal{B}_1$ e $\mathcal{B}_2$ bases ordenadas de $V$ e $W$, respectivamente. Mostre que:
    \begin{enumerate}[label=({\alph*})]
        \item Se uma transforma\c{c}\~ao linear $T : V \to W$ \'e um isomorfismo, ent\~ao a matriz $[T]_{\mathcal{B}_1,\mathcal{B}_2}$ \'e invert{\'\i}vel.
        \item Se uma transforma\c{c}\~ao linear $T : V \to W$ tal que a matriz $[T]_{\mathcal{B}_1,\mathcal{B}_2}$ \'e invert{\'\i}vel , ent\~ao $T$ \'e um isomorfismo.
        \item Mostre que se $T$ \'e um isomorfismo, ent\~ao
        \[
        \left([T]_{\mathcal{B}_1,\mathcal{B}_2}\right)^{-1} = [T^{-1}]_{\mathcal{B}_1,\mathcal{B}_2}.
        \]
    \end{enumerate}
\end{exercicio}

\begin{exercicio}
    Seja $T : \real^3 \to \real^3$ uma transforma\c{c}\~ao linear cuja matriz com rela\c{c}\~ao \`a base ordenada can\^onica, $\mathcal{B} = \{(1,0,0); (0,1,0); (0,0,1)\}$, seja
    \[
    [T]_\mathcal{B} = \begin{pmatrix}
        \phantom{-}1 & \phantom{-}1 & \phantom{-}0\\
        -1 & \phantom{-}0 & \phantom{-}1\\
        \phantom{-}0 & -1 & -1
    \end{pmatrix}.
    \]
    \begin{enumerate}[label=({\alph*})]
        \item Determine $T(x,y,z)$.
        \item Qual \'e a matriz de $T$ com rela\c{c}\~ao \`a base $\mathcal{B}_1 = \{(-1,1,0);(1,-1,1);(0,1,-1)\}$?
        \item A transforma\c{c}\~ao $T$ \'e invert{\'\i}vel? Justifique.
    \end{enumerate}
    \begin{solucao}
        \begin{enumerate}
            \item[b)] $[T]_{\mathcal{B}_1} = \begin{bmatrix}
                0 & 0 & -1\\
                0 & 0 & \phantom{-}0\\
                1 & 0 & \phantom{-}0
            \end{bmatrix}$
            \item[c)] N\~ao.
        \end{enumerate}
    \end{solucao}
\end{exercicio}

\begin{exercicio}
    Seja $T : \z_7^3 \to \z_7^3$ uma transforma\c{c}\~ao linear cuja matriz com rela\c{c}\~ao \`a base ordenada can\^onica, $\mathcal{B} = \{(\overline{1}, \overline{0}, \overline{0}); (\overline{0}, \overline{1}, \overline{0}); (\overline{0}, \overline{0}, \overline{1})\}$, seja
    \[
    [T]_\mathcal{B} = \begin{pmatrix}
        \overline{1} & \overline{1} & \overline{0}\\
        \overline{6} & \overline{0} & \overline{1}\\
        \overline{0} & \overline{6} & \overline{6}
    \end{pmatrix}.
    \]
    \begin{enumerate}[label=({\alph*})]
        \item Determine $T(x,y,z)$.
        \item Qual \'e a matriz de $T$ com rela\c{c}\~ao \`a base $\mathcal{B}_1 = \{(\overline{6},\overline{1},\overline{0});(\overline{1},\overline{6},\overline{1});(\overline{0},\overline{1},\overline{6})\}$?
        \item A transforma\c{c}\~ao $T$ \'e invert{\'\i}vel? Justifique.
    \end{enumerate}
    \begin{solucao}
        \begin{enumerate}
            \item[b)] $[T]_{\mathcal{B}_1} = \begin{pmatrix}
                \overline{5} & \overline{2} & \overline{5}\\
                \overline{5} & \overline{2} & \overline{6}\\
                \overline{6} & \overline{2} & \overline{5}
            \end{pmatrix}$
            \item[c)] Sim
        \end{enumerate}
    \end{solucao}
\end{exercicio}

\begin{exercicio}
    Mostre que cada uma das transforma\c{c}\~oes lineares a seguir \'e invert{\'\i}vel e determine a transforma\c{c}\~ao linear inversa:
    \begin{enumerate}[label=({\alph*})]
        \item $T : \real^3 \to \real^3$ dada por $T(x,y,z) = (x - 3y - 2z, y - 4z, -z)$

        \item $G : \real^3 \to \real^3$ dada por $G(x,y,z) = (x, x - y, 2x + y -z)$

        \item $F : \mathcal{P}_3(\real) \to M_2(\real)$ dada por
        \[
        T(a + bx + cx^2 + dx^3) = \begin{bmatrix}
            a + b & a - 2c\\
            d & b - d
        \end{bmatrix}
        \]

        \item $F : M_2(\real) \to \mathcal{P}_2(\real)$ dada por
        \[
        T\left(\begin{bmatrix}
            a & b\\b & c
        \end{bmatrix}\right) = (a + b + c) + (-a + 2c)x + (2a + 3b + 6c)x^2
        \]
    \end{enumerate}
\end{exercicio}

\begin{exercicio}
    A transforma\c{c}\~ao $T : \complex^3 \to M_2(\complex)$ dada por
    \[
    T(a,b,c) = \begin{bmatrix}
        a - b & 2a + 2b + c\\
        3a + b + c & -2a - 6b -2c
    \end{bmatrix}
    \]
    \'e invers{\'\i}vel? Justifique.
\end{exercicio}


\begin{exercicio}
    Seja $\cp{K}$ um corpo e $T : \cp{K}^2 \to \cp{K}^2$ a transforma\c{c}\~ao linear dada por $T(x_1,x_2) = (x_1 + x_2, x_1)$ para todo $(x_1,x_2) \in \cp{K}^2$. Prove que $T$ \'e um isomorfismo e exiba $T^{-1}$.
\end{exercicio}

\begin{exercicio}
    Seja $T : \complex^3 \to \complex^3$ a transforma\c{c}\~ao linear definida por $T(1,0,0) = (1,0,i)$, $T(0,1,0) = (0,1,1)$ e $T(0,0,1) = (i,1,0)$. Decida se $T$ \'e invert{\'\i}vel.
\end{exercicio}

\begin{exercicio}
    Seja $T : \real^3 \to \real^2$ e $S : \real^2 \to \real^3$ transforma\c{c}\~oes lineares.
    \begin{enumerate}[label=({\alph*})]
        \item Prove que $S \circ T$ n\~ao \'e invert{\'\i}vel.
        \item Achar um exemplo em que $T\circ S$ n\~ao \'e invert{\'\i}vel e um outro em que $T\circ S$ \'e invert{\'\i}vel.
    \end{enumerate}
\end{exercicio}

\begin{exercicio}
    Seja $V$ um $\cp{K}$-espa\c{c}o vetorial de dimens\~ao 2 e seja $\mathcal{B}$ uma base ordenada de $V$. Se $T: V \to V$ \'e uma transforma\c{c}\~ao linear tal que
    \[
    [T]_\mathcal{B} = \begin{bmatrix}
        a & b\\
        c & d
    \end{bmatrix},
    \]
    mostre que $[T]_\mathcal{B}^2 - (a + d)[T]_\mathcal{B} + (ad - bc)I_2 = 0$.
\end{exercicio}

\begin{exercicio}
    Seja $T : \real^3 \to \real^3$ uma transforma\c{c}\~ao linear definida por
    \[
    T(x,y,z) = (3x + z,-2x + y,-x+2y + 4z).
    \]
    \begin{enumerate}[label=({\alph*})]
        \item Qual \'e a matriz de $T$ em rela\c{c}\~ao \`a base ordenada can\^onica, $\mathcal{B} = \{(1,0,0); (0,1,0); (0,0,1)\}$, de $\real^3$?
        \item Qual \'e a matriz de $T$ em rela\c{c}\~ao \`a base ordenada $\mathcal{B}_1 = \{w_1 = (1,0,1); w_2 = (-1,2,1); w_3 = (2,1,1)\}$?
        \item Exiba a matriz $P$ tal que $[T]_{\mathcal{B}} = P^{-1}[T]_{\mathcal{B}_1}P$.
        \item Mostrar que $T$ \'e invert{\'\i}vel e achar uma express\~ao para $T^{-1}$.
    \end{enumerate}
    \begin{solucao}
        \begin{enumerate}[label=({\alph*})]
            \item $[T]_\mathcal{B} = \begin{bmatrix}
                \phantom{-}3 & 0 & 1\\
                -2 & 1 & 0\\
                \phantom{-}1 & 2 & 4
            \end{bmatrix}$

            \item $[T]_{\mathcal{B}_1} = \begin{bmatrix}
                \phantom{-}17/4 & \phantom{-}35/4 & \phantom{-}11/2\\
                -3/4 & \phantom{-}15/4 & -3/2\\
                -1/2 & -7/4 & \phantom{-}0
            \end{bmatrix}$

            \item $P = \begin{bmatrix}
                1 & -1 & 2\\
                0 & \phantom{-}2 & 1\\
                1 & \phantom{-}1 & 1
            \end{bmatrix}$
        \end{enumerate}
    \end{solucao}
\end{exercicio}

\begin{exercicio}
    Sejam $T, S : V \to W$ duas transforma\c{c}\~oes lineares, onde $V$ e $W$ s\~ao $\cp{K}$-espa\c{c}os vetoriais de dimens\~ao finita. Sejam $\mathcal{B}$ e $\mathcal{C}$ bases de $V$ e $W$, respectivamente e $\lambda \in \cp{K}$. Mostre que:
    \begin{enumerate}[label=({\alph*})]
        \item $[T + S]_{\mathcal{B}, \mathcal{C}} = [T]_{\mathcal{B}, \mathcal{C}} + [S]_{\mathcal{B}, \mathcal{C}}$.
        \item $[\lambda T]_{\mathcal{B}, \mathcal{C}} = \lambda [T]_{\mathcal{B}, \mathcal{C}}$.
    \end{enumerate}
\end{exercicio}

\begin{exercicio}
    Seja $T : \real^3 \to \real^3$ uma transforma\c{c}\~ao linear tal que em rela\c{c}\~ao \`a base ordenada can\^onica $\mathcal{B} = \{(1,0,0); (0,1,0); (0,0,1)\}$:
    \[
    [T]_\mathcal{B} =\begin{bmatrix}
        \phantom{-}1 & 2 & 1\\
        \phantom{-} 0 & 1 & 1\\
        -1 & 3 & 4
    \end{bmatrix}.
    \]
    Ache uma base de $\im T$ e uma base de $\ker T$.
\end{exercicio}

\begin{exercicio}
    Seja $T : M_2(\real) \to \mathcal{P}_2(\real)$ uma transforma\c{c}\~ao linear tal que em rela\c{c}\~ao \`as bases ordenadas
    \begin{align*}
        \mathcal{B}_1 &= \left\{\begin{bmatrix}
            \phantom{-}1 & \phantom{-}2\\-1 & -2
        \end{bmatrix}, \begin{bmatrix}
            \phantom{-}1 & \phantom{-}3\\-1 & -4
        \end{bmatrix}, \begin{bmatrix}
            1 & \phantom{-}2\\0 & -2
        \end{bmatrix}, \begin{bmatrix}
            \phantom{-}2 & \phantom{-}5\\-2 & -4
        \end{bmatrix}\right\}\\
        \mathcal{B}_2 &= \{1 + x + x^2, 2 + 3x, -1 - 2x^2\}
    \end{align*}
    de $M_2(\real)$ e $\mathcal{P}_2(\real)$, respectivamente, seja dada por
    \[
    [T]_\mathcal{B} =\begin{bmatrix}
        \phantom{-}2 & -24 & \phantom{-}5 & -8\\
        \phantom{-}0 & \phantom{-}8 & \phantom{-}0 & \phantom{-}4\\
        -2 & -26 & -5 & -17
    \end{bmatrix}.
    \]
    Ache uma base de $\im T$ e uma base de $\ker T$.
\end{exercicio}

\begin{exercicio}
    Seja $T : \cp{M}_2(\complex) \to \cp{M}_2(\complex)$ uma transforma\c{c}\~ao linear dada por
    \[
    T \begin{pmatrix}
        x & y\\
        z & w
    \end{pmatrix} = \begin{pmatrix}
        0 & x\\
        z - w & 0
    \end{pmatrix},
    \]
    onde $\cp{M}_2(\complex)$ \'e um $\complex$-espa\c{c}o vetorial
    e
    \[
    \mathcal{B}_1 = \left\{u_1 = \begin{bmatrix}
        1 & 0\\0 & 0
    \end{bmatrix}, u_2 = \begin{bmatrix}
        0 & 1\\0 & 0
    \end{bmatrix}, u_3 = \begin{bmatrix}
        0 & 0\\1 & 0
    \end{bmatrix}, u_4 = \begin{bmatrix}
        0 & 0\\0 & 1
    \end{bmatrix}\right\}
    \]
    \begin{enumerate}[label=({\alph*})]
        \item Determine a matriz de $T$ com rela\c{c}\~ao \`a base ordenada can\^onica $\mathcal{B}_1$.
        \item Determine a matriz de $T$ com rela\c{c}\~ao \`a base
        \[
        \mathcal{B}_2 = \left\{\begin{pmatrix}
            1 & 0\\
            0 & 1
        \end{pmatrix}; \begin{pmatrix}
            0 & 1\\
            1 & 0
        \end{pmatrix}; \begin{pmatrix}
            1 & 0\\
            1 & 1
        \end{pmatrix}; \begin{pmatrix}
            0 & 1\\
            0 & 1
        \end{pmatrix}\right\}
        \]
        de $\cp{M}_2(\complex)$.
        \item Exiba a matriz $P$ tal que $[T]_{\mathcal{B}_2} = P^{-1}[T]_{\mathcal{B}_1}P$.
    \end{enumerate}
    \begin{solucao}
        \begin{enumerate}[label=({\alph*})]
            \item $[T]_{\mathcal{B}_1} = \begin{bmatrix}
                0 & 0 & 0 & \phantom{-}0\\
                1 & 0 & 0 & \phantom{-}0\\
                0 & 0 & 1 & \phantom{-}0\\
                0 & 0 & 0 & -1
            \end{bmatrix}$

            \item $[T]_{\mathcal{B}_2} = \begin{bmatrix}
                \phantom{-}2 & -1 & \phantom{-}1 & \phantom{-}1\\
                \phantom{-}2 & 0 & \phantom{-}2 & \phantom{-}1\\
                -2 & \phantom{-}1 & -1 & -1\\
                -1 & \phantom{-}0 & -1 & -1
            \end{bmatrix}$

            \item $P = \begin{bmatrix}
                1 & 0 & 1 & 0\\
                0 & 1 & 0 & 1\\
                0 & 1 & 1 & 0\\
                1 & 0 & 1 & 1
            \end{bmatrix}$
        \end{enumerate}
    \end{solucao}
\end{exercicio}

\newpage
\Closesolutionfile{ans}

\begin{center}
    {\large\bf RESPOSTAS}
\end{center}

\input{ans1}


\end{document}