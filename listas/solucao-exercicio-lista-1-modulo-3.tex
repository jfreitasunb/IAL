%!TEX program = xelatex
\def\numeromodulo{3}
\def\numerolista{1}
\documentclass[12pt]{exam}

\def\ano{2023}
\def\semestre{1}
\def\disciplina{Introdução à \'Algebra Linear}
\def\turma{3}

\usepackage{caption}
\usepackage{amssymb}
\usepackage{amsmath,amsfonts,amsthm,amstext}
\usepackage[brazil]{babel}
\usepackage{graphicx}
\graphicspath{{../Pictures/}}
\usepackage{enumitem}
\usepackage{multicol}
\usepackage{answers}
\usepackage[svgnames]{xcolor}
\usepackage{tikz}
\usepackage{ifthen}
\usetikzlibrary{lindenmayersystems}
\usetikzlibrary[shadings]

\Newassociation{solucao}{Solution}{ans}
\newtheorem{exercicio}{}

\setlength{\topmargin}{-1.0in}
\setlength{\oddsidemargin}{0in}
\setlength{\textheight}{10.1in}
\setlength{\textwidth}{6.5in}
\setlength{\baselineskip}{12mm}

\extraheadheight{0.7in}
\firstpageheadrule
\runningheadrule
\lhead{
        \begin{minipage}[c]{1.7cm}
        \includegraphics[width=1.7cm]{unb.pdf}
        \end{minipage}%
        \hspace{0pt}
        \begin{minipage}[c]{4in}
          {Universidade de Bras{\'\i}lia} --
          {Departamento de Matem{\'a}tica}
\end{minipage}
\vspace*{-0.8cm}
}
% \chead{Universidade de Bras{\'\i}lia - Departamento de Matem{\'a}tica}
% \rhead{}
% \vspace*{-2cm}

\extrafootheight{.5in}
\footrule
\lfoot{\disciplina\ - \semestre$^o$/\ano\ - Módulo \numeromodulo}
\cfoot{}
\rfoot{P\'agina \thepage\ de \numpages}

\newcounter{exercicios}
\renewcommand{\theexercicios}{\arabic{exercicios}}

\newenvironment{questao}[1]{
\refstepcounter{exercicios}
\ifx&#1&
\else
   \label{#1}
\fi
\noindent\textbf{Exerc{\'\i}cio {\theexercicios}:}
}

\newcommand{\resp}[1]{
\noindent{\bf Exerc{\'\i}cio #1: }}

\def\ano{2023}
\def\semestre{2}
\def\disciplina{Introdução à Álgebra Linear}
\def\nomeabreviado{IAL}
\def\turma{11}

\newcommand{\im}{{\rm Im\,}}
\newcommand{\dlim}[2]{\displaystyle\lim_{#1\rightarrow #2}}
\newcommand{\minf}{+\infty}
\newcommand{\ninf}{-\infty}
\newcommand{\cp}[1]{\mathbb{#1}}
\newcommand{\sub}{\subseteq}
\newcommand{\n}{\mathbb{N}}
\newcommand{\z}{\mathbb{Z}}
\newcommand{\rac}{\mathbb{Q}}
\newcommand{\real}{\mathbb{R}}
\newcommand{\complex}{\mathbb{C}}

\newcommand{\vesp}[1]{\vspace{ #1  cm}}

\newcommand{\compcent}[1]{\vcenter{\hbox{$#1\circ$}}}
\newcommand{\comp}{\mathbin{\mathchoice
        {\compcent\scriptstyle}{\compcent\scriptstyle}
        {\compcent\scriptscriptstyle}{\compcent\scriptscriptstyle}}}
\renewcommand{\sin}{{\rm sen\,}}
\renewcommand{\tan}{{\rm tg\,}}
\renewcommand{\csc}{{\rm cossec\,}}
\renewcommand{\cot}{{\rm cotg\,}}
\renewcommand{\sinh}{{\rm senh\,}}
\newcommand{\integer}{\mathbb{Z}}
\begin{document}
    \begin{center}
        {\Large\bf \disciplina\ - Turma \turma\ -- \semestre$^{o}$/\ano} \\ \vspace{9pt} {\large\bf
            Solu\c{c}o de Exercício -- $\numerolista^a$ Lista - Módulo \numeromodulo}\\ \vspace{9pt} Prof. José Antônio O. Freitas
    \end{center}
    \hrule

    \vesp{.6}
    Para resolver o exercício:
    \begin{center}
        \begin{flushleft}
            \textit{Considere $\real^4$ e seus subespa\c{c}os $V = [(1,0,1,1);(0,-1,-1,-1)]$ e $W = \{(x,y,z,t) \in \real^4 \mid x + y = 0,\ t + z = 0\}$. Determine uma transforma\c{c}ão linear $T : \real^4 \to \real^4$ tal que $\ker(T) = V$ e $\im(T) = W$.}
         \end{flushleft}
    \end{center}
    devemos lembrar primeiro que no caso de uma transforma\c{c}ão linear $T \colon V \to W$, se $\dim V$ é finita então vale o seguinte teorema:

    \begin{tcolorbox}[colback=green!30, colframe=green!80!blue, title=Teorema do Núcleo e da Imagem]
        Sejam $V$ e $W$ $\cp{K}$-espa\c{c}os vetoriais com $\dim_\cp{K}V$ finita. Seja $T : V \to W$ uma transforma\c{c}ão linear. Então
        \[
            \dim_\cp{K}V = \dim_\cp{K}\ker(T) + \dim_\cp{K}\im(T).
        \]
    \end{tcolorbox}

    Assim, nesse caso devemos ter
    \[
        \dim\ker(T) + \dim\im(T) = 4.
    \]
    Primeiro, observe que os dois vetores que geram $V$ são L.I., logo o conjunto ${(1, 0, 1, 1); (0,-1,-1,-1)}$ é uma base para $V$. Assim $\dim(V) = 2$.

    Para $W$, come\c{c}amos notando que
    \[
        W = \{(x, y, z, t) \in \real \mid x + y = 0,\ t + z = 0\}
    \]
    com isso
    \begin{align*}
        W &= \{(x, y, z, t) \in \real^4 \mid y = -x,\ z = -t\} \\ &= \{(x, -x, -t, t) \in \real^4 \mid x,\ t \in \real\} \\ &= \{(x, -x, 0, 0) + (0, 0, -t, t) \mid x,\ t \in \real\} &= \{x(1, -1, 0, 0) + t(0, 0, -1, 1) \mid x,\ t \in \real\} \\ &= [(1, -1, 0, 0); (0, 0, -1, 1)]
    \end{align*}
    Note que o conjunto ${(1, -1, 0, 0); (0, 0, -1, 1)}$ é L.I., logo uma base para $W$. Assim $\dim(W) = 2$.

    Portanto é possível existir uma transforma\c{c}ão com as condi\c{c}ões solicitadas. Para criar essa transforma\c{c}ão vamos usar o seguinte teorema:
    \begin{tcolorbox}[colback=green!30, colframe=green!80!blue, title=Teorema]
        Sejam $V$ e $W$ $\cp{K}$-espa\c{c}os vetoriais. Se $\{u_1, \dots, u_n\}$ é uma base de $V$ e se $\{w_1, \dots, w_n\} \subseteq W$, então existe uma única transforma\c{c}ão linear $T \colon V \to W$ tal que $T(u_i) = w_i$ para cada $i = 1$, \dots, $n$.
    \end{tcolorbox}
\end{document}
