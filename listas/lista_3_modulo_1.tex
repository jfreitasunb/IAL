%!TEX program = xelatex
\def\numeromodulo{1}
\def\numerolista{3}
\documentclass[12pt]{exam}

\def\ano{2023}
\def\semestre{1}
\def\disciplina{Introdução à \'Algebra Linear}
\def\turma{3}

\usepackage{caption}
\usepackage{amssymb}
\usepackage{amsmath,amsfonts,amsthm,amstext}
\usepackage[brazil]{babel}
\usepackage{graphicx}
\graphicspath{{../Pictures/}}
\usepackage{enumitem}
\usepackage{multicol}
\usepackage{answers}
\usepackage[svgnames]{xcolor}
\usepackage{tikz}
\usepackage{ifthen}
\usetikzlibrary{lindenmayersystems}
\usetikzlibrary[shadings]

\Newassociation{solucao}{Solution}{ans}
\newtheorem{exercicio}{}

\setlength{\topmargin}{-1.0in}
\setlength{\oddsidemargin}{0in}
\setlength{\textheight}{10.1in}
\setlength{\textwidth}{6.5in}
\setlength{\baselineskip}{12mm}

\extraheadheight{0.7in}
\firstpageheadrule
\runningheadrule
\lhead{
        \begin{minipage}[c]{1.7cm}
        \includegraphics[width=1.7cm]{unb.pdf}
        \end{minipage}%
        \hspace{0pt}
        \begin{minipage}[c]{4in}
          {Universidade de Bras{\'\i}lia} --
          {Departamento de Matem{\'a}tica}
\end{minipage}
\vspace*{-0.8cm}
}
% \chead{Universidade de Bras{\'\i}lia - Departamento de Matem{\'a}tica}
% \rhead{}
% \vspace*{-2cm}

\extrafootheight{.5in}
\footrule
\lfoot{\disciplina\ - \semestre$^o$/\ano\ - Módulo \numeromodulo}
\cfoot{}
\rfoot{P\'agina \thepage\ de \numpages}

\newcounter{exercicios}
\renewcommand{\theexercicios}{\arabic{exercicios}}

\newenvironment{questao}[1]{
\refstepcounter{exercicios}
\ifx&#1&
\else
   \label{#1}
\fi
\noindent\textbf{Exerc{\'\i}cio {\theexercicios}:}
}

\newcommand{\resp}[1]{
\noindent{\bf Exerc{\'\i}cio #1: }}

\def\ano{2023}
\def\semestre{2}
\def\disciplina{Introdução à Álgebra Linear}
\def\nomeabreviado{IAL}
\def\turma{11}

\newcommand{\im}{{\rm Im\,}}
\newcommand{\dlim}[2]{\displaystyle\lim_{#1\rightarrow #2}}
\newcommand{\minf}{+\infty}
\newcommand{\ninf}{-\infty}
\newcommand{\cp}[1]{\mathbb{#1}}
\newcommand{\sub}{\subseteq}
\newcommand{\n}{\mathbb{N}}
\newcommand{\z}{\mathbb{Z}}
\newcommand{\rac}{\mathbb{Q}}
\newcommand{\real}{\mathbb{R}}
\newcommand{\complex}{\mathbb{C}}

\newcommand{\vesp}[1]{\vspace{ #1  cm}}

\newcommand{\compcent}[1]{\vcenter{\hbox{$#1\circ$}}}
\newcommand{\comp}{\mathbin{\mathchoice
        {\compcent\scriptstyle}{\compcent\scriptstyle}
        {\compcent\scriptscriptstyle}{\compcent\scriptscriptstyle}}}
\renewcommand{\sin}{{\rm sen\,}}
\renewcommand{\tan}{{\rm tg\,}}
\renewcommand{\csc}{{\rm cossec\,}}
\renewcommand{\cot}{{\rm cotg\,}}
\renewcommand{\sinh}{{\rm senh\,}}
\newcommand{\integer}{\mathbb{Z}}
\begin{document}

  \Opensolutionfile{ans}[ans1]
  \begin{center}
    {\Large\bf \disciplina\ - Turma \turma\ -- \semestre$^{o}$/\ano} \\ \vspace{9pt} {\large\bf
        $\numerolista^a$ Lista de Exerc{\'\i}cios -- Módulo \numeromodulo}\\ \vspace{9pt} Prof. Jos{\'e} Ant{\^o}nio O. Freitas
  \end{center}
  \hrule

\vesp{.6}

\begin{exercicio}
  Considere as seguintes matrizes:
  \begin{align*}
    A &= \begin{bmatrix}1 & \phantom{-}2 & -1\\1 & \phantom{-} 1 & \phantom{-} 1\\1 & -1 & \phantom{-} 0\end{bmatrix},
    B = \begin{bmatrix}1 & -1 & \phantom{-} 0\\1 & \phantom{-} 1 & \phantom{-} 1\\1 & \phantom{-} 2 & -1\end{bmatrix}\\
    C &= \begin{bmatrix}1 & 2 & -1\\1 & 1 & \phantom{-} 1\\2 & 1 & -1\end{bmatrix},
    D = \begin{bmatrix}\phantom{-} 1 & \phantom{-} 2 & -1\\-3 & -1 & \phantom{-} 3\\\phantom{-} 2 & \phantom{-} 1 & -1\end{bmatrix}
  \end{align*}
  Em cada caso, ache a matriz elementar $E$ que satisfaça a equação dada:
  \begin{multicols}{3}
    \begin{enumerate}[label={\alph*})]
      \item $EA = B$
      \item $EB = A$
      \item $EA = C$
      \item $EC = A$
      \item $EC = D$
      \item $ED = C$
      \item Existe uma matriz elementar $E$ tal que $EA = D$? Por quê?
    \end{enumerate}
  \end{multicols}
  \begin{solucao}
    \begin{enumerate}[label={\alph*})]
      \item $E = \begin{bmatrix}0 & 0 & 1\\0 & 1 & 0\\1 & 0 & 0\end{bmatrix}$
      \item $E = \begin{bmatrix}0 & 0 & 1\\0 & 1 & 0\\1 & 0 & 0\end{bmatrix}$
      \item $E = \begin{bmatrix}1 & 0 & 0\\0 & 1 & 0\\1 & 0 & 1\end{bmatrix}$
      \item $E = \begin{bmatrix}\phantom{-} 1 & 0 & 0\\\phantom{-} 0 & 1 & 0\\-1 & 0 & 1\end{bmatrix}$
      \item $E = \begin{bmatrix}1 & 0 & \phantom{-} 0\\0 & 1 & -2\\0 & 0 & \phantom{-} 1\end{bmatrix}$
      \item $E = \begin{bmatrix}1 & 0 & \phantom{-} 0\\0 & 1 & -2\\0 & 0 & \phantom{-} 1\end{bmatrix}$
    \end{enumerate}
  \end{solucao}
\end{exercicio}

Nos exerc{\'\i}cios \ref{matrizinicio} \`a \ref{matrizfim}, mostre que $B$ \'e a inversa de $A$.
\begin{exercicio}\label{matrizinicio}
  \[
    A =\begin{bmatrix}
      1 & -1\\
      2 & \phantom{-} 3
    \end{bmatrix}, B =\begin{bmatrix}
      \phantom{-} 3/5 & 1/5\\
      -2/5 & 1/5
    \end{bmatrix}
  \]
\end{exercicio}

\begin{exercicio}
  \[
    A =\begin{bmatrix}
      -2 & \phantom{-} 2 & 3\\
      \phantom{-} 1 & -1 & 0\\
      \phantom{-} 0 & \phantom{-} 1 & 4
    \end{bmatrix}, B =\begin{bmatrix}
      -4/3 & -5/3 & 1\\
      -4/3 & -8/3 & 1\\
      \phantom{-} 1/3 & \phantom{-} 2/3 & 0
    \end{bmatrix}
  \]
\end{exercicio}

\begin{exercicio}\label{matrizfim}
  \[
    A =\begin{bmatrix}
      \phantom{-} 2 & -17 & \phantom{-} 11\\
      -1 & 11 & -7\\
      \phantom{-} 0 & 3 & -2
    \end{bmatrix}, B =\begin{bmatrix}
      1 & 1 & \phantom{-} 2\\
      2 & 4 & -3\\
      3 & 6 & -5
    \end{bmatrix}
  \]
\end{exercicio}

Nos exerc{\'\i}cios \ref{matrizinversainicio} \`a \ref{matrizinversafim}, encontre a inversa da matriz dada, se existir.

\begin{exercicio}\label{matrizinversainicio}
  $
    A =\begin{bmatrix}
        1 & 2\\
        3 & 7
    \end{bmatrix}
  $
  \begin{solucao}
    $A^{-1} =\begin{bmatrix}
      \phantom{-} 7 & -2\\
      -3 & \phantom{-} 1
    \end{bmatrix}$
  \end{solucao}
\end{exercicio}

\begin{exercicio}
  $
    A =\begin{bmatrix}
        7 & 33\\
        \phantom{-} 4 & 19
    \end{bmatrix}
  $
  \begin{solucao}
    $A^{-1} =\begin{bmatrix}
      19 & -33\\
      -4 & 7
    \end{bmatrix}$
  \end{solucao}
\end{exercicio}

\begin{exercicio}
  $
    A =\begin{bmatrix}
        2 & 4\\
        4 & 8
    \end{bmatrix}
  $
  \begin{solucao}
    N\~ao possui inversa.
  \end{solucao}
\end{exercicio}

\begin{exercicio}
  $
    A =\begin{bmatrix}
        1 & 1 & 1\\
        3 & 5 & 4\\
        3 & 6 & 5
      \end{bmatrix}
    $
  \begin{solucao}
    $
      A^{-1} =\begin{bmatrix}
        \phantom{-} 1 & \phantom{-} 1 & -1\\
        -3 & \phantom{-} 2 & -1\\
        \phantom{-} 3 & -3 & \phantom{-} 2
      \end{bmatrix}
    $
  \end{solucao}
\end{exercicio}

\begin{exercicio}
  $
    A =\begin{bmatrix}
        1 & 2 & -1\\
        3 & 7 & -10\\
        7 & 16 & -21
    \end{bmatrix}
  $
  \begin{solucao}
    N\~ao possui inversa.
  \end{solucao}
\end{exercicio}

\begin{exercicio}
  $
    A =\begin{bmatrix}
        -8 & 0 & 0 & \phantom{-} 0\\
        \phantom{-} 0 & 1 & 0 & \phantom{-} 0\\
        \phantom{-} 0 & 0 & 0 & \phantom{-} 0\\
        \phantom{-} 0 & 0 & 0 & -5
    \end{bmatrix}
  $
  \begin{solucao}
   N\~ao possui inversa.
  \end{solucao}
\end{exercicio}

\begin{exercicio}\label{matrizinversafim}
  $
    A =\begin{bmatrix}
        \phantom{-} 1 & 1 & 2\\
        \phantom{-} 3 & 1 & 0\\
        -2 & 0 & 3
    \end{bmatrix}
  $
  \begin{solucao}
    $
      A^{-1} =\begin{bmatrix}
        -3/2 & \phantom{-} 3/2 & \phantom{-} 1\\
        \phantom{-} 9/2 & -7/2 & -3\\
        -1 & \phantom{-} 1 & \phantom{-} 1
      \end{bmatrix}
    $
  \end{solucao}
\end{exercicio}

\newpage
\Closesolutionfile{ans}
\hrule
\begin{center}
{\large\bf RESPOSTAS}
\end{center}
\hrule
\input{ans1}

\end{document}
