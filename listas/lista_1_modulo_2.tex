%!TEX program = xelatex
\def\numeromodulo{2}
\def\numerolista{1}
\documentclass[12pt]{exam}

\def\ano{2023}
\def\semestre{1}
\def\disciplina{Introdução à \'Algebra Linear}
\def\turma{3}

\usepackage{caption}
\usepackage{amssymb}
\usepackage{amsmath,amsfonts,amsthm,amstext}
\usepackage[brazil]{babel}
\usepackage{graphicx}
\graphicspath{{../Pictures/}}
\usepackage{enumitem}
\usepackage{multicol}
\usepackage{answers}
\usepackage[svgnames]{xcolor}
\usepackage{tikz}
\usepackage{ifthen}
\usetikzlibrary{lindenmayersystems}
\usetikzlibrary[shadings]

\Newassociation{solucao}{Solution}{ans}
\newtheorem{exercicio}{}

\setlength{\topmargin}{-1.0in}
\setlength{\oddsidemargin}{0in}
\setlength{\textheight}{10.1in}
\setlength{\textwidth}{6.5in}
\setlength{\baselineskip}{12mm}

\extraheadheight{0.7in}
\firstpageheadrule
\runningheadrule
\lhead{
        \begin{minipage}[c]{1.7cm}
        \includegraphics[width=1.7cm]{unb.pdf}
        \end{minipage}%
        \hspace{0pt}
        \begin{minipage}[c]{4in}
          {Universidade de Bras{\'\i}lia} --
          {Departamento de Matem{\'a}tica}
\end{minipage}
\vspace*{-0.8cm}
}
% \chead{Universidade de Bras{\'\i}lia - Departamento de Matem{\'a}tica}
% \rhead{}
% \vspace*{-2cm}

\extrafootheight{.5in}
\footrule
\lfoot{\disciplina\ - \semestre$^o$/\ano\ - Módulo \numeromodulo}
\cfoot{}
\rfoot{P\'agina \thepage\ de \numpages}

\newcounter{exercicios}
\renewcommand{\theexercicios}{\arabic{exercicios}}

\newenvironment{questao}[1]{
\refstepcounter{exercicios}
\ifx&#1&
\else
   \label{#1}
\fi
\noindent\textbf{Exerc{\'\i}cio {\theexercicios}:}
}

\newcommand{\resp}[1]{
\noindent{\bf Exerc{\'\i}cio #1: }}

\def\ano{2023}
\def\semestre{2}
\def\disciplina{Introdução à Álgebra Linear}
\def\nomeabreviado{IAL}
\def\turma{11}

\newcommand{\im}{{\rm Im\,}}
\newcommand{\dlim}[2]{\displaystyle\lim_{#1\rightarrow #2}}
\newcommand{\minf}{+\infty}
\newcommand{\ninf}{-\infty}
\newcommand{\cp}[1]{\mathbb{#1}}
\newcommand{\sub}{\subseteq}
\newcommand{\n}{\mathbb{N}}
\newcommand{\z}{\mathbb{Z}}
\newcommand{\rac}{\mathbb{Q}}
\newcommand{\real}{\mathbb{R}}
\newcommand{\complex}{\mathbb{C}}

\newcommand{\vesp}[1]{\vspace{ #1  cm}}

\newcommand{\compcent}[1]{\vcenter{\hbox{$#1\circ$}}}
\newcommand{\comp}{\mathbin{\mathchoice
        {\compcent\scriptstyle}{\compcent\scriptstyle}
        {\compcent\scriptscriptstyle}{\compcent\scriptscriptstyle}}}
\renewcommand{\sin}{{\rm sen\,}}
\renewcommand{\tan}{{\rm tg\,}}
\renewcommand{\csc}{{\rm cossec\,}}
\renewcommand{\cot}{{\rm cotg\,}}
\renewcommand{\sinh}{{\rm senh\,}}
\newcommand{\integer}{\mathbb{Z}}
\begin{document}

  \Opensolutionfile{ans}[ans1]
  \begin{center}
    {\Large\bf \disciplina\ - Turma \turma\ -- \semestre$^{o}$/\ano} \\ \vspace{9pt} {\large\bf
        $\numerolista^a$ Lista de Exerc{\'\i}cios -- Módulo \numeromodulo}\\ \vspace{9pt} Prof. Jos{\'e} Ant{\^o}nio O. Freitas
  \end{center}
  \hrule

\vesp{.6}

\begin{exercicio}
  Verifique se os conjuntos dados com as operações indicadas é ou não um espaço vetorial. Para os que não sã espaços vetoriais apresente um exemplo de qual propriedade falha.
  \begin{enumerate}[label={\alph*)}]
    \item $V = \{(x, 0) \mid x \in \real\}$ e $\cp{K} = \real$, onde
      \begin{itemize}
        \item $(x, 0) + (y, 0) = (x + y, 0)$, para todos $(x, 0)$, $(y, 0) \in V$.
        \item $\alpha(x, 0) = (\alpha x, 0)$, para todos $(x, 0) \in V$ e todo $\alpha \in \real$.
      \end{itemize}

    \item $V = \{(x, y) \mid x, y \in \real,\ y \ge 0\}$ e $\cp{K} = \real$, onde
      \begin{itemize}
        \item $(x, y) + (z, t) = (x + z, y + t)$, para todos $(x, y)$, $(z, t) \in V$.
        \item $\alpha(x, y) = (\alpha x, \alpha y)$, para todos $(x, y) \in V$ e todo $\alpha \in \real$.
      \end{itemize}
  
    \item $V = \{(x, x, x, x) \mid x \in \real\}$ e $\cp{K} = \real$, onde
      \begin{itemize}
        \item $(x, x, x, x) + (y, y, y, y) = (x + y, x + y, x + y, x + y)$, para todos $(x, x, x, x)$, $(y, y, y, y) \in V$.
        \item $\alpha(x, x, x, x) = (\alpha x, \alpha x, \alpha x, \alpha x)$, para todos $(x, x, x, x) \in V$ e todo $\alpha \in \real$.
      \end{itemize}
 
    \item $V = \{(x, y, z) \mid x, y, z \in \real\}$ e $\cp{K} = \real$, onde
      \begin{itemize}
        \item $(x, y, z) + (t, u, w) = (x + t, y + u, z + w)$, para todos $(x, y, z)$, $(t, u, w) \in V$.
        \item $\alpha(x, y, z) = (\alpha^2 x, \alpha^2 y, \alpha^2 z)$, para todos $(x, y, z) \in V$ e todo $\alpha \in \real$.
      \end{itemize}

    \item $V$ é o conjunto de todas as matrizes $2 \times 2$ invertíveis, com entradas em $\rac$ e $\cp{K} = \rac$, onde as operações usuais de adição e multiplicação por escalar.

    \item O conjunto
      \[
        V = \left\{\begin{bmatrix}a & 0\\0 & b \end{bmatrix} \mid a, b \in \real\right\}
      \]
      com as operações usuais de adição e multiplicação por escalar e $\cp{K} = \rac$.

    \item O conjunto
      \[
         V = \left\{\begin{bmatrix}x\\y \end{bmatrix} \mid x, y \in \complex\right\}
      \]
      com as operações usuais de adição e multiplicação por escalar e $\cp{K} = \real$.

    \item O conjunto
      \[
         V = \left\{\begin{bmatrix}x\\y \end{bmatrix} \mid x, y \in \real, xy \ge 0\right\}
      \]
      com as operações usuais de adição e multiplicação por escalar e $\cp{K} = \real$.

    \item O conjunto
      \[
         V = \left\{\begin{bmatrix}x\\y \end{bmatrix} \mid x, y \in \rac\right\}
      \]
      com as operações usuais de adição e multiplicação por escalar e  $\cp{K} = \complex$.

    \item O conjunto
      \[
         V = \left\{\begin{bmatrix}x\\y \end{bmatrix} \mid x, y \in \rac\right\}
      \]
      com a operação usual de adição e multiplicação por escalar definida por
      \[
        \alpha\begin{bmatrix}x\\y\end{bmatrix} = \begin{bmatrix}\alpha x\\y\end{bmatrix}
      \]
      para $\alpha \in \rac$.
      
    \item O conjunto
      \[
         V = \left\{\begin{bmatrix}x\\y \end{bmatrix} \mid x, y \in \complex\right\}
      \]
      com a operação usual multiplicação por escalar e a adição definida por
      \[
        \begin{bmatrix}x\\y\end{bmatrix} + \begin{bmatrix} z\\t\end{bmatrix} = \begin{bmatrix}x + z + 1\\y + t + 1\end{bmatrix}
      \]
      e $\cp{K} = \rac$.
    
    \item O conjunto
      \[
        V = \left\{\begin{bmatrix}a & d\\b & c \end{bmatrix} \mid a, b, c, d \in \real,\ ad = 0 \right\}
      \]
      com as operações usuais de adição e multiplicação por escalar.
    
    \item O conjunto
      \[
         V = \left\{\begin{bmatrix}a + bi\\a - bi \end{bmatrix} \mid a + bi \in \complex\right\}
      \]
      com a operações usuais de adição e multiplicação por escalar e $\cp{K} = \rac$.

    \item O conjunto
      \[
         V = \left\{\begin{bmatrix}x\\y \end{bmatrix} \mid x, y \in \complex\right\}
      \]
      com a operação usual de adição e multiplicação por escalar definida por
      \[
        (a + bi)\begin{bmatrix}x\\y\end{bmatrix} = \begin{bmatrix}(a - bi)x\\(a - bi)y\end{bmatrix}.
      \]


  \end{enumerate}
\end{exercicio}

\begin{exercicio}
  Mostre que os seguintes conjuntos s\~ao espa\c{c}os vetoriais:
  \begin{enumerate}[label={\alph*})]
    \item $\complex^n$ \'e um espa\c{c}o vetorial sobre $\real$ se definirmos:
    \begin{itemize}
      \item $(a_1, a_2, \dots, a_n) \oplus (b_1, b_2, \dots,b_n) = (a_1 + b_1, a_2 + b_2,\dots, a_n + b_n)$ para todos $(a_1, a_2, \dots,a_n)$ ,$(b_1, b_2, \dots,b_n) \in \complex^n$;

      \item $\alpha \otimes (a_1, a_2, \dots,a_n) = (\alpha a_1, \alpha a_2, \dots, \alpha a_n)$ para todo $\alpha \in \real$ e todo $(a_1, a_2, \dots, a_n) \in \complex^n$.
    \end{itemize}

    \item\label{itemb} Defina
    \[
      \mathcal{P}_5(\real) = \{ p(x) = a_5x^5 + a_4x^4 + a_3x^3 + a_2x^2 + a_1x + a_0 \mid a_i \in \real, 0 \le i \le 5 \}.
    \]
     Dados $p(x) = a_5x^5 + a_4x^4 + a_3x^3 + a_2x^2 + a_1x + a_0$ e $q(x) = b_5x^5 + b_4x^4 + b_3x^3 + b_2x^2 + b_1x + b_0$ em $\mathcal{P}_5(\real)$, defina:
    \begin{itemize}
      \item $p(x) \oplus q(x) =  (a_5 + b_n)x^n + (a_4 + b_4)x^4 + (a_3 + b_ 3)x^3 + (a_2 + b_2)x^2 + (a_1 + b_1)x + (a_0 + b_0)$
      \item $\alpha\otimes p(x) = \alpha a_5x^5 + \alpha a_4x^4 + \alpha a_3x^3 + \alpha a_2x^2 + \alpha a_1x + \alpha a_0$.
    \end{itemize}

   Mostre que $\mathcal{P}_5(\real)$ \'e um $\real$-espa\c{c}o vetorial.

    \item Defina
    \[
      \mathcal{P}_5(\complex) = \{ p(x) = a_5x^5 + a_4x^4 + a_3x^3 + a_2x^2 + a_1x + a_0 \mid a_i \in \complex, 0 \le i \le 5 \}.
    \]
    Considere as opera\c{c}\~oes de soma e produto como as definidas no item \ref{itemb}. Mostre que $\mathcal{P}_5(\complex)$ \'e um $\real$-espa\c{c}o vetorial.

    \item Defina
    \[
      \mathcal{P}_5(\complex) = \{ p(x) = a_5x^5 + a_4x^4 + a_3x^3 + a_2x^2 + a_1x + a_0 \mid a_i \in \complex, 0 \le i \le 5 \}.
    \]
    Considere as opera\c{c}\~oes de soma e produto como as definidas em \ref{itemb}. Mostre que $\mathcal{P}_5(\complex)$ \'e um $\complex$-espa\c{c}o vetorial.

    \item Para cada $m \ge 0$ defina
    \[
      \mathcal{P}_m(\cp{K}) = \{ p(x) = a_nx^n + a_{n - 1}x^{n - 1} + \cdots + a_1x + a_0 \mid a_i \in \cp{K}, 0 \le n \le m \}.
    \]
    Considere as opera\c{c}\~oes de soma e produto com as definidas em \ref{itemb}. Mostre que $\mathcal{P}_m(\cp{K})$ é um $\cp{K}$-espaço vetorial.
    
    \item Considere o conjunto dos polin\^omios
    \[
      \mathcal{P}(\cp{K}) = \{ p(x) = a_nx^n + a_{n - 1}x^{n - 1} + \cdots + a_1x + a_0 \mid a_i \in \cp{K}, i = 1, 2, \dots, n; n \ge 0 \}.
    \]
    Dados $p(x) = a_nx^n + a_{n - 1}x^{n - 1} + \cdots + a_1x + a_0$ e $q(x) = b_mx^m + b_{m - 1}x^{m - 1} + \cdots + b_1x + b_0$ em $\mathcal{P}(\cp{K})$, suponha que $n < m$ e defina:
    \begin{itemize}
      \item $p(x) \oplus q(x) = b_mx^m + b_{m - 1}x^{m - 1} + \cdots + b_{n + 1}x^{n + 1} + (a_n + b_n)x^n + (a_1 + b_1)x + (a_0 + b_0)$
      \item $\alpha\otimes p(x) = \alpha a_nx^n + \alpha a_{n - 1}x^{n - 1} + \cdots + \alpha a_1x + \alpha a_0$.
    \end{itemize}

    \item O conjunto das matrizes $\cp{M}_{p \times q}(\cp{K})$ com coeficientes em $\cp{K}$ \'e um $\cp{K}$-espa\c{c}o vetorial com a soma usual de matrizes e a multiplica\c{c}\~ao por escalar usual.

    \item Sejam $X$ um conjunto qualquer n\~ao vazio e $\mathcal{F}(X, \cp{K})$ o conjunto de todas as fun\c{c}\~oes $f : X \to \cp{K}$. Defina as seguintes opera\c{c}\~oes em $\mathcal{F}(X, \cp{K})$:
    \begin{itemize}
      \item  para $f$, $g \in \mathcal{F}(X, \cp{K})$, defina a fun\c{c}\~ao $f + g : X \to \cp{K}$ dada por $(f\oplus g)(x) = f(x) + g(x)$ para cada $x \in X$.
      \item para $f \in \mathcal{F}(X, \cp{K})$ e $\alpha \in \cp{K}$, defina a fun\c{c}\~ao $\alpha \otimes f: X \to \cp{K}$ dada por $(\alpha f)(x) = \alpha f(x)$ para cada $x \in X$.
    \end{itemize}
  \end{enumerate}
\end{exercicio}

\begin{exercicio}
  Suponha que estejam definidas as seguintes opera\c{c}\~oes no conjunto $V = \{(a, b) \in \real^2 \mid a, b > 0\}$:
  \begin{itemize}
    \item $(a, b) \oplus (c, d) = (ac, bd)$, para todos $(a,b)$, $(c,d) \in V$.
    \item $\alpha (a, b) = (a^\alpha, b^\alpha)$ para todo $\alpha \in \real$ e $(a, b) \in V$.
  \end{itemize}
  Prove que $V$ com essa opera\c{c}\~oes \'e um $\real$-espa\c{c}o vetorial.
\end{exercicio}

\begin{exercicio}
  Seja $V$ um espa\c{c}o vetorial sobre um corpo $\cp{K}$.
  \begin{enumerate}[label={\alph*})]
    \item Mostre que o vetor nulo de $V$ \'e \'unico.
    \item Mostre que o vetor oposto \'e \'unico.
    \item Mostre que $0_\cp{K} v = 0_V$ para todo $v \in V$.
    \item Mostre que $\alpha 0_V = 0_V$ para todo $\alpha \in \cp{K}$.
    \item Mostre que se $\alpha v = 0_V$, com $\alpha \in \cp{K}$ e $v \in V$, ent\~ao $\alpha = 0_\cp{K}$ ou $v = 0_V$.
  \end{enumerate}
\end{exercicio}

%\newpage
%\Closesolutionfile{ans}
%\hrule
%\begin{center}
%{\large\bf RESPOSTAS}
%\end{center}
%\hrule
%\input{ans1}


\end{document}
