%!TEX program = xelatex
%!TEX encoding = ISO-8859-1
\def\ano{2023}
\def\semestre{2}
\def\numerosemanas{16}
\def\disciplina{Introdu\c{c}\~ao \`a \'Algebra Linear}
\def\turma{03}
\def\nome{Jos\'e Ant\^onio O. Freitas}
\def\email{jfreitas@unb.br}

\documentclass[12pt]{exam}

%\usepackage[leqno]{amsmath}
%\usepackage{makeidx,graphics}
\usepackage{graphicx}
\graphicspath{{Pictures/}}
\usepackage{amssymb,amsmath,amsfonts,amsthm,amstext}
%\usepackage{color}
%\usepackage[latin1]{inputenc}
\usepackage[portuges]{babel}
\usepackage{enumerate}
\usepackage{hyperref}
\usepackage{url}
\usepackage{enumitem}
\hypersetup{
    colorlinks=true,
    linkcolor=blue,
    filecolor=magenta,
    urlcolor=cyan,
}

\newcommand{\real}{\mathbb{R}}

%\DeclareGraphicsRule{jpg}{*[}{}{`jpeg2eps #1.jpg}
%\input{seteps}
%\input{setbmp-dvips}

\setlength{\topmargin}{-1.0in}
\setlength{\oddsidemargin}{0in}
\setlength{\textheight}{10.1in}
\setlength{\textwidth}{6.5in}
\setlength{\baselineskip}{12mm}

\begin{document}
\extraheadheight{0.7in}
\firstpageheadrule
\runningheadrule
\lhead{
        \begin{minipage}[c]{1.7cm}
        \includegraphics[width=1.7cm]{unb.pdf}
        \end{minipage}%
        \hspace{0pt}
        \begin{minipage}[c]{4in}
          {Universidade de Bras{\'\i}lia} --
          {Departamento de Matem{\'a}tica}
\end{minipage}
\vspace*{-0.8cm}
}
% \chead{Universidade de Bras{\'\i}lia - Departamento de Matem{\'a}tica}
% \rhead{}
% \vspace*{-2cm}

\extrafootheight{.5in}
\footrule
\lfoot{\disciplina\ - Turma \turma\ - \semestre$^o$/\ano}
\cfoot{}
\rfoot{P\'agina \thepage\ de \numpages}

\begin{center}
{\large\bf Plano de Ensino -- \semestre$^{o}$/\ano} \\
{\large\bf \disciplina\ -- Turma \turma}\\
\end{center}
\hrule

\begin{flushleft}
  Prof.: \nome\\
  email: \href{mailto:{\email}}{\email}
  \end{flushleft}
  \hrule
  \bigskip \bigskip
\vspace{0.25cm}
\noindent {\bf{PROGRAMA:}} o curso contar\'a com \numerosemanas\ semanas dividas em 3 m\'odulos. O conte\'udo do curso abordar\'a os seguintes assuntos:
\begin{itemize}
    \item Matrizes e sistemas lineares. Determinantes.

    \item Espa\c{c}os vetoriais. Transforma\c{c}\~oes Lineares.

    \item Autovalores e autovetores. Diagonaliza\c{c}\~ao de operadores. Espa\c{c}os vetoriais com produto interno.

\end{itemize}

\vspace{0.5cm}
\noindent {\bf{BIBLIOGRAFIA:}}
\begin{itemize}

%    \item Arquivo com as \textbf{Notas de Aula} dispon{\'\i}vel em
%        \href{https://moodle.mat.unb.br/mod/resource/view.php?id=905}{Notas de Aula: \disciplina\ - \turma}.

    \item H. Anton, C. Rorres: {\it {\'A}lgebra Linear com Aplica\c{c}\~oes}, $10^a$ Ed., Editora Bookman, 2012.

    \item C. H. Edwards, D. E. Penney: {\it Introdu\c{c}\~ao \`a \'Algebra Linear}, Editora Prentice-Hall do Brasil.

    \item S. Lang: {\it \'Algebra Linear}, Editora Moderna, 2003.

    \item T. S. Blyth, E. F. Robertson: {\it Basic Linear Algebra}, $2^a$ Ed., Springer, 2002.
\end{itemize}

\noindent {\bf{SISTEMA DE AVALIA\c{C}\~{A}O:}} em cada um dos m�dulos o aluno receber� uma nota $M_i$, $i=1$, 2, 3, dada por
\[
    M_i = 15\%T_i + 85\%P_i, \quad 0 \le M_i \le 10,
\]
onde $T_i$ � a m�dia aritm�tica das notas dos testes em sala e $P_i$ � a nota da prova. Os testes $T_i$ ocorrer�o sempre nas quartas-feiras.
A partir das notas dos m�dulos, a nota final ($NF$) de cada estudante ser\'a dada por:
\[
NF = \dfrac{2M_1 + 3M_2 + 4M_3}{9}, \quad 0 \le NF \le 10.
\]
Ser\'a considerado aprovado o estudante que obtiver $NF \ge 5$.

\vspace{0.5cm}

\noindent \textbf{\textit{Testes em sala:}} ocorrer�o sempre �s quartas-feiras, durante os 20 minutos finais da aula. Ser�o compostos por at� 2 exerc�cios.


As provas ser\~ao realizadas no hor\'ario da aula e as datas est\~ao listadas abaixo e podem, a crit\'erio do professor, ser mudadas.

\begin{center}
    \begin{tabular}{c|c|c}
        \hline\hline
        \hspace{1cm}{\bf Prova}\hspace{1cm} & \hspace{3cm}{\bf Data}\hspace{3cm} & \hspace{1.7cm}{\bf Hor\'{a}rio}\hspace{1.7cm} \\
        \hline\hline
        $P_1$ & 04/10/23 (quarta-feira) \phantom{x} & 16:00 - 17:50 \\
        \hline
        $P_2$ & 13/11/23 (segunda-feira) \phantom{x} & 16:00 - 17:50 \\
        \hline
        $P_3$ & 14/12/23 (quarta-feira) \phantom{x} & 16:00 - 17:50 \\
        \hline\hline
    \end{tabular}
\end{center}

\vspace{0.5cm}
{\bf \noindent Men\c{c}\~{a}o Final:} ser\'{a} obtida da $NF$ de
acordo com as normas da UnB.
\begin{center}
    \begin{tabular}{c|c}
        \hline\hline
        \hspace{1cm}{Nota}\hspace{1cm} & \hspace{0.25cm}{Men\c{c}\~{a}o}\hspace{0.25cm}\\
        \hline\hline
        9,00 a 10,0 & SS \\
        \hline
        7,00 a 8,99 & MS \\
        \hline
        5,00 a 6,99 & MM \\
        \hline
        3,00 a 4,99 & MI \\
        \hline
        0,00 a 2,99  & II \\
        \hline\hline
    \end{tabular}
\end{center}
Receber{\'a} a men{\c c}{\~a}o {\bf SR} quem estiver reprovado por faltar mais de 25\%
das aulas.

\vspace{0.5cm}
\noindent {\bf{P\'{A}GINA DA TURMA:}} A p\'agina da disciplina est\'a dispon{\'\i}vel no endere\c{c}o
\begin{center}
    \href{https://moodle.mat.unb.br/course/view.php?id=55}{https://moodle.mat.unb.br}
\end{center}


\begin{itemize}
\item Toda a comunica\c{c}\~{a}o oficial do curso, inclusive a divulga\c{c}\~{a}o de
notas e gabaritos, se dar\'{a} atrav\'{e}s do {\em F\'{o}rum de Not\'{\i}cias} do
\textbf{MoodleMAT}.
\item No {\em F\'{o}rum de Debates} do \textbf{MoodleMAT} poder\~{a}o ser
postadas d\'{u}vidas que ser\~{a}o respondidas on-line pelos seus
colegas, pelo monitor dessa turma ou pelo professor.
\end{itemize}

\noindent {\bf{OBSERVA\c{C}\~{O}ES IMPORTANTES:}}

\begin{enumerate}[label={\arabic*})]
\item As provas ser\~{a}o individuais e sem qualquer tipo de
aux\'{\i}lio (calculadora, livros etc.), sendo vedado o empr\'{e}stimo de
qualquer material entre os alunos durante as avalia\c{c}\~{o}es. As
tentativas de fraude ser\~{a}o reprimidas com m\'{a}ximo rigor.

\item \'{E} vedado o uso de telefones celulares e quaisquer dispositivos
eletr\^{o}nicos pessoais durante a realiza\c{c}\~{a}o das atividades do curso
em sala de aula.

\item Ser\'{a} exigido documento de identifica\c{c}\~{a}o dos estudantes nos
dias de provas e testes.

\item A aus\^{e}ncia acarretar\'{a} nota zero em qualquer uma das
avalia\c{c}\~{o}es.

\item A crit\'{e}rio do professor, as datas das provas poder\~{a}o
ser alteradas.

\item A lista de presen\c{c}a ser\'{a} passada apenas uma vez
durante cada aula e est\'{a} sujeita a confirma\c{c}\~{a}o oral. O
estudante deve assin\'{a}-la de forma leg{\'\i}vel. {\'E} proibido assinar
com suas iniciais e \'{e} proibido assinar por outra pessoa.

\item Haver{\'a} avalia{\c c}{\~a}o quanto {\`a} clareza, apresenta{\c
c}{\~a}o e formaliza{\c c}{\~a}o na  resolu{\c c}{\~a}o das quest{\~o}es de
cada prova. A nota do aluno poder{\'a} ser alterada em raz{\~a}o da
inobserv{\^a}ncia desses par{\^a}metros.
% \item A comunica\c{c}\~ao entre o professor/monitores e estudantes ser\'a, preferencialmente,
%  estabelecida pelo f\'orum do \textbf{MoodleMAT}.
\end{enumerate}
\end{document}
