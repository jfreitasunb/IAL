%!TEX program = xelatex
%!TEX encoding = UTF-8
\def\ano{2023}
\def\semestre{2}
\def\numerosemanas{16}
\def\disciplina{Introdução à Álgebra Linear}
\def\turma{03}
\def\nome{José Antônio O. Freitas}
\def\email{jfreitas@unb.br}

\documentclass[12pt]{exam}

%\usepackage[leqno]{amsmath}
%\usepackage{makeidx,graphics}
\usepackage{graphicx}
\graphicspath{{Pictures/}}
\usepackage{amssymb,amsmath,amsfonts,amsthm,amstext}
%\usepackage{color}
%\usepackage[latin1]{inputenc}
\usepackage[portuges]{babel}
\usepackage{enumerate}
\usepackage{hyperref}
\usepackage{url}
\usepackage{enumitem}
\hypersetup{
    colorlinks=true,
    linkcolor=blue,
    filecolor=magenta,
    urlcolor=cyan,
}

\newcommand{\real}{\mathbb{R}}

%\DeclareGraphicsRule{jpg}{*[}{}{`jpeg2eps #1.jpg}
%\input{seteps}
%\input{setbmp-dvips}

\setlength{\topmargin}{-1.0in}
\setlength{\oddsidemargin}{0in}
\setlength{\textheight}{10.1in}
\setlength{\textwidth}{6.5in}
\setlength{\baselineskip}{12mm}

\begin{document}
\extraheadheight{0.7in}
\firstpageheadrule
\runningheadrule
\lhead{
        \begin{minipage}[c]{1.7cm}
        \includegraphics[width=1.7cm]{unb.pdf}
        \end{minipage}%
        \hspace{0pt}
        \begin{minipage}[c]{4in}
          {Universidade de Brasília} --
          {Departamento de Matemática}
\end{minipage}
\vspace*{-0.8cm}
}
% \chead{Universidade de Brasília - Departamento de Matemática}
% \rhead{}
% \vspace*{-2cm}

\extrafootheight{.5in}
\footrule
\lfoot{\disciplina\ - Turma \turma\ - \semestre$^o$/\ano}
\cfoot{}
\rfoot{Página \thepage\ de \numpages}

\begin{center}
{\large\bf Plano de Ensino -- \semestre$^{o}$/\ano} \\
{\large\bf \disciplina\ -- Turma \turma}\\
\end{center}
\hrule

\begin{flushleft}
  Prof.: \nome\\
  email: \href{mailto:{\email}}{\email}
  \end{flushleft}
  \hrule
  \bigskip \bigskip
\vspace{0.25cm}
\noindent {\bf{PROGRAMA:}} o curso contará com \numerosemanas\ semanas dividas em 3 módulos. O conteúdo do curso abordará os seguintes assuntos:
\begin{itemize}
    \item Matrizes e sistemas lineares. Determinantes.

    \item Espaços vetoriais. Transformações Lineares.

    \item Autovalores e autovetores. Diagonalização de operadores. Espaços vetoriais com produto interno.

\end{itemize}

\vspace{0.5cm}
\noindent {\bf{BIBLIOGRAFIA:}}
\begin{itemize}

%    \item Arquivo com as \textbf{Notas de Aula} disponível em
%        \href{https://moodle.mat.unb.br/mod/resource/view.php?id=905}{Notas de Aula: \disciplina\ - \turma}.

    \item H. Anton, C. Rorres: {\it Álgebra Linear com Aplicações}, $10^a$ Ed., Editora Bookman, 2012.

    \item C. H. Edwards, D. E. Penney: {\it Introdução à álgebra Linear}, Editora Prentice-Hall do Brasil.

    \item S. Lang: {\it Álgebra Linear}, Editora Moderna, 2003.

    \item T. S. Blyth, E. F. Robertson: {\it Basic Linear Algebra}, $2^a$ Ed., Springer, 2002.
\end{itemize}

\noindent {\bf{SISTEMA DE AVALIAÇÃO:}} em cada um dos módulos será atribuída uma nota $M_i$, $i=1$, 2, 3, dada por
\[
    M_i = 15\%T_i + 85\%P_i, \quad 0 \le M_i \le 10,
\]
onde $T_i$ é a média aritmética das notas dos testes em sala e $P_i$ é a nota da prova. Os testes $T_i$ ocorrerão sempre nas quartas-feiras.
A partir das notas dos módulos, a nota final ($NF$) de cada estudante será dada por:
\[
    NF = \dfrac{2M_1 + 3M_2 + 4M_3}{9}, \quad 0 \le NF \le 10.
\]
Será considerado aprovado na disciplina quem obtiver $NF \ge 5$.

\vspace{0.5cm}

\noindent \textbf{\textit{Testes em sala:}} ocorrerão sempre às quartas-feiras, durante os 20 minutos finais da aula. Serão compostos por até 2 exercícios.


As provas serão realizadas no horário da aula e as datas estão listadas abaixo e podem, a critério do professor, sofrer alteração.

\begin{center}
    \begin{tabular}{c|c|c}
        \hline\hline
        \hspace{1cm}{\bf Prova}\hspace{1cm} & \hspace{3cm}{\bf Data}\hspace{3cm} & \hspace{1.7cm}{\bf Horário}\hspace{1.7cm} \\
        \hline\hline
        $P_1$ & 04/10/23 (quarta-feira) \phantom{x} & 16:00 - 17:50 \\
        \hline
        $P_2$ & 13/11/23 (segunda-feira) \phantom{x} & 16:00 - 17:50 \\
        \hline
        $P_3$ & 14/12/23 (quarta-feira) \phantom{x} & 16:00 - 17:50 \\
        \hline\hline
    \end{tabular}
\end{center}

\vspace{0.5cm}
{\bf \noindent Menção Final:} será obtida da $NF$ de acordo com as normas da UnB.
\begin{center}
    \begin{tabular}{c|c}
        \hline\hline
        \hspace{1cm}{Nota}\hspace{1cm} & \hspace{0.25cm}{Menção}\hspace{0.25cm}\\
        \hline\hline
        9,00 a 10,0 & SS \\
        \hline
        7,00 a 8,99 & MS \\
        \hline
        5,00 a 6,99 & MM \\
        \hline
        3,00 a 4,99 & MI \\
        \hline
        0,00 a 2,99  & II \\
        \hline\hline
    \end{tabular}
\end{center}
Receberá a menção {\bf SR} quem estiver reprovado por faltar mais de 25\% das aulas.

\vspace{0.5cm}
\noindent {\bf{PÁGINA DA TURMA:}} A página da disciplina está disponível no endereço
\begin{center}
    \href{https://moodle.mat.unb.br/course/view.php?id=55}{https://moodle.mat.unb.br}
\end{center}

\begin{itemize}
\item Toda a comunicação oficial do curso, inclusive a divulgação de
notas e gabaritos, se dará através do {\em Fórum de Notícias} do
\textbf{MoodleMAT}.
\item No {\em Fórum de Debates} do \textbf{MoodleMAT} poderão ser
postadas dúvidas que serão respondidas on-line pelos seus
colegas, pelo monitor dessa turma ou pelo professor.
\end{itemize}

\noindent {\bf{OBSERVAçõES IMPORTANTES:}}

\begin{enumerate}[label={\arabic*})]
\item As provas serão individuais e sem qualquer tipo de auxílio (calculadora, livros etc.), sendo vedado o empréstimo de
qualquer material entre os alunos durante as avaliações. As tentativas de fraude serão reprimidas com máximo rigor.

\item É vedado o uso de telefones celulares e quaisquer dispositivos eletrônicos pessoais durante a realização das atividades do curso
em sala de aula.

\item Será exigido documento de identificação dos estudantes nos dias de provas e testes.

\item A auência acarretará nota zero em qualquer uma das avaliações.

\item A critério do professor, as datas das provas poderão ser alteradas.

\item A lista de presença será passada apenas uma vez durante cada aula e está sujeita a confirmação oral. O
estudante deve assiná-la de forma legível. {é} proibido assinar com suas iniciais e é proibido assinar por outra pessoa.

\item Haverá avaliação quanto à clareza, apresentação e formalização na  resolução das questões de cada prova. A nota do aluno poderá ser alterada em razão da
inobservância desses parâmetros.
% \item A comunicação entre o professor/monitores e estudantes será, preferencialmente,
%  estabelecida pelo fórum do \textbf{MoodleMAT}.
\end{enumerate}
\end{document}