%!TEX program = xelatex
\def\ano{2023}
\def\semestre{1}
\def\disciplina{Introdução à Álgebra Linear}
\def\turma{3}
\def\autor{Jos\'e Ant\^onio O. Freitas}
\def\instituto{MAT-UnB}

\documentclass{beamer}
\usetheme{Madrid}
\usecolortheme{beaver}
% \mode<presentation>
\usepackage{caption}
\usepackage{amssymb}
\usepackage{amsmath,amsfonts,amsthm,amstext}
\usepackage[brazil]{babel}
\usepackage{graphicx}
\graphicspath{{../Pictures/}}
\usepackage{enumitem}
\usepackage{multicol}
\usepackage{answers}
\usepackage[svgnames]{xcolor}
\usepackage{tikz}
\usepackage{ifthen}
\usetikzlibrary{lindenmayersystems}
\usetikzlibrary[shadings]

\newcounter{exercicios}
\setcounter{exercicios}{0}
\newcommand{\questao}{
    \addtocounter{exercicios}{1}
    \noindent{\bf Quest{\~a}o \arabic{exercicios}: }}

\newcommand{\resp}[1]{
    \noindent{\bf Exerc{\'\i}cio #1: }}

\extrafootheight[.25in]{.5in}
\footrule
\lfoot{Teste \numeroteste\ - Módulo \modulo\ - \nomeabreviado\ - Turma \turma\ - \semestre$^o$/\ano}
\cfoot{}
\rfoot{P\'agina \thepage\ de \numpages}
\def\ano{2023}
\def\semestre{2}
\def\disciplina{Introdução à Álgebra Linear}
\def\nomeabreviado{IAL}
\def\turma{11}

\newcommand{\im}{{\rm Im\,}}
\newcommand{\dlim}[2]{\displaystyle\lim_{#1\rightarrow #2}}
\newcommand{\minf}{+\infty}
\newcommand{\ninf}{-\infty}
\newcommand{\cp}[1]{\mathbb{#1}}
\newcommand{\sub}{\subseteq}
\newcommand{\n}{\mathbb{N}}
\newcommand{\z}{\mathbb{Z}}
\newcommand{\rac}{\mathbb{Q}}
\newcommand{\real}{\mathbb{R}}
\newcommand{\complex}{\mathbb{C}}

\newcommand{\vesp}[1]{\vspace{ #1  cm}}

\newcommand{\compcent}[1]{\vcenter{\hbox{$#1\circ$}}}
\newcommand{\comp}{\mathbin{\mathchoice
        {\compcent\scriptstyle}{\compcent\scriptstyle}
        {\compcent\scriptscriptstyle}{\compcent\scriptscriptstyle}}}
\renewcommand{\sin}{{\rm sen\,}}
\renewcommand{\tan}{{\rm tg\,}}
\renewcommand{\csc}{{\rm cossec\,}}
\renewcommand{\cot}{{\rm cotg\,}}
\renewcommand{\sinh}{{\rm senh\,}}

\title{Sistemas Lineares}
\author[\autor]{\autor}
\institute[\instituto]{\instituto}
\date{}

\begin{document}
    \begin{frame}
        \maketitle
    \end{frame}

    \logo{\includegraphics[scale=.1]{logo-MAT.png}\vspace*{8.5cm}}

    \begin{frame}
        Seja $\cp{K} = \rac$, $\real$ ou $\complex$.\pause

        \vspace{.3cm}

        Vamos trabalhar com \textbf{equações lineares} \pause em $q$ variáveis $x_1$, $x_2$, \dots, $x_q$ \pause que são equações que podem ser escritas na forma
        \begin{equation}\label{equacao_linear}
            a_1x_1 \pause + a_2x_2 + \pause \cdots + a_qx_q \pause = b
        \end{equation}

        onde $a_1$, $a_2$, \dots, $a_q$, \pause $b$ são escalares no corpo $\cp{K}$, \pause sendo que nem todos os coeficientes $a_i$ são nulos.
        
        \vspace{.3cm}

        No caso em que $b = 0$, \pause a equação \eqref{equacao_linear} tem a forma\pause
        \begin{equation}
            a_1x_1 + a_2x_2 + \cdots + a_qx_q = 0\pause
        \end{equation}
        e é chamada de uma \textbf{equação linear homogênea}.
        
    \end{frame}

    \begin{frame}
        Seja $\cp{K} = \real$.\pause

        \vspace{.3cm}
        
        Dado um conjunto de equações da forma
        \begin{align}
            \begin{cases}\label{sistema_linear_2x2}
                ax + by = b_1\\
                cx + dy = b_2
            \end{cases}
        \end{align}
        onde $a$, $b$, $c$, $d$, $b_1$, $b_2 \in \cp{K}$.\pause
    
        \vspace{.3cm}

        Existem valores de $x$, $y \in \cp{K}$ que satisfazem as duas equações simultaneamente?
    \end{frame}

    \begin{frame}
        \definecolor{qqqqff}{rgb}{0,0,1}
        \definecolor{ffqqqq}{rgb}{1,0,0}
        \begin{center}
            \begin{tikzpicture}[line cap=round,line join=round,>=triangle 45,x=1cm,y=1cm]
                \begin{axis}[
                    x=1cm,y=1cm,
                    axis lines=middle,
                    xticklabels={,,},
                    yticklabels={,,},
                    xlabel=$x$,
                    ylabel=$y$,
                    xmin=-2,
                    xmax=5,
                    ymin=-1.3,
                    ymax=3,]
                    \draw [line width=5.2pt,dash pattern=on 1pt off 1pt,color=ffqqqq,domain=-1.348257839721257:5.807994044102577] plot(\x,{(--4-1*\x)/2});
                    \draw [line width=2pt,color=qqqqff,domain=-1.348257839721257:5.807994044102577] plot(\x,{(--8-2*\x)/4});
                    \begin{scriptsize}
                        \draw[color=ffqqqq] (1.8,2) node {$a_{11}x + a_{12}y = b_1$};
                        \draw[color=qqqqff] (3,1.5) node {$a_{21}x + a_{22}y = b_2$};
                    \end{scriptsize}
                \end{axis}
            \end{tikzpicture}
        \end{center}

        Existem infinitos valores de $x$, $y \in \cp{K}$ \pause que satisfazem as duas equações simultanemente. \pause Neste caso dizemos que o sistema \eqref{sistema_linear_2x2} \pause é \textbf{possível e indeterminado}.\pause
    \end{frame}

    \begin{frame}
        \definecolor{ffqqqq}{rgb}{1,0,0}
        \definecolor{qqqqff}{rgb}{0,0,1}
        \begin{center}
            \begin{tikzpicture}[line cap=round,line join=round,>=triangle 45,x=1cm,y=1cm]
                \begin{axis}[
                    x=1cm,y=1cm,
                    axis lines=middle,
                    xticklabels={,,},
                    yticklabels={,,},
                    xlabel=$x$,
                    ylabel=$y$,
                    xmin=-5.3,
                    xmax=6.3,
                    ymin=-1,
                    ymax=4.3,]
                    \draw [line width=2pt,color=qqqqff,domain=-5:6] plot(\x,{(--4-1*\x)/5});
                    \draw [line width=2pt,color=ffqqqq,domain=-5:6] plot(\x,{(--16-1*\x)/5});
                    \begin{scriptsize}
                        \draw[color=qqqqff] (-3.8,2.2) node {$a_{21}x + a_{22}y = b_2$};
                        \draw[color=ffqqqq] (-3.8,3.4) node {$a_{11}x + a_{12}y = b_1$};
                    \end{scriptsize}
                \end{axis}
            \end{tikzpicture}
        \end{center}

        Não existem valores de $x$, $y \in \cp{K}$ \pause satisfazendo as duas equações simultaneamente. \pause Nesse caso dizemos que o sistema \eqref{sistema_linear_2x2} \pause é \textbf{impossível}.

    \end{frame}

    \begin{frame}
        \definecolor{qqqqff}{rgb}{0,0,1}
        \definecolor{ffqqqq}{rgb}{1,0,0}
        \begin{center}
            \begin{tikzpicture}[line cap=round,line join=round,>=triangle 45,x=1cm,y=1cm]
                \begin{axis}[
                    x=1cm,y=1cm,
                    axis lines=middle,
                    xticklabels={,,},
                    yticklabels={,,},
                    xlabel=$x$,
                    ylabel=$y$,
                    xmin=-2,
                    xmax=4.3,
                    ymin=-2,
                    ymax=3.5,]
                    \draw [line width=2pt,color=ffqqqq,domain=-4:5] plot(\x,{(--4.94--7.92*\x)/7.58});
                    \draw [line width=2pt,color=qqqqff,domain=-4:5] plot(\x,{(--4.824-8.66*\x)/8.24});
                    \begin{scriptsize}
                        \draw[color=ffqqqq] (2.9,2) node {$a_{11}x + a_{12}y = b_1$};
                        \draw[color=qqqqff] (2.7,-0.5) node {$a_{21}x + a_{22}y = b_2$};
                    \end{scriptsize}
                \end{axis}
            \end{tikzpicture}
        \end{center}
        
        Nesse caso existe um único conjunto de valores $x = \alpha \in \cp{K}$ e $y = \lambda \in \cp{K}$ \pause que satisfaz as duas equações simultaneamente. \pause Neste caso dizemos que o sistema \eqref{sistema_linear_2x2} \pause é \textbf{possível e determinado}.\pause
    \end{frame}

    \begin{frame}
        Esse mesmo tipo de análise pode ser aplicada num sistema da forma
        \begin{align}
            \begin{cases}\label{sistema_linear_3x3}
                a_1x + a_2y + a_3z = b_1\\
                a_4x + a_5y + a_6z = b_2\\
                a_7x + a_8y + a_9z = b_3
            \end{cases}
        \end{align}
        onde $a_i \in \cp{K}$ para $1 \le i \le 9$ e $b_j \in \cp{K}$ para $1 \le j \le 3$.\pause

        \vspace{.3cm}

        Existem valores de $x$, $y$ e $z \in \cp{K}$ que satisfaçam as três equações simultaneamente?
    \end{frame}
    
    \begin{frame}
        Como no caso anterior, temos as seguintes possibilidades:\pause
        \begin{enumerate}[label={\roman*})]
            \item Existe um único conjunto de valores $x = \alpha \in \cp{K}$, $y = \lambda \in \cp{K}$ e $z = \gamma \in \cp{K}$ \pause que satisfaz as três equações simultaneamente. \pause Neste caso dizemos que o sistema \eqref{sistema_linear_3x3} \pause é \textbf{possível e determinado}.\pause

            \item Existem infinitos valores de $x$, $y$, $z \in \cp{K}$ \pause que satisfazem as três equações simultanemente. \pause Neste caso dizemos que o sistema \eqref{sistema_linear_3x3} \pause é \textbf{possível e indeterminado}.\pause

            \item Não existem valores de $x$, $y$, $z \in \cp{K}$ \pause satisfazendo as três equações simultaneamente. \pause Nesse caso dizemos que o sistema \eqref{sistema_linear_3x3} \pause é \textbf{impossível}.
        \end{enumerate}
    \end{frame}

    \begin{frame}

        Escolha escalares $b_1$, \dots, $b_m$ \pause e $a_{ij}$, \pause $1 \le i \le m$, $1 \le j \le n$ todos em $\cp{K}$.

        \vspace{.3cm}

        Queremos saber se é possível encontrar valores para  $x_1$, $x_2$, \dots, $x_n$ \pause de modo que o seguinte conjunto de equações sejam válidas: \pause
        \begin{equation}\label{sistema_linear_geral}
	    \begin{cases}
                a_{11}x_1 + a_{12}x_2 + \cdots + a_{1n}x_n = b_1\\\pause
                a_{21}x_1 + a_{22}x_2 + \cdots + a_{2n}x_n = b_2\\\pause
                \qquad \vdots\\
                a_{m1}x_1 + a_{m2}x_2 + \cdots + a_{mn}x_n = b_m\pause
            \end{cases}
        \end{equation}
        
        O conjunto de equações em \eqref{sistema_linear_geral} é chamado de um \pause \textbf{sistema de $m$ equa\c{c}\~oes lineares \pause a $n$ inc\'ognitas} $x_1$, $x_2$, \dots, $x_n$\pause , ou simplesmente de um \textbf{sistema linear}, nas incógnitas \pause $x_1$, $x_2$, \dots, $x_n$.
    \end{frame}

    \begin{frame}
        \vspace{.3cm}

        Uma solução de um sistema linear do tipo \eqref{sistema_linear_geral}\pause
        \[
            x_1 = \alpha_1, \pause x_2 = \alpha_2, \pause \dots, x_n = \alpha_n\pause
        \]
        onde $\alpha_1$, $\alpha_2$, \dots, $\alpha_n \in \cp{K}$, \pause pode ser escrita como\pause
        \[
            (\alpha_1, \alpha_2, \dots, \alpha_n)\pause
        \]
        e é chamada de uma \textbf{ênupla ordenada} \pause ou uma \textbf{n-upla ordenada}.
    \end{frame}

    \begin{frame}
        Se $b_1 = b_2 = \cdots = b_m = 0_\cp{K} \in K$, \pause dizemos que o sistema\pause
        \begin{equation}\label{sistemalinearhomogeneo}\index{Sistema Linear}
            \begin{cases}
                a_{11}x_1 + a_{12}x_2 + \cdots + a_{1n}x_n = 0_\cp{K}\\\pause
                a_{21}x_1 + a_{22}x_2 + \cdots + a_{2n}x_n = 0_\cp{K}\\\pause
                \qquad \vdots\\
                a_{m1}x_1 + a_{m2}x_2 + \cdots + a_{mn}x_n = 0_\cp{K}\pause
            \end{cases}
        \end{equation}
        \'e um \textbf{sistema linear homog\^eneo}. \pause

        \vspace{.3cm}

        Observe que tal sistema sempre possui solu\c{c}\~ao, \pause a saber, $x_1 = x_2 = \cdots = x_n = 0_\cp{K}$.
    \end{frame}

    \begin{frame}
        \begin{teorema}
            Todo sistema linear do tipo \eqref{sistema_linear_geral} tem zero, \pause uma \pause ou uma infinidade de soluções. \pause Não existem outras possibilidades.
        \end{teorema}
    \end{frame}

    \begin{frame}
        No caso de um sistema linear da forma \eqref{sistema_linear_geral}, \pause o processo para encontrar suas soluções ser\'a feito mediante o uso de 3 tipos de opera\c{c}\~oes. \pause S\~ao elas:\pause
        \begin{itemize}
	    \item[$e_1$)] Troca da posi\c{c}\~ao de duas equa\c{c}\~oes.\pause
	    \item[$e_2$)] Multiplica\c{c}\~ao de uma equa\c{c}\~ao por um escalar n\~ao nulo.\pause
	    \item[$e_3$)] Substitui\c{c}\~ao de uma equa\c{c}\~ao pela soma desta equa\c{c}\~ao com alguma outra.\pause
        \end{itemize}

        Estas tr\^es opera\c{c}\~oes s\~ao chamadas de \pause \textbf{opera\c{c}\~oes elementares}.
    \end{frame}

    \begin{frame}
        Uma outra matriz que podemos associar ao sistema \eqref{sistema_linear_geral} \'e\pause
        \[
	    \begin{bmatrix}
                a_{11} & a_{12} & \cdots & a_{1n} & b_1\\
		a_{21} & a_{22} & \cdots & a_{2n} & b_2\\
		\vdots & \vdots & \vdots & \vdots & \vdots\\
		a_{m1} & a_{m2} & \cdots & a_{mn} & b_m\\
            \end{bmatrix}
        \]

        que \'e chamada de \textbf{matriz ampliada do sistema} \pause ou \textbf{matriz aumentada do sistema}.
    \end{frame}

    \begin{frame}
        Na forma matricial \pause as opera\c{c}\~oes elementares s\~ao descritas como:\pause

        \vspace{.3cm}

        \begin{itemize}
            \item[$e_1$)] Trocar a $i$-\'esima linha de $A$ \pause pela $j$-\'esima linha de $A$: \pause $L_i \leftrightarrow L_j$;\pause

            \vspace{.3cm}

            \item[$e_2$)] Multiplica\c{c}\~ao da $i$-\'esima linha de $A$ \pause por um escalar $\alpha \in \cp{K}$ n\~ao nulo: \pause $L_i \rightarrow \alpha L_i$;\pause
 
            \vspace{.3cm}

           \item[$e_3$)] Substitui\c{c}\~ao da $i$-\'esima linha de $A$ \pause pela $i$-\'esima linha mais $\alpha$ vezes a $j$-\'esima linha: \pause $L_i \rightarrow L_i + \alpha L_j$.
        \end{itemize}
    \end{frame}

    \begin{frame}
        \begin{definicao}\label{linhareduzida}
            Uma matriz $A$ $m \times n$ \'e \pause dita estar na \textbf{forma escalonada reduzida por linhas} se:\pause
            \begin{enumerate}[label={\roman*})]
                \item O primeiro elemento n\~ao nulo \pause em cada linha n\~ao nula de $A$ \pause \'e $1$. \pause Dizemos que esse número 1 é um \textbf{pivô}.\pause

                \vspace{.3cm}

                \item Toda linha de $A$ cujos elementos s\~ao todos nulos \pause ocorre abaixo de todas as linhas que possuem um elemento n\~ao-nulo.\pause 
 
                \vspace{.3cm}

                \item Se as linhas 1, 2, \dots, $r$ s\~ao as linhas n\~ao-nulas de $A$ \pause e se o \textbf{pivô} da linha $i$ ocorre na coluna $k_i$, \pause $i = 1$, \dots, $r$, \pause ent\~ao $k_1 < k_2 < \cdots < k_r$.\pause
 
                \vspace{.3cm}

                \item Cada coluna de $A$ que cont\'em um \textbf{pivô} \pause tem todos os seus outros elementos nulos.
            \end{enumerate}
        \end{definicao}
    \end{frame}

    \begin{frame}
        \begin{observacao}
            Uma matriz que satisfaz as três primeiras propriedades da definição anterior \pause é dita estar na \textbf{forma escalonada por linhas}, \pause ou simplesmente, em \textbf{forma escalonada}.
        \end{observacao}
    \end{frame}

    \begin{frame}
        \begin{definicao}
	    Se $A$ e $B$ s\~ao matrizes $m \times n$, \pause dizemos que $B$ \'e \textbf{equivalente por linha} \pause a $A$, \pause se $B$ for obtida de $A$ \pause atrav\'es de uma quantidade finita de opera\c{c}\~oes elementares sobre as linhas de $A$.\pause
        \end{definicao}
        
        \vspace{.6cm}

        \begin{notacao}
	    $A \rightarrow B$ \pause ou $A \sim B$.
        \end{notacao}
    \end{frame}

    \begin{frame}
        \begin{teorema}
            Duas matrizes $A$ e $B$ são equivalentes por linha se, e somente, se elas puderem ser reduzidas à mesma forma escalonada por linhas.
        \end{teorema}
    \end{frame}

    \begin{frame}
        \begin{teorema}
            Se as matrizes ampliadas \pause de dois sistemas lineares são equivalentes por linhas, \pause então os dois sistemas possuem as mesmas soluções.
        \end{teorema}
    \end{frame}

    \begin{frame}
        O \textbf{método de eliminação de Gauss} \pause ou \textbf{método de eleminação gaussiana} \pause consiste em substituir um dado sistema de equações lineares \pause por outro \textbf{equivalente}, que seja mais simples de ser solucionado \pause e que tenha a mesma solução do sistema original.\pause

        \begin{definicao}[Método de eliminaçao de Gauss]
            O método de \textbf{eliminação de Gauss}, \pause para solução de um sistema linear, \pause consiste em:\pause
            \begin{enumerate}[label={\roman*})]
                \item Escreva a matriz ampliada do sistema de equações lineares.\pause

                \item Use operações elementares nas linhas de $A$ para reduzir a matriz ampliada à \textbf{forma escalonada por linhas}.\pause
                
                \item Quando a matriz ampliada estiver na forma escalonada, usando substituição de trás para a frente, \pause resolva o sistema equivalente que corresponde a matriz escalonada reduzida por linhas.
            \end{enumerate}
        \end{definicao}
    \end{frame}
    
    \begin{frame}
        \begin{observacao}
            Ao se aplicar o método de eliminação gaussiana, as seguintes dicas podem ser úteis:\pause
            \begin{enumerate}[label=({\alph*})]
                \item Localize a coluna mais à esquerda que não é toda formada por zeros.\pause
                
                \vspace{.3cm}

                \item Crie um \textbf{pivô} no topo desta coluna.\pause
                 
                \vspace{.3cm}

                \item Use o \textbf{pivô} para criar zeros abaixo dele.\pause
 
                \vspace{.3cm}

                \item Faça a linha contendo este \textbf{pivô} ir para a parte de cima \pause e volte ao passo (a) para repetir o procedimento com o restante da submatriz. \pause Pare quando toda a matriz estiver na forma escalonada por linhas.
            \end{enumerate}
        \end{observacao}
    \end{frame}

    \begin{frame}
        \begin{definicao}
            O \textbf{posto}, \pause denotado $\p{A}$ \pause ou $p(A)$, \pause de uma matriz $A$ \pause é número de linhas não nulas \pause de qualquer uma de suas formas escalonadas por linhas.\pause
        \end{definicao}
    \end{frame}

    \begin{frame}
        \begin{definicao}
            Seja $A$ a matriz ampliada de um sistema linear. \pause Se $A$ está na forma escalonada reduzida por linhas, \pause então as variáveis desse sistema \pause que não correspondem aos pivôs \pause são chamadas de \textbf{variáveis livres}.
        \end{definicao}
    \end{frame}

    \begin{frame}
        Considere o sistema linear:
        \begin{equation*}
	        \begin{cases}
                a_{11}x_1 + a_{12}x_2 + \cdots + a_{1n}x_n = b_1\\\pause
                a_{21}x_1 + a_{22}x_2 + \cdots + a_{2n}x_n = b_2\\\pause
                \qquad \vdots\\
                a_{m1}x_1 + a_{m2}x_2 + \cdots + a_{mn}x_n = b_m\pause
            \end{cases}
        \end{equation*}
        A matriz
        \[
            A = \begin{bmatrix}
                a_{11} & a_{12} & \cdots & a_{1n}\\
                a_{21} & a_{22} & \cdots & a_{2n}\\
                \vdots & \vdots & \cdots & \vdots\\
                a_{m1} & a_{m2} & \cdots & a_{mn}
            \end{bmatrix}
        \]

        é chamada de \textbf{matriz dos coeficientes} do sistema linear.
    \end{frame}

    \begin{frame}

        \begin{teorema}[Teorema do Posto]
            Seja $A$ a matriz dos coeficientes de um sistema linear \pause com $n$ variáveis. Então\pause
            \[
                \mbox{número de variáveis livres } = \pause n - \p{A}.
            \]
        \end{teorema}
    \end{frame}

    \begin{frame}
        Durante a aplicação do \textbf{método de eliminação de Gauss}, \pause paramos quando a matriz ampliada do sistema está na forma escalonada.
        Todavia podemos continuar aplicando operações elementares até que a matriz atinja a forma escalonada reduzida por linhas. Esse é o caso do 
        \textbf{método de Gaus-Jordan}.

        \begin{definicao}[Método de eliminaçao de Gauss-Jordan]
            O método de \textbf{eliminação de Gauss-Jordan}, \pause para solução de um sistema linear, consiste em:\pause
            \begin{enumerate}[label={\roman*})]
                \item Escreva a matriz ampliada do sistema de equações lineares.\pause

                \item Use operações elementares nas linhas de $A$ para reduzir a matriz ampliada à \textbf{forma escalonada reduzida por linhas}.\pause
                
                \item Se o sistema resultante for possível, resolva-o para as variáveis dependentes em termos de quaisquer variáveis livres que tenha sobrado.
            \end{enumerate}
        \end{definicao}
    \end{frame}

    \begin{frame}
        \begin{teorema}
            Um sistema linear homogêneo com mais incógnitas que equações \pause tem uma infinidade de soluções.
        \end{teorema}
    \end{frame}


    \begin{frame}
        \begin{definicao}
            Seja $A$ uma matriz quadrada de ordem $n$ \pause e $A \ne 0$. \pause Se for possível encontrar uma matriz quadrada $B$ \pause também de 
            ordem $n$ tal que\pause
            \[
                AB = I_n \pause = BA\pause   
            \]
            onde $I_n$ é a matriz identidade de ordem $n$, \pause então diremos que $A$ é \textbf{invertível}, \pause ou \textbf{não singular}, \pause 
            e que $B$ é a \textbf{inversa} de $A$. \pause Se não pudermos encontrar tal matriz $B$, \pause então diremos que $A$ é \textbf{não invertível} \pause 
            ou \textbf{singular}.
        \end{definicao}
    \end{frame}

    \begin{frame}
        \begin{teorema}
            Se $B$ e $C$ são ambas inversas da matriz $A$, \pause então $B = C$.\pause
        \end{teorema}

        \vspace{1cm}

        \begin{notacao}
            Se $A$ é uma matriz invertível \pause e $B$ é a sua inversa, \pause vamos escrever $B = A^{-1}$.
        \end{notacao}
    \end{frame}

    \begin{frame}
        \begin{proposicao}
            Se $A$ e $B$ são matrizes invertíveis de mesmo tamanho, \pause então $AB$ é invertível \pause e
            \[
                (AB)^{-1} = B^{-1}\pause A^{-1}.
            \]
        \end{proposicao}
    \end{frame}

    \begin{frame}
        \begin{proposicao}
            Seja $A$ uma matriz invertível \pause e $n$ um número inteiro não negativo. \pause Então:
            \begin{enumerate}[label={\roman*})]
                \item $A^{-1}$ é invertível \pause e $(A^{-1})^{-1} = A$.\pause

                \item $A^n$ é invertível \pause e $(A^n)^{-1} = A^{-n} = (A^{1})^n$.\pause

                \item $kA$ é invertível \pause para todo escalar $k$ não nulo \pause e $(kA)^{-1} = k^{-1}A^{-1}$.
            \end{enumerate}
        \end{proposicao}
    \end{frame}

    \begin{frame}
        \begin{enumerate}[label={\arabic*})]
            \item Seja $A$ uma matriz quadrada de ordem $n$. \pause Como podemos encontrar a inversa de $A$, \pause se existir.\pause

            \vspace{1cm}

            \item Dada uma matriz $A$, \pause como podemos decidir se $A$ é invertível?\pause
        \end{enumerate}
    \end{frame}

    \begin{frame}
        Considere o sistema linear: \pause
        \begin{equation}
	    \begin{cases}
                a_{11}x_1 + a_{12}x_2 + \cdots + a_{1n}x_n = b_1\\\pause
                a_{21}x_1 + a_{22}x_2 + \cdots + a_{2n}x_n = b_2\\\pause
                \qquad \vdots\\
                a_{m1}x_1 + a_{m2}x_2 + \cdots + a_{mn}x_n = b_m\pause
            \end{cases}
        \end{equation}
        A esse sistema podemos associar algumas matrizes. \pause A saber:
        \[
            A = \begin{pmatrix}
                a_{11} & a_{12} & \cdots & a_{1n}\\
                a_{21} & a_{22} & \cdots & a_{2n}\\
                \vdots & \vdots & \cdots & \vdots\\
                a_{m1} & a_{m2} & \cdots & a_{mn}
            \end{pmatrix}, \quad\pause
            X = \begin{pmatrix}
                x_1\\
                x_2\\
                \vdots\\
                x_n
            \end{pmatrix},\quad \pause
            B = \begin{pmatrix}
                b_1\\
                b_2\\
                \vdots\\
                b_m
            \end{pmatrix}
        \]
    \end{frame}
    
    \begin{frame}
        A partir dessas matrizes o sistema linear anterior pode ser escrito como a equação matricial:\pause
        \[
            AX = B.\pause
        \]

        No caso em que
        \[
            B = \begin{pmatrix}
                0\\0\\\vdots\\0
            \end{pmatrix}\pause
        \]

        Vamos escrever simplesmente
        \[
            AX = 0.
        \]
    \end{frame}

    \begin{frame}
        \begin{teorema}
            Seja $A$ uma matriz invertível de ordem $n$. \pause Para cada matriz $B$ de ordem $n\times 1$ \pause o sistema linear de forma matricial
            $AX = B$ \pause admite solução única que é dada por\pause
            \[
                X = \begin{pmatrix}
                    x_1 \\ x_2 \\ \vdots \\ x_n
                \end{pmatrix} \pause = A^{-1}B.
            \]
        \end{teorema}
    \end{frame}

    \begin{frame}
        \begin{definicao}
            Uma matriz $n\times n$ \pause que pode ser obtida da matriz identidade $I_n$, \pause de tamanho $n\times n$, \pause efetuando uma 
            única operação elementar \pause sobre linhas \pause é chamada de \textbf{matriz elementar}.
        \end{definicao}
    \end{frame}

    \begin{frame}
        \begin{teorema}
            Se a matriz elementar $E$ \pause é o resultado de efetuar uma certa operação com as linhas de $I_m$ \pause e se $A$ é uma matriz 
            $m \times n$, \pause então o produto $EA$ \pause é a matriz que resulta quando essa mesma operação com linhas é efetuada em $A$.\pause
        \end{teorema}

        \vspace{1cm}

        \begin{teorema}
            Qualquer matriz elementar é invertível, \pause e a inversa também é uma matriz elementar.
        \end{teorema}
    \end{frame}

    \begin{frame}
        \begin{teorema}
            Se $A$ é uma matriz $n \times n$, \pause então as seguintes afirmações são equivalentes, \pause ou seja, são todas verdadeiras \pause 
            ou todas falsas.\pause
            \begin{enumerate}[label={\roman*})]

                \item $A$ é invertível.\pause
                
                \item $AX = 0$ tem somente a solução trivial.\pause

                \item A forma escalonada reduzida por linhas de $A$ é $I_n$.\pause

                \item $A$ pode ser expressa como um produto de matrizes elementares.
            \end{enumerate}
        \end{teorema}
    \end{frame}
\end{document}
