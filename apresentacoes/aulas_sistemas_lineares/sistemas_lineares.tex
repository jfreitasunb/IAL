%!TEX program = xelatex
\def\ano{2023}
\def\semestre{1}
\def\disciplina{Introdução à Álgebra Linear}
\def\turma{3}
\def\autor{Jos\'e Ant\^onio O. Freitas}
\def\instituto{MAT-UnB}

\documentclass{beamer}
\usetheme{Madrid}
\usecolortheme{beaver}
% \mode<presentation>
\usepackage{caption}
\usepackage{amssymb}
\usepackage{amsmath,amsfonts,amsthm,amstext}
\usepackage[brazil]{babel}
\usepackage{graphicx}
\graphicspath{{../Pictures/}}
\usepackage{enumitem}
\usepackage{multicol}
\usepackage{answers}
\usepackage[svgnames]{xcolor}
\usepackage{tikz}
\usepackage{ifthen}
\usetikzlibrary{lindenmayersystems}
\usetikzlibrary[shadings]

\newcounter{exercicios}
\setcounter{exercicios}{0}
\newcommand{\questao}{
    \addtocounter{exercicios}{1}
    \noindent{\bf Quest{\~a}o \arabic{exercicios}: }}

\newcommand{\resp}[1]{
    \noindent{\bf Exerc{\'\i}cio #1: }}

\extrafootheight[.25in]{.5in}
\footrule
\lfoot{Teste \numeroteste\ - Módulo \modulo\ - \nomeabreviado\ - Turma \turma\ - \semestre$^o$/\ano}
\cfoot{}
\rfoot{P\'agina \thepage\ de \numpages}
\def\ano{2023}
\def\semestre{2}
\def\disciplina{Introdução à Álgebra Linear}
\def\nomeabreviado{IAL}
\def\turma{11}

\newcommand{\im}{{\rm Im\,}}
\newcommand{\dlim}[2]{\displaystyle\lim_{#1\rightarrow #2}}
\newcommand{\minf}{+\infty}
\newcommand{\ninf}{-\infty}
\newcommand{\cp}[1]{\mathbb{#1}}
\newcommand{\sub}{\subseteq}
\newcommand{\n}{\mathbb{N}}
\newcommand{\z}{\mathbb{Z}}
\newcommand{\rac}{\mathbb{Q}}
\newcommand{\real}{\mathbb{R}}
\newcommand{\complex}{\mathbb{C}}

\newcommand{\vesp}[1]{\vspace{ #1  cm}}

\newcommand{\compcent}[1]{\vcenter{\hbox{$#1\circ$}}}
\newcommand{\comp}{\mathbin{\mathchoice
        {\compcent\scriptstyle}{\compcent\scriptstyle}
        {\compcent\scriptscriptstyle}{\compcent\scriptscriptstyle}}}
\renewcommand{\sin}{{\rm sen\,}}
\renewcommand{\tan}{{\rm tg\,}}
\renewcommand{\csc}{{\rm cossec\,}}
\renewcommand{\cot}{{\rm cotg\,}}
\renewcommand{\sinh}{{\rm senh\,}}

\title{Sistemas Lineares}
\author[\autor]{\autor}
\institute[\instituto]{\instituto}
\date{}

\begin{document}
    \begin{frame}
        \maketitle
    \end{frame}

    \logo{\includegraphics[scale=.1]{logo-MAT.png}\vspace*{8.5cm}}

    \begin{frame}
        Seja $\cp{K} = \real$.\pause

        \vspace{.3cm}
        
        Dado um conjunto de equações da forma
        \begin{align}
            \begin{cases}\label{sistema_linear_2x2}
                ax + by = b_1\\
                cx + dy = b_2
            \end{cases}
        \end{align}
        onde $a$, $b$, $c$, $d$, $b_1$, $b_2 \in \cp{K}$.\pause
    
        \vspace{.3cm}

        Existem valores de $x$, $y \in \cp{K}$ que satisfazem as duas equações simultaneamente?
    \end{frame}
    
    \begin{frame}
        Neste caso temos as seguintes possibilidades:\pause
        \begin{enumerate}[label={\roman*})]
            \item Existe um único conjunto de valores $x = \alpha \in \cp{K}$ e $y = \lambda \in \cp{K}$ \pause que satisfaz as duas equações simultaneamente. \pause Neste caso dizemos que o sistema \eqref{sistema_linear_2x2} \pause é \textbf{possível e determinado}.\pause

            \item Existem infinitos valores de $x$, $y \in \cp{K}$ \pause que satisfazem as duas equações simultanemente. \pause Neste caso dizemos que o sistema \eqref{sistema_linear_2x2} \pause é \textbf{possível e indeterminado}.\pause

            \item Não existem valores de $x$, $y \in \cp{K}$ \pause satisfazendo as duas equações simultaneamente. \pause Nesse caso dizemos que o sistema \eqref{sistema_linear_2x2} \pause é \textbf{impossível}.
        \end{enumerate}
    \end{frame}
    
    \begin{frame}
        Esse mesmo tipo de análise pode ser aplicada num sistema da forma
        \begin{align}
            \begin{cases}\label{sistema_linear_3x3}
                a_1x + a_2y + a_3z = b_1\\
                a_4x + a_5y + a_6z = b_2\\
                a_7x + a_8y + a_9z = b_3
            \end{cases}
        \end{align}
        onde $a_i \in \cp{K}$ para $1 \le i \le 9$ e $b_j \in \cp{K}$ para $1 \le j \le 3$.\pause

        \vspace{.3cm}

        Existem valores de $x$, $y$ e $z \in \cp{K}$ que satisfaçam as três equações simultaneamente?
    \end{frame}
    
    \begin{frame}
        Como no caso anterior, temos as seguintes possibilidades:\pause
        \begin{enumerate}[label={\roman*})]
            \item Existe um único conjunto de valores $x = \alpha \in \cp{K}$, $y = \lambda \in \cp{K}$ e $z = \gamma \in \cp{K}$ \pause que satisfaz as três equações simultaneamente. \pause Neste caso dizemos que o sistema \eqref{sistema_linear_3x3} \pause é \textbf{possível e determinado}.\pause

            \item Existem infinitos valores de $x$, $y$, $z \in \cp{K}$ \pause que satisfazem as três equações simultanemente. \pause Neste caso dizemos que o sistema \eqref{sistema_linear_3x3} \pause é \textbf{possível e indeterminado}.\pause

            \item Não existem valores de $x$, $y$, $z \in \cp{K}$ \pause satisfazendo as três equações simultaneamente. \pause Nesse caso dizemos que o sistema \eqref{sistema_linear_3x3} \pause é \textbf{impossível}.
        \end{enumerate}
    \end{frame}

    \begin{frame}

        Para o que vamos fazer agora vamos admitir que $\cp{K} = \rac$, \pause $\real$ \pause ou $\complex$.\pause

        \vspace{.3cm}

        Agora considere variáveis $x_1$, $x_2$, \dots, $x_q$ \pause todas assumindo valores em $\cp{K}$.\pause

        \vspace{.3cm}

        Escolha escalares $b_1$, \dots, $b_p$ \pause e $a_{ij}$, \pause $1 \le i \le p$, $1 \le j \le q$ todos em $\cp{K}$.
    \end{frame}
    
    \begin{frame}
        Queremos saber se é possível encontrar valores para  $x_1$, $x_2$, \dots, $x_q$ \pause de modo que o seguinte conjunto de equações sejam válidas: \pause
        \begin{equation}\label{sistemalinear}\index{Sistema Linear}
	    \begin{cases}
                a_{11}x_1 + a_{12}x_2 + \cdots + a_{1q}x_q = b_1\\\pause
                a_{21}x_1 + a_{22}x_2 + \cdots + a_{2q}x_q = b_2\\\pause
                \qquad \vdots\\
                a_{p1}x_1 + a_{p2}x_2 + \cdots + a_{pq}x_q = b_p\pause
            \end{cases}
        \end{equation}
        
        O conjunto de equações em \eqref{sistemalinear} é chamado de um \pause textbf{sistema de $p$ equa\c{c}\~oes lineares \pause a $q$ inc\'ognitas} \pause , ou simplesmente de um \textbf{sistema linear},\pause$x_1$, $x_2$, \dots, $x_q$.\pause

        \vspace{.3cm}

        Toda $q$-upla $(\alpha_1, \alpha_2, \dots, \alpha_q)$ \pause onde $\alpha_i \in \cp{K}$ para $1 \le i \le q$, \pause que satisfazem a cada uma das equa\c{c}\~oes de \eqref{sistemalinear} \pause \'e chamada de uma \textbf{solu\c{c}\~ao} do sistema.
    \end{frame}

    \begin{frame}
        Se $b_1 = b_2 = \cdots = b_p = 0_\cp{K} \in K$, \pause dizemos que o sistema\pause
        \begin{equation}\label{sistemalinearhomogeneo}\index{Sistema Linear}
            \begin{cases}
                a_{11}x_1 + a_{12}x_2 + \cdots + a_{1q}x_q = 0_\cp{K}\\\pause
                a_{21}x_1 + a_{22}x_2 + \cdots + a_{2n}x_q = 0_\cp{K}\\\pause
                \qquad \vdots\\
                a_{p1}x_1 + a_{p2}x_2 + \cdots + a_{pq}x_q = 0_\cp{K}\pause
            \end{cases}
        \end{equation}
        \'e um \textbf{sistema linear homog\^eneo}, \pause ou que cada uma de suas equa\c{c}\~oes \'e homog\^enea. \pause

        \vspace{.3cm}

        Observe que tal sistema sempre possui solu\c{c}\~ao, \pause a saber, $x_1 = x_2 = \cdots = x_q = 0_\cp{K}$.
    \end{frame}

    \begin{frame}
        No caso de um sistema linear da forma \eqref{sistemalinear}, \pause o processo para encontrar suas soluções ser\'a feito mediante o uso de 3 tipos de opera\c{c}\~oes. \pause S\~ao elas:\pause
        \begin{itemize}
	    \item[$e_1$)] Troca da posi\c{c}\~ao de duas equa\c{c}\~oes.\pause
	    \item[$e_2$)] Multiplica\c{c}\~ao de uma equa\c{c}\~ao por um escalar n\~ao nulo.\pause
	    \item[$e_3$)] Substitui\c{c}\~ao de uma equa\c{c}\~ao pela soma desta equa\c{c}\~ao com alguma outra.\pause
        \end{itemize}

        Estas tr\^es opera\c{c}\~oes s\~ao chamadas de \pause \textbf{opera\c{c}\~oes elementares}.
    \end{frame}
\end{document}
