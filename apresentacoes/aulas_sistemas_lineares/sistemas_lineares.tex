%!TEX program = xelatex
\def\ano{2023}
\def\semestre{1}
\def\disciplina{Introdução à Álgebra Linear}
\def\turma{3}
\def\autor{Jos\'e Ant\^onio O. Freitas}
\def\instituto{MAT-UnB}

\documentclass{beamer}
\usetheme{Madrid}
\usecolortheme{beaver}
% \mode<presentation>
\usepackage{caption}
\usepackage{amssymb}
\usepackage{amsmath,amsfonts,amsthm,amstext}
\usepackage[brazil]{babel}
\usepackage{graphicx}
\graphicspath{{../Pictures/}}
\usepackage{enumitem}
\usepackage{multicol}
\usepackage{answers}
\usepackage[svgnames]{xcolor}
\usepackage{tikz}
\usepackage{ifthen}
\usetikzlibrary{lindenmayersystems}
\usetikzlibrary[shadings]

\newcounter{exercicios}
\setcounter{exercicios}{0}
\newcommand{\questao}{
    \addtocounter{exercicios}{1}
    \noindent{\bf Quest{\~a}o \arabic{exercicios}: }}

\newcommand{\resp}[1]{
    \noindent{\bf Exerc{\'\i}cio #1: }}

\extrafootheight[.25in]{.5in}
\footrule
\lfoot{Teste \numeroteste\ - Módulo \modulo\ - \nomeabreviado\ - Turma \turma\ - \semestre$^o$/\ano}
\cfoot{}
\rfoot{P\'agina \thepage\ de \numpages}
\def\ano{2023}
\def\semestre{2}
\def\disciplina{Introdução à Álgebra Linear}
\def\nomeabreviado{IAL}
\def\turma{11}

\newcommand{\im}{{\rm Im\,}}
\newcommand{\dlim}[2]{\displaystyle\lim_{#1\rightarrow #2}}
\newcommand{\minf}{+\infty}
\newcommand{\ninf}{-\infty}
\newcommand{\cp}[1]{\mathbb{#1}}
\newcommand{\sub}{\subseteq}
\newcommand{\n}{\mathbb{N}}
\newcommand{\z}{\mathbb{Z}}
\newcommand{\rac}{\mathbb{Q}}
\newcommand{\real}{\mathbb{R}}
\newcommand{\complex}{\mathbb{C}}

\newcommand{\vesp}[1]{\vspace{ #1  cm}}

\newcommand{\compcent}[1]{\vcenter{\hbox{$#1\circ$}}}
\newcommand{\comp}{\mathbin{\mathchoice
        {\compcent\scriptstyle}{\compcent\scriptstyle}
        {\compcent\scriptscriptstyle}{\compcent\scriptscriptstyle}}}
\renewcommand{\sin}{{\rm sen\,}}
\renewcommand{\tan}{{\rm tg\,}}
\renewcommand{\csc}{{\rm cossec\,}}
\renewcommand{\cot}{{\rm cotg\,}}
\renewcommand{\sinh}{{\rm senh\,}}

\title{Sistemas Lineares}
\author[\autor]{\autor}
\institute[\instituto]{\instituto}
\date{}

\begin{document}
    \begin{frame}
        \maketitle
    \end{frame}

    \logo{\includegraphics[scale=.1]{logo-MAT.png}\vspace*{8.5cm}}

    \begin{frame}
        Seja $\cp{K} = \rac$, $\real$ ou $\complex$.

        \vspace{.3cm}

        Vamos trabalhar com \textbf{equações lineares} em $q$ variáveis $x_1$, $x_2$, \dots, $x_q$ que são equações que podem ser escritas na forma
        \begin{equation}\label{equacao_linear}
            a_1x_1 + a_2x_2 + \cdots + a_qx_q = b
        \end{equation}

        onde $a_1$, $a_2$, \dots, $a_q$, $b$ são escalares no corpo $\cp{K}$, sendo que nem todos os coeficientes $a_i$ são nulos.
        
        \vspace{.3cm}

        No caso em que $b = 0$, a equação \eqref{equacao_linear} tem a forma
        \begin{equation}
            a_1x_1 + a_2x_2 + \cdots + a_qx_q = 0
        \end{equation}
        e é chamada de uma \textbf{equação linear homogênea}.
        
    \end{frame}

    \begin{frame}
        Seja $\cp{K} = \real$.

        \vspace{.3cm}
        
        Dado um conjunto de equações da forma
        \begin{align}
            \begin{cases}\label{sistema_linear_2x2}
                ax + by = b_1\\
                cx + dy = b_2
            \end{cases}
        \end{align}
        onde $a$, $b$, $c$, $d$, $b_1$, $b_2 \in \cp{K}$.
    
        \vspace{.3cm}

        Existem valores de $x$, $y \in \cp{K}$ que satisfazem as duas equações simultaneamente?
    \end{frame}

    \begin{frame}
        \definecolor{qqqqff}{rgb}{0,0,1}
        \definecolor{ffqqqq}{rgb}{1,0,0}
        \begin{center}
            \begin{tikzpicture}[line cap=round,line join=round,>=triangle 45,x=1cm,y=1cm]
                \begin{axis}[
                    x=1cm,y=1cm,
                    axis lines=middle,
                    xticklabels={,,},
                    yticklabels={,,},
                    xlabel=$x$,
                    ylabel=$y$,
                    xmin=-2,
                    xmax=5,
                    ymin=-1.3,
                    ymax=3,]
                    \draw [line width=5.2pt,dash pattern=on 1pt off 1pt,color=ffqqqq,domain=-1.348257839721257:5.807994044102577] plot(\x,{(--4-1*\x)/2});
                    \draw [line width=2pt,color=qqqqff,domain=-1.348257839721257:5.807994044102577] plot(\x,{(--8-2*\x)/4});
                    \begin{scriptsize}
                        \draw[color=ffqqqq] (1.8,2) node {$a_{11}x + a_{12}y = b_1$};
                        \draw[color=qqqqff] (3,1.5) node {$a_{21}x + a_{22}y = b_2$};
                    \end{scriptsize}
                \end{axis}
            \end{tikzpicture}
        \end{center}

        Existem infinitos valores de $x$, $y \in \cp{K}$ que satisfazem as duas equações simultanemente. Neste caso dizemos que o sistema \eqref{sistema_linear_2x2} é \textbf{possível e indeterminado}.
    \end{frame}

    \begin{frame}
        \definecolor{ffqqqq}{rgb}{1,0,0}
        \definecolor{qqqqff}{rgb}{0,0,1}
        \begin{center}
            \begin{tikzpicture}[line cap=round,line join=round,>=triangle 45,x=1cm,y=1cm]
                \begin{axis}[
                    x=1cm,y=1cm,
                    axis lines=middle,
                    xticklabels={,,},
                    yticklabels={,,},
                    xlabel=$x$,
                    ylabel=$y$,
                    xmin=-5.3,
                    xmax=6.3,
                    ymin=-1,
                    ymax=4.3,]
                    \draw [line width=2pt,color=qqqqff,domain=-5:6] plot(\x,{(--4-1*\x)/5});
                    \draw [line width=2pt,color=ffqqqq,domain=-5:6] plot(\x,{(--16-1*\x)/5});
                    \begin{scriptsize}
                        \draw[color=qqqqff] (-3.8,2.2) node {$a_{21}x + a_{22}y = b_2$};
                        \draw[color=ffqqqq] (-3.8,3.4) node {$a_{11}x + a_{12}y = b_1$};
                    \end{scriptsize}
                \end{axis}
            \end{tikzpicture}
        \end{center}

        Não existem valores de $x$, $y \in \cp{K}$ satisfazendo as duas equações simultaneamente. Nesse caso dizemos que o sistema \eqref{sistema_linear_2x2} é \textbf{impossível}.

    \end{frame}

    \begin{frame}
        \definecolor{qqqqff}{rgb}{0,0,1}
        \definecolor{ffqqqq}{rgb}{1,0,0}
        \begin{center}
            \begin{tikzpicture}[line cap=round,line join=round,>=triangle 45,x=1cm,y=1cm]
                \begin{axis}[
                    x=1cm,y=1cm,
                    axis lines=middle,
                    xticklabels={,,},
                    yticklabels={,,},
                    xlabel=$x$,
                    ylabel=$y$,
                    xmin=-2,
                    xmax=4.3,
                    ymin=-2,
                    ymax=3.5,]
                    \draw [line width=2pt,color=ffqqqq,domain=-4:5] plot(\x,{(--4.94--7.92*\x)/7.58});
                    \draw [line width=2pt,color=qqqqff,domain=-4:5] plot(\x,{(--4.824-8.66*\x)/8.24});
                    \begin{scriptsize}
                        \draw[color=ffqqqq] (2.9,2) node {$a_{11}x + a_{12}y = b_1$};
                        \draw[color=qqqqff] (2.7,-0.5) node {$a_{21}x + a_{22}y = b_2$};
                    \end{scriptsize}
                \end{axis}
            \end{tikzpicture}
        \end{center}
        
        Nesse caso existe um único conjunto de valores $x = \alpha \in \cp{K}$ e $y = \lambda \in \cp{K}$ que satisfaz as duas equações simultaneamente. Neste caso dizemos que o sistema \eqref{sistema_linear_2x2} é \textbf{possível e determinado}.
    \end{frame}

    \begin{frame}
        Esse mesmo tipo de análise pode ser aplicada num sistema da forma
        \begin{align}
            \begin{cases}\label{sistema_linear_3x3}
                a_1x + a_2y + a_3z = b_1\\
                a_4x + a_5y + a_6z = b_2\\
                a_7x + a_8y + a_9z = b_3
            \end{cases}
        \end{align}
        onde $a_i \in \cp{K}$ para $1 \le i \le 9$ e $b_j \in \cp{K}$ para $1 \le j \le 3$.

        \vspace{.3cm}

        Existem valores de $x$, $y$ e $z \in \cp{K}$ que satisfaçam as três equações simultaneamente?
    \end{frame}
    
    \begin{frame}
        Como no caso anterior, temos as seguintes possibilidades:
        \begin{enumerate}[label={\roman*})]
            \item Existe um único conjunto de valores $x = \alpha \in \cp{K}$, $y = \lambda \in \cp{K}$ e $z = \gamma \in \cp{K}$ que satisfaz as três equações simultaneamente. Neste caso dizemos que o sistema \eqref{sistema_linear_3x3} é \textbf{possível e determinado}.

            \item Existem infinitos valores de $x$, $y$, $z \in \cp{K}$ que satisfazem as três equações simultanemente. Neste caso dizemos que o sistema \eqref{sistema_linear_3x3} é \textbf{possível e indeterminado}.

            \item Não existem valores de $x$, $y$, $z \in \cp{K}$ satisfazendo as três equações simultaneamente. Nesse caso dizemos que o sistema \eqref{sistema_linear_3x3} é \textbf{impossível}.
        \end{enumerate}
    \end{frame}

    \begin{frame}

        Escolha escalares $b_1$, \dots, $b_m$ e $a_{ij}$, $1 \le i \le m$, $1 \le j \le n$ todos em $\cp{K}$.

        \vspace{.3cm}

        Queremos saber se é possível encontrar valores para  $x_1$, $x_2$, \dots, $x_n$ de modo que o seguinte conjunto de equações sejam válidas: 
        \begin{equation}\label{sistema_linear_geral}
	    \begin{cases}
                a_{11}x_1 + a_{12}x_2 + \cdots + a_{1n}x_n = b_1\\
                a_{21}x_1 + a_{22}x_2 + \cdots + a_{2n}x_n = b_2\\
                \qquad \vdots\\
                a_{m1}x_1 + a_{m2}x_2 + \cdots + a_{mn}x_n = b_m
            \end{cases}
        \end{equation}
        
        O conjunto de equações em \eqref{sistema_linear_geral} é chamado de um \textbf{sistema de $m$ equa\c{c}\~oes lineares a $n$ inc\'ognitas} $x_1$, $x_2$, \dots, $x_n$, ou simplesmente de um \textbf{sistema linear}, nas incógnitas $x_1$, $x_2$, \dots, $x_n$.
    \end{frame}

    \begin{frame}
        \vspace{.3cm}

        Uma solução de um sistema linear do tipo \eqref{sistema_linear_geral}
        \[
            x_1 = \alpha_1, x_2 = \alpha_2, \dots, x_n = \alpha_n
        \]
        onde $\alpha_1$, $\alpha_2$, \dots, $\alpha_n \in \cp{K}$, pode ser escrita como
        \[
            (\alpha_1, \alpha_2, \dots, \alpha_n)
        \]
        e é chamada de uma \textbf{ênupla ordenada} ou uma \textbf{n-upla ordenada}.
    \end{frame}

    \begin{frame}
        Se $b_1 = b_2 = \cdots = b_m = 0_\cp{K} \in K$, dizemos que o sistema
        \begin{equation}\label{sistemalinearhomogeneo}\index{Sistema Linear}
            \begin{cases}
                a_{11}x_1 + a_{12}x_2 + \cdots + a_{1n}x_n = 0_\cp{K}\\
                a_{21}x_1 + a_{22}x_2 + \cdots + a_{2n}x_n = 0_\cp{K}\\
                \qquad \vdots\\
                a_{m1}x_1 + a_{m2}x_2 + \cdots + a_{mn}x_n = 0_\cp{K}
            \end{cases}
        \end{equation}
        \'e um \textbf{sistema linear homog\^eneo}. 

        \vspace{.3cm}

        Observe que tal sistema sempre possui solu\c{c}\~ao, a saber, $x_1 = x_2 = \cdots = x_n = 0_\cp{K}$.
    \end{frame}

    \begin{frame}
        \begin{teorema}
            Todo sistema linear do tipo \eqref{sistema_linear_geral} tem zero, uma ou uma infinidade de soluções. Não existem outras possibilidades.
        \end{teorema}
    \end{frame}

    \begin{frame}
        No caso de um sistema linear da forma \eqref{sistema_linear_geral}, o processo para encontrar suas soluções ser\'a feito mediante o uso de 3 tipos de opera\c{c}\~oes. S\~ao elas:
        \begin{itemize}
	    \item[$e_1$)] Troca da posi\c{c}\~ao de duas equa\c{c}\~oes.
	    \item[$e_2$)] Multiplica\c{c}\~ao de uma equa\c{c}\~ao por um escalar n\~ao nulo.
	    \item[$e_3$)] Substitui\c{c}\~ao de uma equa\c{c}\~ao pela soma desta equa\c{c}\~ao com alguma outra.
        \end{itemize}

        Estas tr\^es opera\c{c}\~oes s\~ao chamadas de \textbf{opera\c{c}\~oes elementares}.
    \end{frame}

    \begin{frame}
        Uma outra matriz que podemos associar ao sistema \eqref{sistema_linear_geral} \'e
        \[
	    \begin{bmatrix}
                a_{11} & a_{12} & \cdots & a_{1n} & b_1\\
		a_{21} & a_{22} & \cdots & a_{2n} & b_2\\
		\vdots & \vdots & \vdots & \vdots & \vdots\\
		a_{m1} & a_{m2} & \cdots & a_{mn} & b_m\\
            \end{bmatrix}
        \]

        que \'e chamada de \textbf{matriz ampliada do sistema} ou \textbf{matriz aumentada do sistema}.
    \end{frame}

    \begin{frame}
        Na forma matricial as opera\c{c}\~oes elementares s\~ao descritas como:

        \vspace{.3cm}

        \begin{itemize}
            \item[$e_1$)] Trocar a $i$-\'esima linha de $A$ pela $j$-\'esima linha de $A$: $L_i \leftrightarrow L_j$;

            \vspace{.3cm}

            \item[$e_2$)] Multiplica\c{c}\~ao da $i$-\'esima linha de $A$ por um escalar $\alpha \in \cp{K}$ n\~ao nulo: $L_i \rightarrow \alpha L_i$;
 
            \vspace{.3cm}

           \item[$e_3$)] Substitui\c{c}\~ao da $i$-\'esima linha de $A$ pela $i$-\'esima linha mais $\alpha$ vezes a $j$-\'esima linha: $L_i \rightarrow L_i + \alpha L_j$.
        \end{itemize}
    \end{frame}

    \begin{frame}
        \begin{definicao}\label{linhareduzida}
            Uma matriz $A$ $m \times n$ \'e dita estar na \textbf{forma escalonada reduzida por linhas} se:
            \begin{enumerate}[label={\roman*})]
                \item O primeiro elemento n\~ao nulo em cada linha n\~ao nula de $A$ \'e $1$. Dizemos que esse número 1 é um \textbf{pivô}.

                \vspace{.3cm}

                \item Toda linha de $A$ cujos elementos s\~ao todos nulos ocorre abaixo de todas as linhas que possuem um elemento n\~ao-nulo.
 
                \vspace{.3cm}

                \item Se as linhas 1, 2, \dots, $r$ s\~ao as linhas n\~ao-nulas de $A$ e se o \textbf{pivô} da linha $i$ ocorre na coluna $k_i$, $i = 1$, \dots, $r$, ent\~ao $k_1 < k_2 < \cdots < k_r$.
 
                \vspace{.3cm}

                \item Cada coluna de $A$ que cont\'em um \textbf{pivô} tem todos os seus outros elementos nulos.
            \end{enumerate}
        \end{definicao}
    \end{frame}

    \begin{frame}
        \begin{observacao}
            Uma matriz que satisfaz as três primeiras propriedades da definição anterior é dita estar na \textbf{forma escalonada por linhas}, ou simplesmente, em \textbf{forma escalonada}.
        \end{observacao}
    \end{frame}

    \begin{frame}
        \begin{definicao}
	    Se $A$ e $B$ s\~ao matrizes $m \times n$, dizemos que $B$ \'e \textbf{equivalente por linha} a $A$, se $B$ for obtida de $A$ atrav\'es de uma quantidade finita de opera\c{c}\~oes elementares sobre as linhas de $A$.
        \end{definicao}
        
        \vspace{.6cm}

        \begin{notacao}
	    $A \rightarrow B$ ou $A \sim B$.
        \end{notacao}
    \end{frame}

    \begin{frame}
        \begin{teorema}
            Duas matrizes $A$ e $B$ são equivalentes por linha se, e somente, se elas puderem ser reduzidas à mesma forma escalonada por linhas.
        \end{teorema}
    \end{frame}

    \begin{frame}
        \begin{teorema}
            Se as matrizes ampliadas de dois sistemas lineares são equivalentes por linhas, então os dois sistemas possuem as mesmas soluções.
        \end{teorema}
    \end{frame}

    \begin{frame}
        O \textbf{método de eliminação de Gauss} ou \textbf{método de eleminação gaussiana} consiste em substituir um dado sistema de equações lineares por outro \textbf{equivalente}, que seja mais simples de ser solucionado e que tenha a mesma solução do sistema original.

        \begin{definicao}[Método de eliminaçao de Gauss]
            O método de \textbf{eliminação de Gauss}, para solução de um sistema linear, consiste em:
            \begin{enumerate}[label={\roman*})]
                \item Escreva a matriz ampliada do sistema de equações lineares.

                \item Use operações elementares nas linhas de $A$ para reduzir a matriz ampliada à \textbf{forma escalonada por linhas}.
                
                \item Quando a matriz ampliada estiver na forma escalonada, usando substituição de trás para a frente, resolva o sistema equivalente que corresponde a matriz escalonada reduzida por linhas.
            \end{enumerate}
        \end{definicao}
    \end{frame}
    
    \begin{frame}
        \begin{observacao}
            Ao se aplicar o método de eliminação gaussiana, as seguintes dicas podem ser úteis:
            \begin{enumerate}[label=({\alph*})]
                \item Localize a coluna mais à esquerda que não é toda formada por zeros.
                
                \vspace{.3cm}

                \item Crie um \textbf{pivô} no topo desta coluna.
                 
                \vspace{.3cm}

                \item Use o \textbf{pivô} para criar zeros abaixo dele.
 
                \vspace{.3cm}

                \item Faça a linha contendo este \textbf{pivô} ir para a parte de cima e volte ao passo (a) para repetir o procedimento com o restante da submatriz. Pare quando toda a matriz estiver na forma escalonada por linhas.
            \end{enumerate}
        \end{observacao}
    \end{frame}

    \begin{frame}
        \begin{definicao}
            O \textbf{posto}, denotado $\p{A}$ ou $p(A)$, de uma matriz $A$ é número de linhas não nulas de qualquer uma de suas formas escalonadas por linhas.
        \end{definicao}
    \end{frame}

    \begin{frame}
        \begin{definicao}
            Seja $A$ a matriz ampliada de um sistema linear. Se $A$ está na forma escalonada reduzida por linhas, então as variáveis desse sistema que não correspondem aos pivôs são chamadas de \textbf{variáveis livres}.
        \end{definicao}
    \end{frame}

    \begin{frame}
        Considere o sistema linear:
        \begin{equation*}
	        \begin{cases}
                a_{11}x_1 + a_{12}x_2 + \cdots + a_{1n}x_n = b_1\\
                a_{21}x_1 + a_{22}x_2 + \cdots + a_{2n}x_n = b_2\\
                \qquad \vdots\\
                a_{m1}x_1 + a_{m2}x_2 + \cdots + a_{mn}x_n = b_m
            \end{cases}
        \end{equation*}
        A matriz
        \[
            A = \begin{bmatrix}
                a_{11} & a_{12} & \cdots & a_{1n}\\
                a_{21} & a_{22} & \cdots & a_{2n}\\
                \vdots & \vdots & \cdots & \vdots\\
                a_{m1} & a_{m2} & \cdots & a_{mn}
            \end{bmatrix}
        \]

        é chamada de \textbf{matriz dos coeficientes} do sistema linear.
    \end{frame}

    \begin{frame}

        \begin{teorema}[Teorema do Posto]
            Seja $A$ a matriz dos coeficientes de um sistema linear com $n$ variáveis. Então
            \[
                \mbox{número de variáveis livres } = n - \p{A}.
            \]
        \end{teorema}
    \end{frame}

    \begin{frame}
        Durante a aplicação do \textbf{método de eliminação de Gauss}, paramos quando a matriz ampliada do sistema está na forma escalonada.
        Todavia podemos continuar aplicando operações elementares até que a matriz atinja a forma escalonada reduzida por linhas. Esse é o caso do 
        \textbf{método de Gaus-Jordan}.

        \begin{definicao}[Método de eliminaçao de Gauss-Jordan]
            O método de \textbf{eliminação de Gauss-Jordan}, para solução de um sistema linear, consiste em:
            \begin{enumerate}[label={\roman*})]
                \item Escreva a matriz ampliada do sistema de equações lineares.

                \item Use operações elementares nas linhas de $A$ para reduzir a matriz ampliada à \textbf{forma escalonada reduzida por linhas}.
                
                \item Se o sistema resultante for possível, resolva-o para as variáveis dependentes em termos de quaisquer variáveis livres que tenha sobrado.
            \end{enumerate}
        \end{definicao}
    \end{frame}

    \begin{frame}
        \begin{teorema}
            Um sistema linear homogêneo com mais incógnitas que equações tem uma infinidade de soluções.
        \end{teorema}
    \end{frame}


    \begin{frame}
        \begin{definicao}
            Seja $A$ uma matriz quadrada de ordem $n$ e $A \ne 0$. Se for possível encontrar uma matriz quadrada $B$ também de 
            ordem $n$ tal que
            \[
                AB = I_n = BA  
            \]
            onde $I_n$ é a matriz identidade de ordem $n$, então diremos que $A$ é \textbf{invertível}, ou \textbf{não singular}, 
            e que $B$ é a \textbf{inversa} de $A$. Se não pudermos encontrar tal matriz $B$, então diremos que $A$ é \textbf{não invertível} 
            ou \textbf{singular}.
        \end{definicao}
    \end{frame}

    \begin{frame}
        \begin{teorema}
            Se $B$ e $C$ são ambas inversas da matriz $A$, então $B = C$.
        \end{teorema}

        \vspace{1cm}

        \begin{notacao}
            Se $A$ é uma matriz invertível e $B$ é a sua inversa, vamos escrever $B = A^{-1}$.
        \end{notacao}
    \end{frame}

    \begin{frame}
        \begin{proposicao}
            Se $A$ e $B$ são matrizes invertíveis de mesmo tamanho, então $AB$ é invertível e
            \[
                (AB)^{-1} = B^{-1}A^{-1}.
            \]
        \end{proposicao}
    \end{frame}

    \begin{frame}
        \begin{proposicao}
            Seja $A$ uma matriz invertível e $n$ um número inteiro não negativo. Então:
            \begin{enumerate}[label={\roman*})]
                \item $A^{-1}$ é invertível e $(A^{-1})^{-1} = A$.

                \item $A^n$ é invertível e $(A^n)^{-1} = A^{-n} = (A^{-1})^n$.

                \item $kA$ é invertível para todo escalar $k$ não nulo e $(kA)^{-1} = k^{-1}A^{-1}$.
            \end{enumerate}
        \end{proposicao}
    \end{frame}

    \begin{frame}
        \begin{enumerate}[label={\arabic*})]
            \item Seja $A$ uma matriz quadrada de ordem $n$. Como podemos encontrar a inversa de $A$, se existir.

            \vspace{1cm}

            \item Dada uma matriz $A$, como podemos decidir se $A$ é invertível?
        \end{enumerate}
    \end{frame}

    \begin{frame}
        Considere o sistema linear: 
        \begin{equation}
	    \begin{cases}
                a_{11}x_1 + a_{12}x_2 + \cdots + a_{1n}x_n = b_1\\
                a_{21}x_1 + a_{22}x_2 + \cdots + a_{2n}x_n = b_2\\
                \qquad \vdots\\
                a_{m1}x_1 + a_{m2}x_2 + \cdots + a_{mn}x_n = b_m
            \end{cases}
        \end{equation}
        A esse sistema podemos associar algumas matrizes. A saber:
        \[
            A = \begin{pmatrix}
                a_{11} & a_{12} & \cdots & a_{1n}\\
                a_{21} & a_{22} & \cdots & a_{2n}\\
                \vdots & \vdots & \cdots & \vdots\\
                a_{m1} & a_{m2} & \cdots & a_{mn}
            \end{pmatrix}, \quad
            x = \begin{pmatrix}
                x_1\\
                x_2\\
                \vdots\\
                x_n
            \end{pmatrix},\quad 
            x = \begin{pmatrix}
                b_1\\
                b_2\\
                \vdots\\
                b_m
            \end{pmatrix}
        \]
    \end{frame}
    
    \begin{frame}
        A partir dessas matrizes o sistema linear anterior pode ser escrito como a equação matricial:
        \[
            Ax = b.
        \]

        No caso em que
        \[
            b = \begin{pmatrix}
                0\\0\\\vdots\\0
            \end{pmatrix}
        \]

        Vamos escrever simplesmente
        \[
            Ax = 0.
        \]
    \end{frame}

    \begin{frame}
        \begin{teorema}
            Seja $A$ uma matriz invertível de ordem $n$. Para cada matriz $B$ de ordem $n\times 1$ o sistema linear de forma matricial
            $Ax = b$ admite solução única que é dada por
            \[
                x = \begin{pmatrix}
                    x_1 \\ x_2 \\ \vdots \\ x_n
                \end{pmatrix} = A^{-1}B.
            \]
        \end{teorema}
    \end{frame}

    \begin{frame}
        \begin{definicao}
            Uma matriz $n\times n$ que pode ser obtida da matriz identidade $I_n$, de tamanho $n\times n$, efetuando uma 
            única operação elementar sobre linhas é chamada de \textbf{matriz elementar}.
        \end{definicao}
    \end{frame}

    \begin{frame}
        \begin{teorema}
            Se a matriz elementar $E$ é o resultado de efetuar uma certa operação com as linhas de $I_m$ e se $A$ é uma matriz 
            $m \times n$, então o produto $EA$ é a matriz que resulta quando essa mesma operação com linhas é efetuada em $A$.
        \end{teorema}

        \vspace{1cm}

        \begin{teorema}
            Qualquer matriz elementar é invertível, e a inversa também é uma matriz elementar.
        \end{teorema}
    \end{frame}

    \begin{frame}
        \begin{teorema}
            Se $A$ é uma matriz $n \times n$, então as seguintes afirmações são equivalentes, ou seja, são todas verdadeiras ou todas falsas:
            \begin{enumerate}[label={\roman*})]
                \item $A$ é invertível.
                
                \vspace{1cm}

                \item $Ax = 0$ tem somente a solução trivial.
                 
                \vspace{1cm}

                \item A forma escalonada reduzida por linhas de $A$ é $I_n$.
                 
                \seti
            \end{enumerate}
        \end{teorema}
    \end{frame}

    \begin{frame}
        \begin{teorema}
            \begin{enumerate}[label={\roman*})]
                \conti

                \item $A$ pode ser expressa como um produto de matrizes elementares.
                 
                \vspace{1cm}

                \item $Ax = b$ é possível para cada matriz $b$ de tamanho $n \times 1$.
                 
                \vspace{1cm}

                \item $Ax = b$ tem exatamente uma solução para cada matriz $b$ de tamanho $n \times 1$.
            \end{enumerate}
        \end{teorema}
    \end{frame}
\end{document}
