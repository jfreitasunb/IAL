%!TEX program = xelatex
%!TEX encoding = UTF-8

\documentclass{beamer}
\usetheme{Madrid}
\usecolortheme{beaver}
% \mode<presentation>
\usepackage{caption}
\usepackage{amssymb}
\usepackage{amsmath,amsfonts,amsthm,amstext}
\usepackage[brazil]{babel}
\usepackage{graphicx}
\graphicspath{{../Pictures/}}
\usepackage{enumitem}
\usepackage{multicol}
\usepackage{answers}
\usepackage[svgnames]{xcolor}
\usepackage{tikz}
\usepackage{ifthen}
\usetikzlibrary{lindenmayersystems}
\usetikzlibrary[shadings]

\newcounter{exercicios}
\setcounter{exercicios}{0}
\newcommand{\questao}{
    \addtocounter{exercicios}{1}
    \noindent{\bf Quest{\~a}o \arabic{exercicios}: }}

\newcommand{\resp}[1]{
    \noindent{\bf Exerc{\'\i}cio #1: }}

\extrafootheight[.25in]{.5in}
\footrule
\lfoot{Teste \numeroteste\ - Módulo \modulo\ - \nomeabreviado\ - Turma \turma\ - \semestre$^o$/\ano}
\cfoot{}
\rfoot{P\'agina \thepage\ de \numpages}
\def\ano{2023}
\def\semestre{2}
\def\disciplina{Introdução à Álgebra Linear}
\def\nomeabreviado{IAL}
\def\turma{11}

\newcommand{\im}{{\rm Im\,}}
\newcommand{\dlim}[2]{\displaystyle\lim_{#1\rightarrow #2}}
\newcommand{\minf}{+\infty}
\newcommand{\ninf}{-\infty}
\newcommand{\cp}[1]{\mathbb{#1}}
\newcommand{\sub}{\subseteq}
\newcommand{\n}{\mathbb{N}}
\newcommand{\z}{\mathbb{Z}}
\newcommand{\rac}{\mathbb{Q}}
\newcommand{\real}{\mathbb{R}}
\newcommand{\complex}{\mathbb{C}}

\newcommand{\vesp}[1]{\vspace{ #1  cm}}

\newcommand{\compcent}[1]{\vcenter{\hbox{$#1\circ$}}}
\newcommand{\comp}{\mathbin{\mathchoice
        {\compcent\scriptstyle}{\compcent\scriptstyle}
        {\compcent\scriptscriptstyle}{\compcent\scriptscriptstyle}}}
\renewcommand{\sin}{{\rm sen\,}}
\renewcommand{\tan}{{\rm tg\,}}
\renewcommand{\csc}{{\rm cossec\,}}
\renewcommand{\cot}{{\rm cotg\,}}
\renewcommand{\sinh}{{\rm senh\,}}

\title{Matriz}
\author[\autor]{\autor}
\institute[\instituto]{\instituto}
\date{}

\begin{document}
    \begin{frame}
        \maketitle
    \end{frame}

    \logo{\includegraphics[scale=.1]{logo-MAT.png}\vspace*{8.5cm}}

    \begin{frame}
        Vamos trabalhar também com o conjunto de matrizes com entradas em $\cp{K}$:\pause
        \[
            M_{p \times q}(\cp{K}) =
            \left\{
                \begin{bmatrix}
                    a_{11} & a_{12} & \cdots & a_{1n}\\
                    a_{21} & a_{22} & \cdots & a_{2n}\\
                    \vdots & & & \vdots\\
                    a_{m1} & a_{m2} & \cdots & a_{mn}
                \end{bmatrix}
                \mid a_{ij} \in \cp{K},\ 1 \le i \le p,\ 1 \le j \le q
            \right\}\pause
        \]
    \end{frame}

    \begin{frame}
        Em $M_{p\times q}(\cp{K})$ vamos admitir definidas as operações de soma de matrizes \pause e multiplicação de matriz por escalar.\pause

        Vamos também admitir as seguintes propriedades como verdadeiras:\pause
        \begin{proposicao}
            Dadas matrizes $A$, $B$ e $C \in M_{p\times q}(\cp{K})$ \pause e escalares $\alpha$, $\lambda \in \cp{K}$ vale:\pause
            \begin{enumerate}[label={\roman*})]
                \item $(A + B) + C = A + (B + C)$\pause
                \item $A + B = B + A$\pause
                \item $0_{p \times q} + A = A \pause = A + 0_{p \times q}$, \pause onde $0_{p \times q}$ é matriz cujas entradas são todas zero.\pause
                \seti
            \end{enumerate}
        \end{proposicao}
    \end{frame}

    \begin{frame}
        \begin{proposicao}
            \begin{enumerate}[label={\roman*})]
                \conti
                \item $A - A = 0_{p \times q}$\pause
                \item $(\alpha\lambda) A = \alpha(\lambda A) \pause = \lambda(\alpha A)$\pause
                \item $(\alpha + \lambda)A = \pause \alpha A + \lambda A$
                \seti
            \end{enumerate}
        \end{proposicao}
    \end{frame}

    \begin{frame}
        \begin{proposicao}
            \begin{enumerate}[label={\roman*})]
                \conti
                \item $\alpha(A + B) = \pause \alpha A + \alpha B$\pause
                \item $1A = A$ e $(-1)A = -A$\pause
                \item  $0A = 0_{p \times q}$
            \end{enumerate}
        \end{proposicao}
    \end{frame}

    \begin{frame}
        Para o produto de matrizes vamos admitir as seguintes propriedades como verdadeiras:\pause
        \begin{proposicao}
            Sejam $A$, $B$ e $C$ \pause matrizes de ordens tais que as operações a seguir estejam definidas. Ent\~ao:\pause
                \begin{enumerate}[label={\roman*})]
                    \item $(A\cdot B)\cdot C = A\cdot(B \cdot C)$\pause
                    \item $(A + B)\cdot C = (A\cdot C) + (B\cdot C)$\pause
                    \item $C\cdot(A + B) = (C\cdot A) + (C\cdot B)$
                    \seti
                \end{enumerate}
        \end{proposicao}
    \end{frame}

    \begin{frame}
        \begin{proposicao}
            \begin{enumerate}[label={\roman*})]
                \conti
                \item $\alpha(A\cdot B) = \pause (\alpha A)\cdot B \pause = A \cdot(\alpha B)$\pause
                \item Se $A$ é uma matriz quadrada de ordem $n$, \pause então \[A\cdot I_n = A = I_n\cdot A,\] \pause onde $I_n$ é a matriz quadrada de ordem $n$ \pause onde todas as entradas da diagonal principal são 1 \pause e as demais são zero.\pause
                \item Se $A$ é uma matriz quadrada de ordem $n$, \pause então \[A\cdot0_n = 0_n \pause = 0_n\cdot A.\]
            \end{enumerate}
        \end{proposicao}
    \end{frame}
\end{document}
