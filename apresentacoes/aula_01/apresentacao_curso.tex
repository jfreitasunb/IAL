%!TEX program = xelatex
\def\ano{2023}
\def\semestre{1}
\def\disciplina{Introdução à Álgebra Linear}
\def\turma{3}
\def\autor{Jos\'e Ant\^onio O. Freitas}
\def\instituto{MAT-UnB}

\documentclass{beamer}
\usetheme{Madrid}
\usecolortheme{beaver}
% \mode<presentation>
\usepackage{caption}
\usepackage{amssymb}
\usepackage{amsmath,amsfonts,amsthm,amstext}
\usepackage[brazil]{babel}
\usepackage{graphicx}
\graphicspath{{../Pictures/}}
\usepackage{enumitem}
\usepackage{multicol}
\usepackage{answers}
\usepackage[svgnames]{xcolor}
\usepackage{tikz}
\usepackage{ifthen}
\usetikzlibrary{lindenmayersystems}
\usetikzlibrary[shadings]

\newcounter{exercicios}
\setcounter{exercicios}{0}
\newcommand{\questao}{
    \addtocounter{exercicios}{1}
    \noindent{\bf Quest{\~a}o \arabic{exercicios}: }}

\newcommand{\resp}[1]{
    \noindent{\bf Exerc{\'\i}cio #1: }}

\extrafootheight[.25in]{.5in}
\footrule
\lfoot{Teste \numeroteste\ - Módulo \modulo\ - \nomeabreviado\ - Turma \turma\ - \semestre$^o$/\ano}
\cfoot{}
\rfoot{P\'agina \thepage\ de \numpages}
\def\ano{2023}
\def\semestre{2}
\def\disciplina{Introdução à Álgebra Linear}
\def\nomeabreviado{IAL}
\def\turma{11}

\newcommand{\im}{{\rm Im\,}}
\newcommand{\dlim}[2]{\displaystyle\lim_{#1\rightarrow #2}}
\newcommand{\minf}{+\infty}
\newcommand{\ninf}{-\infty}
\newcommand{\cp}[1]{\mathbb{#1}}
\newcommand{\sub}{\subseteq}
\newcommand{\n}{\mathbb{N}}
\newcommand{\z}{\mathbb{Z}}
\newcommand{\rac}{\mathbb{Q}}
\newcommand{\real}{\mathbb{R}}
\newcommand{\complex}{\mathbb{C}}

\newcommand{\vesp}[1]{\vspace{ #1  cm}}

\newcommand{\compcent}[1]{\vcenter{\hbox{$#1\circ$}}}
\newcommand{\comp}{\mathbin{\mathchoice
        {\compcent\scriptstyle}{\compcent\scriptstyle}
        {\compcent\scriptscriptstyle}{\compcent\scriptscriptstyle}}}
\renewcommand{\sin}{{\rm sen\,}}
\renewcommand{\tan}{{\rm tg\,}}
\renewcommand{\csc}{{\rm cossec\,}}
\renewcommand{\cot}{{\rm cotg\,}}
\renewcommand{\sinh}{{\rm senh\,}}

\title{Apresentação do curso e noções iniciais}
\author[\autor]{\autor}
\institute[\instituto]{\instituto}
\date{}

\begin{document}
    \begin{frame}
        \maketitle
    \end{frame}

    \logo{\includegraphics[scale=.1]{logo-MAT.png}\vspace*{8.5cm}}

    \begin{frame}
        {\bf Conteúdo a ser visto no curso:}
        \begin{enumerate}[label={\roman*})]
            \item Matrizes e sistemas lineares.

            \item Determinantes.

            \item Espaços vetoriais.

            \item Transformações Lineares.

            \item Autovalores e autovetores.

            \item Diagonalização de operadores.

            \item Espaços vetoriais com produto interno.
        \end{enumerate}
    \end{frame}

    \begin{frame}
        {\bf Bibliografia:}
        \begin{enumerate}[label={\arabic*})]
            \item H. Anton, C. Rorres: {\it {\'A}lgebra Linear com Aplicações}, $10^a$ Ed., Editora Bookman, 2012.

        \item C. H. Edwards, D. E. Penney: {\it Introdução à Álgebra Linear}, Editora Prentice-Hall do Brasil.

            \item S. Lang: {\it Álgebra Linear}, Editora Moderna, 2003.

            \item T. S. Blyth, E. F. Robertson: {\it Basic Linear Algebra}, $2^a$ Ed., Springer, 2002.
        \end{enumerate}
    \end{frame}

    \begin{frame}
        {\bf{Sistema de avaliação:}} Três módulos, 1 avalição por módulo.
            Nota $M_i$, $i=1$, 2, 3. A nota final ($NF$) de cada estudante ser\'a dada por:
        \[
            NF = \dfrac{2M_1 + 3M_2 + 4M_3}{9}, \quad 0 \le NF \le 10.
        \]
        Ser\'a considerado aprovado o estudante que obtiver $NF \ge 5$.
    \end{frame}

    \begin{frame}
        {\bf Datas das provas:}
        \begin{center}
            \begin{tabular}{c|c|c}
                \hline\hline
                \hspace{0.5cm}{\bf Prova}\hspace{0.5cm} & \hspace{1cm}{\bf Data}\hspace{1cm} & \hspace{1cm}{\bf Hor\'{a}rio}\hspace{1cm} \\
                \hline\hline
                $P_1$ & 25/04/23 (terça-feira) & 19:00 - 20:40 \\
                \hline
                $P_2$ & 08/06/23 (quinta-feira) & 19:00 - 20:40 \\
                \hline
                $P_3$ & 20/07/23 (quinta-feira) & 19:00 - 20:40 \\
                \hline\hline
            \end{tabular}
        \end{center}
    \end{frame}

    \begin{frame}
        \begin{definicao}\label{corpo}\index{Corpos}
	Um conjunto n\~ao vazio $\cp{K}$ \'e chamado de \textbf{corpo} se em $\cp{K}$ podemos definir duas opera\c{c}\~oes, denotadas por $+$ (e chamada de \textbf{adi\c{c}\~ao}) e $\cdot$ (e chamada de \textbf{multiplica\c{c}\~ao}) de modo que
	\begin{align*}
            a &+ b \in \cp{K}\\
	    a &\cdot b \in \cp{K}
	\end{align*}
	para todos $a$, $b \in \cp{K}$ e que satisfa\c{c}am as seguintes propriedades:
	    \begin{enumerate}[label={\roman*})]
		\item \textbf{Comutatividade da adi\c{c}\~ao}: $a + b = b + a$ para todos $a$, $b \in \cp{K}$;
		\item \textbf{Associatividade da adi\c{c}\~ao}: $a + (b + c) = (a + b) + c$, para todos $a$, $b$ e $c \in \cp{K}$;
                \seti
            \end{enumerate}
        \end{definicao}
    \end{frame}

    \begin{frame}
        \begin{definicao}
            \begin{enumerate}[label={\roman*})]
                \conti
		\item \textbf{Elemento neutro da adi\c{c}\~ao}: Existe um elemento em $\cp{K}$, denotado por $0_\cp{K}$ ou simplesmente $0$ e chamado de \textbf{elemento neutro da adi\c{c}\~ao}, que satisfaz
		\[
			a + 0_\cp{K} = a = 0_\cp{K} + a
		\]
		para todo $a \in \cp{K}$.
		\item \textbf{Elemento oposto da adi\c{c}\~ao}: Para cada $a \in \cp{K}$, existe um elemento em $\cp{K}$, denotado por $-a$ e chamado de \textbf{oposto} de $a$ ou \textbf{inverso aditivo} de $a$ tal que
		\[
			a + (-a) = 0_\cp{K} = (-a) + a.
		\]

                \seti
            \end{enumerate}
        \end{definicao}
    \end{frame}
    
    \begin{frame}
        \begin{definicao}
            \begin{enumerate}[label={\roman*})]
                \conti

		\item \textbf{Comutatividade da multiplica\c{c}\~ao}: $a \cdot b = b \cdot a$ para todos $a$, $b \in \cp{K}$;
		\item \textbf{Associatividade da multiplica\c{c}\~ao}: $a \cdot (b \cdot c) = (a \cdot b) \cdot c$, para todos $a$, $b$ e $c \in \cp{K}$;
		\item \textbf{Elemento neutro da multiplica\c{c}\~ao}: Existe um elemento em $\cp{K}$, denotado por $1_\cp{K}$ ou simplesmente $1$ e chamado de \textbf{elemento neutro da multiplica\c{c}\~ao} ou \textbf{unidade}, que satisfaz
		\[
			a \cdot 1_\cp{K} = a = 1_\cp{K} \cdot a
		\]
		para todo $a \in \cp{K}$.
	
                \seti
            \end{enumerate}
        \end{definicao}
    \end{frame}
    
    \begin{frame}
        \begin{definicao}
            \begin{enumerate}[label={\roman*})]
                \conti

	\item \textbf{Elemento inverso da multiplica\c{c}\~ao}: Para cada $a \in \cp{K}$, $a \ne 0_{\cp{K}}$, existe um elemento em $\cp{K}$, denotado por $a^{-1}$ e chamado de \textbf{inverso multiplicativo} de $a$
		tal que
		\[
			a \cdot a^{-1} = 1_\cp{K} = a^{-1} \cdot a.
		\]
		\item \textbf{Distributividade da soma em rela\c{c}\~ao \`a multiplica\c{c}\~ao}: $(a + b)\cdot c = a\cdot c + b\cdot c$, para todos $a$, $b$ e $c \in \cp{K}$.
	    \end{enumerate}
        \end{definicao}
    \end{frame}

    \begin{frame}
        Denotamos um corpo $\cp{K}$ pela terna $(\cp{K}, +, \cdot)$. Quando n\~ao houver chance de confus\~ao em rela\c{c}\~ao \`as opera\c{c}\~oes de soma e multiplica\c{c}\~ao envolvidas no corpo $(\cp{K}, +, \cdot)$, vamos simplesmente dizer que $\cp{K}$ \'e um corpo. Os elementos de um corpo $\cp{K}$ s\~ao chamados de \textbf{escalares}.
    \end{frame}
    
    \begin{frame}
    
        Considere $\cp{K} = \rac$, $\real$ ou
        \[
    	    \complex = \{a + bi \mid a, b \in \real, i^2 = -1\}.
        \]
        Então vale sempre que:

        \begin{enumerate}[label={\roman*})]
	    \item $x + y = y + x$, para todos $x$, $y \in \cp{K}$;
            \item $(x + y) + z = x + (y + z)$, para todos $x$, $y$ e $z \in \cp{K}$;
            \item existe $0$ em $\cp{K}$ tal que $x + 0 = x$ para todo $x \in \cp{K}$;
            \item para cada $x \in \cp{K}$, existe $y \in \cp{K}$ tal que $x + y = 0$. Tal $y$ é escrito como $y = -x$;
            \item $xy = yx$ para todos $x$, $y \in \cp{K}$;
            \item $(xy)z = x(yz)$ para todos $x$, $y$ e $z \in \cp{K}$;
            \item existe $1 \in \cp{K}$ tal que $1x = x$ para todo $x \in \cp{K}$
            \item para todo $x \in \cp{K}$, $x \ne 0$, existe $y \in \cp{K}$ tal que $xy =1$. Tal $y$ é escrito como $y = x^{-1}$;
            \item $(x + y)z = xz + yz$ para todos $x$, $y$ e $z \in \cp{K}$.
        \end{enumerate}
    \end{frame}
    
    \begin{frame}
        \begin{enumerate}[label={\roman*})]
            \item Assim $(\cp{K}, +, \cdot)$ é um corpo.

            \item Durante todo esse curso o conjunto $\cp{K}$ será usado para representar um dos conjuntos: $\rac$, $\real$ ou $\complex$.
        \end{enumerate}
    \end{frame}

    \begin{frame}
        Vamos trabalhar também com o conjunto de matrizes com entradas em $\cp{K}$:
        \[
            M_{m \times n}(\cp{K}) = 
            \left\{
                \begin{bmatrix}
                    a_{11} & a_{12} & \cdots & a_{1n}\\
		    a_{21} & a_{22} & \cdots & a_{2n}\\
		    \vdots & & & \vdots\\
		    a_{m1} & a_{m2} & \cdots & a_{mn}
                \end{bmatrix}
                \mid a_{ij} \in \cp{K},\ 1 \le i \le m,\ 1 \le j \le n
	    \right\}
        \]

        Em $M_{r\times s}(\cp{K})$ vamos admitir definidas as operações de soma de matrizes e multiplicação de matriz por escalar.

        Vamos também admitir as seguintes propriedades como verdadeiras:
        \begin{proposicao}
            Dadas matrizes $A$, $B$ e $C \in M_{n\times n}(\cp{K})$ e escalares $\alpha$, $\lambda \in \cp{K}$ vale:
            \begin{enumerate}[label={\roman*})]
                \item $(A + B) + C = A + (B + C)$
                \item $A + B = B + A$
                \item $0_{m \times n} + A = A = A + 0_{m \times n}$, onde $0_{m \times n}$ é matriz cujas entradas são todas zero.
                \seti
            \end{enumerate}
        \end{proposicao}
    \end{frame}

    \begin{frame}
        \begin{proposicao}
            \begin{enumerate}[label={\roman*})]
                \conti
                \item $A - A = 0_{m \times n}$.
                \item $(\alpha\lambda) A = \alpha(\lambda A) = \lambda(\alpha A)$
                \item $(\alpha + \lambda)A = \alpha A + \lambda A$
                \seti
            \end{enumerate}
        \end{proposicao}
    \end{frame}

    \begin{frame}
        \begin{proposicao}
            \begin{enumerate}[label={\roman*})]
                \conti

                \item $\alpha(A + B) = \alpha A + \alpha B$
                \item $1A = A$ e $(-1)A = -A$
                \item  $0A = 0_{m \times n}$
            \end{enumerate}
        \end{proposicao}
    \end{frame}

    \begin{frame}
        Para o produto de matrizes vamos admitir as seguintes propriedades como verdadeiras:
        \begin{proposicao}
	    Sejam $A$, $B$ e $C$ matrizes de ordens tais que as operações a seguir estejam definidas. Ent\~ao:
            \begin{enumerate}[label={\roman*})]
                \item $(A\cdot B)\cdot C = A\cdot(B \cdot C)$
                \item $(A + B)\cdot C = (A\cdot C) + (B\cdot C)$
                \item $C\cdot(A + B) = (C\cdot A) + (C\cdot B)$
                \seti
            \end{enumerate}
        \end{proposicao}
    \end{frame}

    \begin{frame}
        \begin{proposicao}
            \begin{enumerate}[label={\roman*})]
                \conti

                \item $\alpha(A\cdot B) = (\alpha A)\cdot B = A \cdot(\alpha B)$
                \item Se $A$ é uma matriz quadrada de ordem $n$, então \[A\cdot I_n = A = I_n\cdot A,\] onde $I_n$ é a matriz quadrada de ordem $n$ onde todas as entradas da diagonal principal são 1 e as demais são zero.
                \item Se $A$ é uma matriz quadrada de ordem $n$, então \[A\cdot0_n = 0_n = 0_n\cdot A.\]
	    \end{enumerate}
        \end{proposicao}
    \end{frame}
\end{document}
