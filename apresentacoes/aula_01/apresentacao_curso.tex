%!TEX program = xelatex
\def\ano{2023}
\def\semestre{2}
\def\disciplina{Introdução à Álgebra Linear}
\def\turma{3}
\def\autor{Jos\'e Ant\^onio O. Freitas}
\def\instituto{MAT-UnB}

\documentclass{beamer}
\usetheme{Madrid}
\usecolortheme{beaver}
% \mode<presentation>
\usepackage{caption}
\usepackage{amssymb}
\usepackage{amsmath,amsfonts,amsthm,amstext}
\usepackage[brazil]{babel}
\usepackage{graphicx}
\graphicspath{{../Pictures/}}
\usepackage{enumitem}
\usepackage{multicol}
\usepackage{answers}
\usepackage[svgnames]{xcolor}
\usepackage{tikz}
\usepackage{ifthen}
\usetikzlibrary{lindenmayersystems}
\usetikzlibrary[shadings]

\newcounter{exercicios}
\setcounter{exercicios}{0}
\newcommand{\questao}{
    \addtocounter{exercicios}{1}
    \noindent{\bf Quest{\~a}o \arabic{exercicios}: }}

\newcommand{\resp}[1]{
    \noindent{\bf Exerc{\'\i}cio #1: }}

\extrafootheight[.25in]{.5in}
\footrule
\lfoot{Teste \numeroteste\ - Módulo \modulo\ - \nomeabreviado\ - Turma \turma\ - \semestre$^o$/\ano}
\cfoot{}
\rfoot{P\'agina \thepage\ de \numpages}
\def\ano{2023}
\def\semestre{2}
\def\disciplina{Introdução à Álgebra Linear}
\def\nomeabreviado{IAL}
\def\turma{11}

\newcommand{\im}{{\rm Im\,}}
\newcommand{\dlim}[2]{\displaystyle\lim_{#1\rightarrow #2}}
\newcommand{\minf}{+\infty}
\newcommand{\ninf}{-\infty}
\newcommand{\cp}[1]{\mathbb{#1}}
\newcommand{\sub}{\subseteq}
\newcommand{\n}{\mathbb{N}}
\newcommand{\z}{\mathbb{Z}}
\newcommand{\rac}{\mathbb{Q}}
\newcommand{\real}{\mathbb{R}}
\newcommand{\complex}{\mathbb{C}}

\newcommand{\vesp}[1]{\vspace{ #1  cm}}

\newcommand{\compcent}[1]{\vcenter{\hbox{$#1\circ$}}}
\newcommand{\comp}{\mathbin{\mathchoice
        {\compcent\scriptstyle}{\compcent\scriptstyle}
        {\compcent\scriptscriptstyle}{\compcent\scriptscriptstyle}}}
\renewcommand{\sin}{{\rm sen\,}}
\renewcommand{\tan}{{\rm tg\,}}
\renewcommand{\csc}{{\rm cossec\,}}
\renewcommand{\cot}{{\rm cotg\,}}
\renewcommand{\sinh}{{\rm senh\,}}

\title{Apresentação do curso e conceitos iniciais}
\author[\autor]{\autor}
\institute[\instituto]{\instituto}
\date{}

\begin{document}
    \begin{frame}
        \maketitle
    \end{frame}

    \logo{\includegraphics[scale=.1]{logo-MAT.png}\vspace*{8.5cm}}

    \begin{frame}
        {\bf Conteúdo a ser visto no curso:}
        \begin{enumerate}[label={\roman*})]
            \item Matrizes e sistemas lineares.\pause

            \item Determinantes.\pause

            \item Espaços vetoriais.\pause

            \item Transformações Lineares.\pause

            \item Autovalores e autovetores.\pause

            \item Diagonalização de operadores.\pause

            \item Espaços vetoriais com produto interno.\pause
        \end{enumerate}
    \end{frame}

    \begin{frame}
        {\bf Bibliografia:}
        \begin{enumerate}[label={\arabic*})]
            \item H. Anton, C. Rorres: {\it {\'A}lgebra Linear com Aplicações}, $10^a$ Ed., Editora Bookman, 2012.\pause

        \item C. H. Edwards, D. E. Penney: {\it Introdução à Álgebra Linear}, Editora Prentice-Hall do Brasil.\pause

            \item S. Lang: {\it Álgebra Linear}, Editora Moderna, 2003.\pause

            \item T. S. Blyth, E. F. Robertson: {\it Basic Linear Algebra}, $2^a$ Ed., Springer, 2002.\pause
        \end{enumerate}
    \end{frame}

    \begin{frame}
        {\bf{Sistema de avaliação:}} Três módulos: em cada módulo serão realizados testes e 1 avaliação.\pause
        A nota de cada módulo será dada por:
        \[
            M_i = 15\%T_i + 85\%P_i, \quad 0 \le M_i \le 10,
        \]

        A nota final ($NF$) de cada estudante ser\'a dada por:\pause
        \[
            NF = \dfrac{2M_1 + 3M_2 + 4M_3}{9}, \quad 0 \le NF \le 10.\pause
        \]
        Ser\'a considerado aprovado o estudante que obtiver $NF \ge 5$.
    \end{frame}

    \begin{frame}
        {\bf Datas das provas:}
        \begin{center}
            \begin{tabular}{c|c|c}
                \hline\hline
                \hspace{0.5cm}{\bf Prova}\hspace{0.5cm} & \hspace{1cm}{\bf Data}\hspace{1cm} & \hspace{1cm}{\bf Hor\'{a}rio}\hspace{1cm} \\
                \hline\hline
                $P_1$ & 04/10/23 (quarta-feira) \phantom{x} & 16:00 - 17:50 \\
                \hline
                $P_2$ & 13/11/23 (segunda-feira) \phantom{x} & 16:00 - 17:50 \\
                \hline
                $P_3$ & 14/12/23 (quarta-feira) \phantom{x} & 16:00 - 17:50 \\
                \hline\hline
            \end{tabular}
        \end{center}
    \end{frame}

    \begin{frame}
        \begin{definicao}
	    Um conjunto n\~ao vazio $\cp{K}$ \'e chamado de \textbf{corpo} \pause se em $\cp{K}$ podemos definir duas opera\c{c}\~oes, denotadas por $+$ (chamada de \textbf{adi\c{c}\~ao}) \pause e $\cdot$ (chamada de \textbf{multiplica\c{c}\~ao}) \pause de modo que:
            \begin{align*}
                a &+ b \in \cp{K}\\
    	        a &\cdot b \in \cp{K}\pause
            \end{align*}
            para todos $a$, $b \in \cp{K}$ \pause e que satisfa\c{c}am as seguintes propriedades:\pause
	    \begin{enumerate}[label={\roman*})]
                \item \textbf{Comutatividade da adi\c{c}\~ao}: para todos $a$, $b \in \cp{K}$, \pause vale: \pause $a + b = b + a$;\pause
		\item \textbf{Associatividade da adi\c{c}\~ao}: para todos $a$, $b$ e $c \in \cp{K}$ vale: \pause $a + (b + c) = (a + b) + c$;
                \seti
            \end{enumerate}
        \end{definicao}
    \end{frame}

    \begin{frame}
        \begin{definicao}
            \begin{enumerate}[label={\roman*})]
                \conti
		\item \textbf{Elemento neutro da adi\c{c}\~ao}: Existe um elemento em $\cp{K}$, denotado por $0_\cp{K}$ \pause ou simplesmente $0$ e chamado de \textbf{elemento neutro da adi\c{c}\~ao}, \pause que satisfaz\pause
		\[
                    a + 0_\cp{K} = a\pause = 0_\cp{K} + a\pause
		\]
		para todo $a \in \cp{K}$.\pause
		\item \textbf{Elemento oposto da adi\c{c}\~ao}: Para cada $a \in \cp{K}$, \pause existe um elemento em $\cp{K}$, denotado por $-a$ \pause e chamado de \textbf{oposto} de $a$ \pause ou \textbf{inverso aditivo} de $a$ tal que\pause
		\[
                    a + (-a) = 0_\cp{K}\pause = (-a) + a.
		\]

                \seti
            \end{enumerate}
        \end{definicao}
    \end{frame}

    \begin{frame}
        \begin{definicao}
            \begin{enumerate}[label={\roman*})]
                \conti

		\item \textbf{Comutatividade da multiplica\c{c}\~ao}: para todos $a$, $b \in \cp{K}$ \pause vale: $a \cdot b = b \cdot a$;\pause
		\item \textbf{Associatividade da multiplica\c{c}\~ao}: para todos $a$, $b$ e $c \in \cp{K}$\pause vale: $a \cdot (b \cdot c) = (a \cdot b) \cdot c$;\pause

                \seti
            \end{enumerate}
        \end{definicao}
    \end{frame}

    \begin{frame}
        \begin{definicao}
            \begin{enumerate}[label={\roman*})]
                \conti

		\item \textbf{Elemento neutro da multiplica\c{c}\~ao}: Existe um elemento em $\cp{K}$, \pause denotado por $1_\cp{K}$ \pause ou simplesmente $1$ \pause e chamado de \textbf{elemento neutro da multiplica\c{c}\~ao} \pause ou \textbf{unidade}, que satisfaz:\pause
		\[
			a \cdot 1_\cp{K} = a \pause = 1_\cp{K} \cdot a\pause
		\]
		para todo $a \in \cp{K}$.\pause
            \item \textbf{Elemento inverso da multiplica\c{c}\~ao}: Para cada $a \in \cp{K}$, \pause $a \ne 0_{\cp{K}}$, \pause existe um elemento em $\cp{K}$, \pause denotado por $a^{-1}$ \pause e chamado de \textbf{inverso multiplicativo} de $a$ tal que\pause
		\[
			a \cdot a^{-1} = 1_\cp{K} \pause = a^{-1} \cdot a.
		\]
                \seti
            \end{enumerate}
        \end{definicao}
    \end{frame}

    \begin{frame}
        \begin{definicao}
            \begin{enumerate}[label={\roman*})]
                \conti

		\item \textbf{Distributividade da soma em rela\c{c}\~ao \`a multiplica\c{c}\~ao}: para todos $a$, $b$ e $c \in \cp{K}$ \pause vale:
                    \[
                        (a + b)\cdot c = a\cdot c + b\cdot c.
                    \]
	    \end{enumerate}
        \end{definicao}
    \end{frame}

    \begin{frame}
        Denotamos um corpo $\cp{K}$ pela terna $(\cp{K}, +, \cdot)$. \pause

        \vspace{.3cm}

        Quando n\~ao houver chance de confus\~ao em rela\c{c}\~ao \`as opera\c{c}\~oes de soma e multiplica\c{c}\~ao envolvidas no corpo $(\cp{K}, +, \cdot)$, \pause vamos simplesmente dizer que $\cp{K}$ \'e um corpo. \pause

        \vspace{.3cm}

        Os elementos de um corpo $\cp{K}$ s\~ao chamados de \textbf{escalares}.
    \end{frame}

    \begin{frame}

        Considere $\cp{K} = \rac$, \pause $\real$ \pause ou
        \[
    	    \complex = \pause \{a + bi \pause \mid a, b \in \real, \pause i^2 = -1\}.\pause
        \]
        Então vale sempre que:\pause

        \begin{enumerate}[label={\roman*})]
	    \item $x + y = y + x$, para todos $x$, $y \in \cp{K}$;\pause
            \item $(x + y) + z = x + (y + z)$, para todos $x$, $y$ e $z \in \cp{K}$;\pause
            \item existe $0$ em $\cp{K}$ tal que $x + 0 = x$ para todo $x \in \cp{K}$;\pause
            \item para cada $x \in \cp{K}$, existe $y \in \cp{K}$ tal que $x + y = 0$. Tal $y$ é escrito como $y = -x$;\pause
            \item $xy = yx$ para todos $x$, $y \in \cp{K}$;\pause
            \item $(xy)z = x(yz)$ para todos $x$, $y$ e $z \in \cp{K}$;\pause
            \item existe $1 \in \cp{K}$ tal que $1x = x$ para todo $x \in \cp{K}$;\pause
            \item para todo $x \in \cp{K}$, $x \ne 0$, existe $y \in \cp{K}$ tal que $xy =1$. Tal $y$ é escrito como $y = x^{-1}$;\pause
            \item $(x + y)z = xz + yz$ para todos $x$, $y$ e $z \in \cp{K}$.\pause
        \end{enumerate}
    \end{frame}

    \begin{frame}
        \begin{enumerate}[label={\roman*})]
            \item Assim $(\cp{K}, +, \cdot)$ é um corpo.\pause

            \vspace{.3cm}

            \item Durante todo esse curso o conjunto $\cp{K}$ será usado para representar um dos conjuntos: $\rac$, $\real$ ou $\complex$.\pause
        \end{enumerate}
    \end{frame}

    \begin{frame}
        Vamos trabalhar também com o conjunto de matrizes com entradas em $\cp{K}$:\pause
        \[
            M_{p \times q}(\cp{K}) =
            \left\{
                \begin{bmatrix}
                    a_{11} & a_{12} & \cdots & a_{1n}\\
		    a_{21} & a_{22} & \cdots & a_{2n}\\
		    \vdots & & & \vdots\\
		    a_{m1} & a_{m2} & \cdots & a_{mn}
                \end{bmatrix}
                \mid a_{ij} \in \cp{K},\ 1 \le i \le p,\ 1 \le j \le q
	    \right\}\pause
        \]
    \end{frame}

    \begin{frame}
        Em $M_{p\times q}(\cp{K})$ vamos admitir definidas as operações de soma de matrizes \pause e multiplicação de matriz por escalar.\pause

        Vamos também admitir as seguintes propriedades como verdadeiras:\pause
        \begin{proposicao}
            Dadas matrizes $A$, $B$ e $C \in M_{p\times q}(\cp{K})$ \pause e escalares $\alpha$, $\lambda \in \cp{K}$ vale:\pause
            \begin{enumerate}[label={\roman*})]
                \item $(A + B) + C = A + (B + C)$\pause
                \item $A + B = B + A$\pause
                \item $0_{p \times q} + A = A \pause = A + 0_{p \times q}$, \pause onde $0_{p \times q}$ é matriz cujas entradas são todas zero.\pause
                \seti
            \end{enumerate}
        \end{proposicao}
    \end{frame}

    \begin{frame}
        \begin{proposicao}
            \begin{enumerate}[label={\roman*})]
                \conti
                \item $A - A = 0_{p \times q}$\pause
                \item $(\alpha\lambda) A = \alpha(\lambda A) \pause = \lambda(\alpha A)$\pause
                \item $(\alpha + \lambda)A = \pause \alpha A + \lambda A$
                \seti
            \end{enumerate}
        \end{proposicao}
    \end{frame}

    \begin{frame}
        \begin{proposicao}
            \begin{enumerate}[label={\roman*})]
                \conti
                \item $\alpha(A + B) = \pause \alpha A + \alpha B$\pause
                \item $1A = A$ e $(-1)A = -A$\pause
                \item  $0A = 0_{p \times q}$
            \end{enumerate}
        \end{proposicao}
    \end{frame}

    \begin{frame}
        Para o produto de matrizes vamos admitir as seguintes propriedades como verdadeiras:\pause
        \begin{proposicao}
	    Sejam $A$, $B$ e $C$ \pause matrizes de ordens tais que as operações a seguir estejam definidas. Ent\~ao:\pause
            \begin{enumerate}[label={\roman*})]
                \item $(A\cdot B)\cdot C = A\cdot(B \cdot C)$\pause
                \item $(A + B)\cdot C = (A\cdot C) + (B\cdot C)$\pause
                \item $C\cdot(A + B) = (C\cdot A) + (C\cdot B)$
                \seti
            \end{enumerate}
        \end{proposicao}
    \end{frame}

    \begin{frame}
        \begin{proposicao}
            \begin{enumerate}[label={\roman*})]
                \conti
                \item $\alpha(A\cdot B) = \pause (\alpha A)\cdot B \pause = A \cdot(\alpha B)$\pause
                \item Se $A$ é uma matriz quadrada de ordem $n$, \pause então \[A\cdot I_n = A = I_n\cdot A,\] \pause onde $I_n$ é a matriz quadrada de ordem $n$ \pause onde todas as entradas da diagonal principal são 1 \pause e as demais são zero.\pause
                \item Se $A$ é uma matriz quadrada de ordem $n$, \pause então \[A\cdot0_n = 0_n \pause = 0_n\cdot A.\]
	    \end{enumerate}
        \end{proposicao}
    \end{frame}
\end{document}
