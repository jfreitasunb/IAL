%!TEX program = xelatex
\def\ano{2023}
\def\semestre{1}
\def\disciplina{Introdução à Álgebra Linear}
\def\turma{3}
\def\autor{Jos\'e Ant\^onio O. Freitas}
\def\instituto{MAT-UnB}

\documentclass{beamer}
\usetheme{Madrid}
\usecolortheme{beaver}
% \mode<presentation>
\usepackage{caption}
\usepackage{amssymb}
\usepackage{amsmath,amsfonts,amsthm,amstext}
\usepackage[brazil]{babel}
\usepackage{graphicx}
\graphicspath{{../Pictures/}}
\usepackage{enumitem}
\usepackage{multicol}
\usepackage{answers}
\usepackage[svgnames]{xcolor}
\usepackage{tikz}
\usepackage{ifthen}
\usetikzlibrary{lindenmayersystems}
\usetikzlibrary[shadings]

\newcounter{exercicios}
\setcounter{exercicios}{0}
\newcommand{\questao}{
    \addtocounter{exercicios}{1}
    \noindent{\bf Quest{\~a}o \arabic{exercicios}: }}

\newcommand{\resp}[1]{
    \noindent{\bf Exerc{\'\i}cio #1: }}

\extrafootheight[.25in]{.5in}
\footrule
\lfoot{Teste \numeroteste\ - Módulo \modulo\ - \nomeabreviado\ - Turma \turma\ - \semestre$^o$/\ano}
\cfoot{}
\rfoot{P\'agina \thepage\ de \numpages}
\def\ano{2023}
\def\semestre{2}
\def\disciplina{Introdução à Álgebra Linear}
\def\nomeabreviado{IAL}
\def\turma{11}

\newcommand{\im}{{\rm Im\,}}
\newcommand{\dlim}[2]{\displaystyle\lim_{#1\rightarrow #2}}
\newcommand{\minf}{+\infty}
\newcommand{\ninf}{-\infty}
\newcommand{\cp}[1]{\mathbb{#1}}
\newcommand{\sub}{\subseteq}
\newcommand{\n}{\mathbb{N}}
\newcommand{\z}{\mathbb{Z}}
\newcommand{\rac}{\mathbb{Q}}
\newcommand{\real}{\mathbb{R}}
\newcommand{\complex}{\mathbb{C}}

\newcommand{\vesp}[1]{\vspace{ #1  cm}}

\newcommand{\compcent}[1]{\vcenter{\hbox{$#1\circ$}}}
\newcommand{\comp}{\mathbin{\mathchoice
        {\compcent\scriptstyle}{\compcent\scriptstyle}
        {\compcent\scriptscriptstyle}{\compcent\scriptscriptstyle}}}
\renewcommand{\sin}{{\rm sen\,}}
\renewcommand{\tan}{{\rm tg\,}}
\renewcommand{\csc}{{\rm cossec\,}}
\renewcommand{\cot}{{\rm cotg\,}}
\renewcommand{\sinh}{{\rm senh\,}}

\title{Apresentação do curso e noções iniciais}
\author[\autor]{\autor}
\institute[\instituto]{\instituto}
\date{}

\begin{document}
    \begin{frame}
        \maketitle
    \end{frame}

    \logo{\includegraphics[scale=.1]{logo-MAT.png}\vspace*{8.5cm}}

    \begin{frame}
        {\bf Conteúdo a ser visto no curso:}
        \begin{enumerate}[label={\roman*})]
            \item Matrizes e sistemas lineares. Determinantes.

            \item Espaços vetoriais. Transformações Lineares.

            \item Autovalores e autovetores. Diagonalização de operadores. Espaços vetoriais com produto interno.
        \end{enumerate}
    \end{frame}

    \begin{frame}
        {\bf Bibliografia:}
        \begin{enumerate}[label={\arabic*})]
            \item H. Anton, C. Rorres: {\it {\'A}lgebra Linear com Aplicações}, $10^a$ Ed., Editora Bookman, 2012.

        \item C. H. Edwards, D. E. Penney: {\it Introdução à Álgebra Linear}, Editora Prentice-Hall do Brasil.

            \item S. Lang: {\it Álgebra Linear}, Editora Moderna, 2003.

            \item T. S. Blyth, E. F. Robertson: {\it Basic Linear Algebra}, $2^a$ Ed., Springer, 2002.
        \end{enumerate}
    \end{frame}

    \begin{frame}
        {\bf{Sistema de avaliação:}} três módulos, 1 avalição por módulo.
            Nota $M_i$, $i=1$, 2, 3. A nota final ($NF$) de cada estudante ser\'a dada por:
        \[
            NF = \dfrac{2M_1 + 3M_2 + 4M_3}{9}, \quad 0 \le NF \le 10.
        \]
        Ser\'a considerado aprovado o estudante que obtiver $NF \ge 5$.
    \end{frame}

    \begin{frame}
        \bf{Datas das provas:}
        \begin{center}
            \begin{tabular}{c|c|c}
                \hline\hline
                \hspace{1cm}{\bf Prova}\hspace{1cm} & \hspace{3cm}{\bf Data}\hspace{3cm} & \hspace{1.7cm}{\bf Hor\'{a}rio}\hspace{1.7cm} \\
                \hline\hline
                $P_1$ & 25/04/23 (terça-feira) \phantom{x} & 19:00 - 20:40 \\
                \hline
                $P_2$ & 08/06/23 (quinta-feira) \phantom{x} & 19:00 - 20:40 \\
                \hline
                $P_3$ & 20/07/23 (quinta-feira) \phantom{x} & 19:00 - 20:40 \\
                \hline\hline
            \end{tabular}
        \end{center}
    \end{frame}
\end{document}
