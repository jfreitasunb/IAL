%!TEX program = xelatex
\def\ano{2023}
\def\semestre{1}
\def\disciplina{Introdução à Álgebra Linear}
\def\turma{3}
\def\autor{Jos\'e Ant\^onio O. Freitas}
\def\instituto{MAT-UnB}

\documentclass{beamer}
\usetheme{Madrid}
\usecolortheme{beaver}
% \mode<presentation>
\usepackage{caption}
\usepackage{amssymb}
\usepackage{amsmath,amsfonts,amsthm,amstext}
\usepackage[brazil]{babel}
\usepackage{graphicx}
\graphicspath{{../Pictures/}}
\usepackage{enumitem}
\usepackage{multicol}
\usepackage{answers}
\usepackage[svgnames]{xcolor}
\usepackage{tikz}
\usepackage{ifthen}
\usetikzlibrary{lindenmayersystems}
\usetikzlibrary[shadings]

\newcounter{exercicios}
\setcounter{exercicios}{0}
\newcommand{\questao}{
    \addtocounter{exercicios}{1}
    \noindent{\bf Quest{\~a}o \arabic{exercicios}: }}

\newcommand{\resp}[1]{
    \noindent{\bf Exerc{\'\i}cio #1: }}

\extrafootheight[.25in]{.5in}
\footrule
\lfoot{Teste \numeroteste\ - Módulo \modulo\ - \nomeabreviado\ - Turma \turma\ - \semestre$^o$/\ano}
\cfoot{}
\rfoot{P\'agina \thepage\ de \numpages}
\def\ano{2023}
\def\semestre{2}
\def\disciplina{Introdução à Álgebra Linear}
\def\nomeabreviado{IAL}
\def\turma{11}

\newcommand{\im}{{\rm Im\,}}
\newcommand{\dlim}[2]{\displaystyle\lim_{#1\rightarrow #2}}
\newcommand{\minf}{+\infty}
\newcommand{\ninf}{-\infty}
\newcommand{\cp}[1]{\mathbb{#1}}
\newcommand{\sub}{\subseteq}
\newcommand{\n}{\mathbb{N}}
\newcommand{\z}{\mathbb{Z}}
\newcommand{\rac}{\mathbb{Q}}
\newcommand{\real}{\mathbb{R}}
\newcommand{\complex}{\mathbb{C}}

\newcommand{\vesp}[1]{\vspace{ #1  cm}}

\newcommand{\compcent}[1]{\vcenter{\hbox{$#1\circ$}}}
\newcommand{\comp}{\mathbin{\mathchoice
        {\compcent\scriptstyle}{\compcent\scriptstyle}
        {\compcent\scriptscriptstyle}{\compcent\scriptscriptstyle}}}
\renewcommand{\sin}{{\rm sen\,}}
\renewcommand{\tan}{{\rm tg\,}}
\renewcommand{\csc}{{\rm cossec\,}}
\renewcommand{\cot}{{\rm cotg\,}}
\renewcommand{\sinh}{{\rm senh\,}}

\title{Transformações Lineares}
\author[\autor]{\autor}
\institute[\instituto]{\instituto}
\date{}

\begin{document}
\begin{frame}
	\maketitle
\end{frame}

\logo{\includegraphics[scale=.1]{logo-MAT.png}\vspace*{8.5cm}}

\begin{frame}
    \begin{definicao}
        Sejam $(V, +, \cdot)$ e $(W, \oplus, \otimes)$ espaços vetoriais sobre um corpo $\cp{K}$. Uma função $T : V \to W$ é uma \textbf{transformação linear} se
        \begin{enumerate}[label={\roman*})]
            \item $T(u_1 + u_2) = T(u_1) \oplus T(u_2)$ para todos $u_1$, $u_2 \in V$;
            \item $T(\lambda \cdot u) = \lambda \otimes T(u)$ para todo $\lambda \in \cp{K}$ e todo $u \in V$.
        \end{enumerate}
    \end{definicao}
\end{frame}

\begin{frame}
    \begin{observacao}
        Para simplificar a notação, vamos adotar os mesmos símbolos para indicar a soma e o produto por escalar nos espaços vetoriais que aparecerão no decorrer do texto. No entanto, o leitor deve estar ciente que estes símbolos podem ter significados diferentes, dependendo do espaço vetorial em questão.
    \end{observacao}

    \begin{lema}
        Sejam $V$ e $W$ espaços vetoriais sobre $\cp{K}$. Então uma função $T : V \to W$ é uma transformação linear se, e somente se,
        \[
            T(\lambda u_1 + u_2) = \lambda T(u_1) + T(u_2),
        \]
        para todos $u_1$, $u_2 \in V$ e todo $\lambda \in \cp{K}$.
    \end{lema}
\end{frame}

\begin{frame}
    \begin{lema}
        Sejam $V$ e $W$ espaços vetoriais sobre $\cp{K}$ e $T : V \to W$ uma transformação linear. Então:
        \begin{enumerate}[label={\roman*})]\label{transformacao_linear_propriedades_basicas}
            \item $T(0_V) = 0_W$, onde $0_V$ e $0_W$ denotam os vetores nulos de $V$ e $W$, respectivamente.

            \item $T(-u) = -T(u)$, para cada $u \in V$.

            \item $T(\alpha_1u_1 + \alpha_2u_2 + \cdots + \alpha_mu_m) = \alpha_1 T(u_1) + \alpha_2 T(u_2) + \cdots + \alpha_m T(u_m)$, onde $\alpha_i \in \cp{K}$ e $u_i \in V$ para $i = 1$, \dots, $m$.
        \end{enumerate}
    \end{lema}
\end{frame}

\begin{frame}
    \begin{teorema}\label{existencia_de_transformacao_unica_dado_valores}
        Sejam $V$ e $W$ $\cp{K}$-espa\c{c}os vetoriais. Se $\{u_1, \dots, u_n\}$ \'e uma base de $V$ e se $\{w_1, \dots, w_n\} \subseteq W$, ent\~ao existe uma \'unica transforma\c{c}\~ao linear $T : V \to W$ tal que $T(u_i) = w_i$ para cada $i = 1$, \dots, $n$.
    \end{teorema}
\end{frame}

\begin{frame}
    \begin{definicao}
        Sejam $V$ e $W$ $\cp{K}$-espa\c{c}os vetoriais e $T : V \to W$ uma transforma\c{c}\~ao linear.
        \begin{enumerate}[label={\roman*})]
            \item O conjunto
            \[
            \ker T = \{u \in V \mid T(u) = 0_W\}
            \]
            \'e chamado de \textbf{kernel} ou \textbf{n\'ucleo} de $T$. (O n\'ucleo de $T$ tamb\'em pode ser denotado por $Nuc\ T$.)

            \item O conjunto
            \[
            \im T = \{u \in W \mid \mbox{ existe } v \in V \mbox{ tal que } T(v) = u\}
            \]
            \'e chamado de \textbf{imagem} de $T$.
        \end{enumerate}
    \end{definicao}

    \begin{proposicao}
        Sejam $V$ e $W$ $\cp{K}$-espa\c{c}os vetoriais e $T : V \to W$ uma transforma\c{c}\~ao linear. Ent\~ao:
        \begin{enumerate}[label={\roman*})]
            \item $\ker T$ \'e um subespa\c{c}o de $V$;
            \item $\im T$ \'e um subespa\c{c}o de $W$.
        \end{enumerate}
    \end{proposicao}
\end{frame}

\end{document}