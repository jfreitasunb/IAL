%!TEX program = xelatex
%!TEX encoding = UTF-8
\def\ano{2023}
\def\semestre{2}
\def\disciplina{Introdução à Álgebra Linear}
\def\turma{11}
\def\autor{José Antônio O. Freitas}
\def\instituto{MAT-UnB}

\documentclass{beamer}
\usetheme{Madrid}
\usecolortheme{beaver}
% \mode<presentation>
\usepackage{caption}
\usepackage{amssymb}
\usepackage{amsmath,amsfonts,amsthm,amstext}
\usepackage[brazil]{babel}
\usepackage{graphicx}
\graphicspath{{../Pictures/}}
\usepackage{enumitem}
\usepackage{multicol}
\usepackage{answers}
\usepackage[svgnames]{xcolor}
\usepackage{tikz}
\usepackage{ifthen}
\usetikzlibrary{lindenmayersystems}
\usetikzlibrary[shadings]

\newcounter{exercicios}
\setcounter{exercicios}{0}
\newcommand{\questao}{
    \addtocounter{exercicios}{1}
    \noindent{\bf Quest{\~a}o \arabic{exercicios}: }}

\newcommand{\resp}[1]{
    \noindent{\bf Exerc{\'\i}cio #1: }}

\extrafootheight[.25in]{.5in}
\footrule
\lfoot{Teste \numeroteste\ - Módulo \modulo\ - \nomeabreviado\ - Turma \turma\ - \semestre$^o$/\ano}
\cfoot{}
\rfoot{P\'agina \thepage\ de \numpages}
\def\ano{2023}
\def\semestre{2}
\def\disciplina{Introdução à Álgebra Linear}
\def\nomeabreviado{IAL}
\def\turma{11}

\newcommand{\im}{{\rm Im\,}}
\newcommand{\dlim}[2]{\displaystyle\lim_{#1\rightarrow #2}}
\newcommand{\minf}{+\infty}
\newcommand{\ninf}{-\infty}
\newcommand{\cp}[1]{\mathbb{#1}}
\newcommand{\sub}{\subseteq}
\newcommand{\n}{\mathbb{N}}
\newcommand{\z}{\mathbb{Z}}
\newcommand{\rac}{\mathbb{Q}}
\newcommand{\real}{\mathbb{R}}
\newcommand{\complex}{\mathbb{C}}

\newcommand{\vesp}[1]{\vspace{ #1  cm}}

\newcommand{\compcent}[1]{\vcenter{\hbox{$#1\circ$}}}
\newcommand{\comp}{\mathbin{\mathchoice
        {\compcent\scriptstyle}{\compcent\scriptstyle}
        {\compcent\scriptscriptstyle}{\compcent\scriptscriptstyle}}}
\renewcommand{\sin}{{\rm sen\,}}
\renewcommand{\tan}{{\rm tg\,}}
\renewcommand{\csc}{{\rm cossec\,}}
\renewcommand{\cot}{{\rm cotg\,}}
\renewcommand{\sinh}{{\rm senh\,}}

\title{Transformações Lineares}
\author[\autor]{\autor}
\institute[\instituto]{\instituto}
\date{}

\begin{document}
\begin{frame}
    \maketitle
\end{frame}

\logo{\includegraphics[scale=.1]{logo-MAT.png}\vspace*{8.5cm}}

\begin{frame}
    \begin{definicao}
        Sejam $(V, +, \cdot)$  e $(W, \oplus, \otimes)$ \pause espaços vetoriais sobre um corpo $\cp{K}$. \pause Uma função $T \colon V \to
        W$ \pause é uma \textbf{transformação linear} se:\pause
        \begin{enumerate}[label={\roman*})]

            \item $T(u_1 + u_2) \pause = T(u_1) \pause \oplus T(u_2)$ \pause para todos $u_1$, $u_2 \in V$;\pause

            \vspace*{1.5cm}

            \item $T(\lambda \cdot u) \pause = \lambda \pause \otimes T(u)$ \pause para todo $\lambda \in \cp{K}$ e todo $u \in V$.
        \end{enumerate}
    \end{definicao}
\end{frame}

\begin{frame}
    \begin{observacoes}
        \begin{enumerate}[label={\roman*})]
            \item Se $T \colon V \to V$ é uma transformação linear \pause de $V$ em $V$, então \pause $T$ é chamada de um \textbf{operador
                linear}.\pause

            \vspace*{1.5cm}

            \item Para simplificar a notação, \pause vamos adotar os mesmos símbolos para indicar a soma \pause e o produto por escalar
                \pause nos espaços vetoriais que aparecerão no decorrer do conteúdo. \pause No entanto, deve-se estar ciente que estes
                símbolos podem ter significados diferentes, \pause dependendo do espaço vetorial em questão.
        \end{enumerate}
    \end{observacoes}
\end{frame}

\begin{frame}
    \begin{lema}
        Sejam $V$ e $W$ espaços vetoriais sobre $\cp{K}$ \pause e $T \colon V \to W$ uma transformação linear. \pause Então:\pause
        \begin{enumerate}[label={\roman*})]
            \item $T(0_V) \pause = 0_W$, \pause onde $0_V$ e $0_W$ \pause denotam os vetores nulos de $V$ e $W$, respectivamente.\pause

            \vspace*{1cm}

            \item $T(-u) \pause = -T(u)$, \pause para cada $u \in V$.\pause

            \vspace{1cm}

            \item $T(\alpha_1u_1 \pause + \alpha_2u_2 \pause + \cdots + \pause \alpha_mu_m) \pause = \alpha_1 T(u_1) \pause + \alpha_2 T(u_2) \pause + \cdots + \alpha_m T(u_m)$, \pause onde $\alpha_i \in \cp{K}$ e $u_i \in V$ para $i = 1$, \dots, $m$.
        \end{enumerate}
    \end{lema}
\end{frame}

\begin{frame}
    \begin{lema}
        Sejam $V$ e $W$ espaços vetoriais sobre $\cp{K}$. \pause Então uma função $T \colon V \to W$ \pause é uma transformação linear se, e somente se,\pause
        \[
            T(\lambda u_1  + u_2) \pause = \lambda  T(u_1)  + T(u_2),\pause
        \]
        para todos $u_1$, $u_2 \in V$ \pause e todo $\lambda \in \cp{K}$.
    \end{lema}
\end{frame}

\begin{frame}
    \begin{teorema}
        Sejam $V$ e $W$ espaços vetoriais sobre $\cp{K}$. \pause Se $\{u_1, \pause \dots, \pause u_n\}$ \pause é uma base de $V$ \pause e se $\{w_1, \pause \dots, \pause w_n\} \subseteq W$, \pause então existe uma única transformação linear \pause $T \colon V \to W$ \pause tal que $T(u_1) = w_1$, \pause $T(u_2) = w_2$, \pause \dots, $T(u_n) = w_n$.
    \end{teorema}
\end{frame}

\begin{frame}
    \begin{definicao}
        Sejam $V$ e $W$ espaços vetoriais sobre um corpo $\cp{K}$ \pause e $T \colon V \to W$ uma transformação linear.\pause
        \begin{enumerate}[label={\roman*})]
            \item O conjunto\pause
            \[
                \ker(T) \pause = \{u \in V  \mid T(u) = 0_W\}\pause
            \]
            é chamado de \textbf{kernel} \pause ou \textbf{núcleo} de $T$. \pause O núcleo de $T$ também pode ser denotado por
            $Nuc(T)$.\pause

            \vspace*{1.5cm}

            \item O conjunto\pause
            \[
                \im(T)\pause = \{w \in W \pause \mid \mbox{ existe } \pause u \in V \pause \mbox{ tal que } \pause T(u) = w\}\pause
            \]
            é chamado de \textbf{imagem} de $T$.
        \end{enumerate}
    \end{definicao}
\end{frame}

\begin{frame}

    \begin{proposicao}
        Sejam $V$ e $W$ espaços vetoriais sobre $\cp{K}$ \pause e $T \colon V \to W$ uma transformação linear. \pause Então:
        \begin{enumerate}[label={\roman*})]
            \item $\ker(T)$ \pause é um subespaço de $V$;\pause

            \vspace*{1cm}

            \item $\im(T)$ \pause é um subespaço de $W$.
        \end{enumerate}
    \end{proposicao}
\end{frame}

\begin{frame}
    \begin{teorema}[Teorema do N\'ucleo e da Imagem]
        Sejam $V$ e $W$ $\cp{K}$-espa\c{c}os vetoriais \pause com $\dim_\cp{K}V$ finita. \pause Seja $T \colon V \to W$ uma transforma\c{c}\~ao linear. \pause Ent\~ao\pause
        \[
            \dim_\cp{K}V \pause = \dim_\cp{K}\ker(T) \pause + \dim_\cp{K}\im(T).
        \]
    \end{teorema}
\end{frame}

\begin{frame}
    \begin{definicao}
        Sejam $V$ e $W$ $\cp{K}$-espa\c{c}os vetoriais \pause e $T \colon V \to W$ uma transforma\c{c}\~ao linear.\pause
        \begin{enumerate}[label={\roman*})]
            \item Dizemos que $T$ \'e \textbf{injetora} \pause se dados $u_1$, $u_2 \in V$ \pause tais que $T(u_1) = T(u_2)$, \pause ent\~ao $u_1 = u_2$. \pause De modo equivalente, \pause se $u_1$, $u_2 \in V$ s\~ao tais que $u_1 \ne u_2$, \pause ent\~ao $T(u_1) \ne T(u_2)$.

            \vspace{.5cm}

            \item Dizemos que $T$ \'e \textbf{sobrejetora} \pause se $\im(T) = W$. \pause Em outras palavras, \pause $T$ \'e \textbf{sobrejetora} se para todo $w \in W$, \pause existe $u \in V$ tal que $T(u) = w$.
        \end{enumerate}
    \end{definicao}
\end{frame}

\begin{frame}
    \begin{proposicao}
        Sejam $V$ e $W$ espa\c{c}os vetoriais sobre um corpo $\cp{K}$ \pause e $T \colon V \to W$ uma transforma\c{c}\~ao linear.\pause
        \begin{enumerate}[label={\roman*})]
            \item $T$ \'e injetora \pause se, e somente se, $\ker(T) = \{0_V\}$.\pause

            \vspace{.5cm}

            \item Se $\dim V = \dim W$  e $T$ é injetora, \pause então $T$ transforma bases em bases, \pause isto é, se $\mathcal{B} = \{v_1, v_2, \dots, v_n\}$ é uma base de $V$, \pause então $\{T(v_1), \pause T(v_2), \pause \dots, T(v_n)\}$ é uma base de $W$.\pause

            \vspace{.5cm}

            \item Se $\dim V = \dim W$, \pause então $T$ é injetora \pause se, e somente se, \pause $T$ é sobrejetora.
        \end{enumerate}
    \end{proposicao}
\end{frame}

\begin{frame}
    \begin{observacoes}
        Seja $T \colon V \to W$ uma transformação linear, \pause onde $\dim V < \infty$ e $\dim W < \infty$.\pause
        \begin{enumerate}[label={\roman*})]
            \item Se $\dim V > \dim W$, \pause então $T$ não pode ser injetora, \pause isto é, $\ker(T) \ne \emptyset$.\pause

            \vspace{.5cm}

            \item Se $\dim V < \dim W$, \pause então $T$ não pode ser sobrejetora, \pause isto é, $\im(T) \ne W$.
        \end{enumerate}
    \end{observacoes}
\end{frame}

\begin{frame}
    \begin{definicao}
        Se $T \colon V \to W$ é uma transformação linear \pause \textbf{injetora e sobrejetora}, \pause então dizemos que $T$ é uma transformação linear \textbf{bijetora}, \pause ou que $T$ é um \textbf{isomorfismo} \pause do espaço $V$ no espaço $W$. \pause Neste caso, dizemos que o espaço $V$ é isomorfo ao espaço $W$ \pause e denotamos isso, escrevendo $V \cong W$.\pause
    \end{definicao}

    \begin{teorema}
        Seja $V$ um espaço vetorial sobre $\cp{K}$ \pause com $\dim V = n$, \pause então $V \cong \cp{K}^n$.
    \end{teorema}
\end{frame}

\begin{frame}
    \begin{observacoes}
        Seja $V$ um espaço vetorial sobre $\cp{K}$ \pause com $\dim V = n$. \pause Então do teorema anterior temos:\pause
        \begin{enumerate}[label={\roman*})]
            \item Se $\cp{K} = \rac$, \pause então $V \cong \rac^n$.\pause

            \vspace{.5cm}

            \item Se $\cp{K} = \real$, \pause então $V \cong \real^n$.\pause

            \vspace{.5cm}

            \item Se $\cp{K} = \complex$, \pause então $V \cong \complex^n$.
        \end{enumerate}
    \end{observacoes}
\end{frame}

\begin{frame}
    \begin{definicao}
        Dadas $S \colon V \to W$ \pause e $T \colon W \to U$ transformações lineares, \pause definimos a \textbf{composição de $T$ com $S$} \pause como a aplicação:\pause
        \begin{align*}
            T \circ S \colon &V \to U\\
                             &v \mapsto (T\circ S)(u) = T(S(v))
        \end{align*}
    \end{definicao}
\end{frame}

\begin{frame}
    \begin{proposicao}
        Se $S \colon V \to W$ e $T \colon W \to U$ são transformações lineares, \pause então a composição de $T\circ S$ também é uma transformação linear.
    \end{proposicao}
\end{frame}


\begin{frame}
    \begin{definicao}
        Dado $T \colon V \to V$ um operador linear, \pause dizemos que $T$ é \textbf{invertível} \pause se existe outra aplicação $T^{-1} \colon V \to V$ \pause tal que para qualquer $u \in V$,  $T^{-1}(T(u)) = u$.
    \end{definicao}
\end{frame}

\begin{frame}
    \begin{observacoes}
        \begin{enumerate}[label={\roman*})]
            \item Seja $I = I_V = Id = Id_V \colon V \to V$ \pause a transformação linear tal que $I(u) = Id(u) = u$ \pause para todo $u \in V$. \pause Então $I = Id$ é chamado de \textbf{operador identidade}.\pause

            \vspace{.5cm}

            \item Se $T \colon V \to V$ é uma transformação linear invertível, \pause então a aplicação $T^{-1} \colon V \to V$ também é uma transformação linear.\pause

            \vspace{.5cm}

            \item Se $T \colon V \to V$ é um operador invertível, \pause então $T^{-1} \colon V \to V$ é chamado de \textbf{operador inverso de $T$}.
        \end{enumerate}
    \end{observacoes}
\end{frame}

\begin{frame}
    \begin{proposicao}
        Seja $T \colon V \to V$ um operador linear.  Então:
        \begin{enumerate}[label={\roman*})]
            \item Se $T$ é invertível \pause e $T^{-1} \colon V \to V$ é o seu inverso, \pause então
                \[
                    (T^{-1} \circ T) = \pause (T \circ T^{-1}) =  Id.\pause
                \]

            \vspace{.5cm}

            \item Se $V$ tem dimensão finita, \pause então $T$ é invertível \pause se, e somente se, \pause $\ker(T) = \{0_V\}$.\pause

            \vspace{.5cm}

            \item Se $T$ é invertível, \pause então dada uma base $\{v_1, v_2, \dots, v_n\}$ de $V$, \pause o conjunto $\{T(v_1), \pause T(v_2), \pause \dots, T(v_n)\}$ \pause é uma base de $V$.
        \end{enumerate}
    \end{proposicao}
\end{frame}

\begin{frame}
    \begin{proposicao}
        Se $S \colon V \to V$ e $T \colon V \to V$ \pause são operadores invertíveis, \pause então $S \circ T$ é invertível e\pause
        \[
            (S \circ T)^{-1} \pause = T^{-1} \pause \circ S^{-1}.
        \]
    \end{proposicao}
\end{frame}

\begin{frame}
    \begin{teorema}
        Sejam $V$ e $W$ espaços vetoriais sobre $\cp{K}$, \pause de dimensões $p \ge 1$ e $q \ge 1$, respectivamente. \pause Sejam $\mathcal{B}_V$ e $\mathcal{B}_W$ bases ordenadas de $V$ e $W$, respectivamente. \pause Então para cada transformação linear $T \colon V \to W$, \pause existe uma matriz $A \in \cp{M}_{q\times p}(\cp{K})$ tal que\pause
        \[
            [T(x)]_{\mathcal{B}_W} \pause = A[x]_{\mathcal{B}_V}\pause
        \]
        para todo vetor $x \in V$. \pause Além disso, \pause a cada matriz $C \in \cp{M}_{q\times p}(\cp{K})$ \pause corresponde uma transformação linear $T \colon V \to W$ tal que\pause
        \[
            [T(x)]_{\mathcal{B}_W} \pause = C[x]_{\mathcal{B}_V}\pause
        \]
        para todo $x \in V$.
    \end{teorema}
\end{frame}

\begin{frame}
    \begin{definicao}
        A matriz $A \in \cp{M}_{q\times p}(\cp{K})$ do Teorema anterior \pause é chamada de \textbf{matriz da transformação linear} \pause $T$ com respeito às bases ordenadas $\mathcal{B}_V$ e $\mathcal{B}_W$ \pause e será denotada por\pause
        \[
            A = [T]_{\mathcal{B}_{W},\mathcal{B}_{V}}.\pause
        \]
        No caso em que $V = W$ \pause e $\mathcal{B}_V = \mathcal{B}_W = \mathcal{B}$,  denotaremos $[T]_{\mathcal{B}_{V},\mathcal{B}_{W}}$ \pause simplesmente por $[T]_\mathcal{B}$.
    \end{definicao}
\end{frame}

\begin{frame}
    \begin{teorema}
        Sejam $F \colon U \to V$ \pause e $G \colon V \to W$ \pause duas transformações lineares onde $U$, $V$ e $W$ \pause são espaços vetoriais $\cp{K}$ \pause de dimensões $n$, $m$ e $r$, respectivamente. \pause Fixe bases ordenadas $\mathcal{B}_U$, \pause $\mathcal{B}_V$ \pause e $\mathcal{B}_W$ \pause para $U$, $V$ e $W$, respectivamente. \pause Então $(G \circ F) \colon U \to W$ \pause dada por $(G\circ F)(v) = G(F(v))$ \pause é uma transformação linear e\pause
        \[
            [(G \circ F)]_{{\mathcal{B}_W},{\mathcal{B}_U}} \pause = [G]_{{\mathcal{B}_W},{\mathcal{B}_V}}[F]_{{\mathcal{B}_V},{\mathcal{B}_U}}.
        \]
    \end{teorema}
\end{frame}

\begin{frame}
    \begin{teorema}
        Sejam $T \colon V \to V$ é um operador linear \pause e $\mathcal{B}$ é uma base ordenada de $V$. \pause Então as seguintes afirmações são equivalentes:\pause
        \begin{enumerate}[label={\roman*})]
            \item $T$ é injetor\pause

            \vspace{1cm}

            \item $[T]_\mathcal{B}$ é invertível.\pause
        \end{enumerate}

        \vspace{.2cm}

        Além disso, se valem essas condições equivalentes, então\pause
        \[
            [T^{-1}]_\mathcal{B} \pause = [T]_\mathcal{B}^{-1}.
        \]
    \end{teorema}
\end{frame}

\begin{frame}
    \begin{teorema}
        Sejam $V$ um $\cp{K}$-espaço vetorial de dimensão $n \ge 1$, \pause $\mathcal{B}_1$ e $\mathcal{B}_2$ bases ordenadas de $V$. \pause Então existe uma única matriz $P \in \cp{M}_n(\cp{K})$, \pause necessariamente invertível tal que\pause
        \begin{enumerate}[label={\roman*})]
            \item $[x]_{\mathcal{B}_1} = \pause P[x]_{\mathcal{B}_2}$\pause

            \vspace{1cm}

            \item $[x]_{\mathcal{B}_2} = \pause P^{-1}[x]_{\mathcal{B}_1}$\pause
        \end{enumerate}

        \vspace{.2cm}

        para todo $x \in V$. \pause As colunas de $P = \begin{bmatrix}P_1 & P_2 & \cdots & P_n \end{bmatrix}$ \pause são dadas por\pause
        \[
            P_j = [w_j]_{\mathcal{B}_1}\pause
        \]
        para $j = 1$, \dots, $n$.
    \end{teorema}
\end{frame}

\begin{frame}
    \begin{teorema}
        Seja $P \in \cp{M}_n(\cp{K})$ um matriz invertível. \pause Sejam $V$ um $\cp{K}$-espaço vetorial de dimensão $n \ge 1$ \pause e $\mathcal{B}_1$ uma base ordenada de $V$. \pause Então existe uma única base ordenada $\mathcal{B}_2$ \pause de $V$ tal que\pause
        \begin{enumerate}[label={\roman*})]
            \item $[x]_{\mathcal{B}_1} = \pause P[x]_{\mathcal{B}_2}$\pause

            \vspace{1cm}

            \item $[x]_{\mathcal{B}_2} = \pause P^{-1}[x]_{\mathcal{B}_1}$\pause
        \end{enumerate}

        \vspace{.2cm}

        para todo $x \in V$.
\end{teorema}
\end{frame}

\begin{frame}
    \begin{definicao}
        A matriz $P \in \cp{M}_n(\cp{K})$ \pause do Teorema anterior \pause é chamada de \textbf{matriz de mudança de base} \pause e é denotada por $P = [I]_{{\mathcal{B}_1},{\mathcal{B}_2}}$.
    \end{definicao}
\end{frame}
\end{document}
