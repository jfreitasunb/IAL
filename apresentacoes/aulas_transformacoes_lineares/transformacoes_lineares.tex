%!TEX program = xelatex
\def\ano{2023}
\def\semestre{1}
\def\disciplina{Introdução à Álgebra Linear}
\def\turma{3}
\def\autor{Jos\'e Ant\^onio O. Freitas}
\def\instituto{MAT-UnB}

\documentclass{beamer}
\usetheme{Madrid}
\usecolortheme{beaver}
% \mode<presentation>
\usepackage{caption}
\usepackage{amssymb}
\usepackage{amsmath,amsfonts,amsthm,amstext}
\usepackage[brazil]{babel}
\usepackage{graphicx}
\graphicspath{{../Pictures/}}
\usepackage{enumitem}
\usepackage{multicol}
\usepackage{answers}
\usepackage[svgnames]{xcolor}
\usepackage{tikz}
\usepackage{ifthen}
\usetikzlibrary{lindenmayersystems}
\usetikzlibrary[shadings]

\newcounter{exercicios}
\setcounter{exercicios}{0}
\newcommand{\questao}{
    \addtocounter{exercicios}{1}
    \noindent{\bf Quest{\~a}o \arabic{exercicios}: }}

\newcommand{\resp}[1]{
    \noindent{\bf Exerc{\'\i}cio #1: }}

\extrafootheight[.25in]{.5in}
\footrule
\lfoot{Teste \numeroteste\ - Módulo \modulo\ - \nomeabreviado\ - Turma \turma\ - \semestre$^o$/\ano}
\cfoot{}
\rfoot{P\'agina \thepage\ de \numpages}
\def\ano{2023}
\def\semestre{2}
\def\disciplina{Introdução à Álgebra Linear}
\def\nomeabreviado{IAL}
\def\turma{11}

\newcommand{\im}{{\rm Im\,}}
\newcommand{\dlim}[2]{\displaystyle\lim_{#1\rightarrow #2}}
\newcommand{\minf}{+\infty}
\newcommand{\ninf}{-\infty}
\newcommand{\cp}[1]{\mathbb{#1}}
\newcommand{\sub}{\subseteq}
\newcommand{\n}{\mathbb{N}}
\newcommand{\z}{\mathbb{Z}}
\newcommand{\rac}{\mathbb{Q}}
\newcommand{\real}{\mathbb{R}}
\newcommand{\complex}{\mathbb{C}}

\newcommand{\vesp}[1]{\vspace{ #1  cm}}

\newcommand{\compcent}[1]{\vcenter{\hbox{$#1\circ$}}}
\newcommand{\comp}{\mathbin{\mathchoice
        {\compcent\scriptstyle}{\compcent\scriptstyle}
        {\compcent\scriptscriptstyle}{\compcent\scriptscriptstyle}}}
\renewcommand{\sin}{{\rm sen\,}}
\renewcommand{\tan}{{\rm tg\,}}
\renewcommand{\csc}{{\rm cossec\,}}
\renewcommand{\cot}{{\rm cotg\,}}
\renewcommand{\sinh}{{\rm senh\,}}

\title{Transformações Lineares}
\author[\autor]{\autor}
\institute[\instituto]{\instituto}
\date{}

\begin{document}
\begin{frame}
	\maketitle
\end{frame}

\logo{\includegraphics[scale=.1]{logo-MAT.png}\vspace*{8.5cm}}

\begin{frame}
    \begin{definicao}
        Sejam $(V, +, \cdot)$ \pause e $(W, \oplus, \otimes)$ \pause espaços vetoriais sobre um corpo $\cp{K}$. \pause Uma função $T \colon V \to W$ \pause é uma \textbf{transformação linear} se:\pause
        \begin{enumerate}[label={\roman*})]
            \item $T(u_1 + u_2) \pause = T(u_1) \pause \oplus T(u_2)$ \pause para todos $u_1$, $u_2 \in V$;\pause

            \vspace*{1.5cm}

            \item $T(\lambda \cdot u) \pause = \lambda \pause \otimes T(u)$ \pause para todo $\lambda \in \cp{K}$ e todo $u \in V$.
        \end{enumerate}
    \end{definicao}
\end{frame}

\begin{frame}
    \begin{observacoes}
        \begin{enumerate}[label={\roman*})]
            \item Se $T \colon V \to V$ é uma transformação linear \pause de $V$ em $V$, então \pause $T$ é chamada de um \textbf{operador linear}.\pause

            \vspace*{1.5cm}

            \item Para simplificar a notação, \pause vamos adotar os mesmos símbolos para indicar a soma \pause e o produto por escalar \pause nos espaços vetoriais que aparecerão no decorrer do conteúdo. \pause No entanto, deve-se estar ciente que estes símbolos podem ter significados diferentes, \pause dependendo do espaço vetorial em questão.
        \end{enumerate}
    \end{observacoes}
\end{frame}

\begin{frame}
    \begin{lema}
        Sejam $V$ e $W$ espaços vetoriais sobre $\cp{K}$ \pause e $T \colon V \to W$ uma transformação linear. \pause Então:\pause
        \begin{enumerate}[label={\roman*})]\label{transformacao_linear_propriedades_basicas}
            \item $T(0_V) \pause = 0_W$, \pause onde $0_V$ e $0_W$ \pause denotam os vetores nulos de $V$ e $W$, respectivamente.\pause

            \vspace*{1cm}

            \item $T(-u) \pause = -T(u)$, \pause para cada $u \in V$.\pause

            \vspace{1cm}

            \item $T(\alpha_1u_1 \pause + \alpha_2u_2 \pause + \cdots + \pause \alpha_mu_m) \pause = \alpha_1 T(u_1) \pause + \alpha_2 T(u_2) \pause + \cdots + \alpha_m T(u_m)$, \pause onde $\alpha_i \in \cp{K}$ e $u_i \in V$ para $i = 1$, \dots, $m$.
        \end{enumerate}
    \end{lema}
\end{frame}

\begin{frame}
    \begin{lema}
        Sejam $V$ e $W$ espaços vetoriais sobre $\cp{K}$. \pause Então uma função $T \colon V \to W$ \pause é uma transformação linear se, e somente se,\pause
        \[
        T(\lambda u_1 \pause + u_2) \pause = \lambda \pause T(u_1) \pause + T(u_2),\pause
        \]
        para todos $u_1$, $u_2 \in V$ \pause e todo $\lambda \in \cp{K}$.
    \end{lema}
\end{frame}

\begin{frame}
    \begin{teorema}\label{existencia_de_transformacao_unica_dado_valores}
        Sejam $V$ e $W$ espaços vetoriais sobre $\cp{K}$. \pause Se $\{u_1, \pause \dots, \pause u_n\}$ \pause é uma base de $V$ \pause e se $\{w_1, \pause \dots, \pause w_n\} \subseteq W$, \pause então existe uma única transformação linear \pause $T \colon V \to W$ \pause tal que $T(u_1) = w_1$, \pause $T(u_2) = w_2$, \pause \dots, $T(u_n) = w_n$.
    \end{teorema}
\end{frame}

\begin{frame}
    \begin{definicao}
        Sejam $V$ e $W$ espaços vetoriais sobre um corpo $\cp{K}$ \pause e $T \colon V \to W$ uma transformação linear.\pause
        \begin{enumerate}[label={\roman*})]
            \item O conjunto\pause
            \[
            \ker T \pause = \{u \in V \pause \mid T(u) = 0_W\}\pause
            \]
            é chamado de \textbf{kernel} \pause ou \textbf{núcleo} de $T$. \pause (O núcleo de $T$ também pode ser denotado por $Nuc\ T$.)\pause

            \vspace*{1.5cm}

            \item O conjunto\pause
            \[
            \im T \pause = \{w \in W \pause \mid \mbox{ existe } \pause u \in V \pause \mbox{ tal que } \pause T(u) = w\}\pause
            \]
            é chamado de \textbf{imagem} de $T$.
        \end{enumerate}
    \end{definicao}
\end{frame}

\begin{frame}

    \begin{proposicao}
        Sejam $V$ e $W$ espaços vetoriais sobre $\cp{K}$ \pause e $T \colon V \to W$ uma transformação linear. \pause Então:\pause
        \begin{enumerate}[label={\roman*})]
            \item $\ker T$ \pause é um subespaço de $V$;\pause

            \vspace*{1cm}

            \item $\im T$ \pause é um subespaço de $W$.
        \end{enumerate}
    \end{proposicao}
\end{frame}

\begin{frame}
    \begin{teorema}[Teorema do N\'ucleo e da Imagem]\label{teorema_do_nucleo_e_da_imagem}
        Sejam $V$ e $W$ $\cp{K}$-espa\c{c}os vetoriais \pause com $\dim_\cp{K}V$ finita. \pause Seja $T \colon V \to W$ uma transforma\c{c}\~ao linear. \pause Ent\~ao
        \[
            \dim_\cp{K}V \pause = \dim_\cp{K}\ker T \pause + \dim_\cp{K}\im T.
        \]
    \end{teorema}
\end{frame}

\begin{frame}
    \begin{definicao}
        Sejam $V$ e $W$ $\cp{K}$-espa\c{c}os vetoriais \pause e $T \colon V \to W$ uma transforma\c{c}\~ao linear.\pause
        \begin{enumerate}[label={\roman*})]
            \item Dizemos que $T$ \'e \textbf{injetora} \pause se dados $u_1$, $u_2 \in V$ \pause tais que $T(u_1) = T(u_2)$, \pause ent\~ao $u_1 = u_2$. \pause De modo equivalente, \pause se $u_1$, $u_2 \in V$ s\~ao tais que $u_1 \ne u_2$, \pause ent\~ao $T(u_1) \ne T(u_2)$.\pause

            \vspace{.5cm}

            \item Dizemos que $T$ \'e \textbf{sobrejetora} \pause se $\im T = W$. \pause Em outras palavras, \pause $T$ \'e \textbf{sobrejetora} se para todo $w \in W$, \pause existe $u \in V$ tal que $T(u) = w$.\pause

            \vspace{.5cm}

            \item Se $T$ \'e injetora e sobrejetora, \pause ent\~ao dizemos que $T$ \'e um \textbf{isomorfismo}.
        \end{enumerate}
    \end{definicao}
\end{frame}

\begin{frame}
    \begin{proposicao}\label{caracteriza_transformacao_injetora}
        Sejam $V$ e $W$ $\cp{K}$-espa\c{c}os vetoriais \pause e $T \colon V \to W$ uma transforma\c{c}\~ao linear.
        \begin{enumerate}[label={\roman*})]
            \item $T$ \'e injetora \pause se, e somente se, $\ker T = \{0_V\}$.

            \vspace{.5cm}

            \item Se $\dim V = \dim W$ e $T$ é injetora, então $T$ transforma bases em bases, isto é, se $\mathcal{B} = \{v_1, v_2, \dots, v_n\}$ é uma base de $V$, então $\{T(v_1), T(v_2), \dots, T(v_n)\}$ é uma base de $W$.

            \vspace{.5cm}

            \item Se $\dim V = \dim W$, então $T$ é injetora se, e somente se, $T$ é sobrejetora.
        \end{enumerate}
    \end{proposicao}
\end{frame}

\end{document}
