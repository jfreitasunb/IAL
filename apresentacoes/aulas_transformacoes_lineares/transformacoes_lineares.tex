%!TEX program = xelatex
%!TEX encoding = UTF-8
\def\ano{2023}
\def\semestre{2}
\def\disciplina{Introdução à Álgebra Linear}
\def\turma{11}
\def\autor{José Antônio O. Freitas}
\def\instituto{MAT-UnB}

\documentclass{beamer}
\usetheme{Madrid}
\usecolortheme{beaver}
% \mode<presentation>
\usepackage{caption}
\usepackage{amssymb}
\usepackage{amsmath,amsfonts,amsthm,amstext}
\usepackage[brazil]{babel}
\usepackage{graphicx}
\graphicspath{{../Pictures/}}
\usepackage{enumitem}
\usepackage{multicol}
\usepackage{answers}
\usepackage[svgnames]{xcolor}
\usepackage{tikz}
\usepackage{ifthen}
\usetikzlibrary{lindenmayersystems}
\usetikzlibrary[shadings]

\newcounter{exercicios}
\setcounter{exercicios}{0}
\newcommand{\questao}{
    \addtocounter{exercicios}{1}
    \noindent{\bf Quest{\~a}o \arabic{exercicios}: }}

\newcommand{\resp}[1]{
    \noindent{\bf Exerc{\'\i}cio #1: }}

\extrafootheight[.25in]{.5in}
\footrule
\lfoot{Teste \numeroteste\ - Módulo \modulo\ - \nomeabreviado\ - Turma \turma\ - \semestre$^o$/\ano}
\cfoot{}
\rfoot{P\'agina \thepage\ de \numpages}
\def\ano{2023}
\def\semestre{2}
\def\disciplina{Introdução à Álgebra Linear}
\def\nomeabreviado{IAL}
\def\turma{11}

\newcommand{\im}{{\rm Im\,}}
\newcommand{\dlim}[2]{\displaystyle\lim_{#1\rightarrow #2}}
\newcommand{\minf}{+\infty}
\newcommand{\ninf}{-\infty}
\newcommand{\cp}[1]{\mathbb{#1}}
\newcommand{\sub}{\subseteq}
\newcommand{\n}{\mathbb{N}}
\newcommand{\z}{\mathbb{Z}}
\newcommand{\rac}{\mathbb{Q}}
\newcommand{\real}{\mathbb{R}}
\newcommand{\complex}{\mathbb{C}}

\newcommand{\vesp}[1]{\vspace{ #1  cm}}

\newcommand{\compcent}[1]{\vcenter{\hbox{$#1\circ$}}}
\newcommand{\comp}{\mathbin{\mathchoice
        {\compcent\scriptstyle}{\compcent\scriptstyle}
        {\compcent\scriptscriptstyle}{\compcent\scriptscriptstyle}}}
\renewcommand{\sin}{{\rm sen\,}}
\renewcommand{\tan}{{\rm tg\,}}
\renewcommand{\csc}{{\rm cossec\,}}
\renewcommand{\cot}{{\rm cotg\,}}
\renewcommand{\sinh}{{\rm senh\,}}

\title{Transformações Lineares}
\author[\autor]{\autor}
\institute[\instituto]{\instituto}
\date{}

\begin{document}
\begin{frame}
    \maketitle
\end{frame}

\logo{\includegraphics[scale=.1]{logo-MAT.png}\vspace*{8.5cm}}

\begin{frame}
    \begin{definicao}
        Sejam $(V, +, \cdot)$  e $(W, \oplus, \otimes)$  espaços vetoriais sobre um corpo $\cp{K}$.  Uma função $T \colon V \to W$  é uma \textbf{transformação linear} se:
        \begin{enumerate}[label={\roman*})]
            \item $T(u_1 + u_2)  = T(u_1)  \oplus T(u_2)$  para todos $u_1$, $u_2 \in V$;

            \vspace*{1.5cm}

            \item $T(\lambda \cdot u)  = \lambda  \otimes T(u)$  para todo $\lambda \in \cp{K}$ e todo $u \in V$.
        \end{enumerate}
    \end{definicao}
\end{frame}

\begin{frame}
    \begin{observacoes}
        \begin{enumerate}[label={\roman*})]
            \item Se $T \colon V \to V$ é uma transformação linear  de $V$ em $V$, então  $T$ é chamada de um \textbf{operador linear}.

            \vspace*{1.5cm}

            \item Para simplificar a notação,  vamos adotar os mesmos símbolos para indicar a soma  e o produto por escalar  nos espaços vetoriais que aparecerão no decorrer do conteúdo.  No entanto, deve-se estar ciente que estes símbolos podem ter significados diferentes,  dependendo do espaço vetorial em questão.
        \end{enumerate}
    \end{observacoes}
\end{frame}

\begin{frame}
    \begin{lema}
        Sejam $V$ e $W$ espaços vetoriais sobre $\cp{K}$  e $T \colon V \to W$ uma transformação linear.  Então:
        \begin{enumerate}[label={\roman*})]\label{transformacao_linear_propriedades_basicas}
            \item $T(0_V)  = 0_W$,  onde $0_V$ e $0_W$  denotam os vetores nulos de $V$ e $W$, respectivamente.

            \vspace*{1cm}

            \item $T(-u)  = -T(u)$,  para cada $u \in V$.

            \vspace{1cm}

            \item $T(\alpha_1u_1  + \alpha_2u_2  + \cdots +  \alpha_mu_m)  = \alpha_1 T(u_1)  + \alpha_2 T(u_2)  + \cdots + \alpha_m T(u_m)$,  onde $\alpha_i \in \cp{K}$ e $u_i \in V$ para $i = 1$, \dots, $m$.
        \end{enumerate}
    \end{lema}
\end{frame}

\begin{frame}
    \begin{lema}
        Sejam $V$ e $W$ espaços vetoriais sobre $\cp{K}$.  Então uma função $T \colon V \to W$  é uma transformação linear se, e somente se,
        \[
        T(\lambda u_1  + u_2)  = \lambda  T(u_1)  + T(u_2),
        \]
        para todos $u_1$, $u_2 \in V$  e todo $\lambda \in \cp{K}$.
    \end{lema}
\end{frame}

\begin{frame}
    \begin{teorema}\label{existencia_de_transformacao_unica_dado_valores}
        Sejam $V$ e $W$ espaços vetoriais sobre $\cp{K}$.  Se $\{u_1,  \dots,  u_n\}$  é uma base de $V$  e se $\{w_1,  \dots,  w_n\} \subseteq W$,  então existe uma única transformação linear  $T \colon V \to W$  tal que $T(u_1) = w_1$,  $T(u_2) = w_2$,  \dots, $T(u_n) = w_n$.
    \end{teorema}
\end{frame}

\begin{frame}
    \begin{definicao}
        Sejam $V$ e $W$ espaços vetoriais sobre um corpo $\cp{K}$  e $T \colon V \to W$ uma transformação linear.
        \begin{enumerate}[label={\roman*})]
            \item O conjunto
            \[
            \ker(T)  = \{u \in V  \mid T(u) = 0_W\}
            \]
            é chamado de \textbf{kernel}  ou \textbf{núcleo} de $T$. O núcleo de $T$ também pode ser denotado por $Nuc(T)$.

            \vspace*{1.5cm}

            \item O conjunto
            \[
            \im(T)  = \{w \in W  \mid \mbox{ existe }  u \in V  \mbox{ tal que }  T(u) = w\}
            \]
            é chamado de \textbf{imagem} de $T$.
        \end{enumerate}
    \end{definicao}
\end{frame}

\begin{frame}

    \begin{proposicao}
        Sejam $V$ e $W$ espaços vetoriais sobre $\cp{K}$  e $T \colon V \to W$ uma transformação linear.  Então:
        \begin{enumerate}[label={\roman*})]
            \item $\ker(T)$  é um subespaço de $V$;

            \vspace*{1cm}

            \item $\im(T)$  é um subespaço de $W$.
        \end{enumerate}
    \end{proposicao}
\end{frame}

\begin{frame}
    \begin{teorema}[Teorema do N\'ucleo e da Imagem]\label{teorema_do_nucleo_e_da_imagem}
        Sejam $V$ e $W$ $\cp{K}$-espa\c{c}os vetoriais  com $\dim_\cp{K}V$ finita.  Seja $T \colon V \to W$ uma transforma\c{c}\~ao linear.  Ent\~ao
        \[
            \dim_\cp{K}V  = \dim_\cp{K}\ker(T)  + \dim_\cp{K}\im(T).
        \]
    \end{teorema}
\end{frame}

\begin{frame}
    \begin{definicao}
        Sejam $V$ e $W$ $\cp{K}$-espa\c{c}os vetoriais  e $T \colon V \to W$ uma transforma\c{c}\~ao linear.
        \begin{enumerate}[label={\roman*})]
            \item Dizemos que $T$ \'e \textbf{injetora}  se dados $u_1$, $u_2 \in V$  tais que $T(u_1) = T(u_2)$,  ent\~ao $u_1 = u_2$.  De modo equivalente,  se $u_1$, $u_2 \in V$ s\~ao tais que $u_1 \ne u_2$,  ent\~ao $T(u_1) \ne T(u_2)$.

            \vspace{.5cm}

            \item Dizemos que $T$ \'e \textbf{sobrejetora}  se $\im(T) = W$.  Em outras palavras,  $T$ \'e \textbf{sobrejetora} se para todo $w \in W$,  existe $u \in V$ tal que $T(u) = w$.
        \end{enumerate}
    \end{definicao}
\end{frame}

\begin{frame}
    \begin{proposicao}\label{caracteriza_transformacao_injetora}
        Sejam $V$ e $W$ espa\c{c}os vetoriais sobre um corpo $\cp{K}$  e $T \colon V \to W$ uma transforma\c{c}\~ao linear.
        \begin{enumerate}[label={\roman*})]
            \item $T$ \'e injetora  se, e somente se, $\ker(T) = \{0_V\}$.

            \vspace{.5cm}

            \item Se $\dim V = \dim W$  e $T$ é injetora,  então $T$ transforma bases em bases,  isto é, se $\mathcal{B} = \{v_1, v_2, \dots, v_n\}$ é uma base de $V$,  então $\{T(v_1),  T(v_2),  \dots, T(v_n)\}$ é uma base de $W$.

            \vspace{.5cm}

            \item Se $\dim V = \dim W$,  então $T$ é injetora  se, e somente se,  $T$ é sobrejetora.
        \end{enumerate}
    \end{proposicao}
\end{frame}

\begin{frame}
    \begin{observacoes}
        Seja $T \colon V \to W$ uma transformação linear,  onde $\dim V < \infty$ e $\dim W < \infty$.
        \begin{enumerate}[label={\roman*})]
            \item Se $\dim V > \dim W$,  então $T$ não pode ser injetora,  isto é, $\ker(T) \ne \emptyset$.

            \vspace{.5cm}

            \item Se $\dim V < \dim W$,  então $T$ não pode ser sobrejetora,  isto é, $\im(T) \ne W$.
        \end{enumerate}
    \end{observacoes}
\end{frame}

\begin{frame}
    \begin{definicao}
        Se $T \colon V \to W$ é uma transformação linear  \textbf{injetora e sobrejetora},  então dizemos que $T$ é uma transformação linear \textbf{bijetora},  ou que $T$ é um \textbf{isomorfismo}  do espaço $V$ no espaço $W$.  Neste caso, dizemos que o espaço $V$ é isomorfo ao espaço $W$  e denotamos isso, escrevendo $V \cong W$.
    \end{definicao}

    \begin{teorema}
        Seja $V$ um espaço vetorial sobre $\cp{K}$  com $\dim V = n$,  então $V \cong \cp{K}^n$.
    \end{teorema}
\end{frame}

\begin{frame}
    \begin{observacoes}
        Seja $V$ um espaço vetorial sobre $\cp{K}$  com $\dim V = n$.  Então do teorema anterior temos:
        \begin{enumerate}[label={\roman*})]
            \item Se $\cp{K} = \rac$,  então $V \cong \rac^n$.

            \vspace{.5cm}

            \item Se $\cp{K} = \real$,  então $V \cong \real^n$.

            \vspace{.5cm}

            \item Se $\cp{K} = \complex$,  então $V \cong \complex^n$.
        \end{enumerate}
    \end{observacoes}
\end{frame}

\begin{frame}
    \begin{definicao}
        Dadas $S \colon V \to W$  e $T \colon W \to U$ transformações lineares,  definimos a \textbf{composição de $T$ com $S$}  como a aplicação:
        \begin{align*}
            T \circ S \colon &V \to U\\
                             &v \mapsto (T\circ S)(u) = T(S(v))
        \end{align*}
    \end{definicao}
\end{frame}

\begin{frame}
    \begin{proposicao}
        Se $S \colon V \to W$ e $T \colon W \to U$ são transformações lineares,  então a composição de $T\circ S$ também é uma transformação linear.
    \end{proposicao}
\end{frame}


\begin{frame}
    \begin{definicao}
        Dado $T \colon V \to V$ um operador linear,  dizemos que $T$ é \textbf{invertível}  se existe outra aplicação $T^{-1} \colon V \to V$  tal que para qualquer $u \in V$,  $T^{-1}(T(u)) = u$.
    \end{definicao}
\end{frame}

\begin{frame}
    \begin{observacoes}
        \begin{enumerate}[label={\roman*})]
            \item Seja $I = I_V = Id = Id_V \colon V \to V$  a transformação linear tal que $I(u) = Id(u) = u$  para todo $u \in V$.  Então $I = Id$ é chamado de \textbf{operador identidade}.

            \vspace{.5cm}

            \item Se $T \colon V \to V$ é uma transformação linear invertível,  então a aplicação $T^{-1} \colon V \to V$ também é uma transformação linear.

            \vspace{.5cm}

            \item Se $T \colon V \to V$ é um operador invertível,  então $T^{-1} \colon V \to V$ é chamado de \textbf{operador inverso de $T$}.
        \end{enumerate}
    \end{observacoes}
\end{frame}

\begin{frame}
    \begin{proposicao}
        Seja $T \colon V \to V$ um operador linear.  Então:
        \begin{enumerate}[label={\roman*})]
            \item Se $T$ é invertível  e $T^{-1} \colon V \to V$ é o seu inverso,  então
                \[
                    (T^{-1} \circ T) =  (T \circ T^{-1}) =  Id.
                \]

            \vspace{.5cm}

            \item Se $V$ tem dimensão finita,  então $T$ é invertível  se, e somente se,  $\ker(T) = \{0_V\}$.

            \vspace{.5cm}

            \item Se $T$ é invertível,  então dada uma base $\{v_1, v_2, \dots, v_n\}$ de $V$,  o conjunto $\{T(v_1),  T(v_2),  \dots, T(v_n)\}$  é uma base de $V$.
        \end{enumerate}
    \end{proposicao}
\end{frame}

\begin{frame}
    \begin{proposicao}
        Se $S \colon V \to V$ e $T \colon V \to V$  são operadores invertíveis,  então $S \circ T$ é invertível e
        \[
            (S \circ T)^{-1}  = T^{-1}  \circ S^{-1}.
        \]
    \end{proposicao}
\end{frame}

\begin{frame}
    \begin{teorema}\label{teorema_toda_transformacao_matriz}
        Sejam $V$ e $W$ espaços vetoriais sobre $\cp{K}$,  de dimensões $p \ge 1$ e $q \ge 1$, respectivamente.  Sejam $\mathcal{B}_V$ e $\mathcal{B}_W$ bases ordenadas de $V$ e $W$, respectivamente.  Então para cada transformação linear $T \colon V \to W$,  existe uma matriz $A \in \cp{M}_{q\times p}(\cp{K})$ tal que
        \[
            [T(x)]_{\mathcal{B}_W}  = A[x]_{\mathcal{B}_V}
        \]
        para todo vetor $x \in V$.  Além disso,  a cada matriz $C \in \cp{M}_{q\times p}(\cp{K})$  corresponde uma transformação linear $T \colon V \to W$ tal que
        \[
            [T(x)]_{\mathcal{B}_W}  = C[x]_{\mathcal{B}_V}
        \]
        para todo $x \in V$.
    \end{teorema}
\end{frame}

\begin{frame}
    \begin{definicao}
        A matriz $A \in \cp{M}_{q\times p}(\cp{K})$ do Teorema anterior  é chamada de \textbf{matriz da transformação linear}  $T$ com respeito às bases ordenadas $\mathcal{B}_V$ e $\mathcal{B}_W$  e será denotada por
        \[
            A = [T]_{\mathcal{B}_{W},\mathcal{B}_{V}}.
        \]
        No caso em que $V = W$  e $\mathcal{B}_V = \mathcal{B}_W = \mathcal{B}$,  denotaremos $[T]_{\mathcal{B}_{V},\mathcal{B}_{W}}$  simplesmente por $[T]_\mathcal{B}$.
    \end{definicao}
\end{frame}

\begin{frame}
    \begin{teorema}\label{matriz_da_composicao_de_transformacoes}
        Sejam $F \colon U \to V$  e $G \colon V \to W$  duas transformações lineares onde $U$, $V$ e $W$  são espaços vetoriais $\cp{K}$  de dimensões $n$, $m$ e $r$, respectivamente.  Fixe bases ordenadas $\mathcal{B}_U$,  $\mathcal{B}_V$  e $\mathcal{B}_W$  para $U$, $V$ e $W$, respectivamente.  Então $(G \circ F) \colon U \to W$  dada por $(G\circ F)(v) = G(F(v))$  é uma transformação linear e
        \[
            [(G \circ F)]_{{\mathcal{B}_W},{\mathcal{B}_U}}  = [G]_{{\mathcal{B}_W},{\mathcal{B}_V}}[F]_{{\mathcal{B}_V},{\mathcal{B}_U}}.
        \]
    \end{teorema}
\end{frame}

\begin{frame}
    \begin{teorema}
        Sejam $T \colon V \to V$ é um operador linear  e $\mathcal{B}$ é uma base ordenada de $V$.  Então as seguintes afirmações são equivalentes:
        \begin{enumerate}[label={\roman*})]
            \item $T$ é injetor

            \vspace{1cm}

            \item $[T]_\mathcal{B}$ é invertível.
        \end{enumerate}

        \vspace{.2cm}

        Além disso, se valem essas condições equivalentes, então
        \[
            [T^{-1}]_\mathcal{B}  = [T]_\mathcal{B}^{-1}.
        \]
    \end{teorema}
\end{frame}

\begin{frame}
    \begin{teorema}\label{teorema_mudanca_base}
        Sejam $V$ um $\cp{K}$-espaço vetorial de dimensão $n \ge 1$,  $\mathcal{B}_1$ e $\mathcal{B}_2$ bases ordenadas de $V$.  Então existe uma única matriz $P \in \cp{M}_n(\cp{K})$,  necessariamente invertível tal que
        \begin{enumerate}[label={\roman*})]
            \item $[x]_{\mathcal{B}_1} =  P[x]_{\mathcal{B}_2}$

            \vspace{1cm}

            \item $[x]_{\mathcal{B}_2} =  P^{-1}[x]_{\mathcal{B}_1}$
        \end{enumerate}

        \vspace{.2cm}

        para todo $x \in V$.  As colunas de $P = \begin{bmatrix}P_1 & P_2 & \cdots & P_n \end{bmatrix}$  são dadas por
        \[
            P_j = [w_j]_{\mathcal{B}_1}
        \]
        para $j = 1$, \dots, $n$.
    \end{teorema}
\end{frame}

\begin{frame}
    \begin{teorema}\label{teorema_matriz_mudanca_base}
        Seja $P \in \cp{M}_n(\cp{K})$ um matriz invertível.  Sejam $V$ um $\cp{K}$-espaço vetorial de dimensão $n \ge 1$  e $\mathcal{B}_1$ uma base ordenada de $V$.  Então existe uma única base ordenada $\mathcal{B}_2$  de $V$ tal que
        \begin{enumerate}[label={\roman*})]
            \item $[x]_{\mathcal{B}_1} =  P[x]_{\mathcal{B}_2}$

            \vspace{1cm}

            \item $[x]_{\mathcal{B}_2} =  P^{-1}[x]_{\mathcal{B}_1}$
        \end{enumerate}

        \vspace{.2cm}

        para todo $x \in V$.
\end{teorema}
\end{frame}

\begin{frame}
    \begin{definicao}
        A matriz $P \in \cp{M}_n(\cp{K})$  do Teorema anterior  é chamada de \textbf{matriz de mudança de base}  e é denotada por $P = [I]_{{\mathcal{B}_1},{\mathcal{B}_2}}$.
    \end{definicao}
\end{frame}
\end{document}
