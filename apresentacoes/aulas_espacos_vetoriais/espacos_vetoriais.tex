%!TEX program = xelatex
%!TEX encoding = UTF-8
\def\ano{2023}
\def\semestre{1}
\def\disciplina{Introdução à Álgebra Linear}
\def\turma{3}
\def\autor{José Ant\^onio O. Freitas}
\def\instituto{MAT-UnB}

\documentclass{beamer}
\usetheme{Madrid}
\usecolortheme{beaver}
% \mode<presentation>
\usepackage{caption}
\usepackage{amssymb}
\usepackage{amsmath,amsfonts,amsthm,amstext}
\usepackage[brazil]{babel}
\usepackage{graphicx}
\graphicspath{{../Pictures/}}
\usepackage{enumitem}
\usepackage{multicol}
\usepackage{answers}
\usepackage[svgnames]{xcolor}
\usepackage{tikz}
\usepackage{ifthen}
\usetikzlibrary{lindenmayersystems}
\usetikzlibrary[shadings]

\newcounter{exercicios}
\setcounter{exercicios}{0}
\newcommand{\questao}{
    \addtocounter{exercicios}{1}
    \noindent{\bf Quest{\~a}o \arabic{exercicios}: }}

\newcommand{\resp}[1]{
    \noindent{\bf Exerc{\'\i}cio #1: }}

\extrafootheight[.25in]{.5in}
\footrule
\lfoot{Teste \numeroteste\ - Módulo \modulo\ - \nomeabreviado\ - Turma \turma\ - \semestre$^o$/\ano}
\cfoot{}
\rfoot{P\'agina \thepage\ de \numpages}
\def\ano{2023}
\def\semestre{2}
\def\disciplina{Introdução à Álgebra Linear}
\def\nomeabreviado{IAL}
\def\turma{11}

\newcommand{\im}{{\rm Im\,}}
\newcommand{\dlim}[2]{\displaystyle\lim_{#1\rightarrow #2}}
\newcommand{\minf}{+\infty}
\newcommand{\ninf}{-\infty}
\newcommand{\cp}[1]{\mathbb{#1}}
\newcommand{\sub}{\subseteq}
\newcommand{\n}{\mathbb{N}}
\newcommand{\z}{\mathbb{Z}}
\newcommand{\rac}{\mathbb{Q}}
\newcommand{\real}{\mathbb{R}}
\newcommand{\complex}{\mathbb{C}}

\newcommand{\vesp}[1]{\vspace{ #1  cm}}

\newcommand{\compcent}[1]{\vcenter{\hbox{$#1\circ$}}}
\newcommand{\comp}{\mathbin{\mathchoice
        {\compcent\scriptstyle}{\compcent\scriptstyle}
        {\compcent\scriptscriptstyle}{\compcent\scriptscriptstyle}}}
\renewcommand{\sin}{{\rm sen\,}}
\renewcommand{\tan}{{\rm tg\,}}
\renewcommand{\csc}{{\rm cossec\,}}
\renewcommand{\cot}{{\rm cotg\,}}
\renewcommand{\sinh}{{\rm senh\,}}

\title{Espaços Vetoriais}
\author[\autor]{\autor}
\institute[\instituto]{\instituto}
\date{}

\begin{document}
\begin{frame}
  \maketitle
\end{frame}

\logo{\includegraphics[scale=.1]{logo-MAT.png}\vspace*{8.5cm}}

\begin{frame}
  No que se segue o conjunto $\cp{K}$ denotará um corpo, em geral $\rac$, $\real$ ou $\complex$.

  \vspace{.3cm}

  \begin{definicao}
    Um conjunto não vazio $V$ \pause é um \textbf{espaço vetorial}\index{Espaço Vetorial} \pause sobre um corpo $\cp{K}$ \pause se em seus elementos, chamados \textbf{vetores}, \pause estiverem definidas duas
    operações satisfazendo:\pause
    \begin{itemize}
      \item[A)] A cada par $u$, $w \in V$ \pause corresponde um vetor $u + w \in V$, \pause chamado \textbf{soma} de $u$ e $w$, de modo que:\pause

        \vspace{.3cm}

      \item[A1)] $u + w \pause = w + u$, \pause para todos $u$, $w \in V$;\pause

        \vspace{.3cm}

      \item[A2)] $(u + w) \pause + x \pause = u + \pause (w + x)$, \pause para todos $u$, $w$, $x \in V$;
    \end{itemize}
  \end{definicao}
\end{frame}

\begin{frame}
  \begin{definicao}
    \begin{itemize}
      \item[A3)] Existe em $V$ um vetor, \pause denominado \textbf{vetor nulo} \pause e denotado por $0_V$, tal que\pause
        \[
          0_V + u \pause = u\pause
        \]
        para todo $u \in V$.\pause

        \vspace{.3cm}

      \item[A4)] Para cada vetor $u \in V$, \pause existe um vetor em $V$, \pause denotado por $-u$ \pause e chamado de
        \textbf{vetor oposto}, tal que\pause
        \[
          u + (-u) \pause = 0_V.
        \]
    \end{itemize}
  \end{definicao}
\end{frame}

\begin{frame}
  \begin{definicao}
    \begin{itemize}
      \item[M)] A cada par $\alpha \in \cp{K}$ \pause e $u \in V$, \pause corresponde um vetor $\alpha \cdot u \in V$, \pause denominado \textbf{produto por escalar} \pause de $\alpha$ por $u$ de modo que:\pause

        \vspace{.3cm}

      \item[M1)] $(\alpha\beta)\pause \cdot u = \pause \alpha(\pause \beta\cdot u)$ \pause para todos $\alpha$, $\beta \in \cp{K}$ \pause e todo $u \in V$;\pause

        \vspace{.3cm}

      \item[M2)] $1_\cp{K}\cdot u \pause = u$ \pause para todo $u \in V$, \pause onde $1_\cp{K}$ é o elemento neutro da multiplicação em $\cp{K}$.
    \end{itemize}
  \end{definicao}
\end{frame}

\begin{frame}
  \begin{definicao}
    \begin{itemize}
      \item[D1)] $\alpha\pause \cdot(u + w) = \pause \alpha\cdot u \pause + \alpha\cdot w$, \pause para todo $\alpha \in \cp{K}$ \pause e todos $u$, $w \in V$;\pause

        \vspace{.3cm}

      \item[D2)] $(\alpha + \beta)\pause \cdot u \pause = \alpha\cdot u \pause + \beta\cdot u$, \pause para todos $\alpha$, $\beta \in \cp{K}$ \pause e todo $u \in V$.
    \end{itemize}
  \end{definicao}
\end{frame}

\begin{frame}
  \begin{observacao}
    Vamos usar a expressão $\cp{K}$-\textbf{espaço vetorial} \pause para nos referir a um espaço vetorial $V$ sobre um corpo $\cp{K}$.
  \end{observacao}
\end{frame}

\begin{frame}
    \begin{exemplos}[Espaços vetoriais]
        Seja $\cp{K}$ um corpo, \pause $\cp{K} = \rac$, $\real$ ou $\complex$. \pause Considere o conjunto
            \[
                \cp{K}^n = \underbrace{\cp{K} \times \cp{K} \times \cdots \times \cp{K}}_{n\ vezes}\ = \pause \{(a_1, a_2,\dots,a_n) \mid a_i \in \cp{K}, i =1, 2, \dots, n\}
            \]
            e defina
            \begin{itemize}[label=$\color{blue}\bullet$]
                \item $(a_1, a_2, \dots, a_n) + (b_1, b_2, \dots,b_n) =\pause (a_1 + b_1, a_2 + b_2,\dots, a_n + b_n)$\pause

                \item $\alpha (a_1, a_2, \dots,a_n) \pause= (\alpha a_1, \alpha a_2, \dots, \alpha a_n)$\pause
            \end{itemize}
            para todos $(a_1, a_2, \dots,a_n)$ ,$(b_1, b_2, \dots,b_n) \in \cp{K}^n$ \pause e para todo $\alpha \in \cp{K}$.\pause

            \vspace*{.2cm}

            Com estas operações $\cp{K}^n$ é um espaço vetorial sobre $\cp{K}$.\pause

            \vspace*{.2cm}

            Assim temos:\pause
                \begin{itemize}[label=$\color{blue}\blacktriangleright$]
                    \item $\rac^n$ \pause é um espaço vetorial sobre $\rac$;\pause
                    \item $\real^n$ \pause é um espaço vetorial sobre $\real$;\pause
                    \item $\complex^n$ \pause é um espaço vetorial sobre $\complex$.
                \end{itemize}
    \end{exemplos}
\end{frame}

\begin{frame}
    \begin{exemplos}[Espaços vetoriais]
    $\complex^n$ \pause é um espaço vetorial sobre $\real$ \pause se definirmos:\pause
        \begin{itemize}[label=$\color{blue}\bullet$]
            \item $(a_1, a_2, \dots, a_n) + (b_1, b_2, \dots,b_n) \pause = (a_1 + b_1, a_2 + b_2,\dots, a_n + b_n)$\pause
            \item $\alpha (a_1, a_2, \dots,a_n) = (\alpha a_1, \alpha a_2, \dots, \alpha a_n)$\pause
        \end{itemize}
        para todos $(a_1, a_2, \dots,a_n)$ ,$(b_1, b_2, \dots,b_n) \in \complex^n$ \pause e para todo $\alpha \in \real$.
    \end{exemplos}
\end{frame}

\begin{frame}
    \begin{exemplos}[Espaços vetoriais]
        De modo geral, \pause o conjunto das matrizes $\cp{M}_{p \times q}(\cp{K})$ \pause com coeficientes em $\cp{K}$ \pause é um $\cp{K}$-espaço vetorial \pause com a soma usual de matrizes \pause e a multiplicação por escalar usual.
\end{exemplos}
\end{frame}

\begin{frame}
    \begin{exemplos}[Espaços vetoriais]
        Considere o conjunto dos polinômios\pause
        \begin{align*}
            \mathcal{P}(\cp{K}) = &\{ p(x) = a_nx^n + a_{n - 1}x^{n - 1} + \cdots + a_1x + a_0 \mid \\ &a_i \in \cp{K}, i = 0, 1, 2, \dots, n; n \ge 0 \}.
        \end{align*}

    Dados $p(x) = a_nx^n + a_{n - 1}x^{n - 1} + \cdots + a_1x + a_0$ \pause e $q(x) = b_mx^m + b_{m - 1}x^{m - 1} + \cdots + b_1x + b_0$ \pause em $\mathcal{P}(\cp{K})$, \pause suponha que $m < n$ e defina:\pause
    \begin{itemize}[label=$\color{blue}\bullet$]
        \item $(p + q)(x) = \pause (a_nx^n + a_{n - 1}x^{n - 1} + \cdots + a_mx^m + a_{m-1}x^{m-1} + \cdots + a_1x + a_0) \pause + (b_mx^m + b_{m - 1}x^{m - 1} + \cdots + b_1x + b_0\pause = a_nx^n \pause + a_{n-1}x^{n-1} + \pause \cdots + (a_m + b_m)x^m + (a_{m-1} + b_{m-1})x^{m-1} + \cdots \pause + (a_1 + b_1)x + (a_0 + b_0)$\pause
        \vspace*{.2cm}
        \item $(\alpha p)(x) = \pause \alpha(a_nx^n + a_{n - 1}x^{n - 1} + \cdots + a_1x + a_0) \pause = \alpha a_nx^n \pause + \alpha a_{n - 1}x^{n - 1} \pause + \cdots + \alpha a_1x + \alpha a_0$, \pause onde $\alpha \in \cp{K}$.\pause
    \end{itemize}
    Assim $\mathcal{P}(\cp{K})$ é um $\cp{K}$-espaço vetorial.
    \end{exemplos}
\end{frame}

\begin{frame}
  \begin{definicao}
    Seja $V$ um espaço vetorial sobre $\cp{K}$.\pause
    Um vetor $w \in V$ é uma \textbf{combinação linear} \pause dos vetores $u_1$, \pause $u_2$, \pause \dots,
    $u_n \in V$ \pause se existirem escalares $\alpha_1$, \pause $\alpha_2$, \pause \dots, \pause $\alpha_n \in \cp{K}$ tais que\pause
    \[
        w = \alpha_1 u_1 + \pause \alpha_2u_2 + \pause \cdots + \pause \alpha_nu_n.
    \]
  \end{definicao}
\end{frame}

\begin{frame}
  \begin{definicao}
    Seja $V$ um $\cp{K}$-espaço vetorial. \pause Um subconjunto não vazio $W$ \pause de $V$ é um \textbf{subespaço vetorial} de $V$ \pause se $W$ é um espaço vetorial
    sobre $\cp{K}$ \pause com respeito às mesmas operações definidas em $V$.\pause
  \end{definicao}

  \begin{teorema}
    Um subconjunto não vazio $W$ \pause de um $\cp{K}$-espaço vetorial $V$ \pause é um subespaço de $V$ \pause se, e somente se:
    \begin{enumerate}[label={\roman*})]
      \item para todos vetores $u_1$, $u_2 \in W$ \pause temos $u_1 + u_2 \in W$.\pause
      \item para todo escalar $\lambda \in \cp{K}$ \pause e todo vetor $u \in W$, \pause temos $\lambda u\in W$.
    \end{enumerate}
  \end{teorema}
\end{frame}

\begin{frame}
  \begin{definicao}
    Seja $V$ um $\cp{K}$-espaço vetorial. \pause Fixado vetores $v_1$, $v_2$, \dots, $v_n \in V$ \pause o conjunto
    \[
        W = \{\alpha_1v_1 \pause + \alpha_2v_2 \pause + \cdots + \alpha_nv_n \pause \mid \alpha_i \in \cp{K}\}\pause
    \]
    é chamado de \textbf{espaço gerado} \pause por $\{v_1, v_2, \dots, v_n\}$ \pause e os vetores $v_1$, $v_2$, \dots, $v_n$ \pause são chamados de \textbf{geradores} de $W$.\pause
  \end{definicao}

  \begin{notacao}
    O espaço gerado $W = \{\alpha_1v_1 + \alpha_2v_2 + \cdots + \alpha_nv_n \mid \alpha_i \in \cp{K}\}$ \pause pode ser escrito como $W = Span(v_1, v_2, \dots, v_n)$ \pause ou $W = <v_1, v_2, \dots, v_n>$ \pause ou ainda como $W = [v_1, v_2, \dots, v_n]$.
  \end{notacao}
\end{frame}

\begin{frame}
  \begin{proposicao}
    Seja $V$ um $\cp{K}$-espaço vetorial. \pause Fixado vetores $v_1$, $v_2$, \dots, $v_n \in V$ \pause o conjunto
    \[
        W = \{\alpha_1v_1 \pause + \alpha_2v_2 \pause + \cdots + \alpha_nv_n \pause \mid \alpha_i \in \cp{K}\}\pause
    \]
    é um subespaço vetorial de $V$.
  \end{proposicao}
\end{frame}

\begin{frame}
  \begin{definicao}
    Um espaço vetorial $V$ é chamado de \textbf{finitamente gerado} \pause se existe um conjunto finito \pause de vetores $\{v_1, v_2, \dots, v_n\}$ em $V$ \pause tais que\pause
    \[
        V = \pause Span(v_1, v_2, \dots, v_n).
    \]
  \end{definicao}
\end{frame}

\begin{frame}
  \begin{definicao}
    Seja $V$ um $\cp{K}$-espaço vetorial. \pause Dados vetores $v_1$, $v_2$, \dots, $v_n$ \pause em $V$ dizemos que:\pause
    \begin{enumerate}[label={\roman*})]
      \item o conjunto $\{v_1, v_2, \dots, v_n\}$ \pause é \textbf{linearmente dependente} \pause ou que os vetores $v_1$, $v_2$, \dots, $v_n$ \pause são \textbf{linearmente dependentes}, \pause abreviado como L.D, \pause se existem escalares
        $\alpha_1$, $\alpha_2$, \dots, $\alpha_n \in \cp{K}$, \pause \textbf{não todos nulos}, \pause tais que\pause
        \[
            \alpha_1v_1 + \alpha_2v_2 + \cdots + \alpha_nv_n \pause = 0_V.\pause
        \]

      \item o conjunto $\{v_1, v_2, \dots, v_n\}$ \pause é \textbf{linearmente independente} \pause ou que os vetores $v_1$, $v_2$, \dots, $v_n$ \pause são \textbf{linearmente independentes}, \pause abreviado como L.I, \pause se dados escalares
        $\alpha_1$, $\alpha_2$, \dots, $\alpha_n \in \cp{K}$  tais que\pause
        \[
            \alpha_1v_1 + \alpha_2v_2 + \cdots + \alpha_nv_n \pause = 0_V,\pause
        \]
        então $\alpha_1 = \pause \alpha_2 = \pause \cdots = \alpha_n \pause = 0$.
    \end{enumerate}
  \end{definicao}
\end{frame}


\begin{frame}
  \begin{proposicao}
  Seja $V$ um espaço vetorial sobre $\cp{K}$. Então:\pause

  \begin{enumerate}[label={\roman*})]
    \item O vetor nulo é linearmente dependente.\pause

    \vspace{.3cm}

    \item Todo vetor $v \ne 0_V$ \pause é linearmente independente.\pause

    \vspace{.3cm}

    \item Dado um conjunto de vetores $\{v_1, v_2, \dots, v_n\}$ \pause se $v_i = 0_V$ para algum $i$, \pause então $\{v_1, v_2, \dots, v_n\}$ é L.D.

    \seti
  \end{enumerate}
  \end{proposicao}
\end{frame}

\begin{frame}
  \begin{proposicao}
  \begin{enumerate}[label={\roman*})]

    \conti

    \item Se os vetores $\{v_1, v_2, \dots, v_n\}$ são L.D., \pause então qualquer outro conjunto de vetores que contenha $\{v_1, v_2, \dots, v_n\}$ \pause também é L.D.\pause

    \vspace{.3cm}

    \item Se $\{v_1, v_2, \dots, v_n\}$ é L.I., \pause então qualquer subconjunto $\{v_{i_1}, v_{i_2}, \dots, v_{i_s}\}$ \pause com $s < n$ também é L.I.\pause

    \vspace{.3cm}

    \item Seja $\{v_1, v_2, \dots, v_n\}$ um conjunto L.I. \pause Se $w$ é um vetor de $V$ tal que $\{v_1, v_2, \dots, v_n, w\}$ \pause é L.D., \pause então $w$ é uma combinação linear de $v_1$, $v_2$, \dots, $v_n$.
  \end{enumerate}
  \end{proposicao}
\end{frame}

\begin{frame}
  \begin{proposicao}
    Sejam $V$ um $\cp{K}$-espaço vetorial, \pause $n \ge 2$ e $\{v_1, v_2, \dots, v_n\}$ vetores de $V$. \pause Então $\{v_1, v_2, \dots, v_n\}$ é L.D. \pause se, e somente se, pelo menos um vetor $v_j$ \pause for
    combinação linear dos demais $n - 1$ de $\{v_1, v_2, \dots, v_n\}$.
  \end{proposicao}
\end{frame}


\begin{frame}
  \begin{definicao}
    Seja $V$ um $\cp{K}$-espaço vetorial. \pause Um conjunto de vetores $\{v_1, v_2, \dots, v_n\}$ \pause é chamado de uma \textbf{base} \pause de $V$ se:\pause
    \begin{enumerate}[label={\roman*})]
      \item $\{v_1, v_2, \dots, v_n\}$ é L.I.\pause

      \vspace{.3cm}

      \item $V = Span(v_1, v_2, \dots, v_n)$, \pause ou seja, $\{v_1, v_2, \dots, v_n\}$ gera $V$.
    \end{enumerate}
  \end{definicao}
\end{frame}


\begin{frame}
  \begin{teorema}
    Seja $V$ um espaço vetorial sobre $\cp{K}$. \pause Se $\{v_1, v_2, \dots, v_n\}$ é uma base de $V$, \pause então todo conjunto de vetores de $V$ \pause com mais de $n$ elementos \pause é L.D.\pause
  \end{teorema}

  \begin{corolario}
    Dado um $\cp{K}$-espaço vetorial $V$, \pause então quaisquer duas bases de $V$ possuem o mesmo número de vetores.
  \end{corolario}
\end{frame}

\begin{frame}
  \begin{definicao}
    Seja $V$ um espaço vetorial sobre $\cp{K}$. \pause A \textbf{dimensão} de $V$ \pause é o número de vetores de uma base de $V$, \pause quando essa base é constituída de uma quantidade finita de vetores. \pause
    Nesse caso vamos denotar esse número por $\dim_\cp{K}V$ \pause ou simplesmente $\dim V$.\pause
  \end{definicao}

  \begin{observacoes}
    \begin{enumerate}[label={\roman*})]
      \item Se $V$ não possui base, \pause então $\dim V = 0$.\pause

      \item Se $V$ possui uma base com infinitos vetores, \pause então $\dim V = \infty$.
    \end{enumerate}
  \end{observacoes}
\end{frame}

\begin{frame}
  \begin{teorema}
    Seja $V$ um espaço vetorial sobre um corpo $\cp{K}$ \pause com $\dim_\cp{K}V = n$. \pause Se o conjunto $\{v_1, v_2, \dots, v_n\}$ \pause é linearmente independente, \pause então o conjunto $\{v_1, v_2, \dots, v_n\}$ é uma base de $V$.
  \end{teorema}
\end{frame}

\begin{frame}
  \begin{teorema}
    Seja $V$ um $\cp{K}$-espaço vetorial \pause e $\mathcal{B} = \{v_1, v_2, \dots, v_n\}$ uma base de $V$. \pause Então todo vetor $u \in V$ \pause pode ser escrito, de forma única, \pause como uma combinação linear dos vetores de $\mathcal{B}$.
  \end{teorema}
\end{frame}

\begin{frame}
  \begin{definicao}
    Sejam $V$ um $\cp{K}$-espaço vetorial \pause e $\mathcal{B} = \{v_1, v_2, \dots, v_n\}$ uma base de $V$, \pause isto é, $\dim_\cp{K}V = n$. \pause Para cada vetor $u \in V$ \pause podemos escrever
    \[
      u = \alpha_1v_1 + \alpha_2v_2 + \cdots + \alpha_nv_n\pause
    \]
    para únicos escalares \pause $\alpha_1$, $\alpha_2$, \dots, $\alpha_n \in \cp{K}$. \pause Tais escalares são chamados de \textbf{coordenadas de $v$ com respeito à base $\mathcal{B}$}.\pause
  \end{definicao}
\end{frame}

\begin{frame}
  \begin{notacao}
    Se $u = \alpha_1v_1 + \alpha_2v_2 + \cdots + \alpha_nv_n$, \pause então vamos denotar as coordenadas de $u$ \pause em relação à base $\mathcal{B}$ por\pause
    \[
      u_\mathcal{B} = (\alpha_1, \alpha_2, \dots, \alpha_n)_\mathcal{B}\pause
    \]
    ou por
    \[
      [u]_\mathcal{B} = \begin{bmatrix}\alpha_1\\\alpha_2\\ \vdots\\ \alpha_n\end{bmatrix}_\mathcal{B}\pause
    \]

    e vamos dizer que esse vetor \pause é o \textbf{vetor das coordenadas de $u$ com respeito à base $\mathcal{B}$}.
  \end{notacao}
\end{frame}

\begin{frame}
  \begin{teorema}
    Seja $\mathcal{B} = \{v_1, v_2, \dots, v_n\}$ uma base \pause para um $\cp{K}$-espaço vetorial $V$. \pause Dados vetores $u$, $w \in V$, \pause e $\lambda \in \cp{K}$ um escalar, temos: \pause
    \begin{enumerate}[label={\roman*})]
      \vspace{.2cm}

      \item $[u + w]_\mathcal{B} \pause = [u]_\mathcal{B} \pause + [w]_\mathcal{B}$ \pause

      \vspace{1cm}

      \item $[\lambda u]_\mathcal{B} \pause = \lambda[u]_\mathcal{B}$
    \end{enumerate}
  \end{teorema}
\end{frame}

\begin{frame}
  \begin{teorema}
    Seja $\mathcal{B} = \{v_1, v_2, \dots, v_n\}$ uma base \pause para um $\cp{K}$-espaço vetorial $V$. \pause Dados vetores $u_1$, \dots, $u_k \in V$, \pause então $\{u_1, \dots, u_k\}$ é linearmente independente \pause se, e somente se, \pause $\{[u_1]_\mathcal{B}, \dots, [u_k]_\mathcal{B}\}$ \pause é linearmente independente em $\cp{K}^n$.
  \end{teorema}
\end{frame}

\begin{frame}
  \begin{teorema}
    Seja $\mathcal{B} = \{v_1, \dots, v_n\}$ uma base \pause para um espaço vetorial $V$ sobre um corpo $\cp{K}$. \pause
    \begin{enumerate}[label={\roman*})]
      \vspace{.2cm}

      \item Qualquer conjunto  de $V$ com mais de $n$ vetores \pause é linearmente dependente. \pause

      \vspace{1cm}

      \item Qualquer conjunto de $V$ com menos de $n$ vetores \pause não pode gerar $V$.
    \end{enumerate}
  \end{teorema}
\end{frame}


\begin{frame}
  \begin{teorema}
    Seja $V$ um espaço vetorial sobre $\cp{K}$ \pause com $\dim_\cp{K} V = n$. \pause Então: \pause
    \begin{enumerate}[label={\roman*})]
      \vspace{.2cm}

      \item Qualquer conjunto linearmente independente em $V$ \pause contém no máximo $n$ vetores. \pause

      \vspace{1cm}

      \item Qualquer conjunto gerador de $V$ \pause contém no mínimo $n$ vetores. \pause

      \vspace{1cm}

      \item Qualquer conjunto linearmente independente em $V$ \pause contendo exatamente $n$ vetores \pause é uma base de $V$.

      \seti

     \end{enumerate}
  \end{teorema}
\end{frame}

\begin{frame}
  \begin{teorema}
    \begin{enumerate}[label={\roman*})]
      \conti

      \vspace{.2cm}

      \item Qualquer conjunto gerador de $V$ \pause com exatamente $n$ vetores, \pause é uma base de $V$. \pause

      \vspace{1cm}

      \item Qualquer conjunto linearmente independente em $V$ \pause pode ser estendido para uma base de $V$. \pause

      \vspace{1cm}

      \item Qualquer conjunto gerador de $V$ \pause pode ser reduzido a uma base de $V$.
    \end{enumerate}
  \end{teorema}
\end{frame}

\begin{frame}
  \begin{teorema}
    Seja $W$ um subespaço \pause de um espaço vetorial $V$ \pause de dimensão finita. \pause Então: \pause
    \begin{enumerate}[label={\roman*})]
      \vspace{.2cm}

      \item $W$ é de dimensão finita \pause e $\dim W \le \dim V$. \pause

      \vspace{1cm}

      \item $\dim W = \dim V$ \pause se, e somente se, $W = V$.
    \end{enumerate}
  \end{teorema}
\end{frame}

\begin{frame}
	Se mudarmos a base de um espaço vetorial $V$ de alguma base velha $\mathcal{B} = \{u_1, u_2, \dots, u_n\}$ para uma base nova $\mathcal{C} = \{w_1, w_2, \dots, w_n\}$, então, dado qualquer vetor $v \in V$, o velho vetor de coordenadas $[v]_\mathcal{B}$ está relacionado com o novo vetor de coordenadas $[v]_\mathcal{C}$ pela equação
	\begin{equation}
		[v]_\mathcal{B} = P[v]_\mathcal{C}
	\end{equation}
	onde as colunas de $P$ são os vetores de coordenadas dos vetores da base nova em relação à base velha; ou seja, os vetores coluna de $P$ são
	\begin{equation}
		[w_1]_\mathcal{B}, [w_2]_\mathcal{B}, \dots, [w_n]_\mathcal{B}
	\end{equation}
\end{frame}

\begin{frame}
	\begin{teorema}
		Se $P$ for a matriz de transição de uma base $\mathcal{B}$ para uma base $\mathcal{C}$ de um espaço vetorial $V$ de dimensão finita, então $P$ é invertível e $P^{-1}$ é a matriz de transição de $\mathcal{C}$ para $\mathcal{B}$.
	\end{teorema}
\end{frame}

\begin{frame}
	\begin{tcolorbox}[colback=green!30, colframe=green!80!blue, title=Procedimento para calcular $P_{\mathcal{B} \to \mathcal{C}}$]
			\begin{tabular}{ll}
				Passo 1. & Montamos a matriz $[\mathcal{C} | \mathcal{B}]$.\\
				\\
				Passo 2. & Reduzimos a matriz do Passo 1 à forma escalonada\\
				& reduzida usando operações elementares com as linhas.\\
				\\
				Passo 3. & A matriz resultante é$[I | P_{\mathcal{B} \to \mathcal{C}}]$.\\
				\\
				Passo 4. & Extraímos a matriz $P_{\mathcal{B} \to \mathcal{C}}$ do lado direito da matriz do\\
				& Passo 3.
			\end{tabular}
	\end{tcolorbox}
\end{frame}

\begin{frame}
	\begin{teorema}
		Sejam $\mathcal{C} = \{u_1, u_2, \dots, u_n\}$ uma base qualquer do espaço vetorial $\real^n$ e $\mathcal{B} = \{e_1, e_2, \dots, e_n\}$ a base canônica de $\real^n$. Se os vetores dessas bases forem escritos em forma de colunas, então
		\[
			P_{\mathcal{C} \to \mathcal{B}} = [u_1 | u_2 | \dots | u_n].
		\]
	\end{teorema}
\end{frame}

\end{document}
