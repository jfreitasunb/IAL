%!TEX program = xelatex
\def\ano{2023}
\def\semestre{1}
\def\disciplina{Introdução à Álgebra Linear}
\def\turma{3}
\def\autor{Jos\'e Ant\^onio O. Freitas}
\def\instituto{MAT-UnB}

\documentclass{beamer}
\usetheme{Madrid}
\usecolortheme{beaver}
% \mode<presentation>
\usepackage{caption}
\usepackage{amssymb}
\usepackage{amsmath,amsfonts,amsthm,amstext}
\usepackage[brazil]{babel}
\usepackage{graphicx}
\graphicspath{{../Pictures/}}
\usepackage{enumitem}
\usepackage{multicol}
\usepackage{answers}
\usepackage[svgnames]{xcolor}
\usepackage{tikz}
\usepackage{ifthen}
\usetikzlibrary{lindenmayersystems}
\usetikzlibrary[shadings]

\newcounter{exercicios}
\setcounter{exercicios}{0}
\newcommand{\questao}{
    \addtocounter{exercicios}{1}
    \noindent{\bf Quest{\~a}o \arabic{exercicios}: }}

\newcommand{\resp}[1]{
    \noindent{\bf Exerc{\'\i}cio #1: }}

\extrafootheight[.25in]{.5in}
\footrule
\lfoot{Teste \numeroteste\ - Módulo \modulo\ - \nomeabreviado\ - Turma \turma\ - \semestre$^o$/\ano}
\cfoot{}
\rfoot{P\'agina \thepage\ de \numpages}
\def\ano{2023}
\def\semestre{2}
\def\disciplina{Introdução à Álgebra Linear}
\def\nomeabreviado{IAL}
\def\turma{11}

\newcommand{\im}{{\rm Im\,}}
\newcommand{\dlim}[2]{\displaystyle\lim_{#1\rightarrow #2}}
\newcommand{\minf}{+\infty}
\newcommand{\ninf}{-\infty}
\newcommand{\cp}[1]{\mathbb{#1}}
\newcommand{\sub}{\subseteq}
\newcommand{\n}{\mathbb{N}}
\newcommand{\z}{\mathbb{Z}}
\newcommand{\rac}{\mathbb{Q}}
\newcommand{\real}{\mathbb{R}}
\newcommand{\complex}{\mathbb{C}}

\newcommand{\vesp}[1]{\vspace{ #1  cm}}

\newcommand{\compcent}[1]{\vcenter{\hbox{$#1\circ$}}}
\newcommand{\comp}{\mathbin{\mathchoice
        {\compcent\scriptstyle}{\compcent\scriptstyle}
        {\compcent\scriptscriptstyle}{\compcent\scriptscriptstyle}}}
\renewcommand{\sin}{{\rm sen\,}}
\renewcommand{\tan}{{\rm tg\,}}
\renewcommand{\csc}{{\rm cossec\,}}
\renewcommand{\cot}{{\rm cotg\,}}
\renewcommand{\sinh}{{\rm senh\,}}

\title{Espaços Vetoriais}
\author[\autor]{\autor}
\institute[\instituto]{\instituto}
\date{}

\begin{document}
    \begin{frame}
        \maketitle
    \end{frame}

    \logo{\includegraphics[scale=.1]{logo-MAT.png}\vspace*{8.5cm}}

    \begin{frame}
        No que se segue o conjunto $\cp{K}$ denotar\'a um corpo, em geral $\rac$, $real$ ou $\complex$.

        \vspace{.3cm}

        \begin{definicao}
    	    Um conjunto	n\~ao vazio $V$ \'e um \textbf{espa\c{c}o vetorial}\index{Espa\c{c}o Vetorial} sobre um corpo $\cp{K}$ se em seus elementos, chamados \textbf{vetores}, estiverem definidas duas opera\c{c}\~oes satisfazendo:
    	    \begin{itemize}
        		\item[A)] A cada par $u$, $w \in V$ corresponde um vetor $u + w \in V$, chamado \textbf{soma} de $u$ e $w$, de modo que:

                \vspace{.3cm}

		        \item[A1)] $u + w = w + u$, para todos $u$, $w \in V$;
	
                \vspace{.3cm}

        	    \item[A2)] $(u + w) + x = u + (w + x)$, para todos $u$, $w$ e $x \in V$;
	    	\end{itemize}
	    \end{definicao}
	\end{frame}

	\begin{frame}
	    \begin{definicao}
	        \begin{itemize}
        		\item[A3)] Existe em $V$ um vetor, denominado \textbf{vetor nulo} e denotado por $0_V$, tal que
	    	    \[
	        		0_V + u = u
    		    \]
    		    para todo $u \in V$.
	
                \vspace{.3cm}

             	\item[A4)] Para cada vetor $u \in V$, existe um vetor em $V$, denotado por $-u$ tal que
    	    	\[
		        	u + (-u) = 0_V.
    	    	\]
	    	\end{itemize}
	    \end{definicao}
	\end{frame}

	\begin{frame}
	    \begin{definicao}
	        \begin{itemize}
        		\item[M)] A cada par $\alpha \in \cp{K}$ e $u \in V$, corresponde um vetor $\alpha \cdot u \in V$, denominado \textbf{produto por escalar} de $\alpha$ por $u$ de modo que:
		
                \vspace{.3cm}

	            \item[M1)] $(\alpha\beta)\cdot u = \alpha(\beta\cdot u)$ para todos $\alpha$, $\beta \in \cp{K}$ e todo $u \in V$;
	    	
                \vspace{.3cm}

        	    \item[M2)] $1_\cp{K}\cdot u = u$ para todo $u \in V$, onde $1_\cp{K}$ \'e o elemento neutro da multiplica\c{c}\~ao em $\cp{K}$.
	    	\end{itemize}
	    \end{definicao}
	\end{frame}

	\begin{frame}
	    \begin{definicao}
	        \begin{itemize}
        		\item[D1)] $\alpha\cdot(u + w) = \alpha\cdot u + \alpha\cdot w$, para todo $\alpha \in \cp{K}$ e todos $u$, $w \in V$;
	    	
                \vspace{.3cm}

	            \item[D2)] $(\alpha + \beta)\cdot u = \alpha\cdot u + \beta\cdot u$, para todos $\alpha$, $\beta \in \cp{K}$ e todo $u \in V$.
	        \end{itemize}
        \end{definicao}
    \end{frame}

    \begin{frame}
        \begin{observacao}
	        Vamos usar a express\~ao $\cp{K}$-\textbf{espa\c{c}o vetorial} para nos referir a um espa\c{c}o vetorial $V$ sobre um corpo $\cp{K}$.
        \end{observacao} 
    \end{frame}

    \begin{frame}
        \begin{definicao}
	        Seja $V$ um espa\c{c}o vetorial sobre $\cp{K}$.
	        Um vetor $w \in V$ \'e uma \textbf{combina\c{c}\~ao linear}\index{Espa\c{c}o Vetorial!Combina\c{c}\~ao linear} dos vetores $u_1$, $u_2$, \dots, $u_n \in V$ se existirem escalares $\alpha_1$, $\alpha_2$, \dots, $\alpha_n \in V$ tais que
    	    \[
	    	    w = \alpha_1 u_1 + \alpha_2u_2 + \cdots + \alpha_nu_n = \sum_{i = 1}^n \alpha_iu_i.
        	\]
        \end{definicao}
    \end{frame}

    \begin{frame}
        \begin{definicao}
        	Seja $V$ um $\cp{K}$-espa\c{c}o vetorial. Um subconjunto n\~ao vazio $W$ de $V$ \'e um \textbf{subespa\c{c}o vetorial} de $V$ se a restri\c{c}\~ao das opera\c{c}o\~es de $V$ a $W$ torna $W$ um $\cp{K}$-espa\c{c}o vetorial.\index{Subespa\c{c}o}
        \end{definicao}
        
        \begin{teorema}
        	Um subconjunto n\~ao vazio $W$ de um $\cp{K}$-espa\c{c}o vetorial $V$ \'e um subespa\c{c}o de $V$ se, e somente se, para cada par de vetores $u_1$, $u_2 \in W$ e cada escalar $\lambda \in \cp{K}$, temos que $\lambda u_1 + u_2 \in W$.
        \end{teorema}
    \end{frame}
\end{document}
