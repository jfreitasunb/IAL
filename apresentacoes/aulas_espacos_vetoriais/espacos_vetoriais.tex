%!TEX program = xelatex
\def\ano{2023}
\def\semestre{1}
\def\disciplina{Introdução à Álgebra Linear}
\def\turma{3}
\def\autor{Jos\'e Ant\^onio O. Freitas}
\def\instituto{MAT-UnB}

\documentclass{beamer}
\usetheme{Madrid}
\usecolortheme{beaver}
% \mode<presentation>
\usepackage{caption}
\usepackage{amssymb}
\usepackage{amsmath,amsfonts,amsthm,amstext}
\usepackage[brazil]{babel}
\usepackage{graphicx}
\graphicspath{{../Pictures/}}
\usepackage{enumitem}
\usepackage{multicol}
\usepackage{answers}
\usepackage[svgnames]{xcolor}
\usepackage{tikz}
\usepackage{ifthen}
\usetikzlibrary{lindenmayersystems}
\usetikzlibrary[shadings]

\newcounter{exercicios}
\setcounter{exercicios}{0}
\newcommand{\questao}{
    \addtocounter{exercicios}{1}
    \noindent{\bf Quest{\~a}o \arabic{exercicios}: }}

\newcommand{\resp}[1]{
    \noindent{\bf Exerc{\'\i}cio #1: }}

\extrafootheight[.25in]{.5in}
\footrule
\lfoot{Teste \numeroteste\ - Módulo \modulo\ - \nomeabreviado\ - Turma \turma\ - \semestre$^o$/\ano}
\cfoot{}
\rfoot{P\'agina \thepage\ de \numpages}
\def\ano{2023}
\def\semestre{2}
\def\disciplina{Introdução à Álgebra Linear}
\def\nomeabreviado{IAL}
\def\turma{11}

\newcommand{\im}{{\rm Im\,}}
\newcommand{\dlim}[2]{\displaystyle\lim_{#1\rightarrow #2}}
\newcommand{\minf}{+\infty}
\newcommand{\ninf}{-\infty}
\newcommand{\cp}[1]{\mathbb{#1}}
\newcommand{\sub}{\subseteq}
\newcommand{\n}{\mathbb{N}}
\newcommand{\z}{\mathbb{Z}}
\newcommand{\rac}{\mathbb{Q}}
\newcommand{\real}{\mathbb{R}}
\newcommand{\complex}{\mathbb{C}}

\newcommand{\vesp}[1]{\vspace{ #1  cm}}

\newcommand{\compcent}[1]{\vcenter{\hbox{$#1\circ$}}}
\newcommand{\comp}{\mathbin{\mathchoice
        {\compcent\scriptstyle}{\compcent\scriptstyle}
        {\compcent\scriptscriptstyle}{\compcent\scriptscriptstyle}}}
\renewcommand{\sin}{{\rm sen\,}}
\renewcommand{\tan}{{\rm tg\,}}
\renewcommand{\csc}{{\rm cossec\,}}
\renewcommand{\cot}{{\rm cotg\,}}
\renewcommand{\sinh}{{\rm senh\,}}

\title{Espaços Vetoriais}
\author[\autor]{\autor}
\institute[\instituto]{\instituto}
\date{}

\begin{document}
\begin{frame}
  \maketitle
\end{frame}

\logo{\includegraphics[scale=.1]{logo-MAT.png}\vspace*{8.5cm}}

\begin{frame}
  No que se segue o conjunto $\cp{K}$ denotar\'a um corpo, em geral $\rac$, $\real$ ou $\complex$.

  \vspace{.3cm}

  \begin{definicao}
    Um conjunto n\~ao vazio $V$ \pause \'e um \textbf{espa\c{c}o vetorial}\index{Espa\c{c}o Vetorial} \pause sobre um corpo $\cp{K}$ \pause se em seus elementos, chamados \textbf{vetores}, \pause estiverem definidas duas
    opera\c{c}\~oes satisfazendo:\pause
    \begin{itemize}
      \item[A)] A cada par $u$, $w \in V$ \pause corresponde um vetor $u + w \in V$, \pause chamado \textbf{soma} de $u$ e $w$, de modo que:\pause

        \vspace{.3cm}

      \item[A1)] $u + w \pause = w + u$, \pause para todos $u$, $w \in V$;\pause

        \vspace{.3cm}

      \item[A2)] $(u + w) \pause + x \pause = u + \pause (w + x)$, \pause para todos $u$, $w$, $x \in V$;
    \end{itemize}
  \end{definicao}
\end{frame}

\begin{frame}
  \begin{definicao}
    \begin{itemize}
      \item[A3)] Existe em $V$ um vetor, \pause denominado \textbf{vetor nulo} \pause e denotado por $0_V$, tal que\pause
        \[
          0_V + u \pause = u\pause
        \]
        para todo $u \in V$.\pause

        \vspace{.3cm}

      \item[A4)] Para cada vetor $u \in V$, \pause existe um vetor em $V$, \pause denotado por $-u$ \pause e chamado de
        \textbf{vetor oposto}, tal que\pause
        \[
          u + (-u) \pause = 0_V.
        \]
    \end{itemize}
  \end{definicao}
\end{frame}

\begin{frame}
  \begin{definicao}
    \begin{itemize}
      \item[M)] A cada par $\alpha \in \cp{K}$ \pause e $u \in V$, \pause corresponde um vetor $\alpha \cdot u \in V$, \pause denominado \textbf{produto por escalar} \pause de $\alpha$ por $u$ de modo que:\pause

        \vspace{.3cm}

      \item[M1)] $(\alpha\beta)\pause \cdot u = \pause \alpha(\pause \beta\cdot u)$ para todos $\alpha$, $\beta \in \cp{K}$ \pause e todo $u \in V$;\pause

        \vspace{.3cm}

      \item[M2)] $1_\cp{K}\cdot u \pause = u$ para todo $u \in V$, \pause onde $1_\cp{K}$ \'e o elemento neutro da multiplica\c{c}\~ao em $\cp{K}$.
    \end{itemize}
  \end{definicao}
\end{frame}

\begin{frame}
  \begin{definicao}
    \begin{itemize}
      \item[D1)] $\alpha\pause \cdot(u + w) = \pause \alpha\cdot u \pause + \alpha\cdot w$, \pause para todo $\alpha \in \cp{K}$ \pause e todos $u$, $w \in V$;\pause

        \vspace{.3cm}

      \item[D2)] $(\alpha + \beta)\pause \cdot u \pause = \alpha\cdot u \pause + \beta\cdot u$, \pause para todos $\alpha$, $\beta \in \cp{K}$ \pause e todo $u \in V$.
    \end{itemize}
  \end{definicao}
\end{frame}

\begin{frame}
  \begin{observacao}
    Vamos usar a express\~ao $\cp{K}$-\textbf{espa\c{c}o vetorial} \pause para nos referir a um espa\c{c}o vetorial $V$ sobre um corpo $\cp{K}$.
  \end{observacao}
\end{frame}

\begin{frame}
  \begin{definicao}
    Seja $V$ um espa\c{c}o vetorial sobre $\cp{K}$.\pause
    Um vetor $w \in V$ \'e uma \textbf{combina\c{c}\~ao linear}\index{Espa\c{c}o Vetorial!Combina\c{c}\~ao linear} \pause dos vetores $u_1$, \pause $u_2$, \pause \dots,
    $u_n \in V$ \pause se existirem escalares $\alpha_1$, \pause $\alpha_2$, \pause \dots, \pause $\alpha_n \in \cp{K}$ tais que\pause
    \[
      w = \alpha_1 u_1 + \pause \alpha_2u_2 + \pause \cdots + \pause \alpha_nu_n = \sum_{i = 1}^n \alpha_iu_i.
    \]
  \end{definicao}
\end{frame}

\begin{frame}
  \begin{definicao}
    Seja $V$ um $\cp{K}$-espa\c{c}o vetorial. \pause Um subconjunto n\~ao vazio $W$ \pause de $V$ \'e um \textbf{subespa\c{c}o vetorial} de $V$ \pause se $W$ é um espaço vetorial
    sobre $\cp{K}$ \pause com respeito às mesmas operações definidas em $V$.\pause
  \end{definicao}

  \begin{teorema}
    Um subconjunto n\~ao vazio $W$ \pause de um $\cp{K}$-espa\c{c}o vetorial $V$ \pause \'e um subespa\c{c}o de $V$ \pause se, e somente se:
    \begin{enumerate}[label={\roman*})]
      \item para todos vetores $u_1$, $u_2 \in W$ \pause temos $u_1 + u_2 \in W$.\pause
      \item para todo escalar $\lambda \in \cp{K}$ \pause e todo vetor $u \in W$, \pause temos $\lambda u\in W$.
    \end{enumerate}
  \end{teorema}
\end{frame}

\begin{frame}
  \begin{definicao}
    Seja $V$ um $\cp{K}$-espaço vetorial. \pause Fixado vetores $v_1$, $v_2$, \dots, $v_n \in V$ \pause o conjunto
    \[
      W = \{\alpha_1v_1 \pause + \alpha_2v_2 \pause + \cdots + \alpha_nv_n \pause \mid \alpha_i \in \cp{K}\}\pause
    \]
    é chamado de \textbf{espaço gerado} \pause por $\{v_1, v_2, \dots, v_n\}$ \pause e os vetores $\{v_1, v_2, \dots, v_n\}$ \pause são chamados de \textbf{geradores} de $W$.\pause
  \end{definicao}

  \begin{notacao}
    O espaço gerado $W = \{\alpha_1v_1 + \alpha_2v_2 + \cdots + \alpha_nv_n\}$ \pause pode ser escrito como $W = Span(v_1, v_2, \dots, v_n)$ \pause ou $W = <v_1, v_2, \dots, v_n>$ \pause ou ainda como $W = [v_1, v_2, \dots, v_n]$.
  \end{notacao}
\end{frame}

\begin{frame}
  \begin{proposicao}
    Seja $V$ um $\cp{K}$-espaço vetorial. \pause Fixado vetores $v_1$, $v_2$, \dots, $v_n \in V$ \pause o conjunto
    \[
      W = \{\alpha_1v_1 \pause + \alpha_2v_2 \pause + \cdots + \alpha_nv_n \pause \mid \alpha_i \in \cp{K}\}\pause
    \]
    é um subespaço vetorial de $V$.
  \end{proposicao}
\end{frame}

\begin{frame}
  \begin{definicao}
    Um espaço vetorial $V$ é chamado de \textbf{finitamente gerado} \pause se existe um conjunto finito \pause de vetores $\{v_1, v_2, \dots, v_n\}$ em $V$ \pause tais que\pause
    \[
      V = \pause Span(v_1, v_2, \dots, v_n).
    \]
  \end{definicao}
\end{frame}

\begin{frame}
  \begin{definicao}
    Dados dois subespaços vetoriais $W_1$ \pause e $W_2$ \pause de um $\cp{K}$-espaço vetorial $V$, \pause definimos a interseção de $W_1$ e $W_2$ \pause como o subconjunto
    \[
      W_1 \cap W_2 \pause = \{u \in V \pause \mid u\in W_1\pause \mbox{ e } u \in W_2\}.\pause
    \]
  \end{definicao}

  \vspace{1cm}

  \begin{teorema}
    Dados $W_1$, $W_2$ dois subespaços de um $\cp{K}$-espaço vetorial $V$, \pause então $W_1 \cap W_2$ também é um subespaço vetorial de $V$.
  \end{teorema}
\end{frame}

\begin{frame}
  \begin{observacao}
    Se $W_1$ e $W_2$ são subespaços de $V$, \pause então nem sempre o conjunto $W_1 \cup W_2 \pause = \{u \in V \pause \mid u \in W_1 \pause \mbox{ ou } u \in W_2\}$ \pause é um subespaço de $V$.
  \end{observacao}
\end{frame}

\begin{frame}
  \begin{definicao}
    Dados dois subespaços $U$ e $W$ \pause de um $\cp{K}$-espaço vetorial $V$, \pause definimos a \textbf{soma de $U$ e $W$} \pause como o subconjunto
    \[
      U + W \pause = \{u + w \pause \mid u \in U, \pause w \in W\}.\pause
    \]
  \end{definicao}

  \vspace*{1cm}

  \begin{teorema}
    Dados dois subespaços $U$ e $W$ de um $\cp{K}$-espaço vetorial $V$, \pause então $U + W$ é um subespaço de $V$.
  \end{teorema}
\end{frame}

\begin{frame}
  \begin{definicao}
    Seja $V$ um $\cp{K}$-espaço vetorial. \pause Dados vetores $v_1$, $v_2$, \dots, $v_n$ \pause em $V$ dizemos que:\pause
    \begin{enumerate}[label={\roman*})]
      \item o conjunto $\{v_1, v_2, \dots, v_n\}$ \pause é \textbf{linearmente dependentes} \pause ou que os vetores $v_1$, $v_2$, \dots, $v_n$ são \textbf{linearmente dependentes}, \pause abreviado como L.D, \pause se existem escalares
        $\alpha_1$, $\alpha_2$, \dots, $\alpha_n \in \cp{K}$, \pause \textbf{não todos nulos}, \pause tais que\pause
        \[
            \alpha_1v_1 + \alpha_2v_2 + \cdots + \alpha_nv_n \pause = 0_V.\pause
        \]

      \item o conjunto $\{v_1, v_2, \dots, v_n\}$ \pause é \textbf{linearmente independentes} \pause ou que os vetores $v_1$, $v_2$, \dots, $v_n$ são \textbf{linearmente independentes}, \pause abreviado como L.I, \pause se dados escalares
        $\alpha_1$, $\alpha_2$, \dots, $\alpha_n \in \cp{K}$  tais que\pause
        \[
            \alpha_1v_1 + \alpha_2v_2 + \cdots + \alpha_nv_n \pause = 0_V,\pause
        \]
        então $\alpha_1 = \pause \alpha_2 = \pause \cdots = \alpha_n \pause = 0$.
    \end{enumerate}
  \end{definicao}
\end{frame}


\begin{frame}
  \begin{proposicao}
  Seja $V$ um espaço vetorial sobre $\cp{K}$. Então:

  \begin{enumerate}[label={\roman*})]
    \item O vetor nulo é linearmente dependente.

    \vspace{.3cm}
    
    \item Todo vetor $v \ne 0_V$ é linearmente independente.
    
    \vspace{.3cm}

    \item Dado um conjunto de vetores $\{v_1, v_2, \dots, v_n\}$ se $v_i = 0_V$ para algum $i$, então $\{v_1, v_2, \dots, v_n\}$ é L.D.

    \seti
  \end{enumerate}
  \end{proposicao}
\end{frame}

\begin{frame}
  \begin{proposicao}
  \begin{enumerate}[label={\roman*})]
   
    \conti

    \item Se os vetores $\{v_1, v_2, \dots, v_n\}$ são L.D., então qualquer outro conjunto de vetores que contenha $\{v_1, v_2, \dots, v_n\}$ também é L.D.
    
    \vspace{.3cm}

    \item Se $\{v_1, v_2, \dots, v_n\}$ é L.I., então qualquer subconjunto $\{v_{i_1}, v_{i_2}, \dots, v_{i_s}\}$ com $s < n$ também é L.I.
    
    \vspace{.3cm}

    \item Seja $\{v_1, v_2, \dots, v_n\}$ um conjunto L.I. Se $w$ é um vetor de $V$ tal que $\{v_1, v_2, \dots, v_n, w\}$ é L.D., então $w$ é uma combinação linear de $v_1$, $v_2$, \dots, $v_n$.
  \end{enumerate}
  \end{proposicao}
\end{frame}

\begin{frame}
  \begin{proposicao}
    Sejam $V$ um $\cp{K}$-espaço vetorial, $n \ge 2$ e $\{v_1, v_2, \dots, v_n\}$ vetores de $V$. Então $\{v_1, v_2, \dots, v_n\}$ é L.D. se, e somente se, pelo menos um vetor $v_j$ for 
    combinação linear dos demais $n - 1$ de $\{v_1, v_2, \dots, v_n\}$.
  \end{proposicao}
\end{frame}


\begin{frame}
  \begin{definicao}
    Seja $V$ um $\cp{K}$-espaço vetorial. Um conjunto de vetores $\{v_1, v_2, \dots, v_n\}$ é chamado de uma \textbf{base} de $V$ se:
    \begin{enumerate}[label={\roman*})]
      \item $\{v_1, v_2, \dots, v_n\}$ é L.I.

      \vspace{.3cm}

      \item $V = Span(v_1, v_2, \dots, v_n)$, ou seja, $\{v_1, v_2, \dots, v_n\}$ gera $V$.
    \end{enumerate}
  \end{definicao}
\end{frame}


\begin{frame}
  \begin{teorema}
    Seja $V$ um espaço vetorial sobre $\cp{K}$. Se $\{v_1, v_2, \dots, v_n\}$ é uma base de $V$, então todo conjunto de vetores de $V$ com mais de $n$ elementos é L.D.
  \end{teorema}

  \begin{corolario}
    Dado um $\cp{K}$-espaço vetorial $V$, então quaisquer duas bases de $V$ possuem o mesmo número de vetores.
  \end{corolario}
\end{frame}

\begin{frame}
  \begin{definicao}
    Seja $V$ um espaço vetorial sobre $\cp{K}$. A \textbf{dimensão} de $V$ é o número de vetores de uma base de $V$, quando essa base é constituída de uma quantidade finita de vetores.
    Nesse caso vamos denotar esse númeor por $\dim_\cp{K}V$ ou simplesmente $\dim V$.
  \end{definicao}

  \begin{observacoes}
    \begin{enumerate}[label={\roman*})]
      \item Se $V$ não possui base, então $\dim V = 0$.

      \item Se $V$ possui uma base com infinitos vetores, então $\dim V = \infty$.
    \end{enumerate}
  \end{observacoes}
\end{frame}
\end{document}
