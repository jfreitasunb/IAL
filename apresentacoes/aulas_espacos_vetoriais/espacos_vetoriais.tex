%!TEX program = xelatex
\def\ano{2023}
\def\semestre{1}
\def\disciplina{Introdução à Álgebra Linear}
\def\turma{3}
\def\autor{Jos\'e Ant\^onio O. Freitas}
\def\instituto{MAT-UnB}

\documentclass{beamer}
\usetheme{Madrid}
\usecolortheme{beaver}
% \mode<presentation>
\usepackage{caption}
\usepackage{amssymb}
\usepackage{amsmath,amsfonts,amsthm,amstext}
\usepackage[brazil]{babel}
\usepackage{graphicx}
\graphicspath{{../Pictures/}}
\usepackage{enumitem}
\usepackage{multicol}
\usepackage{answers}
\usepackage[svgnames]{xcolor}
\usepackage{tikz}
\usepackage{ifthen}
\usetikzlibrary{lindenmayersystems}
\usetikzlibrary[shadings]

\newcounter{exercicios}
\setcounter{exercicios}{0}
\newcommand{\questao}{
    \addtocounter{exercicios}{1}
    \noindent{\bf Quest{\~a}o \arabic{exercicios}: }}

\newcommand{\resp}[1]{
    \noindent{\bf Exerc{\'\i}cio #1: }}

\extrafootheight[.25in]{.5in}
\footrule
\lfoot{Teste \numeroteste\ - Módulo \modulo\ - \nomeabreviado\ - Turma \turma\ - \semestre$^o$/\ano}
\cfoot{}
\rfoot{P\'agina \thepage\ de \numpages}
\def\ano{2023}
\def\semestre{2}
\def\disciplina{Introdução à Álgebra Linear}
\def\nomeabreviado{IAL}
\def\turma{11}

\newcommand{\im}{{\rm Im\,}}
\newcommand{\dlim}[2]{\displaystyle\lim_{#1\rightarrow #2}}
\newcommand{\minf}{+\infty}
\newcommand{\ninf}{-\infty}
\newcommand{\cp}[1]{\mathbb{#1}}
\newcommand{\sub}{\subseteq}
\newcommand{\n}{\mathbb{N}}
\newcommand{\z}{\mathbb{Z}}
\newcommand{\rac}{\mathbb{Q}}
\newcommand{\real}{\mathbb{R}}
\newcommand{\complex}{\mathbb{C}}

\newcommand{\vesp}[1]{\vspace{ #1  cm}}

\newcommand{\compcent}[1]{\vcenter{\hbox{$#1\circ$}}}
\newcommand{\comp}{\mathbin{\mathchoice
        {\compcent\scriptstyle}{\compcent\scriptstyle}
        {\compcent\scriptscriptstyle}{\compcent\scriptscriptstyle}}}
\renewcommand{\sin}{{\rm sen\,}}
\renewcommand{\tan}{{\rm tg\,}}
\renewcommand{\csc}{{\rm cossec\,}}
\renewcommand{\cot}{{\rm cotg\,}}
\renewcommand{\sinh}{{\rm senh\,}}

\title{Espaços Vetoriais}
\author[\autor]{\autor}
\institute[\instituto]{\instituto}
\date{}

\begin{document}
\begin{frame}
	\maketitle
\end{frame}

\logo{\includegraphics[scale=.1]{logo-MAT.png}\vspace*{8.5cm}}

\begin{frame}
	No que se segue o conjunto $\cp{K}$ denotar\'a um corpo, em geral $\rac$, $\real$ ou $\complex$.

	\vspace{.3cm}

	\begin{definicao}
		Um conjunto	n\~ao vazio $V$ \pause \'e um \textbf{espa\c{c}o vetorial}\index{Espa\c{c}o Vetorial} \pause sobre um corpo $\cp{K}$ \pause se em seus elementos, chamados \textbf{vetores}, \pause estiverem definidas duas
		opera\c{c}\~oes satisfazendo:\pause
		\begin{itemize}
			\item[A)] A cada par $u$, $w \in V$ \pause corresponde um vetor $u + w \in V$, \pause chamado \textbf{soma} de $u$ e $w$, de modo que:\pause

				\vspace{.3cm}

			\item[A1)] $u + w \pause = w + u$, \pause para todos $u$, $w \in V$;\pause

				\vspace{.3cm}

			\item[A2)] $(u + w) \pause + x \pause = u + \pause (w + x)$, \pause para todos $u$, $w$, $x \in V$;
		\end{itemize}
	\end{definicao}
\end{frame}

\begin{frame}
	\begin{definicao}
		\begin{itemize}
			\item[A3)] Existe em $V$ um vetor, \pause denominado \textbf{vetor nulo} \pause e denotado por $0_V$, tal que\pause
				\[
					0_V + u \pause = u\pause
				\]
				para todo $u \in V$.\pause

				\vspace{.3cm}

			\item[A4)] Para cada vetor $u \in V$, \pause existe um vetor em $V$, \pause denotado por $-u$ \pause e chamado de
				\textbf{vetor oposto}, tal que\pause
				\[
					u + (-u) \pause = 0_V.
				\]
		\end{itemize}
	\end{definicao}
\end{frame}

\begin{frame}
	\begin{definicao}
		\begin{itemize}
			\item[M)] A cada par $\alpha \in \cp{K}$ \pause e $u \in V$, \pause corresponde um vetor $\alpha \cdot u \in V$, \pause denominado \textbf{produto por escalar} \pause de $\alpha$ por $u$ de modo que:\pause

				\vspace{.3cm}

			\item[M1)] $(\alpha\beta)\pause \cdot u = \pause \alpha(\pause \beta\cdot u)$ para todos $\alpha$, $\beta \in \cp{K}$ \pause e todo $u \in V$;\pause

				\vspace{.3cm}

			\item[M2)] $1_\cp{K}\cdot u \pause = u$ para todo $u \in V$, \pause onde $1_\cp{K}$ \'e o elemento neutro da multiplica\c{c}\~ao em $\cp{K}$.
		\end{itemize}
	\end{definicao}
\end{frame}

\begin{frame}
	\begin{definicao}
		\begin{itemize}
			\item[D1)] $\alpha\pause \cdot(u + w) = \pause \alpha\cdot u \pause + \alpha\cdot w$, \pause para todo $\alpha \in \cp{K}$ \pause e todos $u$, $w \in V$;\pause

				\vspace{.3cm}

			\item[D2)] $(\alpha + \beta)\pause \cdot u \pause = \alpha\cdot u \pause + \beta\cdot u$, \pause para todos $\alpha$, $\beta \in \cp{K}$ \pause e todo $u \in V$.
		\end{itemize}
	\end{definicao}
\end{frame}

\begin{frame}
	\begin{observacao}
		Vamos usar a express\~ao $\cp{K}$-\textbf{espa\c{c}o vetorial} \pause para nos referir a um espa\c{c}o vetorial $V$ sobre um corpo $\cp{K}$.
	\end{observacao}
\end{frame}

\begin{frame}
	\begin{definicao}
		Seja $V$ um espa\c{c}o vetorial sobre $\cp{K}$.\pause
		Um vetor $w \in V$ \'e uma \textbf{combina\c{c}\~ao linear}\index{Espa\c{c}o Vetorial!Combina\c{c}\~ao linear} \pause dos vetores $u_1$, \pause $u_2$, \pause \dots,
		$u_n \in V$ \pause se existirem escalares $\alpha_1$, \pause $\alpha_2$, \pause \dots, \pause $\alpha_n \in \cp{K}$ tais que\pause
		\[
			w = \alpha_1 u_1 + \pause \alpha_2u_2 + \pause \cdots + \pause \alpha_nu_n = \sum_{i = 1}^n \alpha_iu_i.
		\]
	\end{definicao}
\end{frame}

\begin{frame}
	\begin{definicao}
		Seja $V$ um $\cp{K}$-espa\c{c}o vetorial. \pause Um subconjunto n\~ao vazio $W$ \pause de $V$ \'e um \textbf{subespa\c{c}o vetorial} de $V$ \pause se $W$ é um espaço vetorial
		sobre $\cp{K}$ \pause com respeito às mesmas operações definidas em $V$.\pause
	\end{definicao}

	\begin{teorema}
		Um subconjunto n\~ao vazio $W$ \pause de um $\cp{K}$-espa\c{c}o vetorial $V$ \pause \'e um subespa\c{c}o de $V$ \pause se, e somente se:
		\begin{enumerate}[label={\roman*})]
			\item para todos vetores $u_1$, $u_2 \in W$ \pause temos $u_1 + u_2 \in W$.\pause
			\item para todo escalar $\lambda \in \cp{K}$ \pause e todo vetor $u \in W$, \pause temos $\lambda u\in W$.
		\end{enumerate}
	\end{teorema}
\end{frame}
\end{document}


