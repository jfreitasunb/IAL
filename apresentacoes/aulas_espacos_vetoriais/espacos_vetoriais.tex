%!TEX program = xelatex
%!TEX encoding = UTF-8
\def\ano{2023}
\def\semestre{1}
\def\disciplina{Introdução à Álgebra Linear}
\def\turma{3}
\def\autor{José Ant\^onio O. Freitas}
\def\instituto{MAT-UnB}

\documentclass{beamer}
\usetheme{Madrid}
\usecolortheme{beaver}
% \mode<presentation>
\usepackage{caption}
\usepackage{amssymb}
\usepackage{amsmath,amsfonts,amsthm,amstext}
\usepackage[brazil]{babel}
\usepackage{graphicx}
\graphicspath{{../Pictures/}}
\usepackage{enumitem}
\usepackage{multicol}
\usepackage{answers}
\usepackage[svgnames]{xcolor}
\usepackage{tikz}
\usepackage{ifthen}
\usetikzlibrary{lindenmayersystems}
\usetikzlibrary[shadings]

\newcounter{exercicios}
\setcounter{exercicios}{0}
\newcommand{\questao}{
    \addtocounter{exercicios}{1}
    \noindent{\bf Quest{\~a}o \arabic{exercicios}: }}

\newcommand{\resp}[1]{
    \noindent{\bf Exerc{\'\i}cio #1: }}

\extrafootheight[.25in]{.5in}
\footrule
\lfoot{Teste \numeroteste\ - Módulo \modulo\ - \nomeabreviado\ - Turma \turma\ - \semestre$^o$/\ano}
\cfoot{}
\rfoot{P\'agina \thepage\ de \numpages}
\def\ano{2023}
\def\semestre{2}
\def\disciplina{Introdução à Álgebra Linear}
\def\nomeabreviado{IAL}
\def\turma{11}

\newcommand{\im}{{\rm Im\,}}
\newcommand{\dlim}[2]{\displaystyle\lim_{#1\rightarrow #2}}
\newcommand{\minf}{+\infty}
\newcommand{\ninf}{-\infty}
\newcommand{\cp}[1]{\mathbb{#1}}
\newcommand{\sub}{\subseteq}
\newcommand{\n}{\mathbb{N}}
\newcommand{\z}{\mathbb{Z}}
\newcommand{\rac}{\mathbb{Q}}
\newcommand{\real}{\mathbb{R}}
\newcommand{\complex}{\mathbb{C}}

\newcommand{\vesp}[1]{\vspace{ #1  cm}}

\newcommand{\compcent}[1]{\vcenter{\hbox{$#1\circ$}}}
\newcommand{\comp}{\mathbin{\mathchoice
        {\compcent\scriptstyle}{\compcent\scriptstyle}
        {\compcent\scriptscriptstyle}{\compcent\scriptscriptstyle}}}
\renewcommand{\sin}{{\rm sen\,}}
\renewcommand{\tan}{{\rm tg\,}}
\renewcommand{\csc}{{\rm cossec\,}}
\renewcommand{\cot}{{\rm cotg\,}}
\renewcommand{\sinh}{{\rm senh\,}}

\title{Espaços Vetoriais}
\author[\autor]{\autor}
\institute[\instituto]{\instituto}
\date{}

\begin{document}
\begin{frame}
  \maketitle
\end{frame}

\logo{\includegraphics[scale=.1]{logo-MAT.png}\vspace*{8.5cm}}

\begin{frame}
  No que se segue o conjunto $\cp{K}$ denotará um corpo, em geral $\rac$, $\real$ ou $\complex$.

  \vspace{.3cm}

  \begin{definicao}
    Um conjunto não vazio $V$  é um \textbf{espaço vetorial}\index{Espaço Vetorial}  sobre um corpo $\cp{K}$  se em seus elementos, chamados \textbf{vetores},  estiverem definidas duas
    operações satisfazendo:
    \begin{itemize}
      \item[A)] A cada par $u$, $w \in V$  corresponde um vetor $u + w \in V$,  chamado \textbf{soma} de $u$ e $w$, de modo que:

        \vspace{.3cm}

      \item[A1)] $u + w  = w + u$,  para todos $u$, $w \in V$;

        \vspace{.3cm}

      \item[A2)] $(u + w)  + x  = u +  (w + x)$,  para todos $u$, $w$, $x \in V$;
    \end{itemize}
  \end{definicao}
\end{frame}

\begin{frame}
  \begin{definicao}
    \begin{itemize}
      \item[A3)] Existe em $V$ um vetor,  denominado \textbf{vetor nulo}  e denotado por $0_V$, tal que
        \[
          0_V + u  = u
        \]
        para todo $u \in V$.

        \vspace{.3cm}

      \item[A4)] Para cada vetor $u \in V$,  existe um vetor em $V$,  denotado por $-u$  e chamado de
        \textbf{vetor oposto}, tal que
        \[
          u + (-u)  = 0_V.
        \]
    \end{itemize}
  \end{definicao}
\end{frame}

\begin{frame}
  \begin{definicao}
    \begin{itemize}
      \item[M)] A cada par $\alpha \in \cp{K}$  e $u \in V$,  corresponde um vetor $\alpha \cdot u \in V$,  denominado \textbf{produto por escalar}  de $\alpha$ por $u$ de modo que:

        \vspace{.3cm}

      \item[M1)] $(\alpha\beta) \cdot u =  \alpha( \beta\cdot u)$  para todos $\alpha$, $\beta \in \cp{K}$  e todo $u \in V$;

        \vspace{.3cm}

      \item[M2)] $1_\cp{K}\cdot u  = u$  para todo $u \in V$,  onde $1_\cp{K}$ é o elemento neutro da multiplicação em $\cp{K}$.
    \end{itemize}
  \end{definicao}
\end{frame}

\begin{frame}
  \begin{definicao}
    \begin{itemize}
      \item[D1)] $\alpha \cdot(u + w) =  \alpha\cdot u  + \alpha\cdot w$,  para todo $\alpha \in \cp{K}$  e todos $u$, $w \in V$;

        \vspace{.3cm}

      \item[D2)] $(\alpha + \beta) \cdot u  = \alpha\cdot u  + \beta\cdot u$,  para todos $\alpha$, $\beta \in \cp{K}$  e todo $u \in V$.
    \end{itemize}
  \end{definicao}
\end{frame}

\begin{frame}
  \begin{observacao}
    Vamos usar a expressão $\cp{K}$-\textbf{espaço vetorial}  para nos referir a um espaço vetorial $V$ sobre um corpo $\cp{K}$.
  \end{observacao}
\end{frame}

\begin{frame}
    \begin{exemplos}[Espaços vetoriais]
        Seja $\cp{K}$ um corpo,  $\cp{K} = \rac$, $\real$ ou $\complex$.  Considere o conjunto
            \[
                \cp{K}^n = \underbrace{\cp{K} \times \cp{K} \times \cdots \times \cp{K}}_{n\ vezes}\ =  \{(a_1, a_2,\dots,a_n) \mid a_i \in \cp{K}, i =1, 2, \dots, n\}
            \]
            e defina
            \begin{itemize}[label=$\color{blue}\bullet$]
                \item $(a_1, a_2, \dots, a_n) + (b_1, b_2, \dots,b_n) = (a_1 + b_1, a_2 + b_2,\dots, a_n + b_n)$

                \item $\alpha (a_1, a_2, \dots,a_n) = (\alpha a_1, \alpha a_2, \dots, \alpha a_n)$
            \end{itemize}
            para todos $(a_1, a_2, \dots,a_n)$ ,$(b_1, b_2, \dots,b_n) \in \cp{K}^n$  e para todo $\alpha \in \cp{K}$.

            \vspace*{.2cm}

            Com estas operações $\cp{K}^n$ é um espaço vetorial sobre $\cp{K}$.

            \vspace*{.2cm}

            Assim temos:
                \begin{itemize}[label=$\color{blue}\blacktriangleright$]
                    \item $\rac^n$  é um espaço vetorial sobre $\rac$;
                    \item $\real^n$  é um espaço vetorial sobre $\real$;
                    \item $\complex^n$  é um espaço vetorial sobre $\complex$.
                \end{itemize}
    \end{exemplos}
\end{frame}

\begin{frame}
    \begin{exemplos}[Espaços vetoriais]
    $\complex^n$  é um espaço vetorial sobre $\real$  se definirmos:
        \begin{itemize}[label=$\color{blue}\bullet$]
            \item $(a_1, a_2, \dots, a_n) + (b_1, b_2, \dots,b_n)  = (a_1 + b_1, a_2 + b_2,\dots, a_n + b_n)$
            \item $\alpha (a_1, a_2, \dots,a_n) = (\alpha a_1, \alpha a_2, \dots, \alpha a_n)$
        \end{itemize}
        para todos $(a_1, a_2, \dots,a_n)$ ,$(b_1, b_2, \dots,b_n) \in \complex^n$  e para todo $\alpha \in \real$.
    \end{exemplos}
\end{frame}

\begin{frame}
    \begin{exemplos}[Espaços vetoriais]
        De modo geral,  o conjunto das matrizes $\cp{M}_{p \times q}(\cp{K})$  com coeficientes em $\cp{K}$  é um $\cp{K}$-espaço vetorial  com a soma usual de matrizes  e a multiplicação por escalar usual.
\end{exemplos}
\end{frame}

\begin{frame}
    \begin{exemplos}[Espaços vetoriais]
        Considere o conjunto dos polinômios
        \begin{align*}
            \mathcal{P}(\cp{K}) = &\{ p(x) = a_nx^n + a_{n - 1}x^{n - 1} + \cdots + a_1x + a_0 \mid \\ &a_i \in \cp{K}, i = 0, 1, 2, \dots, n; n \ge 0 \}.
        \end{align*}

    Dados $p(x) = a_nx^n + a_{n - 1}x^{n - 1} + \cdots + a_1x + a_0$  e $q(x) = b_mx^m + b_{m - 1}x^{m - 1} + \cdots + b_1x + b_0$  em $\mathcal{P}(\cp{K})$,  suponha que $m < n$ e defina:
    \begin{itemize}[label=$\color{blue}\bullet$]
        \item $(p + q)(x) =  (a_nx^n + a_{n - 1}x^{n - 1} + \cdots + a_mx^m + a_{m-1}x^{m-1} + \cdots + a_1x + a_0)  + (b_mx^m + b_{m - 1}x^{m - 1} + \cdots + b_1x + b_0 = a_nx^n  + a_{n-1}x^{n-1} +  \cdots + (a_m + b_m)x^m + (a_{m-1} + b_{m-1})x^{m-1} + \cdots  + (a_1 + b_1)x + (a_0 + b_0)$
        \vspace*{.2cm}
        \item $(\alpha p)(x) =  \alpha(a_nx^n + a_{n - 1}x^{n - 1} + \cdots + a_1x + a_0)  = \alpha a_nx^n  + \alpha a_{n - 1}x^{n - 1}  + \cdots + \alpha a_1x + \alpha a_0$,  onde $\alpha \in \cp{K}$.
    \end{itemize}
    Assim $\mathcal{P}(\cp{K})$ é um $\cp{K}$-espaço vetorial.
    \end{exemplos}
\end{frame}

\begin{frame}
  \begin{definicao}
    Seja $V$ um espaço vetorial sobre $\cp{K}$.
    Um vetor $w \in V$ é uma \textbf{combinação linear}  dos vetores $u_1$,  $u_2$,  \dots,
    $u_n \in V$  se existirem escalares $\alpha_1$,  $\alpha_2$,  \dots,  $\alpha_n \in \cp{K}$ tais que
    \[
      w = \alpha_1 u_1 +  \alpha_2u_2 +  \cdots +  \alpha_nu_n = \sum_{i = 1}^n \alpha_iu_i.
    \]
  \end{definicao}
\end{frame}

\begin{frame}
  \begin{definicao}
    Seja $V$ um $\cp{K}$-espaço vetorial.  Um subconjunto não vazio $W$  de $V$ é um \textbf{subespaço vetorial} de $V$  se $W$ é um espaço vetorial
    sobre $\cp{K}$  com respeito às mesmas operações definidas em $V$.
  \end{definicao}

  \begin{teorema}
    Um subconjunto não vazio $W$  de um $\cp{K}$-espaço vetorial $V$  é um subespaço de $V$  se, e somente se:
    \begin{enumerate}[label={\roman*})]
      \item para todos vetores $u_1$, $u_2 \in W$  temos $u_1 + u_2 \in W$.
      \item para todo escalar $\lambda \in \cp{K}$  e todo vetor $u \in W$,  temos $\lambda u\in W$.
    \end{enumerate}
  \end{teorema}
\end{frame}

\begin{frame}
  \begin{definicao}
    Seja $V$ um $\cp{K}$-espaço vetorial.  Fixado vetores $v_1$, $v_2$, \dots, $v_n \in V$  o conjunto
    \[
      W = \{\alpha_1v_1  + \alpha_2v_2  + \cdots + \alpha_nv_n  \mid \alpha_i \in \cp{K}\}
    \]
    é chamado de \textbf{espaço gerado}  por $\{v_1, v_2, \dots, v_n\}$  e os vetores $v_1$, $v_2$, \dots, $v_n$  são chamados de \textbf{geradores} de $W$.
  \end{definicao}

  \begin{notacao}
    O espaço gerado $W = \{\alpha_1v_1 + \alpha_2v_2 + \cdots + \alpha_nv_n \mid \alpha_i \in \cp{K}\}$  pode ser escrito como $W = Span(v_1, v_2, \dots, v_n)$  ou $W = <v_1, v_2, \dots, v_n>$  ou ainda como $W = [v_1, v_2, \dots, v_n]$.
  \end{notacao}
\end{frame}

\begin{frame}
  \begin{proposicao}
    Seja $V$ um $\cp{K}$-espaço vetorial.  Fixado vetores $v_1$, $v_2$, \dots, $v_n \in V$  o conjunto
    \[
      W = \{\alpha_1v_1  + \alpha_2v_2  + \cdots + \alpha_nv_n  \mid \alpha_i \in \cp{K}\}
    \]
    é um subespaço vetorial de $V$.
  \end{proposicao}
\end{frame}

\begin{frame}
  \begin{definicao}
    Um espaço vetorial $V$ é chamado de \textbf{finitamente gerado}  se existe um conjunto finito  de vetores $\{v_1, v_2, \dots, v_n\}$ em $V$  tais que
    \[
      V =  Span(v_1, v_2, \dots, v_n).
    \]
  \end{definicao}
\end{frame}

\begin{frame}
  \begin{definicao}
    Seja $V$ um $\cp{K}$-espaço vetorial.  Dados vetores $v_1$, $v_2$, \dots, $v_n$  em $V$ dizemos que:
    \begin{enumerate}[label={\roman*})]
      \item o conjunto $\{v_1, v_2, \dots, v_n\}$  é \textbf{linearmente dependente}  ou que os vetores $v_1$, $v_2$, \dots, $v_n$  são \textbf{linearmente dependentes},  abreviado como L.D,  se existem escalares
        $\alpha_1$, $\alpha_2$, \dots, $\alpha_n \in \cp{K}$,  \textbf{não todos nulos},  tais que
        \[
            \alpha_1v_1 + \alpha_2v_2 + \cdots + \alpha_nv_n  = 0_V.
        \]

      \item o conjunto $\{v_1, v_2, \dots, v_n\}$  é \textbf{linearmente independente}  ou que os vetores $v_1$, $v_2$, \dots, $v_n$  são \textbf{linearmente independentes},  abreviado como L.I,  se dados escalares
        $\alpha_1$, $\alpha_2$, \dots, $\alpha_n \in \cp{K}$  tais que
        \[
            \alpha_1v_1 + \alpha_2v_2 + \cdots + \alpha_nv_n  = 0_V,
        \]
        então $\alpha_1 =  \alpha_2 =  \cdots = \alpha_n  = 0$.
    \end{enumerate}
  \end{definicao}
\end{frame}


\begin{frame}
  \begin{proposicao}
  Seja $V$ um espaço vetorial sobre $\cp{K}$. Então:

  \begin{enumerate}[label={\roman*})]
    \item O vetor nulo é linearmente dependente.

    \vspace{.3cm}

    \item Todo vetor $v \ne 0_V$  é linearmente independente.

    \vspace{.3cm}

    \item Dado um conjunto de vetores $\{v_1, v_2, \dots, v_n\}$  se $v_i = 0_V$ para algum $i$,  então $\{v_1, v_2, \dots, v_n\}$ é L.D.

    \seti
  \end{enumerate}
  \end{proposicao}
\end{frame}

\begin{frame}
  \begin{proposicao}
  \begin{enumerate}[label={\roman*})]

    \conti

    \item Se os vetores $\{v_1, v_2, \dots, v_n\}$ são L.D.,  então qualquer outro conjunto de vetores que contenha $\{v_1, v_2, \dots, v_n\}$  também é L.D.

    \vspace{.3cm}

    \item Se $\{v_1, v_2, \dots, v_n\}$ é L.I.,  então qualquer subconjunto $\{v_{i_1}, v_{i_2}, \dots, v_{i_s}\}$  com $s < n$ também é L.I.

    \vspace{.3cm}

    \item Seja $\{v_1, v_2, \dots, v_n\}$ um conjunto L.I.  Se $w$ é um vetor de $V$ tal que $\{v_1, v_2, \dots, v_n, w\}$  é L.D.,  então $w$ é uma combinação linear de $v_1$, $v_2$, \dots, $v_n$.
  \end{enumerate}
  \end{proposicao}
\end{frame}

\begin{frame}
  \begin{proposicao}
    Sejam $V$ um $\cp{K}$-espaço vetorial,  $n \ge 2$ e $\{v_1, v_2, \dots, v_n\}$ vetores de $V$.  Então $\{v_1, v_2, \dots, v_n\}$ é L.D.  se, e somente se, pelo menos um vetor $v_j$  for
    combinação linear dos demais $n - 1$ de $\{v_1, v_2, \dots, v_n\}$.
  \end{proposicao}
\end{frame}


\begin{frame}
  \begin{definicao}
    Seja $V$ um $\cp{K}$-espaço vetorial.  Um conjunto de vetores $\{v_1, v_2, \dots, v_n\}$  é chamado de uma \textbf{base}  de $V$ se:
    \begin{enumerate}[label={\roman*})]
      \item $\{v_1, v_2, \dots, v_n\}$ é L.I.

      \vspace{.3cm}

      \item $V = Span(v_1, v_2, \dots, v_n)$,  ou seja, $\{v_1, v_2, \dots, v_n\}$ gera $V$.
    \end{enumerate}
  \end{definicao}
\end{frame}


\begin{frame}
  \begin{teorema}
    Seja $V$ um espaço vetorial sobre $\cp{K}$.  Se $\{v_1, v_2, \dots, v_n\}$ é uma base de $V$,  então todo conjunto de vetores de $V$  com mais de $n$ elementos  é L.D.
  \end{teorema}

  \begin{corolario}
    Dado um $\cp{K}$-espaço vetorial $V$,  então quaisquer duas bases de $V$ possuem o mesmo número de vetores.
  \end{corolario}
\end{frame}

\begin{frame}
  \begin{definicao}
    Seja $V$ um espaço vetorial sobre $\cp{K}$.  A \textbf{dimensão} de $V$  é o número de vetores de uma base de $V$,  quando essa base é constituída de uma quantidade finita de vetores.
    Nesse caso vamos denotar esse número por $\dim_\cp{K}V$  ou simplesmente $\dim V$.
  \end{definicao}

  \begin{observacoes}
    \begin{enumerate}[label={\roman*})]
      \item Se $V$ não possui base,  então $\dim V = 0$.

      \item Se $V$ possui uma base com infinitos vetores,  então $\dim V = \infty$.
    \end{enumerate}
  \end{observacoes}
\end{frame}

\begin{frame}
  \begin{teorema}
    Seja $V$ um espaço vetorial sobre um corpo $\cp{K}$  com $\dim_\cp{K}V = n$.  Se o conjunto $\{v_1, v_2, \dots, v_n\}$  é linearmente independente,  então o conjunto $\{v_1, v_2, \dots, v_n\}$ é uma base de $V$.
  \end{teorema}
\end{frame}

\begin{frame}
  \begin{teorema}
    Seja $V$ um $\cp{K}$-espaço vetorial  e $\mathcal{B} = \{v_1, v_2, \dots, v_n\}$ uma base de $V$.  Então todo vetor $u \in V$  pode ser escrito, de forma única,  como uma combinação linear dos vetores de $\mathcal{B}$.
  \end{teorema}
\end{frame}

\begin{frame}
  \begin{definicao}
    Sejam $V$ um $\cp{K}$-espaço vetorial  e $\mathcal{B} = \{v_1, v_2, \dots, v_n\}$ uma base de $V$,  isto é, $\dim_\cp{K}V = n$.  Para cada vetor $u \in V$  podemos escrever
    \[
      u = \alpha_1v_1 + \alpha_2v_2 + \cdots + \alpha_nv_n
    \]
    para únicos escalares  $\alpha_1$, $\alpha_2$, \dots, $\alpha_n \in \cp{K}$.  Tais escalares são chamados de \textbf{coordenadas de $v$ com respeito à base $\mathcal{B}$}.
  \end{definicao}
\end{frame}

\begin{frame}
  \begin{notacao}
    Se $u = \alpha_1v_1 + \alpha_2v_2 + \cdots + \alpha_nv_n$,  então vamos denotar as coordenadas de $u$  em relação à base $\mathcal{B}$ por
    \[
      u_\mathcal{B} = (\alpha_1, \alpha_2, \dots, \alpha_n)_\mathcal{B}
    \]
    ou por
    \[
      [u]_\mathcal{B} = \begin{bmatrix}\alpha_1\\\alpha_2\\ \vdots\\ \alpha_n\end{bmatrix}_\mathcal{B}
    \]

    e vamos dizer que esse vetor  é o \textbf{vetor das coordenadas de $u$ com respeito à base $\mathcal{B}$}.
  \end{notacao}
\end{frame}

\begin{frame}
  \begin{teorema}
    Seja $\mathcal{B} = \{v_1, v_2, \dots, v_n\}$ uma base  para um $\cp{K}$-espaço vetorial $V$.  Dados vetores $u$, $w \in V$,  e $\lambda \in \cp{K}$ um escalar, temos:
    \begin{enumerate}[label={\roman*})]
      \vspace{.2cm}

      \item $[u + w]_\mathcal{B}  = [u]_\mathcal{B}  + [w]_\mathcal{B}$

      \vspace{1cm}

      \item $[\lambda u]_\mathcal{B}  = \lambda[u]_\mathcal{B}$
    \end{enumerate}
  \end{teorema}
\end{frame}

\begin{frame}
  \begin{teorema}
    Seja $\mathcal{B} = \{v_1, v_2, \dots, v_n\}$ uma base  para um $\cp{K}$-espaço vetorial $V$.  Dados vetores $u_1$, \dots, $u_k \in V$,  então $\{u_1, \dots, u_k\}$ é linearmente independente  se, e somente se,  $\{[u_1]_\mathcal{B}, \dots, [u_k]_\mathcal{B}\}$  é linearmente independente em $\cp{K}^n$.
  \end{teorema}
\end{frame}

\begin{frame}
  \begin{teorema}
    Seja $\mathcal{B} = \{v_1, \dots, v_n\}$ uma base  para um espaço vetorial $V$ sobre um corpo $\cp{K}$.
    \begin{enumerate}[label={\roman*})]
      \vspace{.2cm}

      \item Qualquer conjunto  de $V$ com mais de $n$ vetores  é linearmente dependente.

      \vspace{1cm}

      \item Qualquer conjunto de $V$ com menos de $n$ vetores  não pode gerar $V$.
    \end{enumerate}
  \end{teorema}
\end{frame}


\begin{frame}
  \begin{teorema}
    Seja $V$ um espaço vetorial sobre $\cp{K}$  com $\dim_\cp{K} V = n$.  Então:
    \begin{enumerate}[label={\roman*})]
      \vspace{.2cm}

      \item Qualquer conjunto linearmente independente em $V$  contém no máximo $n$ vetores.

      \vspace{1cm}

      \item Qualquer conjunto gerador de $V$  contém no mínimo $n$ vetores.

      \vspace{1cm}

      \item Qualquer conjunto linearmente independente em $V$  contendo exatamente $n$ vetores  é uma base de $V$.

      \seti

     \end{enumerate}
  \end{teorema}
\end{frame}

\begin{frame}
  \begin{teorema}
    \begin{enumerate}[label={\roman*})]
      \conti

      \vspace{.2cm}

      \item Qualquer conjunto gerador de $V$  com exatamente $n$ vetores,  é uma base de $V$.

      \vspace{1cm}

      \item Qualquer conjunto linearmente independente em $V$  pode ser estendido para uma base de $V$.

      \vspace{1cm}

      \item Qualquer conjunto gerador de $V$  pode ser reduzido a uma base de $V$.
    \end{enumerate}
  \end{teorema}
\end{frame}

\begin{frame}
  \begin{teorema}
    Seja $W$ um subespaço  de um espaço vetorial $V$  de dimensão finita.  Então:
    \begin{enumerate}[label={\roman*})]
      \vspace{.2cm}

      \item $W$ é de dimensão finita  e $\dim W \le \dim V$.

      \vspace{1cm}

      \item $\dim W = \dim V$  se, e somente se, $W = V$.
    \end{enumerate}
  \end{teorema}
\end{frame}

\begin{frame}
	Se mudarmos a base de um espaço vetorial $V$ de alguma base velha $\mathcal{B} = \{u_1, u_2, \dots, u_n\}$ para uma base nova $\mathcal{C} = \{w_1, w_2, \dots, w_n\}$, então, dado qualquer vetor $v \in V$, o velho vetor de coordenadas $[v]_\mathcal{B}$ está relacionado com o novo vetor de coordenadas $[v]_\mathcal{C}$ pela equação
	\begin{equation}
		[v]_\mathcal{B} = P[v]_\mathcal{C}
	\end{equation}
	onde as colunas de $P$ são os vetores de coordenadas dos vetores da base nova em relação à base velha; ou seja, os vetores coluna de $P$ são
	\begin{equation}
		[w_1]_\mathcal{B}, [w_2]_\mathcal{B}, \dots, [w_n]_\mathcal{B}
	\end{equation}
\end{frame}

\begin{frame}
	\begin{teorema}
		Se $P$ for a matriz de transição de uma base $\mathcal{B}$ para uma base $\mathcal{C}$ de um espaço vetorial $V$ de dimensão finita, então $P$ é invertível e $P^{-1}$ é a matriz de transição de $\mathcal{C}$ para $\mathcal{B}$.
	\end{teorema}
\end{frame}

\begin{frame}
	\begin{tcolorbox}[colback=green!30, colframe=green!80!blue, title=Procedimento para calcular $P_{\mathcal{B} \to \mathcal{C}}$]
			\begin{tabular}{ll}
				Passo 1. & Montamos a matriz $[\mathcal{C} | \mathcal{B}]$.\\
				\\
				Passo 2. & Reduzimos a matriz do Passo 1 à forma escalonada\\
				& reduzida usando operações elementares com as linhas.\\
				\\
				Passo 3. & A matriz resultante é$[I | P_{\mathcal{B} \to \mathcal{C}}]$.\\
				\\
				Passo 4. & Extraímos a matriz $P_{\mathcal{B} \to \mathcal{C}}$ do lado direito da matriz do\\
				& Passo 3.
			\end{tabular}
	\end{tcolorbox}
\end{frame}

\begin{frame}
	\begin{teorema}
		Sejam $\mathcal{C} = \{u_1, u_2, \dots, u_n\}$ uma base qualquer do espaço vetorial $\real^n$ e $\mathcal{B} = \{e_1, e_2, \dots, e_n\}$ a base canônica de $\real^n$. Se os vetores dessas bases forem escritos em forma de colunas, então
		\[
			P_{\mathcal{C} \to \mathcal{B}} = [u_1 | u_2 | \dots | u_n].
		\]
	\end{teorema}
\end{frame}

\end{document}
