%!TEX program = xelatex
%!TEX encoding = UTF-8
\def\ano{2023}
\def\semestre{2}
\def\disciplina{Introdução à Álgebra Linear}
\def\turma{11}
\def\autor{José Antônio O. Freitas}
\def\instituto{MAT-UnB}

\documentclass{beamer}
\usetheme{Madrid}
\usecolortheme{beaver}
% \mode<presentation>
\usepackage{caption}
\usepackage{amssymb}
\usepackage{amsmath,amsfonts,amsthm,amstext}
\usepackage[brazil]{babel}
\usepackage{graphicx}
\graphicspath{{../Pictures/}}
\usepackage{enumitem}
\usepackage{multicol}
\usepackage{answers}
\usepackage[svgnames]{xcolor}
\usepackage{tikz}
\usepackage{ifthen}
\usetikzlibrary{lindenmayersystems}
\usetikzlibrary[shadings]

\newcounter{exercicios}
\setcounter{exercicios}{0}
\newcommand{\questao}{
    \addtocounter{exercicios}{1}
    \noindent{\bf Quest{\~a}o \arabic{exercicios}: }}

\newcommand{\resp}[1]{
    \noindent{\bf Exerc{\'\i}cio #1: }}

\extrafootheight[.25in]{.5in}
\footrule
\lfoot{Teste \numeroteste\ - Módulo \modulo\ - \nomeabreviado\ - Turma \turma\ - \semestre$^o$/\ano}
\cfoot{}
\rfoot{P\'agina \thepage\ de \numpages}
\def\ano{2023}
\def\semestre{2}
\def\disciplina{Introdução à Álgebra Linear}
\def\nomeabreviado{IAL}
\def\turma{11}

\newcommand{\im}{{\rm Im\,}}
\newcommand{\dlim}[2]{\displaystyle\lim_{#1\rightarrow #2}}
\newcommand{\minf}{+\infty}
\newcommand{\ninf}{-\infty}
\newcommand{\cp}[1]{\mathbb{#1}}
\newcommand{\sub}{\subseteq}
\newcommand{\n}{\mathbb{N}}
\newcommand{\z}{\mathbb{Z}}
\newcommand{\rac}{\mathbb{Q}}
\newcommand{\real}{\mathbb{R}}
\newcommand{\complex}{\mathbb{C}}

\newcommand{\vesp}[1]{\vspace{ #1  cm}}

\newcommand{\compcent}[1]{\vcenter{\hbox{$#1\circ$}}}
\newcommand{\comp}{\mathbin{\mathchoice
        {\compcent\scriptstyle}{\compcent\scriptstyle}
        {\compcent\scriptscriptstyle}{\compcent\scriptscriptstyle}}}
\renewcommand{\sin}{{\rm sen\,}}
\renewcommand{\tan}{{\rm tg\,}}
\renewcommand{\csc}{{\rm cossec\,}}
\renewcommand{\cot}{{\rm cotg\,}}
\renewcommand{\sinh}{{\rm senh\,}}

\title{Diagonalização}
\author[\autor]{\autor}
\institute[\instituto]{\instituto}
\date{}

\begin{document}
    \begin{frame}
        \maketitle
    \end{frame}

    \logo{\includegraphics[scale=.1]{logo-MAT.png}\vspace*{8.5cm}}

    \begin{frame}
        Considere o operador linear $T \colon \real^3 \to \real^3$ dado por
        \begin{align*}
            T(x, y, z) = (x + 2y - z, -2x -3y + z, 2x + 2y - 2z).
        \end{align*}

        Quando consideramos a base ordenada canônica de $\real^3$, $\mathcal{B} = \{e_1 = (1, 0, 0); e_2 = (0, 1, 0); e_3 = (0, 0, 1)\}$, a matriz desse operador com respeito à essa base é
        \[
        [T]_\mathcal{B} = \begin{bmatrix}
            \phantom{-}1 & \phantom{-}2 & -1\\
            -2 & -3 & \phantom{-}1\\
            \phantom{-}2 & \phantom{-}2 & \phantom{-}2
        \end{bmatrix}.
        \]
    \end{frame}

    \begin{frame}
        Agora, se tomamos a base ordenada $\mathcal{B}_1 = \{w_1 = (1, 0, 2); w_2 = (0, 1, 2); w_3 = (1, -1, 1)\}$ a matriz de $T$ com respeito à essa base é
        \[
        [T]_{\mathcal{B}_1} = \begin{bmatrix}
            -1 & \phantom{-}0 & \phantom{-}0\\
            \phantom{-}0& -1 & \phantom{-}0\\
            \phantom{-}0& \phantom{-}0 & -2
        \end{bmatrix}.
        \]

        Note que para a base $\mathcal{B}_1$ temos
        \begin{align*}
            T(w_1) &= -1w_1\\
            T(w_2) &= -1w_2\\
            T(w_3) &= -2w_3\\
        \end{align*}
    \end{frame}

    \begin{frame}
        No caso geral, seja $T : V \to V$ um operador linear e suponha que exista uma base $\mathcal{B} = \{v_1,\dots,v_n\}$ de $V$ tal que
        \begin{align}\label{formadiagonal}
            [T]_\mathcal{B} = \begin{bmatrix}
                \lambda_1 & 0 & 0 & \dots & 0\\
                0 & \lambda_2 & 0 & \dots & 0\\
                \vdots & & \ddots & & \vdots\\
                0 & 0 & 0 & \dots & \lambda_n
            \end{bmatrix}
        \end{align}
        com $\lambda_i \in \cp{K}$ para $i = 1$, \dots, $n$.
    \end{frame}

    \begin{frame}
        Assim
        \[
        [T(v_i)]_\mathcal{B} = [T]_\mathcal{B}[v_i]_\mathcal{B} = [T]_\mathcal{B}\begin{bmatrix}
            0\\
            0\\
            \vdots\\
            0\\
            1\\
            0\\
            \vdots\\
            0
        \end{bmatrix}_\mathcal{B} = \begin{bmatrix}
            0\\
            0\\
            \vdots\\
            0\\
            \lambda_i\\
            0\\
            \vdots\\
            0
        \end{bmatrix}_\mathcal{B}
        \]
        para $i = 1$, \dots, $n$. Isto é,
        \[
        T(v_1) = \lambda_1 v_1,\ T(v_2) = \lambda_2 v_2,\dots, T(v_n) = \lambda_n v_n.
        \]
    \end{frame}

    \begin{frame}
    \begin{definicao}
        Seja $T : V \to V$ um operador linear.
        \begin{enumerate}[label={\roman*})]
            \item Um \textbf{autovalor} de $T$ é um elemento $\lambda \in \cp{K}$ tal que existe um vetor não nulo $u \in V$ com $T(u) = \lambda u$.\index{Autovalor}
            \item Se $\lambda$ é um autovalor de $T$, então todo vetor não nulo $u \in V$ tal que
            \[
            T(u) = \lambda u
            \]
            é chamado de \textbf{autovetor} de $T$ \textbf{associado} ao autovalor $\lambda$. Denotaremos por $Aut_T(\lambda)$ o subespaço gerado por todos os autovetores associados a $\lambda$. Assim
            \[
            \aut_T(\lambda) = \{u \in V \mid T(u) = \lambda u\}.
            \]
            \item Suponha que $\dim V = n < \infty$. Dizemos que $T$ é \textbf{diagonalizável} se existir uma base $\mathcal{B}$ de $V$ tal que $[T]_\mathcal{B}$ é diagonal, isto é, tem a forma \eqref{formadiagonal}. Tal fato equivale a dizer que existe uma base formada por autovetores.
        \end{enumerate}
    \end{definicao}
    \end{frame}
\end{document}
