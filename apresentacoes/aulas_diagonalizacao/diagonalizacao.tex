%!TEX program = xelatex
%!TEX encoding = UTF-8
\def\ano{2023}
\def\semestre{2}
\def\disciplina{Introdução à Álgebra Linear}
\def\turma{11}
\def\autor{José Antônio O. Freitas}
\def\instituto{MAT-UnB}

\documentclass{beamer}
\usetheme{Madrid}
\usecolortheme{beaver}
% \mode<presentation>
\usepackage{caption}
\usepackage{amssymb}
\usepackage{amsmath,amsfonts,amsthm,amstext}
\usepackage[brazil]{babel}
\usepackage{graphicx}
\graphicspath{{../Pictures/}}
\usepackage{enumitem}
\usepackage{multicol}
\usepackage{answers}
\usepackage[svgnames]{xcolor}
\usepackage{tikz}
\usepackage{ifthen}
\usetikzlibrary{lindenmayersystems}
\usetikzlibrary[shadings]

\newcounter{exercicios}
\setcounter{exercicios}{0}
\newcommand{\questao}{
    \addtocounter{exercicios}{1}
    \noindent{\bf Quest{\~a}o \arabic{exercicios}: }}

\newcommand{\resp}[1]{
    \noindent{\bf Exerc{\'\i}cio #1: }}

\extrafootheight[.25in]{.5in}
\footrule
\lfoot{Teste \numeroteste\ - Módulo \modulo\ - \nomeabreviado\ - Turma \turma\ - \semestre$^o$/\ano}
\cfoot{}
\rfoot{P\'agina \thepage\ de \numpages}
\def\ano{2023}
\def\semestre{2}
\def\disciplina{Introdução à Álgebra Linear}
\def\nomeabreviado{IAL}
\def\turma{11}

\newcommand{\im}{{\rm Im\,}}
\newcommand{\dlim}[2]{\displaystyle\lim_{#1\rightarrow #2}}
\newcommand{\minf}{+\infty}
\newcommand{\ninf}{-\infty}
\newcommand{\cp}[1]{\mathbb{#1}}
\newcommand{\sub}{\subseteq}
\newcommand{\n}{\mathbb{N}}
\newcommand{\z}{\mathbb{Z}}
\newcommand{\rac}{\mathbb{Q}}
\newcommand{\real}{\mathbb{R}}
\newcommand{\complex}{\mathbb{C}}

\newcommand{\vesp}[1]{\vspace{ #1  cm}}

\newcommand{\compcent}[1]{\vcenter{\hbox{$#1\circ$}}}
\newcommand{\comp}{\mathbin{\mathchoice
        {\compcent\scriptstyle}{\compcent\scriptstyle}
        {\compcent\scriptscriptstyle}{\compcent\scriptscriptstyle}}}
\renewcommand{\sin}{{\rm sen\,}}
\renewcommand{\tan}{{\rm tg\,}}
\renewcommand{\csc}{{\rm cossec\,}}
\renewcommand{\cot}{{\rm cotg\,}}
\renewcommand{\sinh}{{\rm senh\,}}

\title{Diagonalização}
\author[\autor]{\autor}
\institute[\instituto]{\instituto}
\date{}

\begin{document}
    \begin{frame}
        \maketitle
    \end{frame}

    \logo{\includegraphics[scale=.1]{logo-MAT.png}\vspace*{8.5cm}}

    \begin{frame}
        Considere o operador linear $T \colon \real^3 \to \real^3$ dado por
        \begin{align*}
            T(x, y, z) = (x + 2y - z, -2x -3y + z, 2x + 2y - 2z).
        \end{align*}

        Quando consideramos a base ordenada canônica de $\real^3$, $\mathcal{B} = \{e_1 = (1, 0, 0); e_2 = (0, 1, 0); e_3 = (0, 0, 1)\}$, a matriz desse operador com respeito à essa base é
        \[
        [T]_\mathcal{B} = \begin{bmatrix}
            \phantom{-}1 & \phantom{-}2 & -1\\
            -2 & -3 & \phantom{-}1\\
            \phantom{-}2 & \phantom{-}2 & \phantom{-}2
        \end{bmatrix}.
        \]
    \end{frame}

    \begin{frame}
        Agora, se tomamos a base ordenada $\mathcal{B}_1 = \{w_1 = (1, 0, 2); w_2 = (0, 1, 2); w_3 = (1, -1, 1)\}$ a matriz de $T$ com respeito à essa base é
        \[
        [T]_{\mathcal{B}_1} = \begin{bmatrix}
            -1 & \phantom{-}0 & \phantom{-}0\\
            \phantom{-}0& -1 & \phantom{-}0\\
            \phantom{-}0& \phantom{-}0 & -2
        \end{bmatrix}.
        \]

        Note que para a base $\mathcal{B}_1$ temos
        \begin{align*}
            T(w_1) &= -1w_1\\
            T(w_2) &= -1w_2\\
            T(w_3) &= -2w_3\\
        \end{align*}
    \end{frame}

    \begin{frame}
        No caso geral, seja $T : V \to V$ um operador linear e suponha que exista uma base $\mathcal{B} = \{v_1,\dots,v_n\}$ de $V$ tal que
        \begin{align}\label{formadiagonal}
            [T]_\mathcal{B} = \begin{bmatrix}
                \lambda_1 & 0 & 0 & \dots & 0\\
                0 & \lambda_2 & 0 & \dots & 0\\
                \vdots & & \ddots & & \vdots\\
                0 & 0 & 0 & \dots & \lambda_n
            \end{bmatrix}
        \end{align}
        com $\lambda_i \in \cp{K}$ para $i = 1$, \dots, $n$.
    \end{frame}

    \begin{frame}
        Assim
        \[
        [T(v_i)]_\mathcal{B} = [T]_\mathcal{B}[v_i]_\mathcal{B} = [T]_\mathcal{B}\begin{bmatrix}
            0\\
            0\\
            \vdots\\
            0\\
            1\\
            0\\
            \vdots\\
            0
        \end{bmatrix}_\mathcal{B} = \begin{bmatrix}
            0\\
            0\\
            \vdots\\
            0\\
            \lambda_i\\
            0\\
            \vdots\\
            0
        \end{bmatrix}_\mathcal{B}
        \]
        para $i = 1$, \dots, $n$. Isto é,
        \[
        T(v_1) = \lambda_1 v_1,\ T(v_2) = \lambda_2 v_2,\dots, T(v_n) = \lambda_n v_n.
        \]
    \end{frame}

    \begin{frame}
    \begin{definicao}
        Seja $T : V \to V$ um operador linear.
        \begin{enumerate}[label={\roman*})]
            \item Um \textbf{autovalor} de $T$ é um elemento $\lambda \in \cp{K}$ tal que existe um vetor não nulo $u \in V$ com $T(u) = \lambda u$.\index{Autovalor}
            \item Se $\lambda$ é um autovalor de $T$, então todo vetor não nulo $u \in V$ tal que
            \[
            T(u) = \lambda u
            \]
            é chamado de \textbf{autovetor} de $T$ \textbf{associado} ao autovalor $\lambda$. Denotaremos por $Aut_T(\lambda)$ o subespaço gerado por todos os autovetores associados a $\lambda$. Assim
            \[
            \aut_T(\lambda) = \{u \in V \mid T(u) = \lambda u\}.
            \]
            \item Suponha que $\dim V = n < \infty$. Dizemos que $T$ é \textbf{diagonalizável} se existir uma base $\mathcal{B}$ de $V$ tal que $[T]_\mathcal{B}$ é diagonal, isto é, tem a forma \eqref{formadiagonal}. Tal fato equivale a dizer que existe uma base formada por autovetores.
        \end{enumerate}
    \end{definicao}
    \end{frame}

    \begin{frame}
        Temos aqui duas questões importantes:
        \begin{enumerate}[label={\arabic*})]
            \item Dado um operador $T \colon V \to V$, existem autovetores para $T$?

            \item Se sim, existe um número suficiente de autovetores para forma uma base de $V$?
        \end{enumerate}
    \end{frame}

    \begin{frame}
        Aqui surgem duas questões:
        \begin{enumerate}[label={\arabic*})]
            \item Como determinar se uma transformação linear $T \colon V \to V$ possui autovetores?

            \item A existência de autovetores de $T$ depende da base escolhida para $V$?
        \end{enumerate}
    \end{frame}

    \begin{frame}
        \begin{teorema}
            As seguintes afirmações são equivalentes:
            \begin{enumerate}[label={\roman*})]
                \item o escalar $\lambda \in \cp{K}$ é um autovalor do operador linear $T \colon V \to V$;

                \item o operador $T - \lambda Id$ não é injetor;

                \item dada uma base ordenada $\mathcal{B}$ qualquer de $V$ temos
                \[
                \det([T]_\mathcal{B} - \lambda[Id]_\mathcal{B}) = 0.
                \]
            \end{enumerate}
        \end{teorema}
    \end{frame}

    \begin{frame}
        \begin{teorema}
            Se $A \in M_n(\cp{K})$ for uma matriz $n \times n$, então $\lambda \in \cp{K}$ é um autovalor de $A$ se, e somente se, $\lambda$ satisfaz a equação
            \[
            \det(A - \lambda I_n) = 0.
            \]
            Essa equação é chamada de \textbf{equação característica} de $A$.
        \end{teorema}
    \end{frame}

    \begin{frame}
        Seja $A \in M_n(\cp{K})$ onde $n \ge 1$. Vamos definir o \textbf{determinante da matriz $A$}, denotado por $\det(A)$, de modo indutivo.

        Se $n = 1$, então a matriz $A \in \cp{M}_1(\cp{K})$ é da forma
        \[
        A = (a_{11})
        \]
        e neste caso definimos
        \[
        \det(A) = a_{11} \in \cp{K}.
        \]
    \end{frame}

\begin{frame}
        Suponha que $n > 1$ e que $\det(B)$ esteja definido para todas as matrizes $B \in \cp{M}_p(\cp{K})$ com $p < n$ e seja $A \in \cp{M}_n(\cp{K})$. Para cada $(i,j)$, defina a matriz $A_{ij}$ formada a partir de $A$ retirando-se a sua $i$-ésima linha e a sua $j$-ésima coluna. É claro que $A \in \cp{M}_{n - 1}(\cp{K})$ e portanto $\det(A_{ij})$ está definido. Defina então
        \begin{align*}
            \det(A) &= \sum_{j = 1}^n(-1)^{i + j}a_{ij}\det(A_{ij}) \\ &= (-1)^{i + 1}a_{i1}\det(A_{i1}) + (-1)^{i + 2}a_{i2}\det(A_{i2}) \\ & + (-1)^{i + 3}a_{i3}\det(A_{i3}) + \cdots + (-1)^{i + n}a_{in}\det(A_{in}) \in \cp{K}
        \end{align*}
    \end{frame}

    \begin{frame}
        \begin{proposicao}
            Sejam $A$, $B \in \cp{M}_n(\cp{K})$ e $\lambda \in \cp{K}$. Temos:
            \begin{enumerate}[label={\roman*})]
                \item $\det(AB) = \det(A) \det(B)$,
                \item $\det(\lambda A) = \lambda^n \det(A)$,
                \item Se $A$ é invertível, então $\det(A^{-1}) = [(\det(A)]^{-1}$.
            \end{enumerate}
        \end{proposicao}
    \end{frame}

    \begin{frame}
        \begin{proposicao}
            Seja $A$ uma matriz $n \times n$ com entradas num corpo $\cp{K}$.
            \begin{enumerate}[label={\roman*})]
                \item Se $B$ é a matriz resultante da permutação de duas linhas de $A$, então $\det (B) = -\det (A)$.
                \item Se $B$ é a matriz resultante da multiplicação de uma linha de $A$ por um escalar não nulo $\alpha \in \cp{K}$, então $\det(B) = \alpha\det(A)$.
                \item Se $B$ é a matriz resultante da soma da linha $i$ de $A$ com um múltiplo não nulo $\alpha \in \cp{K}$ da linha $j$ de $A$, então $\det(B) = \det(A)$.
            \end{enumerate}
        \end{proposicao}
    \end{frame}

    \begin{frame}
        \begin{teorema}
            Uma matriz $A \in \cp{M}_n(\cp{K})$ é invertível se, e somente se, $\det(A) \ne 0$.
        \end{teorema}
    \end{frame}

    \begin{frame}
        Dado $V$ um espaço vetorial sobre $\cp{K}$, com $\dim_\cp{K} V = n < \infty$ e um operador linear $T \colon V \to V$, queremos tentar encontrar, se existir, uma base ordenada $\mathcal{D}$ de $V$ formada por autovetores. Para isso devemos proceder da seguinte maneira:
        \begin{enumerate}[label={\arabic*})]
            \item Se não foi dada, encontre a matriz de $T$ com relação a alguma base ordenada $\mathcal{B}$ de $V$. Pode-se usar a base canônica $\mathcal{B}$ de $V$ para facilitar o cálculo de $[T]_\mathcal{B}$.
            \seti
        \end{enumerate}
    \end{frame}

    \begin{frame}
        \begin{enumerate}[label={\arabic*})]
            \conti
            \item Uma vez encontrada $[T]_\mathcal{B}$ calcule
            \[
            [T]_\mathcal{B} - \lambda I_n = [T]_\mathcal{B} - \begin{bmatrix}\lambda & 0 & 0 & \cdots & 0\\0 & \lambda & 0 & \cdots & 0\\0 & 0 & \lambda & \cdots & 0\\\vdots & \vdots & \vdots & \cdots & \vdots\\0 & 0 & 0 & \cdots & \lambda\end{bmatrix}.
            \]

            \item\label{passo4diag} Determine todos os valores, se existem, de $\lambda \in \cp{K}$ tais que
            \[
            \det([T]_\mathcal{B} - \lambda I_n) = 0.
            \]
            \seti
        \end{enumerate}
    \end{frame}

    \begin{frame}
        \begin{enumerate}[label={\arabic*})]
            \conti
            \item Se não existir $\lambda \in \cp{K}$ no passo anterior, então $T$ não é diagonalizável.

            \item Para cada $\lambda_1$, $\lambda_2$, \dots, $\lambda_r$ encontrando em \ref{passo4diag} resolva o sistema linear homogêneo
            \[
            ([T]_\mathcal{B} - \lambda_jI_n)X = 0.
            \]
            O subespaço $Aut_T(\lambda_j)$ será dada pelas soluções desse sistema.
            \seti
        \end{enumerate}
    \end{frame}

\begin{frame}
        \begin{enumerate}[label={\arabic*})]
        \conti
            \item Encontre uma base para cada $Aut_T(\lambda_j)$, com $j = 1$, 2, \dots, $r$.

            \item Se o conjunto $\mathcal{D}$ resultante a união de todas as bases encontradas no passo anterior possuir $\dim_\cp{K}V$ elementos, então $\mathcal{D}$ será uma base de $V$ tal que a matriz de $T$ com relação à essa base será diagonal. Se $\mathcal{D}$ possuir menos elementos que $\dim_\cp{K}V$ então $T$ não é diagonalizável.
        \end{enumerate}
    \end{frame}
\end{document}
