%!TEX program = xelatex
%!TEX root = IAL.tex

\chapter{Espaços Vetoriais}

Em todo este cap{\'\i}tulo $\cp{K}$ denotar\'a um corpo.

\begin{definicao}
	Um conjunto	n\~ao vazio $V$ \'e um \textbf{espa\c{c}o vetorial}\index{Espaço Vetorial} sobre um corpo $\cp{K}$ se em seus elementos, 
	chamados \textbf{vetores}\index{Espaço Vetorial!vetores}, estiverem definidas duas opera\c{c}\~oes satisfazendo:
	\begin{itemize}
		\item[A)] A cada par $u$, $w \in V$ corresponde um vetor $u + w \in V$, chamado \textbf{soma} de $u$ e $w$, de modo que:
		\item[A1)] $u + w = w + u$, para todos $u$, $w \in V$;
		\item[A2)] $(u + w) + x = u + (w + x)$, para todos $u$, $w$, $x \in V$;
		\item[A3)] Existe em $V$ um vetor, denominado \textbf{vetor nulo}\index{Espaço Vetorial!vetor nulo} e denotado por $0_V$, tal que
		\[
			0_V + u = u
		\]
		para todo $u \in V$.
	\item[A4)] Para cada vetor $u \in V$, existe um vetor em $V$, denotado por $-u$ e chamado de \textbf{vetor oposto},\index{Espaço Vetorial!vetor oposto} tal que
		\[
			u + (-u) = 0_V.
		\]
	\item[M)] A cada par $\alpha \in \cp{K}$ e $u \in V$, corresponde um vetor $\alpha \cdot u \in V$, denominado \textbf{produto por escalar}\index{Espaço Vetorial!produto por escalar} de $\alpha$ por $u$ de modo que:
		\item[M1)] $(\alpha\beta)\cdot u = \alpha(\beta\cdot u)$ para todos $\alpha$, $\beta \in \cp{K}$ e todo $u \in V$;
		\item[M2)] $1_\cp{K}\cdot u = u$ para todo $u \in V$, onde $1_\cp{K}$ \'e o elemento neutro da multiplica\c{c}\~ao em $\cp{K}$.
		\item[D1)] $\alpha\cdot(u + w) = \alpha\cdot u + \alpha\cdot w$, para todo $\alpha \in \cp{K}$ e todos $u$, $w \in V$;
		\item[D2)] $(\alpha + \beta)\cdot u = \alpha\cdot u + \beta\cdot u$, para todos $\alpha$, $\beta \in \cp{K}$ e todo $u \in V$.
	\end{itemize}
\end{definicao}

\begin{observacao}
	Vamos usar a express\~ao $\cp{K}$-\textbf{espa\c{c}o vetorial} para nos referir a um espa\c{c}o vetorial $V$ sobre um corpo $\cp{K}$.
\end{observacao}

\begin{exemplo}
	\begin{enumerate}[label={\arabic*})]
		\item Seja $\cp{K} = \real$ um corpo. Considere o conjunto
			\[
			\real^3 = \underbrace{\real \times \real \times \real}_{3 vezes} = \{(a_1, a_2, a_3) \mid a_i \in \real, i =1, 2, \dots, n\}
			\]
			e defina
			\begin{itemize}
				\item $(a_1, a_2, a_3) + (b_1, b_2, b_3) = (a_1 + b_1, a_2 + b_2, a_3 + b_3)$ para todos $(a_1, a_2, a_3)$, $(b_1, b_2, b_3) \in \real^3$;
				\item $\alpha (a_1, a_2, a_3) = (\alpha a_1, \alpha a_2, \alpha a_3)$ para todo $\alpha \in \real$ e todo $(a_1, a_2, a_3) \in \real^3$.
				Com estas opera\c{c}\~oes $\real^3$ \'e um espa\c{c}o vetorial sobre $\real$.
			\end{itemize}
			\begin{solucao}
				De fato,
				\begin{enumerate}
					\item[A1)] $(a_1, a_2, a_3) + (b_1, b_2, b_3) = (a_1 + b_1, a_2 + b_2, a_3 + b_3) = (b_1 + a_1, b_2 + a_2, b_3 + a_3) = (b_1, b_2, b_3) + (a_1, a_2, a_3)$, para todos $(a_1, a_2, a_3)$, $(b_1, b_2, b_3) \in \real^3$.

					\item[A2)] $[(a_1, a_2, a_3) + (b_1, b_2, b_3)] + (c_1, c_2, c_3) = (a_1 + b_1, a_2 + b_2, a_3 + b_3) + (c_1, c_2, c_3) = ((a_1 + b_1) + c_1, (a_2 + b_2) + c_2, (a_3 + b_3) + c_3) = (a_1 + (b_1 + c_1), a_2 + (b_2 + c_2), a_3 + (b_3 + c_3)) = (a_1, a_2, a_3) + (b_1 + c_1, b_2 + c_2, b_3 + c_3) = (a_1, a_2, a_3) + [(b_1, b_2, b_3) + (c_1, c_2, c_3)]$, para todos $(a_1, a_2, a_3)$, $(b_1, b_2, b_3)$, $(c_1, c_2, c_3) \in \real^3$.
					
					\item[A3)] Tome $0_V = (0, 0, 0)$. Ent\~ao para todo $(a_1, a_2, a_3) \in \real^3$ temos
					\[
						(a_1, a_2, a_3) + 0_V = (a_1, a_2, a_3) + (0, 0, 0) = (a_1 + 0, a_2 + 0, a_3 + 0) = (a_1, a_2, a_3).
					\]
					Logo $0_V = (0, 0, 0)$ \'e o vetor nulo de $\real^3$.
				
					\item[A4)] Dado $u = (a_1, a_2, a_3) \in \real^3$, tome $-u = (-a_1, -a_2, -a_3) \in \real^3$. Assim
					\[
						u + (-u) = (a_1, a_2, a_3) + (-a_1, -a_2, -a_3) = (a_1 - a_1, a_2 - a_2, a_3 - a_3) = (0, 0, 0) = 0_V.
					\]
					Logo $-u = (-a_1, -a_2, -a-3)$ é o vetor oposto de $u = (a_1, a_2, a_3)$ em $\real^3$.

					\item[M1)] Dados $\alpha$, $\beta \in \real$ e $u = (a_1, a_2, a_3) \in \real^3$ temos:
						\begin{align*}
							(\alpha\beta)\cdot u &= (\alpha \beta)\cdot (a_1, a_2, a_3) = ((\alpha\beta)a_1, (\alpha\beta)a_2, (\alpha\beta)a_3) \\ &= (\alpha(\beta a_1), \alpha(\beta a_2), \alpha(\beta a_3)) = \alpha\cdot(\beta a_1, \beta a_2, \beta a_3)\\ & = \alpha\cdot(\beta\cdot(a_1, a_2, a_3)) = \alpha\cdot(\beta\cdot u)
						\end{align*}

					\item[M2)] $1_\real \cdot (a_1, a_2, a_3) = 1 \cdot (a_1, a_2, a_3) = (1\cdot a_1, 1\cdot a_2, 1\cdot a_3) = (a_1, a_2, a_3)$ para todo $(a_1, a_2, a_3) \in \real^3$.
					
					\item[D1)] Dados $\alpha \in \real$ e $u = (a_1, a_2, a_3)$, $w = (b_1, b_2, b_3) \in \real^3$ temos:
						\begin{align*}
							\alpha\cdot(u + w) &= \alpha\cdot[(a_1, a_2, a_3) + (b_1, b_2, b_3)] = \alpha\cdot(a_1 + b_1, a_2 + b_2, a_3 + b_3) \\ &= (\alpha(a_1 + b_2), \alpha(a_2 + b_2), \alpha(a_3 + b_3))\\ &= (\alpha a_1 + \alpha b_1, \alpha a_2 + \alpha b_2, \alpha a_3 + \alpha b_3) \\ &= (\alpha a_1, \alpha a_2, \alpha a_3) + (\alpha b_1, \alpha b_2, \alpha b_3)\\ &= \alpha\cdot(a_1, a_2, a_3) + \alpha\cdot(b_1, b_2, b_3) \\ &= \alpha\cdot u + \alpha\cdot w
						\end{align*}

					\item[D2)] Sejam $\alpha$, $\beta \in \real$ e $u = (a_1, a_2, a_3) \in \real^3$. Temos
						\begin{align*}
							(\alpha + \beta)\cdot u &= (\alpha + \beta)\cdot(u_1, u_2, u_ 3) = ((\alpha + \beta)a_1, (\alpha + \beta)a_2, (\alpha + \beta)a_3) \\ &= (\alpha a_1 + \beta a_1, \alpha a_2 + \beta a_2, \alpha a_3 + \beta a_3) \\ &= (\alpha a_1, \alpha a_2, \alpha a_3) + (\beta a_1, \beta a_2, \beta a_3) \\ &= \alpha\cdot(a_1, a_2, a_3) + \beta\cdot(a_1, a_2, a_3) \\ &= \alpha\cdot u + \beta\cdot u
						\end{align*}
			\end{enumerate}
			Portanto $V = \real^3$ \'e um $\real$-espa\c{c}o vetorial.
		\end{solucao}

		\item De modo geral, seja $\cp{K}$ um corpo. Considere o conjunto
		\[
			\cp{K}^n = \underbrace{\cp{K} \times \cp{K} \times \cdots \times \cp{K}}_{n\ vezes} = \{(a_1, a_2,\dots,a_n) \mid a_i \in \cp{K}, i =1, 2, \dots, n\}
		\]
		e defina
		\begin{itemize}
			\item $(a_1, a_2, \dots, a_n) + (b_1, b_2, \dots,b_n) = (a_1 + b_1, a_2 + b_2,\dots, a_n + b_n)$ para todos $(a_1, a_2, \dots,a_n)$ ,$(b_1, b_2, \dots,b_n) \in \cp{K}^n$;
			\item $\alpha (a_1, a_2, \dots,a_n) = (\alpha a_1, \alpha a_2, \dots, \alpha a_n)$ para todo $\alpha \in \cp{K}$ e todo $(a_1, a_2, \dots, a_n) \in \cp{K}^n$.
			Com estas opera\c{c}\~oes $\cp{K}^n$ \'e um espa\c{c}o vetorial sobre $\cp{K}$. Assim temos:
			\begin{itemize}
				\item $\rac^n$ \'e um espa\c{c}o vetorial sobre $\rac$;
				\item $\real^n$ \'e um espa\c{c}o vetorial sobre $\real$;
				\item $\complex^n$ \'e um espa\c{c}o vetorial sobre $\complex$.
			\end{itemize}
		\end{itemize}
		\item $\complex^n$ \'e um espa\c{c}o vetorial sobre $\real$ se definirmos:
		\begin{itemize}
			\item $(a_1, a_2, \dots, a_n) + (b_1, b_2, \dots,b_n) = (a_1 + b_1, a_2 + b_2,\dots, a_n + b_n)$ para todos $(a_1, a_2, \dots,a_n)$ ,$(b_1, b_2, \dots,b_n) \in \complex^n$;
			\item $\alpha (a_1, a_2, \dots,a_n) = (\alpha a_1, \alpha a_2, \dots, \alpha a_n)$ para todo $\alpha \in \real$ e todo $(a_1, a_2, \dots, a_n) \in \complex^n$.
		\end{itemize}
		\item O conjunto $\rac$ \'e um espa\c{c}o vetorial sobre $\rac$, mas n\~ao \'e um espa\c{c}o vetorial sobre $\real$. (Por qu\^e?)
		\item O conjunto das matrizes $\cp{M}_{p \times q}(\cp{K})$ com coeficientes em $\cp{K}$ \'e um $\cp{K}$-espa\c{c}o vetorial com a soma usual de matrizes e a multiplica\c{c}\~ao por escalar usual.
		\item Considere o conjunto dos polin\^omios
		\[
			\mathcal{P}(\cp{K}) = \{ p(x) = a_nx^n + a_{n - 1}x^{n - 1} + \cdots + a_1x + a_0 \mid a_i \in \cp{K}, i = 0, 1, 2, \dots, n; n \ge 0 \}.
		\]
		Dados $p(x) = a_nx^n + a_{n - 1}x^{n - 1} + \cdots + a_1x + a_0$ e $q(x) = b_mx^m + b_{m - 1}x^{m - 1} + \cdots + b_1x + b_0$ em $\mathcal{P}(\cp{K})$, suponha que $n < m$ e defina:
		\begin{itemize}
			\item $(p + q)(x) = b_mx^m + b_{m - 1}x^{m - 1} + \cdots + b_{n + 1}x^{n + 1} + (a_n + b_n)x^n + \cdots + (a_1 + b_1)x + (a_0 + b_0)$
			\item $(\alpha p)(x) = \alpha a_nx^n + \alpha a_{n - 1}x^{n - 1} + \cdots + \alpha a_1x + \alpha a_0$, onde $\alpha \in \cp{K}$.
		\end{itemize}
		Assim $\mathcal{P}(\cp{K})$ \'e um $\cp{K}$-espa\c{c}o vetorial.

		\item Seja $V = \{x \in \real \mid x > 0\} = \real_+^*$. Defina em $V$ as seguintes opera\c{c}\~oes:
		\begin{align*}
			x \oplus y &= xy\\
			\lambda \otimes x &= x^\lambda
		\end{align*}
		para todos $x$, $y \in V$ e para todo $\lambda \in \real$. Ent\~ao com essas opera\c{c}\~oes $V$ \'e um $\real$-espa\c{c}o vetorial.
		\begin{solucao}
			Temos:
			\begin{enumerate}
				\item[A1)] $x \oplus y = xy = yx = y \oplus x$, para todos $x$, $y \in V$.
				\item[A2)] $(x \oplus y) \oplus z = (xy) \oplus z = (xy)z = x(yz) = x \oplus (yz) = x \oplus (y \oplus z)$, para todos $x$, $y$, $z \in V$.
				\item[A3)] Tome $0_V = 1$. Ent\~ao para todo $x \in V$ temos
				\[
					x\oplus 0_V = x\oplus 1 = x\cdot 1 = x.
				\]
				Logo $0_V = 1$ \'e o vetor nulo de $V$.
				\item[A4)] Dado $x \in V$, tome $-x = \dfrac{1}{x} \in V$. Assim
				\[
					x \oplus (-x) = x \oplus \dfrac{1}{x} = x \cdot \dfrac{1}{x} = 1 = 0_V.
				\]
				\item[M1)] $(\alpha \beta)\otimes x = x^{\alpha \beta} = (x^\beta)^\alpha = \alpha \otimes (x^\beta) = \alpha \otimes (\beta \otimes x)$ para todos $\alpha$, $\beta \in \real$ e todo $x \in V$.
				\item[M2)] $1_\real \otimes x = 1\otimes x = x^1 = x$ para todo $x \in V$.
				\item[D1)] $\alpha \otimes (x \oplus y) = \alpha \otimes (xy) = (xy)^\alpha = (x^\alpha) (y^\alpha) = (\alpha \otimes x) (\alpha \otimes y) = (\alpha \otimes x) \oplus (\alpha \otimes y)$ para todos $x$, $y \in V$ e para todo $\alpha \in \real$.
				\item[D2)] $(\alpha + \beta) \otimes x = x^{\alpha + \beta} = x^\alpha x^\beta = (\alpha \otimes x)(\beta \otimes x) = (\alpha \otimes x) \oplus (\beta \otimes x)$, para todos $\alpha$, $\beta \in \real$ e para todo $x \in V$.
			\end{enumerate}
			Portanto $V$ \'e um $\real$-espa\c{c}o vetorial.
		\end{solucao}
		\end{enumerate}
\end{exemplo}

\begin{definicao}
	Seja $V$ um espa\c{c}o vetorial sobre $\cp{K}$.
	\begin{enumerate}[label={\roman*})]
		\item Um vetor $w \in V$ \'e uma \textbf{combina\c{c}\~ao linear}\index{Espaço Vetorial!combina\c{c}\~ao linear} dos vetores $u_1$, $u_2$, \dots, $u_n \in V$ se existirem escalares 
			$\alpha_1$, $\alpha_2$, \dots, $\alpha_n \in \cp{K}$ tais que
		\[
			w = \alpha_1 u_1 + \alpha_2u_2 + \cdots + \alpha_nu_n = \sum_{i = 1}^n \alpha_iu_i.
		\]
		\item Seja $\mathcal{B}$ um subconjunto de $V$. Dizemos que $\mathcal{B}$ \'e um \textbf{conjunto gerador}\index{Espaço Vetorial!conjunto gerador} (ou que $\mathcal{B}$ gera $V$) se todo elemento de $V$ for uma combina\c{c}\~ao linear de uma quantidade finita de elementos de $\mathcal{B}$.
	\end{enumerate}
\end{definicao}

\begin{exemplo}
	\begin{enumerate}[label={\arabic*})]
		\item O vetor $(1, 1, 1) \in \real^3$ \'e uma combina\c{c}\~ao linear dos vetores $u_1 = (1, 2, 3)$, $u_2 = (0, 1, 2)$ e $u_3 = (-1, 0, 1)$. De fato, precisamos encontrar $\alpha$, $\beta$ e $\gamma \in \real$ tais que:
		\[
				(1, 1, 1) = \alpha u_1 + \beta u_2 + \gamma u_3 = (\alpha - \gamma, 2\alpha + \beta, 3\alpha + 2\beta + \gamma).
		\]
		Assim resolvendo o sistema
		\[
		\begin{cases}
			\alpha - \gamma = 1\\
			2\alpha + \beta = 1\\
			3\alpha + 2\beta + \gamma = 1
		\end{cases}
		\]
		encontramos que suas solu\c{c}\~oes s\~ao dadas por
		\begin{align*}
			\alpha &= 1 + \gamma\\
			\beta &= -1 - 2\gamma
		\end{align*}
		onde $\gamma \in \real$. Logo o vetor $(1, 1, 1)$ \'e de fato uma combina\c{c}\~ao linear dos vetores $u_1$, $u_2$ e $u_3$. Podemos tomar, por exemplo, $\gamma = 2$ e escrever
		\[
			(1, 1, 1) = 3(1, 2, 3) - 5(0, 1, 2) + 2(-1, 0 , 1).
		\]
		\item O vetor $(1, -2, 2) \in \real^3$ n\~ao \'e uma combina\c{c}\~ao linear dos vetores $u_1 = (1, 2, 3)$, $u_2 = (0, 1, 2)$ e $u_3 = (-1, 0, 1)$. De fato, suponha que $\alpha$, $\beta$ e $\gamma \in \real$ s\~ao tais que:
		\[
			(1, -2, 2) = \alpha u_1 + \beta u_2 + \gamma u_3 = (\alpha - \gamma, 2\alpha + \beta, 3\alpha + 2\beta + \gamma).
		\]
		Assim resolvendo o sistema
		\[
		\begin{cases}
			\alpha - \gamma = 1\\
			2\alpha + \beta = -2\\
			3\alpha + 2\beta + \gamma = 2
		\end{cases}.
		\]
		A matriz escalonada reduzida por linha desse sistema \'e
		\[
			\begin{bmatrix}
				1 & 0 & -1 & 0\\
				0 & 1 & \phantom{-}2 & 0\\
				0 & 0 & \phantom{-}0 & 1
			\end{bmatrix}
		\]
		e portanto o sistema \'e imposs{\'\i}vel. Logo o vetor $(1, -2, 2)$ n\~ao \'e uma combina\c{c}\~ao linear dos vetores $u_1$, $u_2$ e $u_3$.
	\item Considere o $\real$-espa\c{c}o vetorial $\real^3$. Dado um vetor $(a, b, c) \in \real^3$ podemos escrever
	\[
		(a, b, c) = a(1, 0, 0) + b(0, 1, 0) + c(0, 0, 1).
	\]
	Assim qualquer vetor de $\real^3$ \'e uma combina\c{c}\~ao linear dos vetores $(1, 0 , 0)$, $(0, 1 , 0)$ e $(0, 0 , 1)$. Logo o conjunto $\mathcal{B} = \{(1, 0 , 0), (0, 1 , 0), (0, 0 , 1)\}$ \'e um conjunto gerador para $\real^3$.
	\item $\mathcal{B}_1 = \{(1, 0 , 1), (1, 1 , 0), (1, 1 , 1), (-1, 0, 0), (-1, -1, 0), (-1, -1, -1)\}$ tamb\'em \'e um conjunto gerador para $\real^3$ como $\real$-espa\c{c}o vetorial. De fato, dado $(a, b, c) \in \real^3$, queremos encontrar $x_1$, $x_2$, $x_3$, $x_4$, $x_5$, $x_6 \in \real$ tais que
	\[
		(a, b, c) = x_1(1, 0 , 1) + x_2(1, 1 , 0) + x_3(1, 1 , 1) + x_4(-1, 0, 0) + x_5(-1, -1, 0) + x_6(-1, -1, -1).
	\]
	O que fornece o sistema
	\[
		\begin{cases}
			x_1 + x_2 + x_3 - x_4 - x_5 - x_6 = a\\
			x_2 + x_3 - x_5 - x_6 = b\\
			x_1 + x_3 - x_6 = c
		\end{cases}
	\]
	cuja matriz escalonada reduzida por linhas \'e
	\[
		\begin{bmatrix}
			1 & 0 & 0 & -1 & \phantom{-}0 & \phantom{-}0 & a - b\\
			0 & 1 & 0 & -1 & -1 & \phantom{-}0 & a - c\\
			0 & 0 & 1 & \phantom{-}1 & \phantom{-}0 & -1 & b + c - a
		\end{bmatrix},
	\]
	assim o sistema \'e poss{\'\i}vel e indeterminado, com posto 3 e nulidade 3. Sua solu\c{c}\~ao \'e descrita pela equa\c{c}\~oes
	\begin{align*}
		x_1 &= a - b + x_4\\
		x_2 &= a - c + x_4 + x_5\\
		x_3 &= b + c - a - x_4 + x_6
	\end{align*}
	onde $x_4$, $x_5$ e $x_6 \in \real$. Portanto $\mathcal{B}_1$ \'e um conjunto gerador para $\real^3$.
	\item Seja $\mathcal{P}(\real)$ o conjunto dos polin\^omios em $\real$. O subconjunto $\mathcal{B} = \{1, x, x^2, \dots, x^n, \dots\}$ \'e um conjunto gerador de $\mathcal{P}(\real)$ visto como um $\real$-espa\c{c}o vetorial.
	\item O conjunto $\mathcal{B} = \{(1, 0), (0, 1)\}$ gera $\complex^2$ como $\complex$-espa\c{c}o vetorial. No entanto, este conjunto n\~ao gera $\complex^2$ como $\real$-espa\c{c}o vetorial pois, por exemplo, $(i, 0)$ n\~ao \'e combina\c{c}\~ao linear dos vetores em $\mathcal{B}$. De fato, se $\alpha$ e $\beta$ s\~ao tais que
	\[
		(i, 0) = \alpha (1, 0) + \beta (0, 1)
	\]
	ent\~ao $\alpha = i$ o que \'e imposs{\'\i}vel em $\real$.
	\end{enumerate}
\end{exemplo}

\begin{observacao}
	\begin{enumerate}\label{conjuntogerador}
		\item Por conven\c{c}\~ao diremos que o conjunto vazio gera o $\cp{K}$-espa\c{c}o vetorial $\{0\}$.\label{geradorconjuntonulo}
		\item Todo espa\c{c}o vetorial possui um conjunto gerador. (Qual?)
	\end{enumerate}
\end{observacao}

\begin{definicao}
	Seja $V$ um $\cp{K}$-espa\c{c}o vetorial e $\mathcal{B}$ um subconjunto de $V$.
	\begin{enumerate}[label={\roman*})]
		\item Dizemos que $\mathcal{B}$ \'e \textbf{linearmente independente}\index{Espaço Vetorial!conjunto L.I.} ou simplesmente \textbf{L.I.}, se o \'unico meio de escrevermos
		\[
			0_V = \alpha_1 u_1 + \alpha_2 u_2 + \cdots + \alpha_n u_n
		\]
		onde $u_i \in \mathcal{B}$ e $\alpha_i \in \cp{K}$ com $i = 1$, 2, \dots, n \'e tomando $\alpha_1 = \alpha_2 = \cdots = \alpha_n = 0_\cp{K}$.

		\item Dizemos que o conjunto $\mathcal{B}$ \'e \textbf{linearemente dependente}\index{Espaço Vetorial!conjunto L.D.} ou simplesmente
		\textbf{L.D.}, se existem vetores distintos $u_1$, $u_2$, \dots, $u_n \in \mathcal{B}$ e escalares $\alpha_1$, $\alpha_2$, \dots, $\alpha_n$ n\~ao todos nulos, tais que
		\[
			\alpha_1 u_1 + \alpha_2 u_2 + \cdots + \alpha_n u_n = 0_V.
		\]
	\end{enumerate}
\end{definicao}

\begin{exemplo}
	\begin{enumerate}[label={\arabic*})]
		\item Seja $\mathcal{B} = \{(1, 0), (i, 0), (0, 1), (0, i)\}\sub \complex^2$. Considerando $\complex^2$ como um $\complex$-espa\c{c}o vetorial, ent\~ao $\mathcal{B}$ \'e L.D. pois
		\[
			(0, 0) = 0(1, 0) + 0(i, 0) + 1(0, 1) + i(0, i).
		\]
		Agora, considerando $\complex^2$ com um $\real$-espa\c{c}o vetorial, o conjunto $\mathcal{B}$ \'e L.I.. De fato, se $\alpha$, $\beta$, $\gamma$, $\delta \in \real$ s\~ao tais que
		\[
			(0, 0 ) = \alpha(1, 0) + \beta(i, 0) + \gamma(0, 1) + \delta(0, i)
		\]
		ent\~ao $\alpha = \beta = \gamma = \delta = 0$.
		\item Considerando $\real^2$ como um $\real$-espa\c{c}o vetorial, o conjunto $\mathcal{B} = \{(1, -1), (1, 0), (1, 1)\}$ \'e L.D. pois
		\[
			(0, 0 ) = \dfrac{1}{2}(1, -1) - (1, 0) + \dfrac{1}{2}(1, 1).
		\]
	\end{enumerate}
\end{exemplo}

\begin{observacao}\label{conjutoLDLI}
	Por conven\c{c}\~ao, o conjunto vazio \'e L.I.
\end{observacao}

\begin{propriedades}
	\begin{enumerate}[label={\roman*})]
		\item Todo conjunto contendo o vetor nulo \'e L.D.
		\item Todo espa\c{c}o vetorial n\~ao-nulo possui um conjunto L.I. n\~ao vazio. (Qual?)
		\item Todo subconjunto de um conjunto L.I. \'e L.I.
	\end{enumerate}
\end{propriedades}

\begin{definicao}
	Seja $V$ um espa\c{c}o vetorial sobre o corpo $\cp{K}$. Dizemos que um subconjunto $\mathcal{B}$ de $V$ \'e uma \textbf{base}\index{Espaço Vetorial!base} de $V$ se:
	\begin{enumerate}[label={\roman*})]
		\item $\mathcal{B}$ for um conjunto gerador de $V$;
		\item $\mathcal{B}$ for L.I.
	\end{enumerate}
\end{definicao}

\begin{exemplo}
	\begin{enumerate}[label={\arabic*})]
		\item $\mathcal{B} = \{(1, 0 , 0), (0, 1, 0), (0, 0 ,1)\}$ \'e uma base de $\real^3$ como $\real$-espa\c{c}o vetorial.
		\item $\mathcal{B} = \{(1, 0), (0, 1)\}$ \'e uma base de $\complex^2$ como $\complex$-espa\c{c}o vetorial.
		\item $\mathcal{B} = \{(1, 0), (i , 0), (0, 1) (0, 1), (0, i)\}$ \'e uma base de $\complex^2$ como $\real$-espa\c{c}o vetorial.
		\item Considere $\mathcal{P}(\complex) = \{p(x) = a_nx^n + \cdots + a_1x + a_0 \mid a_i \in \complex\}$ como $\complex$-espa\c{c}o vetorial. Sabemos que o conjunto
		\[
			\mathcal{B} = \{1, x, x^2, \dots, x^n, \dots\}
		\]
		gera $\mathcal{P}(\complex)$ como $\complex$-espa\c{c}o vetorial. Suponha que existam $\alpha_1$, $\alpha_2$, \dots, $\alpha_n \in \complex$ tais que
		\[
			\alpha_0 + \alpha_1x + \cdots + \alpha_nx^n = 0
		\]
		para todo $x \in \complex$. Como um polin\^omio de grau $n$ em $\complex$ n\~ao pode ter mais do que $n$ ra{\'\i}zes, ent\~ao $\alpha_0 = \alpha_1 = \cdots = \alpha_n = 0$ e portanto $\mathcal{B}$ \'e L.I. Logo $\mathcal{B}$ \'e uma base de $\mathcal{P}(\complex)$.
	\end{enumerate}
\end{exemplo}

\begin{observacao}
	Pelas Observa\c{c}\~oes \ref{conjuntogerador} item \ref{geradorconjuntonulo} e Observa\c{c}\~ao \ref{conjutoLDLI}, o conjunto vazio \'e uma base do $\cp{K}$-espa\c{c}o vetorial $\{0\}$.
\end{observacao}

\begin{definicao}
	Dizemos que um espa\c{c}o vetorial $V$ sobre um corpo $\cp{K}$ \'e \textbf{finitamente gerado}\index{Espaço Vetorial!finitamente gerado} se ele possui um conjunto gerador finito.
\end{definicao}

\begin{proposicao}\label{conjuntoLI}
	Seja $V$ um $\cp{K}$-espa\c{c}o vetorial n\~ao nulo e finitamente gerado. Suponha que $\{v_1, v_2, \dots, v_m\}$ seja um conjunto gerador de $V$. Ent\~ao todo conjunto L.I de vetores de $V$ tem no m\'aximo $m$ elementos.
\end{proposicao}
\begin{prova}
	Vamos provar que todo conjunto de elementos de $V$ que contenha mais do que $m$ vetores \'e L.D. Para tanto, seja $\mathcal{A} = \{u_1, u_2, \dots, u_n\}$ com $n > m$. Como $\{v_1, v_2, \dots, v_m\}$ \'e um conjunto gerador de $V$, ent\~ao existem escalares $\alpha_{ij} \in \cp{K}$ tais que para cada $j = 1$, 2, \dots, $n$ temos
	\begin{align}\label{combinacao_linear_u_v}
		u_j = \alpha_{1j}v_1 + \alpha_{2j}v_2 + \cdots + \alpha_{mj}v_m = \sum_{i = 1}^m\alpha_{ij}v_i.
	\end{align}
	Assim queremos mostrar que existem escalares $\lambda_1$, $\lambda_2$, \dots, $\lambda_n$ n\~ao todos nulos de modo que
	\begin{align}\label{vetores_u_LI}
		\sum_{j = 1}^n \lambda_ju_j = \lambda_1 u_1 + \lambda_2 u_2 + \cdots + \lambda_n u_n = 0_V.
	\end{align}
	Agora, substituindo os valores em \eqref{combinacao_linear_u_v} na equa\c{c}\~ao \eqref{vetores_u_LI} obtemos
	\[
		\sum_{j = 1}^n\lambda_ju_j = \sum_{j = 1}^n\lambda_j\left(\sum_{i = 1}^m\alpha_{ij}v_i\right) = \sum_{j = 1}^n\sum_{i = 1}^m\lambda_j\alpha_{ij}v_i = \sum_{i = 1}^m\left(\sum_{j = 1}^n\lambda_j\alpha_{ij}\right)v_i.
	\]
	Como queremos mostrar que os vetores no conjunto $\mathcal{A}$ s\~ao L.D vamos supor ent\~ao que
	\[
		\sum_{j = 1}^n\lambda_j\alpha_{ij} = 0_\cp{K}
	\]
	para $i = 1$, 2, \dots, $n$. Assim obtemos o sistema
	\begin{equation}\label{sistemabaseEV}
		\begin{cases}
			\lambda_1\alpha_{11} + \lambda_2\alpha_{12} + \cdots + \lambda_n\alpha_{1n} = 0_\cp{K}\\
			\qquad \vdots\\
			\lambda_1\alpha_{m1} + \lambda_2\alpha_{m2} + \cdots + \lambda_n\alpha_{mn} = 0_\cp{K}\\
		\end{cases}
	\end{equation}
	onde $\lambda_1$, \dots, $\lambda_n$ s\~ao as inc\'ognitas e $\alpha_{ij} \in \cp{K}$. Como o n\'umero de equa\c{c}\~oes \'e menor que o n\'umero de inc\'ognitas, ent\~ao a matriz linha-reduzida \`a forma em escada do sistema \eqref{sistemabaseEV} tem posto menor que $n$, ou seja, o sistema \eqref{sistemabaseEV} tem solu\c{c}\~ao n\~ao trivial. Assim existem $\lambda_1$, \dots, $\lambda_n$ n\~ao todos nulos tais que
	\[
		\sum_{j = 1}^n\lambda_j\alpha_{ij} = 0_\cp{K},
	\]
	ou seja, existem $\lambda_1$, \dots, $\lambda_n$ n\~ao todos nulos de modo que
	\[
		\lambda_1 u_1 + \lambda_2 u_2 + \cdots + \lambda_n u_n = 0_V.
	\]
	Portanto o conjunto $\mathcal{A}$ \'e L.D., como quer{\'\i}amos.
\end{prova}

\begin{corolario}
	Seja $V$ um $\cp{K}$-espa\c{c}o vetorial n\~ao nulo e finitamente gerado. Ent\~ao duas bases quaisquer de $V$ t\^em o mesmo n\'umero de elementos.
\end{corolario}
\begin{prova}
	Sejam $\mathcal{B}$ e $\mathcal{B}'$ duas bases de $V$. Como $V$ \'e finitamente gerado, segue da Proposi\c{c}\~ao \ref{conjuntoLI} que $\mathcal{B}$ e $\mathcal{B}'$ s\~ao finitos pois s\~ao L.I e possuem $m$ e $m'$ elementos, respectivamente.

	Considerando $\mathcal{B}$ como conjunto gerador de $V$ e $\mathcal{B}'$ L.I., segue da Proposi\c{c}\~ao \ref{conjuntoLI} que $m' \le m$. Por outro lado, considerando $\mathcal{B}'$ como conjunto gerador de $V$ e $\mathcal{B}$ L.I., segue da Proposi\c{c}\~ao \ref{conjuntoLI} que $m
	\le m'$.

	Logo $m = m'$, ou seja, duas bases t\^em sempre o mesmo n\'umero de elementos.
\end{prova}

\begin{definicao}
	Seja $V$ um $\cp{K}$-espa\c{c}o vetorial. Se $V$ admite uma base finita, ent\~ao chamamos de \textbf{dimens\~ao}\index{Espaço Vetorial!dimens\~ao} de $V$, e denotamos por $\dim_\cp{K} V$, o n\'umero de elementos em tal base.
\end{definicao}

\begin{exemplo}
	\begin{enumerate}[label={\arabic*})]
		\item $\dim_\cp{K}\cp{K}^n = n$;
		\item $\dim_\real\real^3 = 3$;
		\item $\dim_\complex\complex^2 = 2$;
		\item $\dim_\real\complex^2 = 4$.
	\end{enumerate}
\end{exemplo}


\begin{proposicao}\label{ampliarconjuntoLI}
	Seja $V$ um $\cp{K}$-espa\c{c}o vetorial e considere $\mathcal{B} = \{u_1, u_2, \dots, u_n\}$ um conjunto L.I. em $V$. Se existir $u \in V$ que n\~ao seja combina\c{c}\~ao linear dos elementos de $\mathcal{B}$, ent\~ao $\{u_1, u_2, \dots, u_n, u\}$ \'e L.I.
\end{proposicao}
\begin{prova}
	Sejam $\alpha_1$, $\alpha_2$, \dots, $\alpha_n$, $\alpha_{n + 1}$ escalares tais que
	\[
		\alpha_1u_1 + \alpha_2u_2 + \cdots + \alpha_nu_n + \alpha_{n + 1}u = 0_u.
	\]
	Se $\alpha_{n + 1} \ne 0_\cp{K}$ ent\~ao podemos escrever
	\[
		u = \left(-\dfrac{\alpha_1}{\alpha_{n + 1}}\right)u_1 + \left(-\dfrac{\alpha_2}{\alpha_{n + 1}}\right)u_2 + \cdots + \left(-\dfrac{\alpha_n}{\alpha_{n + 1}}\right)u_n
	\]
	e ent\~ao $u$ \'e combina\c{c}\~ao linear de $u_1$, \dots, $u_n$ o que contradiz nossa hip\'otese. Assim $\alpha_{n + 1} = 0_\cp{K}$ e da{\'\i} obtemos
	\[
		\alpha_1u_1 + \alpha_2u_2 + \cdots + \alpha_nu_n = 0_u.
	\]
	Mas $\{u_1, u_2, \dots, u_n\}$ \'e L.I, logo $\alpha_1 = \cdots = \alpha_n = 0_\cp{K}$. Portanto, $\{u_1, u_2, \dots, u_n, u\}$ \'e L.I.
\end{prova}

\begin{teorema}
	Todo espa\c{c}o vetorial n\~ao-nulo e finitamente gerado possui uma base.
\end{teorema}
\begin{prova}
	Seja $V$ um $\cp{K}$-espa\c{c}o vetorial n\~ao-nulo e finitamente gerado. Ent\~ao $V$ possui um conjunto gerador finito, digamos com $m$ elementos, $m > 1$. Seja $u_1 \in V$, $u_1 \ne 0_V$. Ent\~ao $\mathcal{B}_1 = \{u_1\}$ \'e L.I. Se $\mathcal{B}_1$ gera $V$ ent\~ao $\mathcal{B}_1$ \'e uma base de $V$. Caso contr\'ario, existe $u_2 \in V$ tal que $u_2$ n\~ao \'e um m\'ultiplo escalar de $u_1$. Pela Proposi\c{c}\~ao \ref{ampliarconjuntoLI}, $\mathcal{B}_2 = \{u_1, u_2\}$ \'e L.I. Novamente, se $\mathcal{B}_2$ gera $V$, ent\~ao $\mathcal{B}_2$ \'e uma base de $V$. Caso contr\'ario, existe $u_3 \in V$ tal que $u_3$ n\~ao \'e uma combina\c{c}\~ao linear de $u_1$ e $u_2$. Da{\'\i}, $\mathcal{B}_3 = \{u_1, u_2, u_3\}$ \'e L.I. Continuando com este processo chegamos a uma base de $V$ ou obtemos conjuntos L.I. com um n\'umero arbitr\'ario de elementos. A segunda op\c{c}\~ao n\~ao \'e poss{\'\i}vel por causa da Proposi\c{c}\~ao \ref{conjuntoLI}. Portanto, com este processo obtemos uma base de $V$.
\end{prova}

\begin{corolario}
	Seja $V$ um espa\c{c}o vetorial sobre um corpo $\cp{K}$ e de dimens\~ao $n \ge 1$. Seja $\mathcal{B}$ um subconjunto de $V$ com $n$ elementos. As seguintes afirma\c{c}\~oes s\~ao equivalentes:
	\begin{enumerate}[label={\roman*})]
		\item $\mathcal{B}$ \'e uma base de $V$;
		\item $\mathcal{B}$ \'e L.I.;
		\item $\mathcal{B}$ \'e um conjunto gerador de $V$.
	\end{enumerate}
\end{corolario}

\begin{teorema}\label{basecontendoconjuntoLI}
	Seja $V$ uma $\cp{K}$-espa\c{c}o vetorial finitamente gerado e seja $\mathcal{B}$ um conjunto L.I. em $V$. Ent\~ao existe uma base de $V$ contendo $\mathcal{B}$.
\end{teorema}

\begin{proposicao}
	Seja $V$ uma $\cp{K}$-espa\c{c}o vetorial de dimens\~ao $n \ge 1$ e seja $\mathcal{B} \sub V$. As seguintes afirma\c{c}\~oes s\~ao equivalentes:
	\begin{enumerate}[label={\roman*})]
		\item $\mathcal{B}$ \'e uma base de $V$;
		\item Cada elemento de $V$ se escreve de maneira \'unica como combina\c{c}\~ao linear de elementos de $\mathcal{B}$.
	\end{enumerate}
\end{proposicao}
\begin{prova}
	\begin{itemize}
		\item[$i) \Rightarrow ii)$] Suponha que $\mathcal{B} = \{u_1, \dots,u_n\}$ seja um base de $V$. Em particular $\mathcal{B}$ gera $V$. Seja $u \in V$ e suponha que existam escalares $\alpha_1$, \dots, $\alpha_n$, $\beta_1$, \dots, $\beta_n$ tais que
		\[
			\sum_{i = 1}^n \alpha_iu_i = u = \sum_{i = 1}^n \beta_iu_i.
		\]
		Da{\'\i}
		\[
			\sum_{i = 1}^n (\alpha_i - \beta_i)u_i	 = 0_V
		\]
		e como $\mathcal{B}$ \'e L.I segue que $\alpha_1 = \beta_1$, \dots, $\alpha_n = \beta_n$ como quer{\'\i}amos.
		\item[$ii) \Rightarrow i)$] Agora suponha que cada elemento de $V$ se escreva de maneira \'unica como combina\c{c}\~ao linear de elementos de $\mathcal{B}$. Assim, $\mathcal{B}$ gera $V$. Resta mostrar que $\mathcal{B}$ \'e L.I. Sejam ent\~ao $\alpha_1$, \dots, $\alpha_n \in \cp{K}$ tais que
		\[
			\alpha_1u_1 + \cdots + \alpha_nu_n = 0_V.
		\]
		Como $0_V = 0_\cp{K}u_1 + \cdots + 0_\cp{K}u_n$, segue ent\~ao da unicidade que $\alpha_1 = \dots = \alpha_n = 0_\cp{K}$ e da{\'\i} $\mathcal{B}$ \'e L.I. Portanto, $\mathcal{B}$ \'e uma base de $V$.
	\end{itemize}
\end{prova}

\section{Subespa\c{c}os} % (fold)
\label{sec:subespacos}

\begin{definicao}
	Seja $V$ um $\cp{K}$-espa\c{c}o vetorial. Um subconjunto n\~ao vazio $W$ de $V$ \'e um \textbf{subespa\c{c}o vetorial} de $V$ se a restri\c{c}\~ao das opera\c{c}o\~es de $V$ a $W$ torna $W$ um $\cp{K}$-espa\c{c}o vetorial.\index{Subespa\c{c}o}
\end{definicao}

\begin{exemplo}
	\begin{enumerate}[label={\arabic*})]
		\item O subconjunto $W = \{0_V\}$ \'e um subespa\c{c}o vetorial de qualquer espa\c{c}o vetorial $V$. O pr\'oprio $V$ como subconjunto de $V$ \'e tamb\'em um subespa\c{c}o vetorial. Estes dois subespa\c{c}os s\~ao chamados de \textbf{subespa\c{c}os triviais}.
		\item Seja $V$ um $\cp{K}$-espa\c{c}o vetorial e $w \in V$. Ent\~ao o conjunto $\cp{K}w = \{\alpha w \mid \alpha \in \cp{K}\}$ \'e um subespa\c{c}o vetorial de $V$.
	\end{enumerate}
\end{exemplo}

\begin{teorema}
	Um subconjunto n\~ao vazio $W$ de um $\cp{K}$-espa\c{c}o vetorial $V$ \'e um subespa\c{c}o de $V$ se, e somente se, para cada par de vetores $u_1$, $u_2 \in W$ e cada escalar $\lambda \in \cp{K}$, temos que $\lambda u_1 + u_2 \in W$.
\end{teorema}
\begin{prova}
	Exerc{\'\i}cio!
\end{prova}

\begin{exemplo}
	\begin{enumerate}[label={\arabic*})]
		\item $V = \real^5$; $W = \{(0,x_2,x_3,x_4,x_5) \mid x_i \in \real\}$ \'e um subespa\c{c}o de $V$.
		\item $V = M_n(\cp{K})$; $UT_n(\cp{K}) = \left\{\begin{bmatrix}
			a_{11} & a_{12} & \cdots & a_{1n}\\
			0_\cp{K} & a_{22} & \cdots & a_{2n}\\
			\vdots\\
			0_\cp{K} & 0_\cp{K} & \cdots & a_{nn}
		\end{bmatrix} \mid a_{ij} \in \cp{K}\right\}$ \'e um subespa\c{c}o de $V$.
		\item $V = \real^2$; $W = \{(x,x^2) \mid x \in \real\}$ n\~ao \'e subespa\c{c}o. Por exemplo, $u = (1,1)$, $v = (2,4) \in W$ e no entanto $u + v \notin W$.
	\end{enumerate}
\end{exemplo}

\begin{proposicao}
	Se $W_1$ e $W_1$ s\~ao subespa\c{c}os de um $\cp{K}$-espa\c{c}o vetorial $V$ ent\~ao
	\begin{enumerate}[label={\roman*})]
		\item $W_1 \cap W_2$ \'e um subespa\c{c}o vetorial de $V$. De fato, $W_1 \cap W_2$ \'e um subespa\c{c}o tanto de $W_1$ quanto de $W_2$.
		\item $W_1 + W_2 = \{u_1 + u_2 \mid u_1 \in W_1, u_2 \in W_2\}$ \'e um subespa\c{c}o vetorial de $V$.
	\end{enumerate}
\end{proposicao}
\begin{prova}
	Exerc{\'\i}cio!
\end{prova}

\begin{proposicao}
	Seja $V$ um $\cp{K}$-espa\c{c}o vetorial n\~ao nulo e de dimens\~ao finita. Se $W$ \'e um subespa\c{c}o n\~ao trivial de $V$, ent\~ao $\dim_\cp{K} W < \dim_\cp{K} V$.
\end{proposicao}
\begin{prova}
	Seja $\mathcal{B} = \{w_1, \dots,w_n\}$ uma base de $W$. Em particular $\mathcal{B}$ \'e um conjunto L.I. de $V$. Como $W \ne V$, existe $u \in V$ tal que $u \notin W$ e assim $u$ n\~ao \'e gerado pelos elementos de $\mathcal{B}$. Da{\'\i} pela Proposi\c{c}\~ao \ref{ampliarconjuntoLI}, $\{w_1, \dots,w_n,u\}$ \'e L.I..Logo, uma base de $V$ conter\'a mais elementos que o conjunto $\mathcal{B}$, isto \'e, $\dim_\cp{K} W < \dim_\cp{K} V$.
\end{prova}

\begin{proposicao}
	Sejam $V$ um $\cp{K}$-espa\c{c}o vetorial, $W_1$ e $W_2$ subespa\c{c}os vetoriais de $V$, ambos de dimens\~ao finita. Ent\~ao
	\[
		\dim_\cp{K}(W_1 + W_2) = \dim_\cp{K}W_1 + \dim_\cp{K}W_2 - \dim_\cp{K}(W_1 \cap W_2).
	\]
\end{proposicao}
\begin{prova}
	Vamos supor primeiro que $W_1 \cap W_2 \ne \{0_V\}$. Como $W_1$ e $W_2$ s\~ao de dimens\~ao finita, ent\~ao $W_1 \cap W_2$ e $W_1 + W_2$ tamb\'em o s\~ao. Assim seja $\mathcal{B} = \{w_1, \dots,w_n\}$ uma base de $W_1 \cap W_2$. Como $W_1 \cap W_2$ \'e um subespa\c{c}o vetorial tanto de $W_1$ como de $W_2$, pelo Teorema \ref{basecontendoconjuntoLI} podemos estender $\mathcal{B}$ a uma base de $W_1$ e a uma base de $W_2$. Sejam $\mathcal{B}_1 = \{w_1,\dots,w_n,v_1,\dots,v_r\}$ uma base de $W_1$ e $\mathcal{B}_2 = \{w_1,\dots,w_n,u_1,\dots,u_p\}$ uma base de $W_2$. Note que $\mathcal{B} \sub \mathcal{B}_1$ e $\mathcal{B} \sub \mathcal{B}_2$. Vamos ent\~ao mostrar que $\mathcal{A} = \{w_1,\dots,w_n,v_1,\dots,v_r,u_1,\dots,u_p\}$ \'e uma base de $W_1 + W_2$. Primeiro mostraremos que $\mathcal{A}$ gera $W_1 + W_2$.

	Seja $v \in W_1 + W_2$. Ent\~ao $v = x + y$ onde $x \in W_1$ e $y \in W_2$. Mas $\mathcal{B}_1$ e $\mathcal{B}_2$ s\~ao bases de $W_1$ e $W_2$, respectivamente. Assim
	\begin{align*}
		x = \sum_{i = 1}^n \alpha_i w_i + \sum_{j = 1}^r\beta_j v_j\\
		y = \sum_{i = 1}^n \delta_i w_i + \sum_{l = 1}^p\gamma_l u_l
	\end{align*}
	com $\alpha_i$, $\beta_j$, $\delta_i$, $\gamma_l \in \cp{K}$. Da{\'\i}
	\[
		v = x + y = \sum_{i = 1}^n(\alpha_i + \delta_i)w_i + \sum_{j = 1}^r\beta_jv_j + \sum_{l = 1}^p\gamma_lu_l
	\]
	e ent\~ao $\mathcal{A}$ gera $W_1 + W_2$.

	Agora, precisamos mostrar que $\mathcal{A}$ \'e L.I.. Considere ent\~ao a soma
	\[
		\sum_{i = 1}^n\alpha_iw_i + \sum_{j = 1}^r\beta_jv_j + \sum_{l = 1}^p\gamma_lu_l = 0_V
	\]
	onde $\alpha_i$, $\beta_j$, $\gamma_l \in \cp{K}$. Assim
	\begin{equation}\label{equacaoauxiliar}
		\sum_{l = 1}^p\gamma_lu_l = \sum_{i = 1}^n(-\alpha_i)w_i + \sum_{j = 1}^r(-\beta_j)v_j \in W_1 \cap W_2
	\end{equation}
	pois \'e uma combina\c{c}\~ao linear de elementos de $\mathcal{B}_1$ e de $\mathcal{B}_2$, simultaneamente. Logo, existem $\lambda_1$, \dots, $\lambda_n \in \cp{K}$ tais que
	\[
		\sum_{l = 1}^p\gamma_lu_l = \sum_{i = 1}^n\lambda_iw_i
	\]
	isto \'e,
	\[
		\sum_{l = 1}^p\gamma_lu_l + \sum_{i = 1}^n(-\lambda_i)w_i = 0_V.
	\]
	Mas $\mathcal{B}$ \'e L.I., da{\'\i} $\gamma_1 = \cdots = \gamma_p = \lambda_1 = \cdots = \lambda_n = 0_\cp{K}$. Assim podemos reescrever \eqref{equacaoauxiliar} como
	\[
		\sum_{i = 1}^n(-\alpha_i)w_i + \sum_{j = 1}^r(-\beta_j)v_j = 0_\cp{K}.
	\]
	Mas $\mathcal{B}_1$ \'e L.I., donde $\alpha_1 = \cdots = \alpha_n = \beta_1 = \cdots = \beta_r = 0_\cp{K}$. Isto \'e, $\mathcal{A}$ \'e L.I..

	Portanto $\mathcal{A}$ \'e uma base de $W_1 + W_2$ e assim
	\[
		\dim_\cp{K}(W_1 + W_2) = \dim_\cp{K}W_1 + \dim_\cp{K}W_2 - \dim_\cp{K}(W_1 \cap W_2).
	\]

	Se $W_1 \cap W_2 = \{0_V\}$, sejam $\mathcal{B}_1$ e $\mathcal{B}_2$ bases de $W_1$ e $W_2$, respectivamente. Analogamente ao caso anterior, mostra-se que $\mathcal{B}_1 \cup \mathcal{B}_2$ \'e uma base de $W_1 + W_2$.
\end{prova}

\begin{definicao}\index{Subespa\c{c}o!gerado}
	Seja $V$ um $\cp{K}$-espa\c{c}o vetorial e $S = \{u_1,\dots,u_n\} \sub V$. O \textbf{subespa\c{c}o gerado} por $S$ \'e definido como o subconjunto de $V$ formado por todas as combina\c{c}\~oes lineares de $u_1$, \dots, $u_n$. Denotaremos tal conjunto por
	\[
		[u_1,\dots,u_n] = \{\alpha_1u_1 + \cdots + \alpha_nu_n \mid \alpha_1, \dots, \alpha_n \in \cp{K}\}.
	\]
\end{definicao}

\begin{proposicao}
	Seja $V$ um $\cp{K}$-espa\c{c}o vetorial e $\{v_1,\dots,v_n\} \sub V$. O conjunto $[v_1,\dots,v_n]$ \'e um $\cp{K}$-subespa\c{c}o vetorial de $V$.
\end{proposicao}

\begin{exemplo}
	\begin{enumerate}[label={\arabic*})]
		\item Dado $\mathcal{P}(\complex)$ o $\complex$-espa\c{c}o vetorial dos polin\^omios com coeficientes em $\complex$, seja $\{1,x,x^2,x^3,x^4\} \sub \mathcal{P}(\complex)$. Ent\~ao
		\[
			[1,x,x^2,x^3,x^4] = \{a_0 + a_1x + a_2x^2 + a_3x^3 + a_4x^4\} = \{f(x) \in \mathcal{P}(\complex) \mid \deg f(x) \le 4\}
		\]
		\'e um subespa\c{c}o de $\mathcal{P}(\complex)$.
		\item Considere $\real^5$ como um $\real$-espa\c{c}o vetorial e seja $\{(1,2,0,3,0);(0,0,1,4,0);(0,0,0,0,1)\}$. Ent\~ao
		\[
			[(1,2,0,3,0);(0,0,1,4,0);(0,0,0,0,1)] = \{(\alpha, 2\alpha,\beta,3\alpha + 4\beta,\gamma) \mid \alpha, \beta, \gamma \in \real\}
		\]
		\'e um subespa\c{c}o vetorial de $\real^5$.
		\item Considere $W = [(1,2,3);(0,1,2);(-1,0,1)]$ um $\real$-espa\c{c}o vetorial. Determinar $\dim_\real W$.
		\begin{solucao}
			Inicialmente, vamos verificar se o conjunto $\{(1,2,3);(0,1,2);(-1,0,1)\}$ \'e L.I. ou L.D.. Para isso, sejam $x$, $y$ e $z \in \real$ tais que
			\[
				x(1,2,3) + y(0,1,2) + z(-1,0,1) = (0,0,0).
			\]
			Assim vemos que $x = z$ e $y = -2z$ s\~ao solu\c{c}\~oes para o sistema
			\[
				\begin{cases}
					x - z = 0\\
					2x + y = 0\\
					3x + 2y + z = 0
				\end{cases}
			\]
			e com isso $\{(1,2,3);(0,1,2);(-1,0,1)\}$ \'e L.D. Da{\'\i} $\dim_\real W \le 2$. Agora note que $(1,2,3)$ n\~ao \'e m\'ultiplo escalar de $(0,1,2)$. Logo, $\{(1,2,3); (0,1,2)\}$ \'e L.I. e ent\~ao $\dim_\real W = 2$.
		\end{solucao}
		
		\item Seja $V = \mathcal{P}_4(\real)$. Determine uma base de $V$ contendo os polin\^omios
		\begin{align*}
			&p_1(x) = 1 + 2x - x^2 + 3x^3 + 2x^4\\
			&p_2(x) = 2 + 4x + x^2 + 6x^3 + 3x^4\\
			&p_3(x) = 1 + 2x + 2x^2 + 3x^3 + 2x^4.
		\end{align*}
		\begin{solucao}
			Sabemos que $\mathcal{B} = \{1,x,x^2,x^3,x^4\}$ \'e uma base de $\mathcal{P}_4(\real)$. Tal base \'e chamada de \textbf{base can\^onica} de $\mathcal{P}_4(\real)$. Assim $\dim_\real \mathcal{P}_4(\real) = 5$. Para determinar uma base contendo $p_1(x)$, $p_2(x)$ e $p_3(x)$, come\c{c}amos determinando se tais vetores s\~ao L.I. ou L.D.. Para isso montamos a matriz
			\[
				A = \begin{bmatrix}
					\phantom{-}1 & 2 & 1\\
					\phantom{-}2 & 4 & 2\\
					-1 & 1 & 2\\
					\phantom{-}3 & 6 & 3\\
					\phantom{-}2 & 3 & 2
				\end{bmatrix}.
			\]
			Aplicando as opera\c{c}\~oes elementares em $A$ para reduz{\'\i}-la a forma em escada
			\begin{align*}
				A &=
					\left[
						\begin{array}{ccc}
		  					\phantom{-}1 & 2 & 1\\
							\phantom{-}2 & 4 & 2\\
							-1 & 1 & 2\\
							\phantom{-}3 & 6 & 3\\
							\phantom{-}2 & 3 & 2
     					\end{array}
     				\right]
     				\begin{array}{l}
     					\phantom{x}\\
     					L_2 \to L_2 - 2L_1\\
     					L_3 \to L_3 + L_1\\
     					L_4 \to L_2 - 3L_1\\
     					L_5 \to L_5 - 2L_1
     				\end{array} \sim
     				\left[
	     				\begin{array}{ccc}
		  					1 & \phantom{-}2 & 1\\
							0 & \phantom{-}0 & 0\\
							0 & \phantom{-}3 & 3\\
							0 & \phantom{-}0 & 0\\
							0 & -1 & 0
	     				\end{array}
     				\right] = M.
     		\end{align*}
     		Como a matrix $M$ possui posto 3, ent\~ao $\{p_1(x), p_2(x), p_3(x)\}$ \'e L.I. em $\mathcal{P}_4(\real)$. Observe que $M$ n\~ao tem 1 na segunda e quarta linhas. Assim vamos adicionar a $M$ uma coluna com 1 na segunda linha e outra com 1 na quarta linha, obtendo
     		\[
     			M' = \begin{bmatrix}
  					1 & \phantom{-}2 & 1 & 0 & 0\\
					0 & \phantom{-}0 & 0 & 1 & 0\\
					0 & \phantom{-}3 & 3 & 0 & 0\\
					0 & \phantom{-}0 & 0 & 0 & 1\\
					0 & -1 & 0 & 0 & 0
     			\end{bmatrix}.
     		\]
     		Assim $\mathcal{B}' = \{p_1(x), p_2(x), p_3(x),x, x^3\}$ forma uma base de $\mathcal{P}_4(\real)$ contendo $p_1(x)$, $p_2(x)$ e $p_3(x)$.
		\end{solucao}
		\item Considere $V = \cp{M}_{3\times 2}(\real)$. Verifique se o conjunto formado pelas matrizes
		\[
			A_1 = \begin{bmatrix}
				\phantom{-}1 & 0\\
				-1 & 3\\
				\phantom{-}3 & 2
			\end{bmatrix}; A_2 = \begin{bmatrix}
				0 & -1\\
				2 & \phantom{-}4\\
				3 & \phantom{-}2
			\end{bmatrix}; A_3 = \begin{bmatrix}
				1 & -2\\
				3 & \phantom{-}11\\
				9 & \phantom{-}6
			\end{bmatrix}.
		\]
		\'e L.D. ou L.I. em $V$.
		\begin{solucao}
			Considere a matriz $M$ formada pelas entradas das matrizes $A_1$, $A_2$ e $A_3$ escritas como colunas de $M$
			\[
				M = \begin{bmatrix}
					\phantom{-}1 & \phantom{-}0 & \phantom{-}1\\
					\phantom{-}0 & -1 & -2\\
					-1 & \phantom{-}2 & \phantom{-}3\\
					\phantom{-}3 & \phantom{-}4 & \phantom{-}11\\
					\phantom{-}3 & \phantom{-}3 & \phantom{-}9\\
					\phantom{-}2 & \phantom{-}2 & \phantom{-}6
				\end{bmatrix}.
			\]
			Aplicando as opera\c{c}\~oes elementares em $M$:
			\begin{align*}
				M &=
					\left[
						\begin{array}{ccc}
		  					\phantom{-}1 & \phantom{-}0 & \phantom{-}1\\
							\phantom{-}0 & -1 & -2\\
							-1 & \phantom{-}2 & \phantom{-}3\\
							\phantom{-}3 & \phantom{-}4 & \phantom{-}11\\
							\phantom{-}3 & \phantom{-}3 & \phantom{-}9\\
							\phantom{-}2 & \phantom{-}2 & \phantom{-}6
     					\end{array}
     				\right]
     				\begin{array}{l}
     					\phantom{x}\\
     					\phantom{x}\\
     					L_3 \to L_3 + L_1\\
     					L_4 \to L_4 - 3L_1\\
     					L_5 \to L_5 - 3L_1\\
     					L_6 \to L_6 - 2L_1
     				\end{array} \sim
     				\left[
     					\begin{array}{ccc}
		  					1 & \phantom{-}0 & \phantom{-}1\\
							0 & -1 & -2\\
							0 & \phantom{-}2 & \phantom{-}4\\
							0 & \phantom{-}4 & \phantom{-}8\\
							0 & \phantom{-}3 & \phantom{-}6\\
							0 & \phantom{-}2 & \phantom{-}4
     					\end{array}
     				\right]
     				\begin{array}{l}
     					\phantom{x}\\
     					\phantom{x}\\
     					L_3 \to L_3 + 2L_2\\
     					L_4 \to L_4 + 4L_2\\
     					L_5 \to L_5 + 3L_2\\
     					L_6 \to L_6 + 2L_2
     				\end{array}\\ &\sim
     				\left[
     					\begin{array}{ccc}
		  					1 & \phantom{-}0 & \phantom{-}1\\
							0 & -1 & -2\\
							0 & \phantom{-}0 & \phantom{-}0\\
							0 & \phantom{-}0 & \phantom{-}0\\
							0 & \phantom{-}0 & \phantom{-}0\\
							0 & \phantom{-}0 & \phantom{-}0
     					\end{array}
     				\right]
     		\end{align*}
     		Como a matriz possui posto 2, ent\~ao o vetor $A_3$ \'e uma combina\c{c}\~ao linear de $A_1$ e $A_2$, ou seja, $\{A_1, A_2, A_3\}$ \'e L.D..
		\end{solucao}
	\end{enumerate}
\end{exemplo}
% section subespacos (end)
