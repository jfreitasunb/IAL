%!TEX program = xelatex
%!TEX root = IAL.tex
%%Usar makeindex -s indexstyle.ist arquivo no terminal para gerar o {\'\i}ndice remissivo agrupado por inicial
%%Ap\'os executar pdflatex arquivo
\chapter{Matrizes}
\section{Introdução}

\subsection{Corpos}
Durante esse curso iremos trabalhar com o conceito de corpos e alguns casos específicos desse conceito.

Para mais detalhes sobre corpos veja \cite{Hygino:2019}.

\begin{definicao}\label{corpo}\index{Corpo}
  Um conjunto não vazio $\cp{K}$ é chamado de \textbf{corpo} se em $\cp{K}$ podemos definir duas operações, denotadas por $\oplus$ (e chamada de \textbf{adição}) e $\otimes$ (e chamada de \textbf{multiplicação}) de modo que
  \begin{align*}
    (1)\ a \oplus b \in \cp{K}\\
    (2)\ a \otimes b \in \cp{K}
  \end{align*}
  para todos $a$, $b \in \cp{K}$ e que satisfaçam as seguintes propriedades:
  \begin{enumerate}[label={\roman*})]
    \item \textbf{Comutatividade da adição}: $a \oplus b = b \oplus a$ para todos $a$, $b \in \cp{K}$;
    \item \textbf{Associatividade da adição}: $a \oplus (b \oplus c) = (a \oplus b) \oplus c$, para todos $a$, $b$ e $c \in \cp{K}$;
    \item \textbf{Elemento neutro da adição}: Existe um elemento em $\cp{K}$, denotado por $0_\cp{K}$ ou simplesmente $0$ e chamado de \textbf{elemento neutro da adição}, que satisfaz
    \[
      a \oplus 0_\cp{K} = a = 0_\cp{K} \oplus a
    \]
    para todo $a \in \cp{K}$.
    \item \textbf{Elemento oposto da adição}: Para cada $a \in \cp{K}$, existe um elemento em $\cp{K}$, denotado por $-a$ e chamado de \textbf{oposto} de $a$ ou \textbf{inverso aditivo} de $a$ tal que
    \[
      a \oplus (-a) = 0_\cp{K} = (-a) \oplus a.
    \]
    \item \textbf{Comutatividade da multiplicação}: $a \otimes b = b \otimes a$ para todos $a$, $b \in \cp{K}$;
    \item \textbf{Associatividade da multiplicação}: $a \otimes (b \otimes c) = (a \otimes b) \otimes c$, para todos $a$, $b$ e $c \in \cp{K}$;
    \item \textbf{Elemento neutro da multiplicação}: Existe um elemento em $\cp{K}$, denotado por $1_\cp{K}$ ou simplesmente $1$ e chamado de \textbf{elemento neutro da multiplicação} ou \textbf{unidade}, que satisfaz
    \[
      a \otimes 1_\cp{K} = a = 1_\cp{K} \otimes a
    \]
    para todo $a \in \cp{K}$.
    \item \textbf{Elemento inverso da multiplicação}: Para cada $a \in \cp{K}$, $a \ne 0_{\cp{K}}$, existe um elemento em $\cp{K}$, denotado por $a^{-1}$ e chamado de \textbf{inverso multiplicativo} de $a$
    tal que
    \[
      a \otimes a^{-1} = 1_\cp{K} = a^{-1} \otimes a.
    \]
    \item \textbf{Distributividade da soma em relação \`a multiplicação}: $(a \oplus b)\otimes c = a\otimes c \oplus b\otimes c$, para todos $a$, $b$ e $c \in \cp{K}$.
  \end{enumerate}
\end{definicao}

Para simplificar a notação vamos escrever
\[
  a \otimes b = ab.
\]

Denotamos um corpo $\cp{K}$ pela terna $(\cp{K}, \otimes, \oplus)$. Quando não houver chance de confusão em relação \`as operações de soma e multiplicação envolvidas no corpo $(\cp{K}, \otimes, \oplus)$, vamos simplesmente dizer que $\cp{K}$ é um corpo. Os elementos de um corpo $\cp{K}$ são chamados de \textbf{escalares}.

Considere $\cp{K} = \rac$ ou $\real$. Então vale sempre que:
\begin{enumerate}[label={\roman*})]
  \item $x + y = y + x$, para todos $x$, $y \in \cp{K}$;

  \item $(x + y) + z = x + (y + z)$, para todos $x$, $y$ e $z \in \cp{K}$;

  \item existe $0$ em $\cp{K}$ tal que $x + 0 = x$ para todo $x \in \cp{K}$;

  \item para cada $x \in \cp{K}$, existe $y \in \cp{K}$ tal que $x + y = 0$. Tal $y$ é escrito como $y = -x$;

  \item $xy = yx$ para todos $x$, $y \in \cp{K}$;

  \item $(xy)z = x(yz)$ para todos $x$, $y$ e $z \in \cp{K}$;

  \item existe $1 \in \cp{K}$ tal que $1x = x$ para todo $x \in \cp{K}$

  \item para todo $x \in \cp{K}$, $x \ne 0$, existe $y \in \cp{K}$ tal que $xy =1$. Tal $y$ é escrito como $y = x^{-1}$;

  \item $(x + y)z = xz + yz$ para todos $x$, $y$ e $z \in \cp{K}$.
\end{enumerate}

\subsection{Números Complexos}\index{Números Complexos}
Um outro conjunto importante com o qual iremos trabalhar é \textbf{conjunto dos números complexos}. Tal conjunto é definido como
\[
  \complex = \{a + bi \mid a, b \in \real, i^2 = -1\}.
\]
Dados $z = a + bi$, $w = c + di \in \complex$ definimos a soma e a multiplicação em $\complex$ por
\begin{align}
  z + w &= (a + c) + (b + d)i\label{soma_complexos}\\
  z\cdot w &= (ac - bd) + (ad + bc)i.\label{multiplicacao_complexos}
\end{align}
Além disso, $a + bi = c + di$ se, e somente se, $a = c$ e $b = d$.

Dado um número complexo $z = a + bi \ne 0$, seja
\[
  w = \dfrac{a - bi}{a^2 + b^2} \ne 0.
\]
Note que
\begin{align*}
  z\cdot w = (a + bi)\cdot \dfrac{a - bi}{a^2 + b^2} = \dfrac{(a^2 + b^2) + (-ab + ba)i}{a^2 + b^2} = 1
\end{align*}
assim $w$ é o inverso multiplicativo de $z$ em $\complex$.

O \textbf{m\'odulo}\index{Números Complexos!módulo} do número complexo $z = a + bi$ é definido como
\[
  |z| = \sqrt{a^2 + b^2}
\]
e o \textbf{conjugado complexo} \index{Números Complexos!conjugado complexo}de $z = a + bi$ é definido como
\[
  \overline{z} = a - bi.
\]

\begin{proposicao}
  Para todos $z$, $w \in \complex$ valem:
  \begin{enumerate}[label={\roman*})]
    \item $\overline{\overline{z}} = z$;
    \item $|\overline{z}| = |z|$;
    \item $z\cdot\overline{z} = |z|^2$;
    \item $\overline{z^{-1}} = \overline{z}^{-1}$;
    \item $\overline{z + w} = \overline{z} + \overline{w}$;
    \item $\overline{z \cdot w} = \overline{z} \cdot \overline{w}$;
    \item $|z \cdot w| = |z| \cdot |w|$;
  \end{enumerate}
\end{proposicao}

\begin{proposicao}
  O conjunto $\complex$ com as operações definidas em \eqref{soma_complexos} e \eqref{multiplicacao_complexos} é um corpo.
\end{proposicao}

Uma propriedade importante dos números complexos que iremos utilizar é a seguinte:
\begin{teorema}
  Todo polin\^omio com coeficientes em $\complex$ possui ra{\'\i}zes complexas.
\end{teorema}

\begin{observacao}
  Um corpo satisfazendo a propriedade do teorema anterior é chamado de \textbf{algebricamente fechado}\index{Corpo!algebricamente fechado}. é fácil ver que $\rac$ e $\real$ não são corpos algebricamente fechados, isto é, existem polin\^omios com coeficientes em $\rac$ e em $\real$ que não possuem ra{\'\i}zes nesses corpos.
\end{observacao}
\section{Matrizes}

Durante o restante dessas notas o conjunto $\cp{K}$ representará um dos conjuntos $\rac$, $\real$ ou $\complex$.

Para mais informações sobre matrizes veja \cite{Boldrini:1980, Callioli:2009, Anton:2012}.

Aqui vamos recordar algumas propriedades básicas de matrizes. Para isso seja $\cp{K}$ um corpo.\index{Matriz}

Sejam $p$, $q$ dois inteiros positivos. Uma \textbf{matriz $p$ por $q$ $A$ sobre $\cp{K}$} é dada por $p \times q$ valores $a_{ij} \in K$, com $1 \le i \le p$, $1 \le j \le q$ agrupadas em $p$ linhas e $q$ colunas, que será representada por
\[
  A = (a_{ij})_{p\times q} = \begin{pmatrix}
    a_{11} & a_{12} & \cdots & a_{1q}\\
    a_{21} & a_{22} & \cdots & a_{2q}\\
    \vdots & & & \vdots\\
    a_{p1} & a_{p2} & \cdots & a_{pq}
  \end{pmatrix} \quad\mbox{ou}\quad
  A = [a_{ij}]_{p\times q} = \begin{bmatrix}
    a_{11} & a_{12} & \cdots & a_{1q}\\
    a_{21} & a_{22} & \cdots & a_{2q}\\
    \vdots & & & \vdots\\
    a_{p1} & a_{p2} & \cdots & a_{pq}
  \end{bmatrix}.
\]

Neste caso $A$ denota a matriz em si e o elemento de $A$ que ocorre na linha \textbf{i} e na coluna \textbf{j} é representado por $a_{ij}$.

Uma matriz $A$ com $p$ linhas e $q$ colunas é dita possuir \textbf{ordem}\index{Matriz!ordem}, ou \textbf{tamanho}, $p\times q$.

\begin{exemplo}
  A matriz $A$ dada a seguir:
  \[
    A = \begin{bmatrix}
      -1 & 0 & 1/2 & 1\\
      2 & \sqrt{2} & 3 & 3\\
      0 & 0 & \pi & 0
    \end{bmatrix}
  \]
  é uma matriz de ordem $3 \times 4$.

  Nessa matriz
  \begin{itemize}
    \item O elemento $a_{12}$ de $A$ é $a_{12} = 0$;

    \item O elemento $a_{24}$ de $A$ é $a_{24} = 3$;

    \item O elemento $a_{33}$ de $A$ é $a_{12} = \pi$.
  \end{itemize}
\end{exemplo}

\begin{definicao}
  Seja
  \[
    A = (a_{ij})_{p\times q} = \begin{bmatrix}
      a_{11} & a_{12} & \cdots & a_{1q}\\
      a_{21} & a_{22} & \cdots & a_{2q}\\
      \vdots & & & \vdots\\
      a_{p1} & a_{p2} & \cdots & a_{pq}
    \end{bmatrix}
  \]
  uma matriz de ordem $p\times q$ com entradas num corpo $\cp{K}$.

  \begin{enumerate}
    \item Se $p \ne q$, $A$ é chamada de uma \textbf{matriz retangular}.\index{Matriz!retangular}

    \item Se $A$ tem ordem $p \times 1$, isto é, se $A$ possui $p$ linhas e somente \textbf{uma coluna}, dizemos que $A$ é uma \textbf{matriz coluna}\index{Matriz!coluna}. Nesse caso $A$ tem a forma:
    \[
      A = \begin{pmatrix}a_{11}\\a_{21}\\\vdots\\a_{p - 11}\\a_{p1}\end{pmatrix}.
    \]
    \item Se $A$ tem ordem $1 \times q$, isto é, se $A$ possui somente \textbf{uma linhas} e $q$ colunas, dizemos que $A$ é uma \textbf{matriz linha}\index{Matriz!linha}. Nesse caso $A$ tem a forma:
    \[
      A = \begin{pmatrix}a_{11} & a_{12} & \cdots & a_{1n-1} & a_{1q}\end{pmatrix}.
    \]
    \item Se $p = q$, então $A$ é chamada de \textbf{matriz quadrada} e a ordem de $A$ é $p \times p$ ou simplesmente $p$. Neste caso $A$ tem a forma:
    \[
      A = (a_{ij})_{p\times p} = \begin{pmatrix}
        a_{11} & a_{12} & \cdots & a_{1p}\\
        a_{21} & a_{22} & \cdots & a_{2p}\\
        \vdots & & & \vdots\\
        a_{p1} & a_{p2} & \cdots & a_{pp}
      \end{pmatrix}.
    \]
  \end{enumerate}
\end{definicao}

\begin{exemplo}
  \begin{enumerate}
    \item A matrix
    \[
      A = \begin{pmatrix}
        -1 & 2 & -3/7 & 9
      \end{pmatrix}
    \]
    é uma matriz-linha de ordem $1 \times 4$.
    \item A matrix
    \[
      B = \begin{bmatrix}
        -1 \\ \phantom{-}2 \\ -3/7 \\ \phantom{-}9
      \end{bmatrix}
    \]
    é uma matriz-coluna de ordem $4 \times 1$.
    \item A matrix
    \[
      C = \begin{pmatrix}
        -1 & 2 & -3/7 & 9\\
        -1 & 2 & -3/7 & 9\\
        -1 & 2 & -3/7 & 9\\
        -1 & 2 & -3/7 & 9\\
      \end{pmatrix}
    \]
    é uma matriz quadrada de ordem $4 \times 4$.
    \item A matrix
    \[
      D = \begin{bmatrix}
        -1 & 2 & -3/7 & 9\\
        -9 & 3/7 & -2 & 1
      \end{bmatrix}
    \]
    é uma matriz retangular de ordem $2 \times 4$.
  \end{enumerate}
\end{exemplo}

Seja
\[
  A = \begin{bmatrix}
    a_{11} & a_{21} & \cdots & a_{1q}\\
    a_{21} & a_{22} & \cdots & a_{2q}\\
    \vdots & \vdots & & \vdots\\
    a_{p1} & a_{p2} & \cdots & a_{pq}
  \end{bmatrix}
\]
uma matriz de ordem $p\times q$. Denotamos a $i$-ésima linha, $1 \le i \le p$, de $A$ por
\[
  A_i = \begin{bmatrix}a_{i1} & a_{i2} & \cdots & a_{iq}\end{bmatrix}.
\]
Denotamos a $j$-ésima coluna, $1 \le j \le q$, de $A$ por
\[
  A^j = \begin{bmatrix}a_{1j} \\ a_{2j} \\ \vdots \\ a_{pj}\end{bmatrix}.
\]

\begin{exemplo}
  Considere a matriz
  \[
    A = \begin{pmatrix}
      \phantom{-} 1 & 2 + i & 2 - 3i\\
      -2 & 0 & \pi - i\sqrt{2}
    \end{pmatrix}
  \].
  Neste caso as linhas de $A$ são:
  \begin{align*}
    A_1 &= \begin{pmatrix}\phantom{-} 1 & 2 + i & 2 - 3i\end{pmatrix}\\
    A_2 &= \begin{pmatrix}-2 & 0 & \pi - i\sqrt{2}\end{pmatrix}.
  \end{align*}
  As colunas de $A$ são:
  \begin{align*}
    A^1 &= \begin{pmatrix}
      \phantom{-} 1\\-2
      \end{pmatrix}\\
    A^2 &= \begin{pmatrix}
      2 + i\\0
      \end{pmatrix}\\
    A^3 &= \begin{pmatrix}
      2 - 3i\\\pi - i\sqrt{2}
      \end{pmatrix}\\
  \end{align*}
\end{exemplo}

Dada uma matriz quadrada de ordem $p$
\[
  A = \begin{pmatrix}
    a_{11} & a_{21} & \cdots & a_{1p}\\
    a_{21} & a_{22} & \cdots & a_{2p}\\
    \vdots & \vdots & & \vdots\\
    a_{p1} & a_{p2} & \cdots & a_{pp}
  \end{pmatrix}
\]
os elementos $a_{ij}$ de $A$ com $i = j$ formam a \textbf{diagonal princial} de $A$:\index{Matriz!diagonal principal}
\[
  \begin{pmatrix}
    \tikzmarknode{top}{a_{11}} & a_{12} & a_{13} & \cdots  & a_{1p}\\
    a_{21} & a_{22} \phantom{-} & a_{23} & \cdots  & a_{2n}\\
    a_{31} & a_{32}\phantom{-} & a_{33} \phantom{-1} & \cdots  & a_{3p}\\
    \vdots  & \vdots   & \vdots  & \vdots  & \vdots\\
    a_{p1} & a_{p2} & a_{p3} & \cdots  & \tikzmarknode{bottom}{a_{pp}}\\
  \end{pmatrix}
\]

\begin{tikzpicture}[overlay, remember picture]
  \draw[red, opacity=.5, line width=5mm, line cap=round] (top.north west) -- (bottom.south east);
\end{tikzpicture}

Os elementos $a_{ij}$ com $i + j = p + 1$ formam a \textbf{diagonal secundária} de $A$:\index{Matriz!diagonal secundária}
\[
  \begin{pmatrix}
    a_{11} & a_{12} & a_{13} & \cdots  & a_{1p-1} & \tikzmarknode{top_right}{a_{1p}}\\
    a_{21} & a_{22} & a_{23} & \cdots  & a_{2p-1} & a_{2p}\\
    a_{31} & a_{32} & a_{33} & \cdots  & \phantom{-}a_{3p-1} & a_{3p}\\
    \vdots  & \vdots   & \vdots  & \vdots  & \vdots\\
    \tikzmarknode{bottom_left}{a_{p1}} & \phantom{-} a_{p2} & a_{p3} & \cdots  & a_{pp-1} & {a_{pp}}\\
  \end{pmatrix}
\]

\begin{tikzpicture}[overlay, remember picture]
  \draw[blue, opacity=.5, line width=5mm, line cap=round] (top_right.north east) -- (bottom_left.south west);
\end{tikzpicture}

\begin{definicao}
    Seja
    \[
        A = \begin{bmatrix}
                a_{11} & \cdots & a_{1n}\\
                a_{21} & \cdots & a_{2n}\\
                \vdots & \ddots & \vdots\\
                a_{n1} & \cdots & a_{nn}
            \end{bmatrix}
    \]
    uma matriz quadrada de ordem $p$. Dizemos que:
    \begin{enumerate}[label={\roman*})]
        \item $A$ é uma \textbf{matriz diagonal}\index{Matriz!diagonal} se $a_{ij} = 0$ para todo $i \ne j$. Assim $A$ é uma matriz da forma
            \[
                A = \begin{bmatrix}
                    a_{11} & 0 & \cdots & 0\\
                    0 & a_{22} & \cdots & 0\\
                    \vdots & \vdots & \ddots & \vdots\\
                    0 & \cdots & 0 & a_{nn}
                \end{bmatrix}.
            \]

        \item $A$ é uma \textbf{matriz unidade} ou \textbf{matriz identidade} ou ainda que $A$ é uma \textbf{matriz unitária} de
            \textbf{ordem} $n$ se $A$ é uma matriz diagonal e $a_{ii} = 1$ para $1 \le i \le n$. Neste caso denotamos tal matriz por $I_n$ e
        \[
            I_n = \begin{bmatrix}
                1 & 0 & \cdots & 0\\
                0 & 1 & \cdots & 0\\
                \vdots & \vdots & \ddots & \vdots\\
                0 & \cdots & 0 & 1
            \end{bmatrix}.
        \]
    \item $A$ é uma \textbf{matriz triangular inferior}\index{Matriz!triangular inferior} se $a_{ij} = 0$ para $i < j$. Assim $A$ será uma matriz da forma
    \[
        A = \begin{bmatrix}
                a_{11} & 0 & 0 & \cdots & 0\\
                a_{21} & a_{22} & 0 & \cdots & 0\\
                \vdots & \vdots & \ddots & & \vdots\\
                a_{n1} & a_{n2} & \cdots & \cdots &  a_{nn}
        \end{bmatrix}.
    \]
    \item $A$ é uma \textbf{matriz triangular superior}\index{Matriz!triangular superior} se $a_{ij} = 0$ para $i > j$. Assim $A$ será uma matriz da forma
    \[
        A = \begin{bmatrix}
                a_{11} & a_{12} & a_{13} & \cdots & a_{1n}\\
                0 & a_{22} & a_{23} & \cdots & a_{2n}\\
                \vdots & \vdots & \ddots & & \vdots\\
                0 & 0 & \cdots & 0 &  a_{nn}
        \end{bmatrix}.
    \]
    \end{enumerate}
\end{definicao}

\begin{exemplos}
    \begin{enumerate}[label={\arabic*})]
        \item A matriz
            \[
                A = \begin{pmatrix}
                    2 & -1/2 & 3 & \phantom{-}4\\
                    0 & \phantom{-}0 & 1 & -2\\
                    0 & \phantom{-}0 & 4 & \phantom{-}\sqrt{2}\\
                    0 & \phantom{-}0 & 0 & \phantom{-}1
                \end{pmatrix}
            \]
        é uma matriz triangular superior de ordem 4.
        \item A matriz
            \[
                B = \begin{pmatrix}
                    2 & \phantom{-}0 & \phantom{-}0 & 0\\
                    1 & -3i & \phantom{-}0 & 0\\
                    i & -1 & \phantom{-}4 & 0\\
                    7 & \phantom{-}3 & -2 & 5
                \end{pmatrix}
            \]
        é uma matriz triangular inferior de ordem 4.
        \item A matriz
            \[
                C = \begin{pmatrix}
                    1 & 0 & 0\\
                    0 & 1 & 0\\
                    0 & 0 & 1
                \end{pmatrix}
            \]
        é a matriz identidade de ordem 3.
        \item A matriz
            \[
                D = \begin{pmatrix}
                    2 & 0\\
                    0 & 3
                \end{pmatrix}
            \]
        é uma matriz diagonal de ordem 2.
    \end{enumerate}
\end{exemplos}

\begin{definicao}
    Seja $A$ uma matriz de ordem $p\times q$. Dizemos que $A$ é uma \textbf{matriz nula}\index{Matriz!nula} se $a_{ij} = 0$ para todo $1 \le i \le p$ e todo $1 \le j \le p$. Neste caso escrevemos $A_{p\times q} = 0_{p\times q}$.
\end{definicao}

\begin{exemplo}
    \begin{enumerate}[label={\arabic*})]
        \item $0_2 = \begin{pmatrix} 0 & 0\\0 & 0\end{pmatrix}$ é a matrz nula de ordem 2.

        \item $0_{2 \times 3} = \begin{pmatrix}0 & 0 & 0\\0 & 0 & 0\end{pmatrix}$ é a matriz nula de ordem $2 \times 3$.
    \end{enumerate}
\end{exemplo}

\begin{notacao}
  Denotaremos o conjunto de todas as matrizes $p\times q$ sobre $\cp{K}$ por $M_{p\times q}(\cp{K})$, onde $\cp{K} = \rac$, $\real$ ou $\complex$. Assim
  \[
    M_{p\times q}(\cp{K}) = \left\{\begin{bmatrix}
      a_{11} & a_{12} & \cdots & a_{1p}\\
      a_{21} & a_{22} & \cdots & a_{2p}\\
      \vdots & & & \vdots\\
      a_{p1} & a_{p2} & \cdots & a_{pq}\end{bmatrix} \mid a_{ij} \in \cp{K},\ 1 \le i \le p,\ 1 \le j \le q
    \right\}
  \]
\end{notacao}

Por exemplo,
\[
  M_{3\times 2}(\real) = \left\{\begin{pmatrix}a_{11} & a_{12}\\a_{21} & a_{22}\\ a_{31} & a_{32}\end{pmatrix} \mid a_{ij} \in \real,\ 1 \le i \le 3,\ 1 \le j \le 2\right\}.
\]


No caso em que $p = q = n$ iremos escrever simplesmente
\[
  M_{n\times n}(\cp{K}) = M_n(\cp{K}).
\]

Nesse conjunto podemos definir as seguintes operações:
\begin{enumerate}[label={\roman*})]
    \item \textit{Igualdade de matrizes:}\index{Matriz!igualdade} Sejam $A = (a_{ij})$ e $B = (b_{ij})$ matrizes em $M_{p\times q}(\cp{K})$. Então dizemos que $A$
        e $B$ são \textbf{iguais} quando $a_{ij} = b_{ij}$ para todo $1 \le i \le p$ e todo $1 \le j \le q$.

    \item \textit{Soma de matrizes:}\index{Matrizes!soma} Se $A = (a_{ij})$, $B = (b_{ij}) \in M_{p\times q}(\cp{K})$, então a
        \textbf{soma} $A + B$ é a matriz $C = (c_{ij}) \in M_{p\times q}(\cp{K})$, tal que, para cada par $(i,j)$ temos $c_{ij} = a_{ij} + b_{ij}$, ou seja,
  \begin{align*}
    C = A + B &= \begin{pmatrix}
    a_{11} & a_{12} & \cdots & a_{1q}\\
    a_{21} & a_{22} & \cdots & a_{2q}\\
    \vdots & & & \vdots\\
    a_{p1} & a_{p2} & \cdots & a_{pq}
  \end{pmatrix} + \begin{pmatrix}
    b_{11} & b_{12} & \cdots & b_{1q}\\
    b_{21} & b_{22} & \cdots & b_{2q}\\
    \vdots & & & \vdots\\
    b_{p1} & b_{p2} & \cdots & b_{pq}
  \end{pmatrix}\\ &= \begin{pmatrix}
    a_{11} + b_{11} & a_{12} + b_{12} & \cdots & a_{1q} + b_{1q}\\
    a_{21} + b_{21} & a_{22} + b_{22}& \cdots & a_{2q} + b_{1q}\\
    \vdots & & & \vdots\\
    a_{p1} + b_{p1} & a_{p2} + b_{p2}& \cdots & a_{pq} + b_{pq}
  \end{pmatrix}
  \end{align*}

    \item \textit{Diferença de matrizes:}\index{Matrizes!diferença} Se $A = (a_{ij})$, $B = (b_{ij}) \in M_{p\times q}(\cp{K})$, então a
        \textbf{diferença} $A - B$ é a matriz $C = (c_{ij}) \in M_{p\times q}(\cp{K})$, tal que, para cada par $(i,j)$ temos $c_{ij} =
        a_{ij} - b_{ij}$, ou seja,
  \begin{align*}
    C = A - B &= \begin{pmatrix}
    a_{11} & a_{12} & \cdots & a_{1q}\\
    a_{21} & a_{22} & \cdots & a_{2q}\\
    \vdots & & & \vdots\\
    a_{p1} & a_{p2} & \cdots & a_{pq}
  \end{pmatrix} - \begin{pmatrix}
    b_{11} & b_{12} & \cdots & b_{1n}\\
    b_{21} & b_{22} & \cdots & b_{2n}\\
    \vdots & & & \vdots\\
    b_{p1} & b_{p2} & \cdots & b_{pq}
  \end{pmatrix}\\ &= \begin{pmatrix}
    a_{11} - b_{11} & a_{12} - b_{12} & \cdots & a_{1q} - b_{1q}\\
    a_{21} - b_{21} & a_{22} - b_{22}& \cdots & a_{2q} - b_{1q}\\
    \vdots & & & \vdots\\
    a_{p1} - b_{p1} & a_{p2} - b_{p2}& \cdots & a_{pq} - b_{pq}
  \end{pmatrix}
  \end{align*}
\end{enumerate}

\begin{exemplo}
    Sejam
    \[
        A = \begin{pmatrix}1 & \phantom{-}2 & 1/3\\0 & -3 & 2/3\end{pmatrix}\\
        \mbox{e}\\
        B = \begin{pmatrix}2 & -3 & 3\\0 & \phantom{-}2 & 2/3\end{pmatrix}.
    \]
    Então
    \begin{align*}
        A + B &= \begin{pmatrix} 3 & -1 & 10/3\\0 & -1 & 4/3\end{pmatrix}\\
        A - B &= \begin{pmatrix} -1 & \phantom{-}5 & -8/3\\ \phantom{-}0 & -5 & \phantom{-}0\end{pmatrix}
    \end{align*}
\end{exemplo}

\begin{proposicao}
  Dadas matrizes $A$, $B$ e $C \in M_{p\times q}(\cp{K})$ vale:
  \begin{enumerate}[label={\roman*})]
    \item $(A + B) + C = A + (B + C)$
    \item $A + B = B + A$
        \item $0_{p\times q} + A = A = A + 0_{p\times q}$
        \item $A - A = 0_{p\times q}$.
  \end{enumerate}
\end{proposicao}

\begin{observacao}
    Os elementos de um corpo $\cp{K}$ serão chamados de \textbf{escalares}.
\end{observacao}

\begin{definicao}
    Seja $A = (a_{ij}) \in M_{p\times q}(\cp{K})$ e $\lambda \in \cp{K}$ definimos o \textbf{produto escalar}\index{Matriz!produto escalar} de $\lambda$ por $A$ como sendo a matriz $B = (b_{ij}) \in M_{p\times q}(\cp{K})$ onde para cada para $(i,j)$ temos $b_{ij} = \lambda a_{ij}$:
  \begin{align*}
    \lambda A = \lambda \begin{pmatrix}
    a_{11} & a_{12} & \cdots & a_{1q}\\
    a_{21} & a_{22} & \cdots & a_{2q}\\
    \vdots & & & \vdots\\
    a_{p1} & a_{p2} & \cdots & a_{pq}
  \end{pmatrix} = \begin{pmatrix}
    \lambda a_{11} & \lambda a_{12} & \cdots & \lambda a_{1q}\\
    \lambda a_{21} & \lambda a_{22} & \cdots & \lambda a_{2q}\\
    \vdots & & & \vdots\\
    \lambda a_{p1} & \lambda a_{p2} & \cdots & \lambda a_{pq}
  \end{pmatrix}.
  \end{align*}
\end{definicao}
\begin{exemplo}
    Seja $\lambda = 2 \in \real$ e
    \[
        A = \begin{bmatrix}
            1 & \sqrt{2} & \phantom{-}1/2\\
            \pi & 2 & -1
        \end{bmatrix}.
    \]
    Então
    \begin{align*}
        \lambda A = 2A = \begin{bmatrix}
            2 & 2\sqrt{2} & \phantom{-}1\\
            2\pi & 4 & -2
        \end{bmatrix}
    \end{align*}
\end{exemplo}

\begin{proposicao}
    Sejam $A \in M_{p\times q}(\cp{K})$ e  $B \in M_{p\times q}(\cp{K})$ e $\alpha$, $\lambda \in \cp{K}$, então:
  \begin{enumerate}[label={\roman*})]
    \item $(\alpha\lambda) A = \alpha(\lambda A) = \lambda(\alpha A)$
    \item $(\alpha + \lambda)A = \alpha A + \lambda A$
        \item $\alpha(A + B) = \alpha A + \alpha B$
        \item $1A = A$ e $(-1)A = -A$
        \item  $0A = 0_{p\times q}$
  \end{enumerate}
\end{proposicao}

\begin{observacao}
    Dadas matrizes $A_1$, $A_2$, \dots, $A_l \in M_{p\times q}(\cp{K})$ e escalares $\lambda_1$, $\lambda_2$, \dots, $\lambda_l \in \cp{K}$
    podemos definir a soma
    \[
        \lambda_1 A_1 + \lambda_2 A_2 + \cdots + \lambda_l A_l
    \]
    que é chamada de \textbf{combinação linear} de $A_1$, $A_2$, \dots, $A_l$ com os escalares $\lambda_1$, $\lambda_2$, \dots, $\lambda_l$. Por exemplo, tomando
    \[
        A_1 = \begin{pmatrix}1\\2\\3\end{pmatrix}, A_2 = \begin{pmatrix}-1\\\phantom{-}0\\-3\end{pmatrix}, A_3 = \begin{pmatrix}-1\\\phantom{-}2\\\phantom{-}3\end{pmatrix}
    \]
    e $\lambda_1 = 2$, $\lambda_2 = -1$ e $\lambda_3 = 3$ obtemos
    \begin{align*}
        \lambda_1 A_1 + \lambda_2 A_2 + \lambda_3 A_3 &= 2A_1 + (-1)A_2 + 3A_3\\ &= 2\begin{pmatrix}1\\2\\3\end{pmatrix} + (-1)
        \begin{pmatrix}-1\\\phantom{-}0\\-3\end{pmatrix} +3\begin{pmatrix}-1\\\phantom{-}2\\\phantom{-}3\end{pmatrix}\\ &= \begin{pmatrix}2 + 1 - 3\\4 + 0 + 6\\6 + 3 +
        9\end{pmatrix} = \begin{pmatrix}-2\\\phantom{-}10\\\phantom{-}18\end{pmatrix}
    \end{align*}
\end{observacao}

\begin{definicao}
    Seja $A = (a_{ij})_{p\times q}$ uma matriz de ordem $p\times q$. A \textbf{matriz transposta}\index{Matriz!transposta} é a matriz
    denotada por $A^T = (b_{ij})$ com $b_{ij} = a_{ji}$ para todo $1 \le i \le p$, $1 \le j \le q$, de ordem $q \times p$.
\end{definicao}

\begin{exemplos}
    \begin{enumerate}[label={\arabic*})]
        \item Seja $A$ uma matriz qualquer, de ordem $2 \times 3$.
            Assim escrevendo
            \[
                A = \begin{pmatrix}
                        a_{11} & a_{12} & a_{13}\\
                        a_{21} & a_{22} & a_{23}
                    \end{pmatrix}
            \]
            então a \textbf{matriz transposta} de $A$ será
            \[
                A^T = \begin{pmatrix}
                        a_{11} & a_{21}\\
                        a_{12} & a_{22}\\
                        a_{13} & a_{23}
                    \end{pmatrix}
            \]
            que é uma matriz de ordem $3 \times 2$.

        \item Seja $B$ uma matriz qualquer, de ordem $3 \times 3$.
            Assim escrevendo
            \[
                B = \begin{pmatrix}
                        b_{11} & b_{12} & b_{13}\\
                        b_{21} & b_{22} & b_{23}\\
                        b_{31} & b_{32} & b_{33}
                    \end{pmatrix}
            \]
            então a \textbf{matriz transposta} de $B$ será
            \[
                B^T = \begin{pmatrix}
                        b_{11} & b_{21} & b_{31}\\
                        b_{12} & b_{22} & b_{32}\\
                        b_{13} & b_{23} & b_{33}
                    \end{pmatrix}
            \]
            que é uma matriz de ordem $3 \times 3$.

        \item Se $A = \begin{pmatrix}1 \\ 3\\ 0\end{pmatrix}$ é uma matriz $3 \times 1$, então
            \[
                A^T = \begin{pmatrix}1 & 3 & 0\end{pmatrix}
            \]
            é uma matriz $1 \times 3$.

        \item Seja $B = \begin{pmatrix}1 & -2\\3 & \sqrt{3}\end{pmatrix}$, então
            \[
                B^T = \begin{pmatrix}
                    \phantom{-}1 & 3\\
                    -2 & \sqrt{3}
                \end{pmatrix}
            \]
    \end{enumerate}
\end{exemplos}


\begin{proposicao}
    Sejam $A$, $B \in M_{p\times q}(\cp{K})$ e $\alpha \in \cp{K}$. Então:
    \begin{enumerate}[label={\roman*})]
        \item $(A^T)^T = A$

        \item $(\alpha A)^T = \alpha(A^T)$

        \item $(A + B)^T = A^T + B^T$

        \item $(A - B)^T = A^T - B^T$
    \end{enumerate}
\end{proposicao}

\begin{definicao}
    Seja $A \in M_n(\cp{K})$. Definimos o \textbf{traço}\index{Matriz!traço} da matriz $A$ como o escalar dado por
    \[
        \tr{A} = a_{11} + a_{22} + \cdots + a_{nn},
    \]
    isto é, o traço de $A$ é a \textbf{soma} dos elementos da \textbf{diagonal princial} de $A$.
\end{definicao}

\begin{proposicao}
    Dadas matrizes $A$, $B \in M_n(\cp{K})$ e $\lambda \in \cp{K}$, então
    \begin{enumerate}[label={\roman*})]
        \item $\tr{A^T} = \tr{A}$

        \item $\tr{\lambda A} = \lambda\tr{A}$

        \item $\tr{A + B} = \tr{A} + \tr{B}$
    \end{enumerate}
\end{proposicao}

\begin{definicao}
    Seja $A = (a_{ij})\in M_n(\cp{K})$.
    \begin{enumerate}[label={\roman*})]
        \item Dizemos que $A$ é uma \textbf{matriz simétrica}\index{Matriz!simétrica} se $A = A^T$, ou seja, se $a_{ij} = a_{ji}$ para todos $i, j$.
        \item Dizemos que $A$ é uma \textbf{matriz anti-simétrica}\index{Matriz!anti-simétrica} se $A^T = -A$, ou seja, se $a_{ji} =
            -a_{ij}$ para todos $i, j$. Neste caso devemos ter $a_{ii} = 0$ para $1 \le i \le n$.
    \end{enumerate}
\end{definicao}

\begin{exemplos}
    \begin{enumerate}[label={\arabic*})]
        \item A matriz
            \[
                A = \begin{pmatrix}
                    1 & 2\\
                    2 & 1
                \end{pmatrix}
            \]
            é uma matriz simétrica pois
            \[
                A^T = \begin{pmatrix}
                    1 & 2\\
                    2 & 1
                \end{pmatrix} = A.
            \]

        \item A matriz
            \[
                B = \begin{pmatrix}
                    \phantom{-}1 & \phantom{-}0 & -2\\
                    \phantom{-}0 & -2 & \phantom{-}3\\
                    -2 & \phantom{-}3 & \phantom{-}4
                \end{pmatrix}
            \]
            é uma matriz simétrica pois
            \[
                B^T = \begin{pmatrix}
                    \phantom{-}1 & \phantom{-}0 & -2\\
                    \phantom{-}0 & -2 & \phantom{-}3\\
                    -2 & \phantom{-}3 & \phantom{-}4
                \end{pmatrix} = B
            \]

        \item A matriz
            \[
                C = \begin{pmatrix}
                   \phantom{-} 0 & \phantom{-}1 & -2\\
                    -1 & \phantom{-}0 & \phantom{-}3\\
                    \phantom{-}2 & -3 & \phantom{-}0
                \end{pmatrix}
            \]
            é uma matriz anti-simétrica pois
             \[
                C^T = \begin{pmatrix}
                    \phantom{-}0 & -1 & \phantom{-}2\\
                    \phantom{-}1 & \phantom{-}0 & -3\\
                    -2 & \phantom{-}3 & \phantom{-}0
                \end{pmatrix} = -C
            \]

        \item A matriz
            \[
                D = \begin{pmatrix}
                    0 & -2\\
                    2 & \phantom{-}0
                \end{pmatrix}
            \]
            é uma matriz anti-simétrica pois
            \[
                D^T = \begin{pmatrix}
                    \phantom{-}0 & 2\\
                    -2 & 0
                \end{pmatrix} = -D.
            \]

        \item A matriz
            \[
                E = \begin{pmatrix}
                    1 & 0 & -1 & 5\\
                    2 & 0 & \phantom{-}1 & 3\\
                    4 & 0 & -1 & 5\\
                    1 & 0 & -1 & 5
                \end{pmatrix}
            \]
            não é nem uma matriz simétrica e nem uma matriz anti-simétrica pois
            \[
                E^T = \begin{pmatrix}
                    \phantom{-}1 & \phantom{-}2 & \phantom{-}4 & \phantom{-}1\\
                    \phantom{-}0 & \phantom{-}0 & \phantom{-}0 & \phantom{-}0\\
                    -1 & \phantom{-}1 & -1 & -1\\
                    \phantom{-}5 & \phantom{-}3 & \phantom{-}5 & \phantom{-}5
                \end{pmatrix}
            \]
            e nesse caso $E \ne E^T$ e $E^T  \ne -E$.
    \end{enumerate}
\end{exemplos}


\begin{definicao}
  Sejam $A = (a_{ij}) \in M_{p\times q}(\cp{K})$ e $B = (b_{ij}) \in M_{q\times r}(\cp{K})$, definimos o \textbf{produto de $A$ por
    $B$}\index{Matrizes!produto de} como sendo a matriz $C = (c_{ij}) \in M_{p\times r}(\cp{K})$ tal que
  \[
        c_{ij} = a_{i1}b_{1j} + a_{i2}b_{2j} + \cdots + a_{iq}b_{qj},
  \]
  para $i = 1$, \dots, $p$ e $j = 1$, \dots, $r$. Isto é,
  \begin{align*}
    A\cdot B &= \begin{pmatrix}
    a_{11} & a_{12} & \cdots & a_{1q}\\
    a_{21} & a_{22} & \cdots & a_{2q}\\
    \vdots & & & \vdots\\
    a_{p1} & a_{p2} & \cdots & a_{pq}
  \end{pmatrix}\cdot \begin{pmatrix}
    b_{11} & b_{12} & \cdots & b_{1r}\\
    b_{21} & b_{22} & \cdots & b_{2r}\\
    \vdots & & & \vdots\\
    b_{q1} & b_{q2} & \cdots & b_{qr}
        \end{pmatrix} \\ &= \begin{pmatrix}
        a_{11}b_{11} + a_{12}b_{21} + \cdots + a_{1q}b_{q1} & a_{11}b_{12} + a_{12}b_{22} + \cdots + a_{1q}b_{q2} & \dots & a_{11}b_{1r} +
        a_{12}b_{2r} + \\ & & & \cdots + a_{1r}b_{qr}\\
        a_{21}b_{11} + a_{22}b_{21} + \cdots + a_{2q}b_{q1} & a_{21}b_{12} + a_{22}b_{22} + \cdots + a_{2q}b_{q2} & \dots & a_{21}b_{1r} +
        a_{12}b_{2r} +\\ & & & \cdots + a_{1q}b_{qr}\\
        \vdots & \vdots & \cdots & \vdots\\
         a_{p1}b_{11} + a_{p2}b_{21} + \cdots + a_{pq}b_{n1} & a_{p1}b_{12} + a_{p2}b_{22} + \cdots + a_{pq}b_{q2} & \dots & a_{p1}b_{1r} +
         a_{p2}b_{2r} + \\ & & & \cdots + a_{pq}b_{qr}\\
        \end{pmatrix}
  \end{align*}
\end{definicao}
\begin{exemplos}
    \begin{enumerate}[label={\arabic*})]
        \item Considere as matrizes
            \[
                A = \begin{pmatrix}
                    1 & \phantom{-}2 & -3\\
                    0 & -1 & -2
                \end{pmatrix},
                B = \begin{pmatrix}
                    \phantom{-}3 & 1\\
                    -2 & 2\\
                    \phantom{-}4 & 0
                \end{pmatrix}.
            \]
            Como $A$ é uma matriz $2 \times 3$ e $B$ é uma matriz $3 \times 2$, então o produto $AB$ está definido e o resultado será uma
            matriz $C$ de ordem $2 \times 2$.
            \begin{align*}
                C &= AB = \begin{pmatrix}
                    1 & \phantom{-}2 & -3\\
                    0 & -1 & -2
                \end{pmatrix}
                \begin{pmatrix}
                    \phantom{-}3 & 1\\
                    -2 & 2\\
                    \phantom{-}4 & 0
                \end{pmatrix} \\ &=
                \begin{pmatrix}
                    1\cdot 3 + 2\cdot(-2) + (-3)\cdot4 & 1\cdot 1 + 2\cdot 2 + (-3)\cdot 0\\
                    0\cdot 3 + (-1)\cdot(-2) + (-2) \cdot 4 & 0\cdot 1 + (-1)\cdot 2 + (-2)\cdot 0
                    \end{pmatrix}\\ &= \begin{pmatrix}
                        -13 & 5\\
                        -5 & -2
                    \end{pmatrix}
                \end{align*}
        \item Considere as matrizes
            \[
                A = \begin{pmatrix}
                    -2 & \phantom{-}2\\
                    \phantom{-}2 & -1\\
                    \phantom{-}1/2 & \phantom{-}3
                \end{pmatrix},
                B = \begin{pmatrix}
                    \phantom{-}3 & -1 & \phantom{-}1 &2\\
                    -2 & \phantom{-}1 & -3 & 2
                \end{pmatrix}.
            \]
            Como $A$ é uma matriz $3 \times 2$ e $B$ é uma matriz $2 \times 4$, então o produto $AB$ está definido e o resultado será uma
            matriz $C$ de ordem $3 \times 4$.
            \begin{align*}
                C &= AB = \begin{pmatrix}
                    -2 & \phantom{-}2\\
                    \phantom{-}2 & -1\\
                    \phantom{-}1/2 & \phantom{-}3
                \end{pmatrix}
                \begin{pmatrix}
                    \phantom{-}3 & -1 & \phantom{-}1 & 2\\
                    -2 & \phantom{-}1 & -3 & 2
                \end{pmatrix} \\ &=
                \begin{pmatrix}
                    -2\cdot 3 + 2\cdot(-2) & -2\cdot (-1) + 2\cdot 2 & -2\cdot 1 + 2\cdot (-3) & -2\cdot 2 + 2\cdot 2\\
                    2\cdot 3 + (-1)\cdot(-2) & 2\cdot(-1) + (-1) \cdot 1 & 2\cdot 1 + (-1)\cdot(-3) & 2\cdot 2 + (-1)\cdot 2\\
                    1/2\cdot(3) + 3\cdot(-2) & 1/2\cdot(-1) + 3\cdot 1 & 1/2\cdot 1 + 3\cdot(-3) & 1/2\cdot 2 + 3\cdot 2
                    \end{pmatrix}\\ &=
                    \begin{pmatrix}
                        -10 & \phantom{-}2 & -8 & 0\\
                        \phantom{-}8 & -3 & \phantom{-}5 & 2\\
                        -15/2 & \phantom{-}5/2 & -17/2 & 7
                    \end{pmatrix}
                \end{align*}
    \end{enumerate}
\end{exemplos}

\begin{observacoes}
    \begin{enumerate}[label={\arabic*})]
        \item Dadas duas matrizes $A$ e $B$ quaisquer, nem sempre o produto $AB$ ou $BA$ estarão definidos. Mesmo quando $AB$ existe, $BA$
            pode não existir ou o contrário. Por exemplo, tome
            \[
                A = \begin{pmatrix}
                    1 & 2 & 3\\
                    3 & 0 & 1\\
                    9 & 2 & 0
                \end{pmatrix},
                B = \begin{pmatrix}
                    9 & \pi
                \end{pmatrix}
            \]
            Neste caso $A$ é uma matriz de ordem $3\times 3$ e $B$ é uma matriz $2 \times 1$. Desse modo, nem $AB$ e nem $BA$ estão
            definidas.
            Agora tomando
            \[
                C = \begin{pmatrix}
                    1 & -2 & 2
                \end{pmatrix}
            \]
            o produto $AC$ não está definido, mas o produto $CA$ está e
            \begin{align*}
                CA = \begin{pmatrix}
                    1 & -2 & 2
                \end{pmatrix}
                \begin{pmatrix}
                    1 & 2 & 3\\
                    3 & 0 & 1\\
                    9 & 2 & 0
                    \end{pmatrix} = \begin{pmatrix}
                    1\cdot 1 + (-2)\cdot 3 + 2\cdot 9\\
                    1\cdot 2 + (-2)\cdot 0 + 2\cdot 2\\
                    1\cdot 3 + (-2)\cdot 1 + 2\cdot 0
                \end{pmatrix}
                = \begin{pmatrix}
                    13\\
                    4\\
                    1
                \end{pmatrix}
            \end{align*}

        \item Dadas duas matrizes $A$ e $B$ tais que $AB$ e $BA$ estejam definidas. Em geral, $AB \ne BA$. Por exemplo, tome
            \[
                A = \begin{pmatrix}\phantom{-}1& \phantom{-}2\\-1 & -2\end{pmatrix},
                B = \begin{pmatrix}3 & 1 \\ 1 & 2\end{pmatrix}.
            \]
            Assim os produtos $AB$ e $BA$ estão definidos, ambos resultam em uma matriz de ordem $2 \times 2$. Mas veja que
            \begin{align*}
                AB &= \begin{pmatrix}\phantom{-}1 & \phantom{-}2\\-1 & -2\end{pmatrix}\begin{pmatrix}3 & 1\\1 & 2\end{pmatrix} =
                \begin{pmatrix}1\cdot 3 + 2 \cdot 1 & 1\cdot 1 + 2 \cdot 2\\-1\cdot3 + (-2)\cdot 2 & -1\cdot 1+ (-2)\cdot 2\end{pmatrix} =
                \begin{pmatrix}\phantom{-}5 & \phantom{-}5\\-7 & -5\end{pmatrix}\\
                BA &=\begin{pmatrix}3 & 1\\1 & 2\end{pmatrix}\begin{pmatrix}\phantom{-}1 & \phantom{-}2\\-1 & -2\end{pmatrix} = \begin{pmatrix}3\cdot 1 +
                1\cdot(-1) & 3\cdot 2 + 1\cdot(-1)\\1\cdot 1 + 2\cdot(-1) & 1\cdot 2 + 2 \cdot(-2)\end{pmatrix} =
                \begin{pmatrix}\phantom{-}2 & \phantom{-}5\\-1 & -3\end{pmatrix},
            \end{align*}
            logo $AB \ne BA$.
        \item Se $A$ e $B$ são matrizes tais que $AB = 0$, não é verdade que $A = 0$ ou $B = 0$. Por exemplo, tome
    \[
      A = \begin{pmatrix}1 & -1\\1 & -1\end{pmatrix} \ne 0_{2\times 2}\\
      B = \begin{pmatrix}2 & -2\\2 & -2\end{pmatrix} \ne 0_{2 \times 2}.
    \]
  Temos
  \begin{align*}
    AB = \begin{pmatrix}1 & -1\\1 & -1\end{pmatrix}\begin{pmatrix}2 & -2\\2 & -2\end{pmatrix} = \begin{pmatrix}1\cdot 2 + (-1)\cdot 2 & 1 \cdot (-2) + (-1)\cdot(-2)\\ 1\cdot 2 + (-1)\cdot 2 & 1 \cdot (-2) + (-1)\cdot (-2)\end{pmatrix} = \begin{pmatrix}0 & 0\\0 & 0\end{pmatrix}.
  \end{align*}

        \item Se $A$, $B$ e $C$ são matrizes tais que $AB = AC$, não é verdade que $B = C$. Isto é, não vale a lei do cancelamento no produto. Por exemplo, tome
  \[
    A = \begin{pmatrix}0 & 1\\0 & 1\end{pmatrix}\\
    B = \begin{pmatrix}1 & 1\\0 & 0\end{pmatrix}\\
    C = \begin{pmatrix}3 & 3\\0 & 0\end{pmatrix}.
  \]
  Temos
  \[
    AB = AC = \begin{pmatrix}0 & 0\\0 & 0\end{pmatrix}
  \]
  mas $B \ne C$.
    \end{enumerate}
\end{observacoes}

\begin{proposicao}
  Sejam $A$, $B$ e $C$ matrizes de ordens tais que as operações a seguir estejam definidas. Entao:
  \begin{enumerate}[label={\roman*})]
    \item $(A\cdot B)\cdot C = A\cdot(B \cdot C)$
    \item $(A + B)\cdot C = (A\cdot C) + (B\cdot C)$
    \item $C\cdot(A + B) = (C\cdot A) + (C\cdot B)$
    \item $\alpha(A\cdot B) = (\alpha A)\cdot B = A \cdot(\alpha B)$
    \item Se $A$ é uma matriz quadrada de ordem $n$, então
      \[A\cdot I = A = I\cdot A.\]
    \item Se $A$ é uma matriz quadrada de ordem $n$, então
      \[A\cdot0_n = 0_n = 0_n\cdot A.\]
    \item $(AB)^T = B^TA^T$.
  \end{enumerate}
\end{proposicao}
