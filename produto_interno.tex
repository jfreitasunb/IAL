%!TEX program = xelatex
%!TEX root = IAL.tex

\chapter{Espa\c{c}os com produto interno}

Durante este cap{\'\i}tulo o corpo base $\cp{K}$ de um espa\c{c}o vetorial $V$ ser\'a sempre $\real$ ou $\complex$.

\begin{definicao}
	Seja $V$ um $\cp{K}$-espa\c{c}o vetorial. Um \textbf{produto interno} sobre $V$ \'e uma fun\c{c}\~ao $\langle\ ,\ \rangle : V \times V \to \cp{K}$ que satisfaz as propriedades seguintes:\index{Produto interno}
	\begin{enumerate}[label={\roman*})]
		\item $\langle u + v , w\rangle = \langle u, w\rangle + \langle v, w\rangle$ para todos $u$, $v$ e $w \in V$.
		\item $\langle \lambda u, w\rangle = \lambda \langle u, w\rangle$ para todo $\lambda \in \cp{K}$ e todos $u$, $w \in V$.
		\item $\langle u, w\rangle = \overline{\langle w, u\rangle}$ para todos $u$, $w \in V$. (Aqui, $\overline{a + bi} = a - bi$.)
		\item $\langle u, u\rangle > 0_\cp{K}$ se $u \ne 0_V$.
	\end{enumerate}
\end{definicao}

\begin{observacao}
	Seja $V$ um $\cp{K}$-espa\c{c}o vetorial com produto interno $\langle\ ,\ \rangle$. Temos
	\begin{enumerate}[label={\arabic*})]
		\item $\langle 0_V, w\rangle = \langle w, 0_V\rangle = 0_\cp{K}$ para todo $w \in V$.

		De fato,
		\[
			\langle 0_V, w\rangle = \langle 0_V + 0_V, w\rangle = \langle 0_V, w\rangle + \langle 0_V, w\rangle
		\]
		logo $\langle 0_V, w\rangle = 0_\cp{K}$. Como $\langle w, 0_V\rangle = \overline{\langle 0_V, w\rangle}$, ent\~ao $\langle w, 0_V\rangle = 0_\cp{K}$.
		\item $\langle w, w\rangle = 0_\cp{K}$ se, e somente se, $w = 0_V$.
		\item No caso em $\cp{K} = \real$, a propriedade 3 da defini\c{c}\~ao de produto interno torna-se: $\langle u, w\rangle = \langle w, u\rangle$.
	\end{enumerate}
\end{observacao}

\begin{exemplo}
	\begin{enumerate}[label={\arabic*})]
		\item Em $\real^3$ a fun\c{c}\~ao $\langle\ ;\ \rangle : \real^3\times\real^3 \to \real$ dada por
		\[
			\langle (a, b, c); (x, y , z)\rangle = ax + by + cz
		\]
		\'e um produto interno.
		\begin{solucao}
			De fato, dados $(a, b, c)$, $(x, y, z)$, $(u,v,w) \in \real^3$ e $\lambda \in \real$ temos:
			\begin{enumerate}[label={\roman*})]
				\item $\langle (a, b, c) + (x, y, z); (u, v, w)\rangle = \langle (a + x, b + y, c+ z); (u, v, w)\rangle = (a + x)u + (b + y)v + (c + z)w = \langle (a, b, c); (u, v, w)\rangle + \langle (x, y, z); (u, v, w)\rangle$.
				\item $\langle \lambda(a, b, c), (x, y, z)\rangle = \langle (\lambda a, \lambda b, \lambda c), (x, y, z)\rangle = \lambda a x + \lambda b y + \lambda cz = \lambda\langle (a, b, c); (x, y, z)\rangle$
				\item $\langle (a, b, c); (x, y, z)\rangle = ax + by + cz = \langle (x, y, z), (a, b, c)\rangle$.
				\item $\langle (a, b, c); (a, b, c)\rangle = a^2 + b^2 + c^2 > 0$ para todo $(a, b, c) \ne (0, 0, 0)$.
			\end{enumerate}
		\end{solucao}
		\item Em $\complex^3$ a fun\c{c}\~ao $\langle\ ;\ \rangle : \complex^3\times\complex^3 \to \complex$ dada por
		\[
			\langle (a, b, c); (x, y , z)\rangle = ax + by + cz
		\]
		n\~ao \'e um produto interno.
		\begin{solucao}
			De fato, por exemplo
			\[
				\langle (i, 0, 0); (i, 0 , 0)\rangle = i^2 = -1 < 0.
			\]
		\end{solucao}
		\item Em $\complex^3$ a fun\c{c}\~ao $\langle\ ;\ \rangle : \complex^3\times\complex^3 \to \complex$ dada por
		\[
			\langle (a, b, c); (x, y , z)\rangle = a\overline{x} + b\overline{y} + c\overline{z}
		\]
		\'e um produto interno.
		\item Se $V = \real^2$ ent\~ao $\langle\ ,\ \rangle : \real^2 \times\real^2 \to \real$ dado por
		\[
			\langle (a,b); (x,y) \rangle = 2ax - ay - bx + by
		\]
		\'e um produto interno.
		\item De modo geral, em $V = \cp{K}^n$ a fun\c{c}\~ao definida por
		\[
			\langle (x_1, \dots, x_n) ; (y_1, \dots, y_n) \rangle = x_1\overline{y_1} + \cdots + x_n\overline{y_n}
		\]
		\'e um produto interno em $\cp{K}^n$. Tal produto interno \'e chamada de \textbf{produto interno can\^onico} em $\cp{K}^n$.
		\item Considere o $\cp{K}$-espa\c{c}o vetorial $V = \mathcal{C}([a, b], \cp{K})$ o espa\c{c}o das fun\c{c}\~oes cont{\'\i}nuas de $[a,b]$ em $\cp{K}$. Defina
   		 \[
        	\langle f , g \rangle = \int_a^bf(t)\overline{g(t)}dt
    	\]
    	para $f$, $g \in V$. Tal fun\c{c}\~ao \'e um produto interno em $V$ e \'e chamado de \textbf{produto interno can\^onico} em $\mathcal{C}([a, b], \cp{K})$.
    	\item O \textbf{produto interno can\^onico} em $V = \cp{M}_n(\cp{K})$ \'e dado por
    	\[
    		\langle A, B \rangle = \sum_{i, j = 1}^n a_{ij}\overline{b_{ij}}
    	\]
    	onde $A = (a_{ij})_{n \times n}$ e $B = (b_{ij})_{n \times n}$.
	\end{enumerate}
\end{exemplo}

\begin{definicao}
	Seja $V$ um $\cp{K}$-espa\c{c}o vetorial munido de um produto interno $\langle\ ,\ \rangle$. Para cada $u \in V$, chamados de \textbf{norma} de $u$ ao n\'umero real dado por
	\[
		\norma{u} = \sqrt{\langle u, u \rangle}.
	\]
\end{definicao}

\begin{observacao}
	Segue diretamente da defini\c{c}\~ao de norma que:
	\begin{enumerate}
		\item $\norma{u} \ge 0$ para todo $u \in V$.
		\item $\norma{u} = 0$ se, e somente se, $u = 0_V$.
		\item $\norma{\alpha u} = |\alpha|\norma{u}$ para todo $\alpha \in \cp{K}$ e todo $u \in V$.
	\end{enumerate}
\end{observacao}

\begin{exemplo}
	\begin{enumerate}[label={\arabic*})]
		\item Em $\real^3$ considere o produto interno can\^onico. Dados $u = (a, b, c)$ e $v = (x, y, z)$ temos
		\[
			\norma{u - v} = \norma{(a - x, b - y, c - z)} = \sqrt{(a - x)^2 + (b - y)^2 + (c - z)^2}
		\]
		indicar\'a a dist\^ancia entre $u$ e $v$.
		\item A norma de um vetor depende do produto interno escolhido. Por exemplo, em $\real^2$ se considerarmos o produto interno dado por
		\[
			\langle (x_1, x_2); (y_1, y_1)\rangle_1 = 2x_1y_1 + 25x_2y_2.
		\]
		Ent\~ao $\norma{(1, 0)}_1 = \langle (1, 0); (1, 0)\rangle_1 = \sqrt{2}$ e $\norma{(0, 1)}_1 = \langle (0, 1); (0, 1)\rangle_1 = 5$.

		Agora, se considerarmos o produto interno can\^onico de $\real^2$, temos $\norma{(1, 0)} = \norma{(0, 1)} = 1$.
	\end{enumerate}
\end{exemplo}

\begin{definicao}
	Seja $V$ um $\cp{K}$-espa\c{c}o vetorial munido de um produto interno $\langle\ , \rangle$.
	\begin{enumerate}[label={\roman*})]
		\item Dizemos que $u$ e $w$ s\~ao vetores \textbf{ortogonais} se $\langle u, w \rangle = 0$.
		\item Um subconjunto $\mathcal{A}$ de $V$ \'e chamado de \textbf{ortogonal} se os seus elementos s\~ao ortogonais dois a dois.
		\item Dizemos que um subconjunto $\mathcal{A}$ de $V$ \'e \textbf{ortonormal} se for um conjunto ortogonal e se $\norma{u} = 1$ para todo $u \in \mathcal{A}$.
	\end{enumerate}
\end{definicao}

\begin{notacao}
	Usaremos a nota\c{c}\~ao $u \perp w$ para indicar que os vetores $u$ e $v$ s\~ao ortogonais.
\end{notacao}

\begin{observacao}
	O vetor nulo $0_V$ \'e ortogonal a todos os elementos de $V$ pois $\langle 0_V, u \rangle = 0$ para todo $u \in V$. Al\'em disso, o vetor nulo \'e o \'unico vetor com esta propriedade.
\end{observacao}

\begin{exemplo}
	As bases can\^onicas de $\real^n$, $\complex^n$, $\cp{M}_n(\real)$ e $\cp{M}_n(\complex)$ s\~ao conjunto ortonormais, considerando o produto interno can\^onico nestes espa\c{c}os vetoriais.
\end{exemplo}

\begin{proposicao}\label{ortogonalLI}
	Seja $V$ um $\cp{K}$-espa\c{c}o vetorial com produto interno e seja $\mathcal{A}$ um sobconjunto ortogonal de $V$ formado por vetores n\~ao nulos.
	\begin{enumerate}[label={\roman*})]
		\item Se $u \in [v_1, \dots, v_n]$ com $v_i \in \mathcal{A}$ para $i = 1$, \dots, $n$, ent\~ao
		\[
			u = \sum_{i = 1}^n \dfrac{\langle u, v_i \rangle}{\norma{v_i}^2}v_i.
		\]
		\item O conjunto $\mathcal{A}$ \'e L.I.
	\end{enumerate}
\end{proposicao}
\begin{prova}
	\begin{enumerate}[label={\roman*})]
		\item  Seja $u = \alpha_1 v_1 + \cdots + \alpha_n v_n$ com $\alpha_i \in \cp{K}$ para $i = 1$, \dots, $n$. Como $\{v_1, \dots, v_n\}$ \'e ortogonal temos
		\[
			\langle u, v_j \rangle = \langle \sum_{i = 1}^n\alpha_i v_i, v_j \rangle = \sum_{i = 1}^n \alpha_i\langle v_i, v_j \rangle = \alpha_j \langle v_j, v_j \rangle
		\]
		para $j = 1$, \dots, $n$. Portanto
		\[
			\alpha_j = \dfrac{\langle u, v_j \rangle}{\langle v_j, v_j \rangle} = \dfrac{\langle u, v_j \rangle}{\norma{v_j}^2}
		\]
		para $j = 1$, \dots, $n$. Assim
		\[
			u = \sum_{i = 1}^n \dfrac{\langle u, v_i \rangle}{\norma{v_i}^2}v_i.
		\]
		como quer{\'\i}amos.
		\item Suponha que existam escalares $\alpha_1$, \dots, $\alpha_n \in \cp{K}$ e vetores n\~ao nulos $v_1$, \dots, $v_n \in \mathcal{A}$ tais que
		\[
			\alpha_1 v_1 + \cdots + \alpha_n v_n = 0_V.
		\]
		Pelo item (a) temos
		\[
			\alpha_i = \dfrac{\langle 0_V, v_j \rangle}{\norma{v_j}^2} = 0_K
		\]
		para $i = 1$, \dots, $n$. Portanto $\mathcal{A}$ \'e L.I..
	\end{enumerate}
\end{prova}

\begin{corolario}
	Seja $V$ um $\cp{K}$-espa\c{c}o vetorial com produto interno e seja $\mathcal{A}$ uma base ortonormal de $V$. Ent\~ao para $u \in V$ temos
	\[
		u = \sum_{i = 1}^n \langle u, v_i\rangle v_i
	\]
	onde $v_i \in \mathcal{A}$ para $i = 1$, \dots, $n$.
\end{corolario}

\section{Processo de Ortogonaliza\c{c}\~ao de Gram-Schmidt} % (fold)
\label{sec:processo_de_ortogonaliza_cao_de_gram_schmidt}

Seja $V$ um $\cp{K}$-espa\c{c}o vetorial com um produto interno. Considere $\mathcal{A} = \{v_1, \dots, v_n\} \subset V$ um conjunto L.I. Vamos obter um novo conjunto L.I. $\mathcal{B} = \{w_1, \dots, w_n\}$ ortogonal e que gera o mesmo espa\c{c}o vetorial que $\mathcal{A}$. Para tal fazemos:
\begin{enumerate}[label={\roman*})]
	\item $w_1 = v_1$
	\item $w_2 = v_2 - \dfrac{\langle v_2, w_1\rangle}{\norma{w_1}^2}w_1$.

	Como $\{v_1, v_2\}$ \'e L.I., ent\~ao $w_2 \ne 0_V$. Al\'em disso,
	\begin{align*}
		\langle w_2, w_1 \rangle = \left\langle v_2 - \dfrac{\langle v_2, w_1 \rangle}{\norma{w_1}^2}w_1; w_1 \right\rangle = \langle v_2, w_1\rangle - \dfrac{\langle v_2, w_1\rangle}{\norma{w_1}^2}\langle w_1, w_1\rangle = \langle v_2, w_1\rangle - \langle v_2, w_1 \rangle = 0
	\end{align*}
	e ent\~ao $w_1 \perp w_2$.
	\item Definidos $w_1$, \dots, $w_k$, $1 < k < n$ podemos definir $w_{k + 1}$ por
	\begin{align*}
		w_{k + 1} = v_{k + 1} - \dfrac{\langle v_{k + 1}, w_1\rangle}{\norma{w_1}^2}w_1 - \cdots - \dfrac{\langle v_{k + 1}, w_k\rangle}{\norma{w_k}^2}w_k\\
		w_{k + 1} = v_{k + 1} - \sum_{j = 1}^k\dfrac{\langle v_{k + 1}, w_j\rangle}{\norma{w_j}^2}w_j.
	\end{align*}
\end{enumerate}

\'E f\'acil verificar que os vetores $w_1$, \dots, $w_n$ definidos anteriormente s\~ao dois a dois ortogonais, logo $\mathcal{B} = \{w_1, \dots, w_n\}$ \'e um conjunto ortogonal. Mais ainda, pela Proposi\c{c}\~ao \ref{ortogonalLI}, o conjunto $\mathcal{B}$ \'e L.I. e $w_i \in [v_1, \dots, v_n] = U$ para $i = 1$, \dots, $n$. Agora como $\dim_\cp{K} U = n$, segue que $\mathcal{B}$ \'e uma base para $U$, como quer{\'\i}amos.

\begin{exemplo}
	\begin{enumerate}[label={\arabic*})]
		\item Seja $V = \complex^3$ com o produto interno can\^onico. Determinar uma base ortogonal de $\complex^3$ contendo o vetor $(1, 2i, 0)$.
		\begin{solucao}
			Primeiro obtemos uma base de $\complex^3$ contedo $(1, 2i, 0)$. Por exemplo, podemos considerar a base $\{v_1 = (1, 2i, 0); v_2 = (0, 1, 0); v_3 = (0, 0, 1)\}$. Defina $w_1 = v_1 = (1, 2i, 0)$, assim
			\begin{align*}
				w_2 &= v_2 - \dfrac{\langle v_2, w_1\rangle}{\norma{w_1}^2} = (0, 1, 0) - \dfrac{\langle (0, 1, 0), (1, 2i, 0)\rangle}{\norma{(1, 2i, 0)}^2}(1, 2i, 0) = (0, 1, 0) + \dfrac{2i}{5}(1, 2i, 0)\\ w_2 &= \left(\dfrac{2i}{5}, \dfrac{1}{5}, 0\right)\\
				w_3 &= v_3 - \dfrac{\langle v_3, w_1\rangle}{\norma{w_1}^2}w_1 - \dfrac{\langle v_3, w_2\rangle}{\norma{w_2}^2}w_2 = (0, 0, 1) - \dfrac{\langle (0, 0, 1), (1, 2i, 0)\rangle}{\norma{(1, 2i, 0)}^2}(1, 2i, 0) \\ &- \dfrac{\langle (0, 0, 1), (2i/5, 1/5, 0)\rangle}{\norma{(2i/5, 1/5, 0)}^2}(2i/5, 1/5, 0) \\ w_3 &= (0, 0, 1).
			\end{align*}
			Portanto o conjunto
			\[
				\left\{(1, 2i, 0); \left(\dfrac{2i}{5}, \dfrac{1}{5},0\right); (0, 0, 1)\right\}
			\]
			\'e uma base ortogonal de $\complex^3$.
		\end{solucao}
		\item Seja $\{(1, 1, 1); (0, 2, 1); (0, 0, 1)\}$ uma base de $\real^3$. Encontrar uma base ortogonal de $\real^3$, em rela\c{c}\~ao ao produto interno can\^onico.
		\begin{solucao}
			Defina $w_1 = (1, 1, 1)$. Temos
			\begin{align*}
				w_2 &= (0, 2, 1) - \dfrac{\langle (0, 2, 1) , (1, 1, 1)\rangle}{\norma{(1, 1, 1)}^2}(1, 1, 1) = (-1, 1, 0)\\
				w_3 &= (0, 0 ,1) - \dfrac{\langle (0, 0, 1), (-1, 1, 0)\rangle}{\norma{(-1, 1, 0)}^2}(-1, 1, 0) - \dfrac{\langle (0, 0, 1), (1, 1, 1)\rangle}{\norma{(1, 1, 1)}^2}(1, 1, 1)\\ w_3 &= (-1/3, -1/3, 2/3).
			\end{align*}
			Portanto o conjunto $\mathcal{A} = \{(1, 1, 1); (-1, 1, 0); (-1/3, -1/3, 2/3)\}$ \'e uma base ortogonal de $\real^3$.
		\end{solucao}
	\end{enumerate}
\end{exemplo}

\begin{teorema}
	Todo espa\c{c}o vetorial de dimens\~ao finita $n \ge 1$ com produto interno possui uma base ortonormal.
\end{teorema}
\begin{prova}
	Seja $V$ um $\cp{K}$-espa\c{c}o vetorial de dimens\~ao finita $n \ge 1$ com produto interno. Pelo processo de ortogonaliza\c{c}\~ao de Gram-Schmidt, existe uma base ortogonal para $V$. Seja $\{w_1, \dots, w_n\}$ tal base. Assim
	\[
		\left\{\dfrac{w_1}{\norma{w_1}}, \dots, \dfrac{w_n}{\norma{w_n}}\right\}
	\]
	\'e uma base ortonormal, como quer{\'\i}amos.
\end{prova}

Deste teorema temos o seguinte: Seja $V$ um $\cp{K}$-espa\c{c}o vetorial de dimens\~ao finita e com produto interno. Seja $\{w_1, \dots, w_n\}$ uma base ortonormal de $V$. Se
\begin{align*}
	u = \sum_{i = 1}^n \alpha_i w_i\\
	v = \sum_{j = 1}^n \beta_j w_j
\end{align*}
ent\~ao
\begin{align}\label{produtointerno}
	\langle u, v \rangle = \langle \sum_{i = 1}^n \alpha_i w_i, \sum_{j = 1}^n \beta_j w_j \rangle = \sum_{i, j = 1}^n \alpha_i \overline{\beta_j} \langle w_i, w_j \rangle = \sum_{i = 1}^n \alpha_i \overline{\beta_i}.
\end{align}

Deste fato segue que
\begin{corolario}
	Sejam $V$ um $\cp{K}$-espa\c{c}o vetorial de dimens\~ao finita munido de um produto interno e $\mathcal{B} = \{u_1, \dots, u_n\}$ e $\mathcal{A} = \{w_1, \dots, w_n\}$ duas bases ortonormais de $V$. Se $M$ \'e a matriz mudan\c{c}a de base de $\mathcal{B}$ para $\mathcal{A}$, ent\~ao
	\[
		M\overline{M}^t = \overline{M}^tM = I_n.
	\]
\end{corolario}
\begin{prova}
	Seja $M = (\alpha_{ij}) \in \cp{M}_n(\cp{K})$ a matriz mudan\c{c}a de base de $\mathcal{B}$ para $\mathcal{A}$. Ent\~ao para $i$, $j = 1$, \dots, $n$ temos
	\begin{align*}
		v_j = \sum_{k = 1}^n\alpha_{kj}u_k\\
		v_i = \sum_{k = 1}^n\alpha_{ki}u_k.
	\end{align*}
	Agora, de \eqref{produtointerno} temos
	\[
		\delta_{ij} = \langle v_i, v_j \rangle = \sum_{k = 1}^n\alpha_ki\overline{\alpha_{kj}}
	\]
	para cada $1 \le i,\ j \le n$. Logo
	\[
		M\overline{M}^t = \overline{M}^tM = I_n
	\]
	como quer{\'\i}amos.
\end{prova}
% section processo_de_ortogonaliza_cao_de_gram_schmidt (end)