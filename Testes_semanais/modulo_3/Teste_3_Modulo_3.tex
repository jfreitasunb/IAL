%!TEX program = xelatex
%!TEX encoding = UTF-8
%Definições de Ano, Semestre, número e data da avaliação
\def\numeroteste{3}
\def\modulo{3}
\def\dataavaliacao{06/12/2023}

\documentclass[12pt]{exam}

\usepackage{caption}
\usepackage{amssymb}
\usepackage{amsmath,amsfonts,amsthm,amstext}
\usepackage[brazil]{babel}
\usepackage{graphicx}
\graphicspath{{../Pictures/}}
\usepackage{enumitem}
\usepackage{multicol}
\usepackage{answers}
\usepackage[svgnames]{xcolor}
\usepackage{tikz}
\usepackage{ifthen}
\usetikzlibrary{lindenmayersystems}
\usetikzlibrary[shadings]

\def\ano{2023}
\def\semestre{2}
\def\disciplina{Introdução à Álgebra Linear}
\def\nomeabreviado{IAL}
\def\turma{11}

\newcommand{\im}{{\rm Im\,}}
\newcommand{\dlim}[2]{\displaystyle\lim_{#1\rightarrow #2}}
\newcommand{\minf}{+\infty}
\newcommand{\ninf}{-\infty}
\newcommand{\cp}[1]{\mathbb{#1}}
\newcommand{\sub}{\subseteq}
\newcommand{\n}{\mathbb{N}}
\newcommand{\z}{\mathbb{Z}}
\newcommand{\rac}{\mathbb{Q}}
\newcommand{\real}{\mathbb{R}}
\newcommand{\complex}{\mathbb{C}}

\newcommand{\vesp}[1]{\vspace{ #1  cm}}

\newcommand{\compcent}[1]{\vcenter{\hbox{$#1\circ$}}}
\newcommand{\comp}{\mathbin{\mathchoice
        {\compcent\scriptstyle}{\compcent\scriptstyle}
        {\compcent\scriptscriptstyle}{\compcent\scriptscriptstyle}}}
\renewcommand{\sin}{{\rm sen\,}}
\renewcommand{\tan}{{\rm tg\,}}
\renewcommand{\csc}{{\rm cossec\,}}
\renewcommand{\cot}{{\rm cotg\,}}
\renewcommand{\sinh}{{\rm senh\,}}
\newcounter{exercicios}
\setcounter{exercicios}{0}
\newcommand{\questao}{
    \addtocounter{exercicios}{1}
    \noindent{\bf Quest{\~a}o \arabic{exercicios}: }}

\newcommand{\resp}[1]{
    \noindent{\bf Exerc{\'\i}cio #1: }}

\extrafootheight[.25in]{.5in}
\footrule
\lfoot{Teste \numeroteste\ - Módulo \modulo\ - \nomeabreviado\ - Turma \turma\ - \semestre$^o$/\ano}
\cfoot{}
\rfoot{P\'agina \thepage\ de \numpages}

\begin{document}

    \begin{figure}[h]
        \begin{minipage}[c]{1.7cm}
            \includegraphics[width=1.7cm]{unb.pdf}
        \end{minipage}
        \hspace{0pt}
        \begin{minipage}[c]{4in}
            {Universidade de Brasília} \\
            {Departamento de Matemática}
        \end{minipage}
    \end{figure}
    \vesp{-1}
    \hrule
    \begin{center}
        {\Large\bf \disciplina\ - Turma \turma}  \\
        \vesp{0.2} {\large\bf Teste \numeroteste\ - Módulo\ \modulo\ -\ \dataavaliacao}
    \end{center}

    \noindent{\bf \underline{Nome: \hspace{9.8cm} Mat.:\qquad \
            \hspace{2.1cm}}}\\
    \vspace*{.01cm}

    \noindent{\bf \underline{Nome: \hspace{9.8cm} Mat.:\qquad \
            \hspace{2.1cm}}}
    \vspace{.4cm}

    \questao{} Encontre a matriz da transformação linear $G : \real^2 \to \mathcal{P}_2(\real)$ dada por $G(a,b) = ax^2 + bx + (a + b)$.
    Considere $\real^2$ e $\mathcal{P}_2(\real)$ como $\real$-espaços vetoriais, $\mathcal{B}_1 = \{(1,3); (2,5)\}$ e
    $\mathcal{B}_2 = \{2, 1 + x, x^2\}$ bases ordenadas de $\real^2$ e $\mathcal{P}_2(\real)$, respectivamente.

    \vspace*{.5cm}

    \solucao Para encontrar a matriz da transformação $G$ com respeito às bases dadas devemos aplicar a transformação $G$ nos vetores
    de $\mathcal{B}_1$ e escrever o resultado como combinação linear dos vetores da base $\mathcal{B}_2$. Começamos calculando $G$ em
    cada vetor de $\mathcal{B}_1$:
    \begin{align*}
        G(1, 3) = 1x^2 + 3x + 4\\
        G(2, 5) = 2x^2 + 5x + 7
    \end{align*}
    Agora vamos escrever cada um desses resultados como combinação linear dos vetores em $\mathcal{B}_2$:
    \begin{align*}
        x^2 + 3x + 4 &= \alpha 2 + \beta(1 + x) + \gamma x^2\\
        x^2 + 3x + 4 &= \alpha 2 + \beta + \beta x + \gamma x^2\\
        x^2 + 3x + 4 &= (\alpha 2 + \beta) + \beta x + \gamma x^2
    \end{align*}
    daí devemos ter
    \[
        \begin{cases}
            2\alpha + \beta = 4\\
            \beta = 3\\
            \gamma = 1
        \end{cases}
    \]
    logo $\alpha = 1/2$, $\beta = 3$ e $\gamma = 1$.

    Repetindo o procedimento para o segundo vetor:
    \begin{align*}
        2x^2 + 5x + 7 &= \alpha 2 + \beta(1 + x) + \gamma x^2\\
        2x^2 + 5x + 7 &= \alpha 2 + \beta + \beta x + \gamma x^2\\
        2x^2 + 5x + 7 &= (\alpha 2 + \beta) + \beta x + \gamma x^2
    \end{align*}
    daí devemos ter
    \[
        \begin{cases}
            2\alpha + \beta = 7\\
            \beta = 5\\
            \gamma = 2
        \end{cases}
    \]
    logo $\alpha = 1$, $\beta = 5$ e $\gamma = 2$.

    Portanto a matriz de $G$ será:
    \[
        [G]_{\mathcal{B}_1, \mathcal{B}_2} = \begin{bmatrix}1/2 & 1\\ 3 & 5\\ 1 & 2\end{bmatrix}.
    \]

    \vspace*{2cm}

    \questao{}  Seja $T : \cp{M}_2(\complex) \to \cp{M}_2(\complex)$ uma transformação linear dada por
    \[
    T \begin{pmatrix}
        x & y\\
        z & w
    \end{pmatrix} = \begin{pmatrix}
        0 & x\\
        z - w & 0
    \end{pmatrix},
    \]
    onde $\cp{M}_2(\complex)$ é um $\complex$-espaço vetorial
    e
    \[
    \mathcal{B}_1 = \left\{u_1 = \begin{bmatrix}
        1 & 0\\0 & 0
    \end{bmatrix}, u_2 = \begin{bmatrix}
        0 & 1\\0 & 0
    \end{bmatrix}, u_3 = \begin{bmatrix}
        0 & 0\\1 & 0
    \end{bmatrix}, u_4 = \begin{bmatrix}
        0 & 0\\0 & 1
    \end{bmatrix}\right\}
    \]
    Determine a matriz de $T$ com relação à base ordenada canônica $\mathcal{B}_1$.

    \vspace*{.5cm}

    \solucao Nesse caso vamos aplicar a transformação $T$ nos vetores de $\mathcal{B}_1$ e escrever o resultado como combinação linear
    dos vetores de $\mathcal{B}_1$:
    \begin{align*}
        T(u_1) &= \begin{bmatrix}0 & 1\\ 0 & 0\end{bmatrix} = 0u_1 + 1u_2 + 0u_3 + 0u_4\\
        T(u_2) &= \begin{bmatrix}0 & 0\\ 0 & 0\end{bmatrix} = 0u_1 + 0u_2 + 0u_3 + 0u_4\\
        T(u_3) &= \begin{bmatrix}0 & 0\\ 1 & 0\end{bmatrix} = 0u_1 + 0u_2 + 1u_3 + 0u_4\\
        T(u_1) &= \begin{bmatrix}\phantom{-}0 & 0\\ -1 & 0\end{bmatrix} = 0u_1 + 0u_2 - 1u_3 + 0u_4
    \end{align*}

    Portanto a matriz de $T$ com respeito à base $\mathcal{B}_1$ é:
    \[
        [T]_{\mathcal{B}_1} = \begin{bmatrix}0 & 0 & 0 & \phantom{-}0\\1 & 0 & 0 & \phantom{-}0\\0 & 0 & 1 & -1\\0 & 0 & 0 & \phantom{-}0\end{bmatrix}
    \]
\end{document}
