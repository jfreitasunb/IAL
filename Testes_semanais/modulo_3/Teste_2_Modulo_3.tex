%!TEX program = xelatex
%!TEX encoding = UTF-8
%Definições de Ano, Semestre, número e data da avaliação
\def\numeroteste{2}
\def\modulo{3}
\def\dataavaliacao{27/11/2023}

\documentclass[12pt]{exam}

\usepackage{caption}
\usepackage{amssymb}
\usepackage{amsmath,amsfonts,amsthm,amstext}
\usepackage[brazil]{babel}
\usepackage{graphicx}
\graphicspath{{../Pictures/}}
\usepackage{enumitem}
\usepackage{multicol}
\usepackage{answers}
\usepackage[svgnames]{xcolor}
\usepackage{tikz}
\usepackage{ifthen}
\usetikzlibrary{lindenmayersystems}
\usetikzlibrary[shadings]

\def\ano{2023}
\def\semestre{2}
\def\disciplina{Introdução à Álgebra Linear}
\def\nomeabreviado{IAL}
\def\turma{11}

\newcommand{\im}{{\rm Im\,}}
\newcommand{\dlim}[2]{\displaystyle\lim_{#1\rightarrow #2}}
\newcommand{\minf}{+\infty}
\newcommand{\ninf}{-\infty}
\newcommand{\cp}[1]{\mathbb{#1}}
\newcommand{\sub}{\subseteq}
\newcommand{\n}{\mathbb{N}}
\newcommand{\z}{\mathbb{Z}}
\newcommand{\rac}{\mathbb{Q}}
\newcommand{\real}{\mathbb{R}}
\newcommand{\complex}{\mathbb{C}}

\newcommand{\vesp}[1]{\vspace{ #1  cm}}

\newcommand{\compcent}[1]{\vcenter{\hbox{$#1\circ$}}}
\newcommand{\comp}{\mathbin{\mathchoice
        {\compcent\scriptstyle}{\compcent\scriptstyle}
        {\compcent\scriptscriptstyle}{\compcent\scriptscriptstyle}}}
\renewcommand{\sin}{{\rm sen\,}}
\renewcommand{\tan}{{\rm tg\,}}
\renewcommand{\csc}{{\rm cossec\,}}
\renewcommand{\cot}{{\rm cotg\,}}
\renewcommand{\sinh}{{\rm senh\,}}
\newcounter{exercicios}
\setcounter{exercicios}{0}
\newcommand{\questao}{
    \addtocounter{exercicios}{1}
    \noindent{\bf Quest{\~a}o \arabic{exercicios}: }}

\newcommand{\resp}[1]{
    \noindent{\bf Exerc{\'\i}cio #1: }}

\extrafootheight[.25in]{.5in}
\footrule
\lfoot{Teste \numeroteste\ - Módulo \modulo\ - \nomeabreviado\ - Turma \turma\ - \semestre$^o$/\ano}
\cfoot{}
\rfoot{P\'agina \thepage\ de \numpages}

\begin{document}

    \begin{figure}[h]
        \begin{minipage}[c]{1.7cm}
            \includegraphics[width=1.7cm]{unb.pdf}
        \end{minipage}
        \hspace{0pt}
        \begin{minipage}[c]{4in}
            {Universidade de Brasília} \\
            {Departamento de Matemática}
        \end{minipage}
    \end{figure}
    \vesp{-1}
    \hrule
    \begin{center}
        {\Large\bf \disciplina\ - Turma \turma}  \\
        \vesp{0.2} {\large\bf Teste \numeroteste\ - Módulo\ \modulo\ -\ \dataavaliacao}
    \end{center}

    \noindent{\bf \underline{Nome: \hspace{9.8cm} Mat.:\qquad \
            \hspace{2.1cm}}}\\
    \vspace*{.01cm}

    \noindent{\bf \underline{Nome: \hspace{9.8cm} Mat.:\qquad \
            \hspace{2.1cm}}}
    \vspace{.4cm}

    \questao{} Seja $F : \real^4 \to \real^3$ a transformação linear definida por
    \[
        F(x,y,s,t) = (x - y + s + t, x + 2s - t, x + y + 3s - 3t).
    \]
    Encontre
    \begin{enumerate}[label={\alph*})]
        \item $\ker(F)$

        \item $\im(F)$
    \end{enumerate}
    e uma base para cada subespaço.

    \solucao
    \begin{enumerate}[label={\alph*})]
        \item O kernel de $F$ é definido como
            \[
                \ker(F) = \{(x, y, s, t) \in \real^4 \mid F(x, y, s, t) = (0, 0, 0)\}
            \]
            Assim para encontrar $\ker(F)$ devemos encontrar todos os vetores $(x, y, s, t) \in \real^4$ tais que:
            \begin{align*}
                F(x, y, s, t) = (0, 0, 0)\\
                (x - y + s + t, x + 2s - t, x + y + 3s - 3t) = (0, 0, 0)\\
            \end{align*}
            Assim obtemos o sistema linear:
            \[
                \begin{cases}
                    x - y + s + t = 0\\
                    x + 2s - t = 0\\
                    x + y + 3s - 3t = 0
                \end{cases}
            \]
            Cuja matriz de coneficientes é:
            \[
                \begin{bmatrix}
                    1 & -1 & 1 & \phantom{-} 1\\
                    1 & \phantom{-} 0 & 2 & -1\\
                    1 & \phantom{-} 1 & 3 & -3
                \end{bmatrix}.
            \]
            Escalonando essa matriz:
            \begin{align*}
                \begin{bmatrix}
                    1 & -1 & 1 & \phantom{-} 1\\
                    1 & \phantom{-} 0 & 2 & -1\\
                    1 & \phantom{-} 1 & 3 & -3
                \end{bmatrix}
                \begin{array}{l}
                    L_1 \leftrightarrow L_2\\
                    \phantom{x}\\
                    \phantom{x}
                \end{array}&\sim
                \begin{bmatrix}
                    1 & \phantom{-} 0 & 2 & -1\\
                    1 & -1 & 1 & \phantom{-} 1\\
                    1 & \phantom{-} 1 & 3 & -3
                \end{bmatrix}
                \begin{array}{l}
                    \phantom{x}\\
                    L_2 \to L_2 - L_1\\
                    L_3 \to L_3 - L_1
                \end{array}\\&\sim
                \begin{bmatrix}
                    1 & \phantom{-} 0 & \phantom{-} 2 & -1\\
                    0 & -1 & -1 & \phantom{-} 2\\
                    0 & \phantom{-} 1 & \phantom{-} 1 & -2
                \end{bmatrix}
                \begin{array}{l}
                    \phantom{x}\\
                    \phantom{x}\\
                    L_3 \to L_3 + L_2
                \end{array}\\&\sim
                \begin{bmatrix}
                    1 & \phantom{-} 0 & \phantom{-} 2 & -1\\
                    0 & -1 & -1 & \phantom{-} 2\\
                    0 & \phantom{-} 0 & \phantom{-} 0 & \phantom{-} 0
                \end{bmatrix}
                \begin{array}{l}
                    \phantom{x}\\
                    L_2 \to -L_2
                    \phantom{x}\\
                \end{array}\\&\sim
                \begin{bmatrix}
                    1 & 0 & 2 & -1\\
                    0 & 1 & 1 & -2\\
                    0 & 0 & 0 & \phantom{-} 0
                \end{bmatrix}
            \end{align*}
            Assim a solução desse sistema é dada por
            \[
                x = t - 2s \quad \mbox{e}\quad y = 2t - s.
            \]
            Logo
            \[
                \ker(F) = \{(t - 2s, 2t - s, s, t) \in \real^4 \mid s, t \in \real\}.
            \]
            Para encontrar uma base do kernel primeiro fazemos:
            \begin{align*}
                (t - 2s, 2t - s, s, t) = (t, 2t, 0, t) + (-2s, -s, s, 0) = t(1, 2, 0, 1) + s(-2, -1, 1, 0)
            \end{align*}
            logo
            \[
                \ker(F) = [(1, 2, 0, 1); (-2, -1, 1, 0)].
            \]
            Mas esses dois vetores são L.I. pois um não é múltiplo escalar do outro (isso só vale pois estamos lidando com somente
            dois vetores). Assim uma base para $\ker(F)$ é dada pelo conjunto
            \[
                \{(1, 2, 0, 1); (-2, -1, 1, 0)\}.
            \]

        \item Considere a base canônica de $\real^4$ dada por
            \[
                \{(1, 0, 0, 0); (0, 1, 0, 0); (0, 0, 1, 0); (0, 0, 0, 1)\}.
            \]
            Então
            \[
                \im(F) = [F(1, 0, 0, 0); F(0, 1, 0, 0); F(0, 0, 1, 0); F(0, 0, 0, 1)]
            \]
            Agora
            \begin{align*}
                F(1, 0, 0, 0) &= (1, 1, 1);\\
                F(0, 1, 0, 0) &= (-1, 0, 1);\\
                F(0, 0, 1, 0) &= (1, 2, 3);\\
                F(0, 0, 0, 1) &= (1, -1, -3)
            \end{align*}
            Agora observe que
            \begin{align*}
                (1, 2, 3) &= 2(1, 1, 1) + (-1, 0, 1)\\
                (1, -1, -3) &= -(1, 1, 1) - 2(-1, 0, 1)
            \end{align*}
            daí
            \[
                \im(F) = [(1, 1, 1); (-1, 0, 1)]
            \]
            Esses dois vetores são L.I., pois um não é múltiplo escalar do outro (isso só vale pois estamos lidando com somente
            dois vetores). Assim uma base para $\im(F)$ é dada pelo conjunto
            \[
                \{(1, 1, 1); (-1, 0, 1)\}.
            \]


    \end{enumerate}


\end{document}
