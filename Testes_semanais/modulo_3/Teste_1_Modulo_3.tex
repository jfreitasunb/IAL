%!TEX program = xelatex
%!TEX encoding = UTF-8
%Definições de Ano, Semestre, número e data da avaliação
\def\numeroteste{1}
\def\modulo{3}
\def\dataavaliacao{22/11/2023}

\documentclass[12pt]{exam}

\usepackage{caption}
\usepackage{amssymb}
\usepackage{amsmath,amsfonts,amsthm,amstext}
\usepackage[brazil]{babel}
\usepackage{graphicx}
\graphicspath{{../Pictures/}}
\usepackage{enumitem}
\usepackage{multicol}
\usepackage{answers}
\usepackage[svgnames]{xcolor}
\usepackage{tikz}
\usepackage{ifthen}
\usetikzlibrary{lindenmayersystems}
\usetikzlibrary[shadings]

\def\ano{2023}
\def\semestre{2}
\def\disciplina{Introdução à Álgebra Linear}
\def\nomeabreviado{IAL}
\def\turma{11}

\newcommand{\im}{{\rm Im\,}}
\newcommand{\dlim}[2]{\displaystyle\lim_{#1\rightarrow #2}}
\newcommand{\minf}{+\infty}
\newcommand{\ninf}{-\infty}
\newcommand{\cp}[1]{\mathbb{#1}}
\newcommand{\sub}{\subseteq}
\newcommand{\n}{\mathbb{N}}
\newcommand{\z}{\mathbb{Z}}
\newcommand{\rac}{\mathbb{Q}}
\newcommand{\real}{\mathbb{R}}
\newcommand{\complex}{\mathbb{C}}

\newcommand{\vesp}[1]{\vspace{ #1  cm}}

\newcommand{\compcent}[1]{\vcenter{\hbox{$#1\circ$}}}
\newcommand{\comp}{\mathbin{\mathchoice
        {\compcent\scriptstyle}{\compcent\scriptstyle}
        {\compcent\scriptscriptstyle}{\compcent\scriptscriptstyle}}}
\renewcommand{\sin}{{\rm sen\,}}
\renewcommand{\tan}{{\rm tg\,}}
\renewcommand{\csc}{{\rm cossec\,}}
\renewcommand{\cot}{{\rm cotg\,}}
\renewcommand{\sinh}{{\rm senh\,}}
\newcounter{exercicios}
\setcounter{exercicios}{0}
\newcommand{\questao}{
    \addtocounter{exercicios}{1}
    \noindent{\bf Quest{\~a}o \arabic{exercicios}: }}

\newcommand{\resp}[1]{
    \noindent{\bf Exerc{\'\i}cio #1: }}

\extrafootheight[.25in]{.5in}
\footrule
\lfoot{Teste \numeroteste\ - Módulo \modulo\ - \nomeabreviado\ - Turma \turma\ - \semestre$^o$/\ano}
\cfoot{}
\rfoot{P\'agina \thepage\ de \numpages}

\begin{document}

    \begin{figure}[h]
        \begin{minipage}[c]{1.7cm}
            \includegraphics[width=1.7cm]{unb.pdf}
        \end{minipage}
        \hspace{0pt}
        \begin{minipage}[c]{4in}
            {Universidade de Brasília} \\
            {Departamento de Matemática}
        \end{minipage}
    \end{figure}
    \vesp{-1}
    \hrule
    \begin{center}
        {\Large\bf \disciplina\ - Turma \turma}  \\
        \vesp{0.2} {\large\bf Teste \numeroteste\ - Módulo\ \modulo\ -\ \dataavaliacao}
    \end{center}

    \noindent{\bf \underline{Nome: \hspace{9.8cm} Mat.:\qquad \
            \hspace{2.1cm}}}\\
    \vspace*{.01cm}

    \noindent{\bf \underline{Nome: \hspace{9.8cm} Mat.:\qquad \
            \hspace{2.1cm}}}
    \vspace{.4cm}

    \questao{}  Considere a função $T : \complex \to \cp{M}_2(\real)$ dada por
  \[
      T(x + yi) = \begin{bmatrix}
                        x + 7y & 5y\\
                       -10y & x - 7y
                   \end{bmatrix},
  \]
    onde $\complex$ é um espaço vetorial sobre $\real$. Prove que $T$ é uma transformação linear.

    \solucao Precisamos mostrar que valem as seguintes propriedades:
    \begin{align}
        T(u_1 + u_2) &= T(u_1) + T(u_2)\\
        T(\lambda u_1) &= \lambda T(u_1)
    \end{align}
    ou então que
    \begin{align}\label{condicaoTL}
        T(\lambda u_1 + u_2) = \lambda T(u_1) + T(u_2)
    \end{align}
    para quaisquer $u_1$, $u_2 \in \complex$ e todo $\lambda \in \real$.
    Vamos fazer a verificação de que a condição \eqref{condicaoTL} é verdadeira. Para isso sejam
    \begin{align*}
        u_1 = x + yi\\
        u_2 = a + bi
    \end{align*}
    e $\lambda \in \real$. Vamos primeiro calcular o lado esquerdo da equação \eqref{condicaoTL}:
    \begin{align}\label{parte1}
        T(\lambda u_1 + u_2) &= T(\lambda(x + yi) + a + bi) = T(\lambda x + \lambda yi + a + bi)\nonumber \\ &= T((\lambda x + a) + (\lambda y + b)i)\nonumber \\ &=
        \begin{bmatrix}
            (\lambda x + a) + 7(\lambda y + b) & 5(\lambda y + b)\\
            -10(\lambda y + b) & (\lambda x + a) - 7(\lambda y + b)
        \end{bmatrix}\nonumber\\
        &= \begin{bmatrix}
            \lambda x + a + 7\lambda y + 7b & 5\lambda y + 5b\\
            -10\lambda y - 10b & \lambda x + a - 7\lambda y + 7b
        \end{bmatrix}
    \end{align}

    Agora calculando o lado direito de \eqref{condicaoTL}:
    \begin{align}\label{parte2}
        \lambda T(u_1) + T(u_2) & = \lambda T(x + yi) + T(a + bi)\nonumber \\ &= \lambda\begin{bmatrix}x + 7y & 5y\\-10y & x - 7y\end{bmatrix} + \begin{bmatrix}a + 7b & 5b\\-10b & a - 7b\end{bmatrix}\nonumber \\ &= \begin{bmatrix}\lambda x + 7\lambda y & 5\lambda y\\-10\lambda y & \lambda x - 7\lambda y\end{bmatrix} + \begin{bmatrix}a + 7b & 5b\\-10b & a - 7b\end{bmatrix}\nonumber \\ &= \begin{bmatrix}\lambda x + a + 7\lambda y + 7b & 5\lambda y + 5b\\-10\lambda y -10b & \lambda x + a - 7\lambda y - 7b\end{bmatrix}
    \end{align}

    Comparando os resultados obtidos em \eqref{parte1} e \eqref{parte2}, vemos que
    \[
        T(\lambda u_1 + u_2) = \lambda T(u_1) + T(u_2).
    \]

    Portanto $T$ é uma transformação linear.
\end{document}
