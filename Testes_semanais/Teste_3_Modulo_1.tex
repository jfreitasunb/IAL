%!TEX program = xelatex
%!TEX encoding = UTF-8
%Defini\c{c}\~oes de Ano, Semestre, n\'umero e data da avalia\c{c}\~ao
\def\numeroteste{3}
\def\modulo{1}
\def\dataavaliacao{20/09/2023}

\documentclass[12pt]{exam}

\usepackage{caption}
\usepackage{amssymb}
\usepackage{amsmath,amsfonts,amsthm,amstext}
\usepackage[brazil]{babel}
\usepackage{graphicx}
\graphicspath{{../Pictures/}}
\usepackage{enumitem}
\usepackage{multicol}
\usepackage{answers}
\usepackage[svgnames]{xcolor}
\usepackage{tikz}
\usepackage{ifthen}
\usetikzlibrary{lindenmayersystems}
\usetikzlibrary[shadings]

\def\ano{2023}
\def\semestre{2}
\def\disciplina{Introdução à Álgebra Linear}
\def\nomeabreviado{IAL}
\def\turma{11}

\newcommand{\im}{{\rm Im\,}}
\newcommand{\dlim}[2]{\displaystyle\lim_{#1\rightarrow #2}}
\newcommand{\minf}{+\infty}
\newcommand{\ninf}{-\infty}
\newcommand{\cp}[1]{\mathbb{#1}}
\newcommand{\sub}{\subseteq}
\newcommand{\n}{\mathbb{N}}
\newcommand{\z}{\mathbb{Z}}
\newcommand{\rac}{\mathbb{Q}}
\newcommand{\real}{\mathbb{R}}
\newcommand{\complex}{\mathbb{C}}

\newcommand{\vesp}[1]{\vspace{ #1  cm}}

\newcommand{\compcent}[1]{\vcenter{\hbox{$#1\circ$}}}
\newcommand{\comp}{\mathbin{\mathchoice
        {\compcent\scriptstyle}{\compcent\scriptstyle}
        {\compcent\scriptscriptstyle}{\compcent\scriptscriptstyle}}}
\renewcommand{\sin}{{\rm sen\,}}
\renewcommand{\tan}{{\rm tg\,}}
\renewcommand{\csc}{{\rm cossec\,}}
\renewcommand{\cot}{{\rm cotg\,}}
\renewcommand{\sinh}{{\rm senh\,}}
\newcounter{exercicios}
\setcounter{exercicios}{0}
\newcommand{\questao}{
    \addtocounter{exercicios}{1}
    \noindent{\bf Quest{\~a}o \arabic{exercicios}: }}

\newcommand{\resp}[1]{
    \noindent{\bf Exerc{\'\i}cio #1: }}

\extrafootheight[.25in]{.5in}
\footrule
\lfoot{Teste \numeroteste\ - Módulo \modulo\ - \nomeabreviado\ - Turma \turma\ - \semestre$^o$/\ano}
\cfoot{}
\rfoot{P\'agina \thepage\ de \numpages}

\begin{document}

    \begin{figure}[h]
        \begin{minipage}[c]{1.7cm}
            \includegraphics[width=1.7cm]{unb.pdf}
        \end{minipage}
        \hspace{0pt}
        \begin{minipage}[c]{4in}
            {Universidade de Bras{\'\i}lia} \\
            {Departamento de Matem{\'a}tica}
        \end{minipage}
    \end{figure}
    \vesp{-1}
    \hrule
    \begin{center}
        {\Large\bf \disciplina\ - Turma \turma}  \\
        \vesp{0.2} {\large\bf Teste \numeroteste\ - Módulo\ \modulo\ -\ \dataavaliacao}
    \end{center}

    \noindent{\bf \underline{Nome: \hspace{9.8cm} Mat.:\qquad \
            \hspace{2.1cm}}}\\
    \vspace*{.01cm}

    \noindent{\bf \underline{Nome: \hspace{9.8cm} Mat.:\qquad \
            \hspace{2.1cm}}}
    \vspace{.4cm}

    \questao{} Considere o seguinte sistema linear:
    \[
        \begin{cases}
            x_1 + \lambda x_2 + (1 + 4\lambda )x_3 = 1 + 4\lambda \\
            2x_1 + (\lambda  + 1)x_2 + (2 + 7\lambda )x_3 = 1 + 7\lambda \\
            3x_1 + (\lambda  + 2)x_2 + (3 + 9\lambda )x_3 = 1 + 9\lambda
        \end{cases}
    \]
em $\real$. Encontre o(s) valor(es) de $\lambda$ tal(is) que:
\begin{enumerate}[label={\alph*})]
    \item o sistema tem solução única;

    \item o sistema tem várias soluções;

    \item o sistema não tem solução.
\end{enumerate}
\end{document}
