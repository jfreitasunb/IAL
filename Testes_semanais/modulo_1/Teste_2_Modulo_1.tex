%!TEX program = xelatex
%!TEX encoding = UTF-8
%Definições de Ano, Semestre, número e data da avaliação
\def\numeroteste{2}
\def\modulo{1}
\def\dataavaliacao{13/09/2023}

\documentclass[12pt]{exam}

\usepackage{caption}
\usepackage{amssymb}
\usepackage{amsmath,amsfonts,amsthm,amstext}
\usepackage[brazil]{babel}
\usepackage{graphicx}
\graphicspath{{../Pictures/}}
\usepackage{enumitem}
\usepackage{multicol}
\usepackage{answers}
\usepackage[svgnames]{xcolor}
\usepackage{tikz}
\usepackage{ifthen}
\usetikzlibrary{lindenmayersystems}
\usetikzlibrary[shadings]

\def\ano{2023}
\def\semestre{2}
\def\disciplina{Introdução à Álgebra Linear}
\def\nomeabreviado{IAL}
\def\turma{11}

\newcommand{\im}{{\rm Im\,}}
\newcommand{\dlim}[2]{\displaystyle\lim_{#1\rightarrow #2}}
\newcommand{\minf}{+\infty}
\newcommand{\ninf}{-\infty}
\newcommand{\cp}[1]{\mathbb{#1}}
\newcommand{\sub}{\subseteq}
\newcommand{\n}{\mathbb{N}}
\newcommand{\z}{\mathbb{Z}}
\newcommand{\rac}{\mathbb{Q}}
\newcommand{\real}{\mathbb{R}}
\newcommand{\complex}{\mathbb{C}}

\newcommand{\vesp}[1]{\vspace{ #1  cm}}

\newcommand{\compcent}[1]{\vcenter{\hbox{$#1\circ$}}}
\newcommand{\comp}{\mathbin{\mathchoice
        {\compcent\scriptstyle}{\compcent\scriptstyle}
        {\compcent\scriptscriptstyle}{\compcent\scriptscriptstyle}}}
\renewcommand{\sin}{{\rm sen\,}}
\renewcommand{\tan}{{\rm tg\,}}
\renewcommand{\csc}{{\rm cossec\,}}
\renewcommand{\cot}{{\rm cotg\,}}
\renewcommand{\sinh}{{\rm senh\,}}
\newcounter{exercicios}
\setcounter{exercicios}{0}
\newcommand{\questao}{
    \addtocounter{exercicios}{1}
    \noindent{\bf Quest{\~a}o \arabic{exercicios}: }}

\newcommand{\resp}[1]{
    \noindent{\bf Exerc{\'\i}cio #1: }}

\extrafootheight[.25in]{.5in}
\footrule
\lfoot{Teste \numeroteste\ - Módulo \modulo\ - \nomeabreviado\ - Turma \turma\ - \semestre$^o$/\ano}
\cfoot{}
\rfoot{P\'agina \thepage\ de \numpages}

\begin{document}

    \begin{figure}[h]
        \begin{minipage}[c]{1.7cm}
            \includegraphics[width=1.7cm]{unb.pdf}
        \end{minipage}
        \hspace{0pt}
        \begin{minipage}[c]{4in}
            {Universidade de Brasília} \\
            {Departamento de Matemática}
        \end{minipage}
    \end{figure}
    \vesp{-1}
    \hrule
    \begin{center}
        {\Large\bf \disciplina\ - Turma \turma}  \\
        \vesp{0.2} {\large\bf Teste \numeroteste\ - Módulo\ \modulo\ -\ \dataavaliacao}
    \end{center}

    \noindent{\bf \underline{Nome: \hspace{9.8cm} Mat.:\qquad \
            \hspace{2.1cm}}}\\
    \vspace*{.01cm}

    \noindent{\bf \underline{Nome: \hspace{9.8cm} Mat.:\qquad \
            \hspace{2.1cm}}}
    \vspace{.4cm}

    \questao{} Considere o seguinte sistema linear:
    \[
        \begin{cases}
            x_1 + 3x_2 - 2x_3 + 2x_5 = 0\\
            2x_1 + 6x_2 - 5x_3 - 2x_4 + 4x_5 - 3x_6 = -1\\
            5x_3 + 10x_4 + 15x_6 = 5\\
            2x_1 + 6x_2 + 8x_4 + 4x_5 + 18x_6 = 6
        \end{cases}
    \]
em $\real$. Usando eliminação gaussiana, encontre a solução desse sistema.

\solucao A matriz ampliada do sistema é:
\[
    \begin{amatrix}{6}
        1 & 3 & -2 & 0 & 2 & 0 & 0\\
        2 & 6 & -5 & -2 & 4 & -3 & -1\\
        0 & 0 & 5 & 10 & 0 & 15 & 5\\
        2 & 6 & 0 & 8 & 4 & 18 & 6
    \end{amatrix}
\]
Aplicando a eliminação gaussiana:
\begin{align*}
    \begin{amatrix}{6}
        1 & 3 & -2 & 0 & 2 & 0 & 0\\
        2 & 6 & -5 & -2 & 4 & -3 & -1\\
        0 & 0 & 5 & 10 & 0 & 15 & 5\\
        2 & 6 & 0 & 8 & 4 & 18 & 6
    \end{amatrix}
    \begin{array}{l}
        \phantom{x}\\
        L_2 \to L_2 - 2L_1\\
        \phantom{x}\\
        L_4 \to L_4 - 2L_1
    \end{array}&\sim
    \begin{amatrix}{6}
        1 & 3 & -2 & 0 & 2 & 0 & 0\\
        0 & 0 & -1 & -2 & 0 & -3 & -1\\
        0 & 0 & 5 & 10 & 0 & 15 & 5\\
        0 & 0 & 4 & 8 & 0 & 18 & 6
    \end{amatrix}
    \begin{array}{l}
        \phantom{x}\\
        \phantom{x}\\
        L_3 \to L_3 + 5L_2\\
        L_4 \to L_4 + 4L_2
    \end{array}\\&\sim
    \begin{amatrix}{6}
        1 & 3 & -2 & 0 & 2 & 0 & 0\\
        0 & 0 & -1 & -2 & 0 & -3 & -1\\
        0 & 0 & 0 & 0 & 0 & 0 & 0\\
        0 & 0 & 0 & 0 & 0 & 6 & 2
    \end{amatrix}
    \begin{array}{l}
        \phantom{x}\\
        L_2 \to -L_2\\
        L_3 \leftrightarrow L_4\\
        \phantom{x}
    \end{array}\\&\sim
    \begin{amatrix}{6}
        1 & 3 & -2 & 0 & 2 & 0 & 0\\
        0 & 0 & 1 & 2 & 0 & 3 & 1\\
        0 & 0 & 0 & 0 & 0 & 6 & 2\\
        0 & 0 & 0 & 0 & 0 & 0 & 0
    \end{amatrix}
    \begin{array}{l}
        \phantom{x}\\
        \phantom{x}\\
        L_3 \to \dfrac{1}{6}L_3\\
        \phantom{x}
    \end{array}\\&\sim
    \begin{amatrix}{6}
        1 & 3 & -2 & 0 & 2 & 0 & 0\\
        0 & 0 & 1 & 2 & 0 & 3 & 1\\
        0 & 0 & 0 & 0 & 0 & 1 & 1/3\\
        0 & 0 & 0 & 0 & 0 & 0 & 0
    \end{amatrix}
\end{align*}
Essa última matriz está na forma escalonada. Nesse caso vemos que a matriz dos coeficientes tem posto 3 e como o sistema possui 6 variáveis teremos 3 variáveis livres, a saber, $x_2$, $x_4$ e $x_5$. A solução é dada por:
\[
    \begin{cases}
        x_1 + 3x_2 - 2x_3 + 2x_5 = 0\\
        x_3 + 2x_4 + 3x_6 = 1\\
        x_6 = 1/3
    \end{cases}
\]
Substituindo, obtemos
\[
    x_3 = -2x_4 \mbox{e}\ x_1 = -3x_2 - 4x_4 - 2x_5.
\]
Logo a solução geral é dada pelo conjunto:
\begin{align*}
    S &= \{(x_1, x_2, x_3, x_4, x_5, x_6) \mid x_1 = -3x_2 - 4x_4 - 2x_5, x_3 = -2x_4, x_6 = 1/3, x_2, x_4, x_5 \in \real\}\\
    S &= \{(-3x_2 - 4x_4 - 2x_5, x_2, -2x_4, x_4, x_5, 1/3) \mid x_2, x_4, x_5 \in \real.\}
\end{align*}
\end{document}
