%!TEX program = xelatex
%!TEX encoding = UTF-8
%Definições de Ano, Semestre, número e data da avaliação
\def\numeroteste{4}
\def\modulo{2}
\def\dataavaliacao{01/11/2023}

\documentclass[12pt]{exam}

\usepackage{caption}
\usepackage{amssymb}
\usepackage{amsmath,amsfonts,amsthm,amstext}
\usepackage[brazil]{babel}
\usepackage{graphicx}
\graphicspath{{../Pictures/}}
\usepackage{enumitem}
\usepackage{multicol}
\usepackage{answers}
\usepackage[svgnames]{xcolor}
\usepackage{tikz}
\usepackage{ifthen}
\usetikzlibrary{lindenmayersystems}
\usetikzlibrary[shadings]

\def\ano{2023}
\def\semestre{2}
\def\disciplina{Introdução à Álgebra Linear}
\def\nomeabreviado{IAL}
\def\turma{11}

\newcommand{\im}{{\rm Im\,}}
\newcommand{\dlim}[2]{\displaystyle\lim_{#1\rightarrow #2}}
\newcommand{\minf}{+\infty}
\newcommand{\ninf}{-\infty}
\newcommand{\cp}[1]{\mathbb{#1}}
\newcommand{\sub}{\subseteq}
\newcommand{\n}{\mathbb{N}}
\newcommand{\z}{\mathbb{Z}}
\newcommand{\rac}{\mathbb{Q}}
\newcommand{\real}{\mathbb{R}}
\newcommand{\complex}{\mathbb{C}}

\newcommand{\vesp}[1]{\vspace{ #1  cm}}

\newcommand{\compcent}[1]{\vcenter{\hbox{$#1\circ$}}}
\newcommand{\comp}{\mathbin{\mathchoice
        {\compcent\scriptstyle}{\compcent\scriptstyle}
        {\compcent\scriptscriptstyle}{\compcent\scriptscriptstyle}}}
\renewcommand{\sin}{{\rm sen\,}}
\renewcommand{\tan}{{\rm tg\,}}
\renewcommand{\csc}{{\rm cossec\,}}
\renewcommand{\cot}{{\rm cotg\,}}
\renewcommand{\sinh}{{\rm senh\,}}
\newcounter{exercicios}
\setcounter{exercicios}{0}
\newcommand{\questao}{
    \addtocounter{exercicios}{1}
    \noindent{\bf Quest{\~a}o \arabic{exercicios}: }}

\newcommand{\resp}[1]{
    \noindent{\bf Exerc{\'\i}cio #1: }}

\extrafootheight[.25in]{.5in}
\footrule
\lfoot{Teste \numeroteste\ - Módulo \modulo\ - \nomeabreviado\ - Turma \turma\ - \semestre$^o$/\ano}
\cfoot{}
\rfoot{P\'agina \thepage\ de \numpages}

\begin{document}

    \begin{figure}[h]
        \begin{minipage}[c]{1.7cm}
            \includegraphics[width=1.7cm]{unb.pdf}
        \end{minipage}
        \hspace{0pt}
        \begin{minipage}[c]{4in}
            {Universidade de Brasília} \\
            {Departamento de Matemática}
        \end{minipage}
    \end{figure}
    \vesp{-1}
    \hrule
    \begin{center}
        {\Large\bf \disciplina\ - Turma \turma}  \\
        \vesp{0.2} {\large\bf Teste \numeroteste\ - Módulo\ \modulo\ -\ \dataavaliacao}
    \end{center}

    \noindent{\bf \underline{Nome: \hspace{9.8cm} Mat.:\qquad \
            \hspace{2.1cm}}}\\
    \vspace*{.01cm}

    \noindent{\bf \underline{Nome: \hspace{9.8cm} Mat.:\qquad \
            \hspace{2.1cm}}}
    \vspace{.4cm}

    \questao{} Seja $S \subseteq M_2(\real)$ dado por:
    \[
    S = \left\{
    \begin{bmatrix}
        a & b\\
        c & d
    \end{bmatrix}
    \mid c = a+b \mbox{ e } d = a
    \right\}.
    \]

    \begin{enumerate}[label={\alph*})]
        \item Mostre que $S$ é um subespaço vetorial de $M_2(\real)$.

        \item Qual a dimensão de $S$?

        \item O conjunto
        \[
        \left\{
        \begin{bmatrix}
            1 & -1\\
            0 & \phantom{x} 1
        \end{bmatrix},
        \begin{bmatrix}
            2 & 1\\
            3 & 4
        \end{bmatrix}
        \right\}
        \]
        é uma base de $S$? Justifique.
    \end{enumerate}

    \solucao Primeiro observe que o conjunto $S$ pode ser escrito como
    \[
        S = \left\{\begin{bmatrix}
            a & b\\a + b & a
        \end{bmatrix} \mid a, b \in \real\right\}.
    \]
    \begin{enumerate}[label={\alph*})]
        \item Primeiro observe que
        \[
            \begin{bmatrix}
                0 & 0\\ 0 & 0
            \end{bmatrix} \in S,
        \]
        logo $S$ é um conjunto não vazio.

        Agora sejam $u_1$, $u_2 \in S$ e $\lambda \in \real$. Temos
        \begin{align*}
            u_1 = \begin{bmatrix}a & b\\a + b & a\end{bmatrix}\\
            u_2 = \begin{bmatrix}c & d\\c + d & c\end{bmatrix}\\
        \end{align*}
        daí:
        \begin{enumerate}[label={\roman*})]
            \item $u_1 + u_2 = \begin{bmatrix}a & b\\a + b & a\end{bmatrix} + \begin{bmatrix}c & d\\c + d & c\end{bmatrix} = \begin{bmatrix}a + c & b + d\\a + b + c + d & a + c\end{bmatrix}\in S$;
            \item $\lambda\cdot u_1 = \lambda\cdot \begin{bmatrix}a & b\\a + b & a\end{bmatrix} = \begin{bmatrix}\lambda a & \lambda b\\\lambda(a + b) & \lambda a\end{bmatrix} = \begin{bmatrix}\lambda a & \lambda b\\\lambda a + \lambda b & \lambda a\end{bmatrix} \in S$.
        \end{enumerate}
        Logo $S$ é de fato um subespaço de $M_2(\real)$.

        \item Para encontrar a dimensão de $S$ precisamos encontrar um base para esse espaço. Para isso, seja $\begin{bmatrix}a & b\\ a + b & a\end{bmatrix} \in S$ um elemento qualquer. Podemos escrever
        \begin{align*}
            \begin{bmatrix}a & b\\a + b & a\end{bmatrix} &= \begin{bmatrix}a & 0\\a & a\end{bmatrix} + \begin{bmatrix}0 & b\\b & 0\end{bmatrix} = a\begin{bmatrix}1 & 0\\1 & 1\end{bmatrix} + b\begin{bmatrix}0 & 1\\1 & 0\end{bmatrix}.
        \end{align*}
        Com isso todo elemento de $S$ é uma combinação linear de $\begin{bmatrix}1 & 0\\1 & 1\end{bmatrix}$ e de $\begin{bmatrix}0 & 1\\1 & 0\end{bmatrix}$. Portanto o conjunto
        \[
            \mathcal{A} = \left\{\begin{bmatrix}1 & 0\\1 & 1\end{bmatrix}, \begin{bmatrix}0 & 1\\1 & 0\end{bmatrix}\right\}
        \]
        gera $S$. Para ser uma base basta que seja L.I. e isso é verdade uma vez que esse conjunto possui somente 2 vetores e é fácil ver que um não é múltiplo escalar do outro. Logo o conjunto $\mathcal{A}$ é uma base para $S$ e como esse conjunto possui 2 vetores então
        \[
            \dim_\real S = 2.
        \]

        \item Como $S$ tem dimensão 2, então qualquer conjunto L.I. contendo 2 vetores será uma base para $S$. Nesse caso é fácil ver que os vetores no
        \[
        \left\{
        \begin{bmatrix}
            1 & -1\\
            0 & \phantom{x} 1
        \end{bmatrix},
        \begin{bmatrix}
            2 & 1\\
            3 & 4
        \end{bmatrix}
        \right\}
        \]
        são L.I. pois um não é múltiplo escalar no outro, no entanto esse conjunto \textbf{NÃO É UMA BASE} para $S$ pois $ \begin{bmatrix}2 & 1\\3 & 4\end{bmatrix} \notin S$.

        Um conjunto para ser base de um espaço ou subespaço além de ser L.I. e gerar também deve conter vetores desse conjunto.
    \end{enumerate}
\end{document}