%!TEX program = xelatex
%!TEX encoding = UTF-8
%Definições de Ano, Semestre, número e data da avaliação
\def\numeroteste{1}
\def\modulo{2}
\def\dataavaliacao{11/10/2023}

\documentclass[12pt]{exam}

\usepackage{caption}
\usepackage{amssymb}
\usepackage{amsmath,amsfonts,amsthm,amstext}
\usepackage[brazil]{babel}
\usepackage{graphicx}
\graphicspath{{../Pictures/}}
\usepackage{enumitem}
\usepackage{multicol}
\usepackage{answers}
\usepackage[svgnames]{xcolor}
\usepackage{tikz}
\usepackage{ifthen}
\usetikzlibrary{lindenmayersystems}
\usetikzlibrary[shadings]

\def\ano{2023}
\def\semestre{2}
\def\disciplina{Introdução à Álgebra Linear}
\def\nomeabreviado{IAL}
\def\turma{11}

\newcommand{\im}{{\rm Im\,}}
\newcommand{\dlim}[2]{\displaystyle\lim_{#1\rightarrow #2}}
\newcommand{\minf}{+\infty}
\newcommand{\ninf}{-\infty}
\newcommand{\cp}[1]{\mathbb{#1}}
\newcommand{\sub}{\subseteq}
\newcommand{\n}{\mathbb{N}}
\newcommand{\z}{\mathbb{Z}}
\newcommand{\rac}{\mathbb{Q}}
\newcommand{\real}{\mathbb{R}}
\newcommand{\complex}{\mathbb{C}}

\newcommand{\vesp}[1]{\vspace{ #1  cm}}

\newcommand{\compcent}[1]{\vcenter{\hbox{$#1\circ$}}}
\newcommand{\comp}{\mathbin{\mathchoice
        {\compcent\scriptstyle}{\compcent\scriptstyle}
        {\compcent\scriptscriptstyle}{\compcent\scriptscriptstyle}}}
\renewcommand{\sin}{{\rm sen\,}}
\renewcommand{\tan}{{\rm tg\,}}
\renewcommand{\csc}{{\rm cossec\,}}
\renewcommand{\cot}{{\rm cotg\,}}
\renewcommand{\sinh}{{\rm senh\,}}
\newcounter{exercicios}
\setcounter{exercicios}{0}
\newcommand{\questao}{
    \addtocounter{exercicios}{1}
    \noindent{\bf Quest{\~a}o \arabic{exercicios}: }}

\newcommand{\resp}[1]{
    \noindent{\bf Exerc{\'\i}cio #1: }}

\extrafootheight[.25in]{.5in}
\footrule
\lfoot{Teste \numeroteste\ - Módulo \modulo\ - \nomeabreviado\ - Turma \turma\ - \semestre$^o$/\ano}
\cfoot{}
\rfoot{P\'agina \thepage\ de \numpages}

\begin{document}

    \begin{figure}[h]
        \begin{minipage}[c]{1.7cm}
            \includegraphics[width=1.7cm]{unb.pdf}
        \end{minipage}
        \hspace{0pt}
        \begin{minipage}[c]{4in}
            {Universidade de Brasília} \\
            {Departamento de Matemática}
        \end{minipage}
    \end{figure}
    \vesp{-1}
    \hrule
    \begin{center}
        {\Large\bf \disciplina\ - Turma \turma}  \\
        \vesp{0.2} {\large\bf Teste \numeroteste\ - Módulo\ \modulo\ -\ \dataavaliacao}
    \end{center}

    \noindent{\bf \underline{Nome: \hspace{9.8cm} Mat.:\qquad \
            \hspace{2.1cm}}}\\
    \vspace*{.01cm}

    \noindent{\bf \underline{Nome: \hspace{9.8cm} Mat.:\qquad \
            \hspace{2.1cm}}}
    \vspace{.4cm}

    \questao{} Seja
    \[
        V = M_2(\real) = \left\{\begin{bmatrix}a_{11} & a_{12}\\a_{21} & a_{22} \end{bmatrix} \mid a_{11}, a_{12}, a_{21}, a_{22} \in \real\right\}.
    \]
    Considere em $V$ as seguintes operações
    \begin{align*}
        A + B &= \begin{bmatrix}a_{11} & a_{12}\\a_{21} & a_{22} \end{bmatrix} + \begin{bmatrix}b_{11} & b_{12}\\b_{21} & b_{22} \end{bmatrix} = \begin{bmatrix}a_{11} + b_{11 }& a_{12} + b_{12}\\a_{21} + b_{21} & a_{22} + b_{22}\end{bmatrix}\\
        \alpha \cdot A &= \alpha \cdot \begin{bmatrix}a_{11} & a_{12}\\a_{21} & a_{22} \end{bmatrix} = \begin{bmatrix}\alpha a_{11} & \alpha a_{12}\\\alpha a_{21} & \alpha a_{22} \end{bmatrix}
    \end{align*}
    para todos $A = \begin{bmatrix}a_{11} & a_{12}\\a_{21} & a_{22} \end{bmatrix}$, $B = \begin{bmatrix}b_{11} & b_{12}\\b_{21} & b_{22} \end{bmatrix} \in V$ e todo $\alpha \in \cp{K} = \real$.
    Mostre que com essas operações $V = M_2(\real)$ é um espaço vetorial sobre $\cp{K} = \real$.

    \solucao

    De fato,
    \begin{itemize}
        \item[A1)] Sejam
        \[
        A = \begin{pmatrix} a_{11} & a_{12}\\a_{21} & a_{22}\end{pmatrix},
        B = \begin{pmatrix} b_{11} & b_{12}\\b_{21} & b_{22}\end{pmatrix}.
        \]
        Temos:
        \begin{align*}
            A + B & = \begin{pmatrix} a_{11} & a_{12}\\a_{21} & a_{22}\end{pmatrix} + \begin{pmatrix} b_{11} & b_{12}\\b_{21} & b_{22}\end{pmatrix}
            \\ &= \begin{pmatrix} a_{11} + b_{11} & a_{12} + b_{12}\\a_{21} + b_{21} & a_{22} + b_{22}\end{pmatrix}
            \\ &= \begin{pmatrix} b_{11} + a_{11} & b_{12} + a_{12}\\b_{21} + a_{21} & b_{22} + a_{22}\end{pmatrix}
            \\ &= \begin{pmatrix} b_{11} & b_{12}\\b_{21} & b_{22}\end{pmatrix} + \begin{pmatrix} a_{11} & a_{12}\\a_{21} & a_{22}\end{pmatrix}
            \\ &= B + A
        \end{align*}

        \item[A2)] Sejam
        \[
        A = \begin{pmatrix} a_{11} & a_{12}\\a_{21} & a_{22}\end{pmatrix},
        B = \begin{pmatrix} b_{11} & b_{12}\\b_{21} & b_{22}\end{pmatrix},
        C = \begin{pmatrix} c_{11} & c_{12}\\c_{21} & c_{22}\end{pmatrix}.
        \]
        Temos:
        \begin{align*}
            (A + B) + C & = \left(\begin{pmatrix} a_{11} & a_{12}\\a_{21} & a_{22}\end{pmatrix} +
            \begin{pmatrix} b_{11} & b_{12}\\b_{21} & b_{22}\end{pmatrix}\right) +
            \begin{pmatrix} c_{11} & c_{12}\\c_{21} & c_{22}\end{pmatrix}
            \\ &= \begin{pmatrix} a_{11} + b_{11} & a_{12} + b_{12}\\a_{21} + b_{21} & a_{22} + b_{22}\end{pmatrix} +
            \begin{pmatrix} c_{11} & c_{12}\\c_{21} & c_{22}\end{pmatrix}
            \\ &= \begin{pmatrix} (a_{11} + b_{11}) + c_{11} & (a_{12} + b_{12}) + c_{12}\\(a_{21} + b_{21}) + c_{21} & (a_{22} + b_{22}) + c_{22}\end{pmatrix}
            \\ &= \begin{pmatrix} a_{11} + (b_{11} + c_{11}) & a_{12} + (b_{12} + c_{12})\\a_{21} + (b_{21} + c_{21}) & a_{22} + (b_{22} + c_{22})\end{pmatrix}
            \\ &= \begin{pmatrix} a_{11} & a_{12}\\a_{21} & a_{22}\end{pmatrix} + \begin{pmatrix} b_{11} + c_{11} & b_{12} + c_{12}\\b_{21} + c_{21} & b_{22} + c_{22}\end{pmatrix}
            \\ &= \begin{pmatrix} a_{11} & a_{12}\\a_{21} & a_{22}\end{pmatrix} +
            \left(\begin{pmatrix} b_{11} & b_{12}\\b_{21} & b_{22}\end{pmatrix} +
            \begin{pmatrix} c_{11} & c_{12}\\c_{21} & c_{22}\end{pmatrix}\right)
            \\ &= A + (B + C).
        \end{align*}

        \item[A3)] Tome
        \[
        0_V = \begin{pmatrix}0 & 0\\0 & 0\end{pmatrix}.
        \]
        Para toda matriz
        \[
        A = \begin{pmatrix} a_{11} & a_{12}\\a_{21} & a_{22}\end{pmatrix},
        \]
        temos
        \begin{align*}
            A + 0_V & = \begin{pmatrix} a_{11} & a_{12}\\a_{21} & a_{22}\end{pmatrix} +
            \begin{pmatrix}0 & 0\\0 & 0\end{pmatrix}
            = \begin{pmatrix}a_{11} + 0 & a_{12} + 0\\a_{21} + 0  & a_{22} + 0\end{pmatrix}
            \\ &= \begin{pmatrix} a_{11} & a_{12}\\a_{21} & a_{22}\end{pmatrix} = A.
        \end{align*}
        Portanto a matriz
        \[
        0_V = \begin{pmatrix}0 & 0\\0 & 0\end{pmatrix}
        \]
        é o vetor nulo.

        \item[A4)] Dada
        \[
        A = \begin{pmatrix} a_{11} & a_{12}\\a_{21} & a_{22}\end{pmatrix} \in \cp{M}_{2 \times 2}(\real),
        \]
        tome
        \[
        -A = \begin{pmatrix} -a_{11} & -a_{12}\\-a_{21} & -a_{22}\end{pmatrix} \in \cp{M}_{2 \times 2}(\real).
        \]
        Assim
        \begin{align*}
            A + (-A) & = \begin{pmatrix} a_{11} & a_{12}\\a_{21} & a_{22}\end{pmatrix} +
            \begin{pmatrix} -a_{11} & -a_{12}\\-a_{21} & -a_{22}\end{pmatrix}
            \\ &= \begin{pmatrix} a_{11} - a_{11} & a_{12} - a_{12}\\a_{21} - a_{21} & a_{22} - a_{22}\end{pmatrix}
            \\ &= \begin{pmatrix}0 & 0\\0 & 0\end{pmatrix} = 0_V.
        \end{align*}
        Logo a matriz
        \[
        -A = \begin{pmatrix} -a_{11} & -a_{12}\\-a_{21} & -a_{22}\end{pmatrix}
        \]
        é o vetor oposto.

        \item[M1)] Sejam $\alpha$, $\beta \in \real$ e
        \[
        A = \begin{pmatrix} a_{11} & a_{12}\\a_{21} & a_{22}\end{pmatrix} \in \cp{M}_{2 \times 2}(\real).
        \]
        Então:
        \begin{align*}
            (\alpha\beta)\cdot A & = (\alpha\beta)\begin{bmatrix} a_{11} & a_{12}\\a_{21} & a_{22}\end{bmatrix}
            \\ &= \begin{bmatrix} (\alpha\beta)a_{11} & (\alpha\beta)a_{12}\\(\alpha\beta)a_{21} & (\alpha\beta)a_{22}\end{bmatrix}
            \\ &= \begin{bmatrix} \alpha(\beta a_{11}) & \alpha(\beta a_{12})\\\alpha(\beta a_{21}) & \alpha(\beta a_{22})\end{bmatrix}
            \\ &= \alpha\cdot\begin{bmatrix} \beta a_{11} & \beta a_{12}\\\beta a_{21} & \beta a_{22}\end{bmatrix}
            \\ &= \alpha\cdot\left(\beta\cdot\begin{bmatrix} a_{11} & a_{12}\\a_{21} & a_{22}\end{bmatrix}\right)
            \\ &= \alpha\cdot(\beta\cdot A)
        \end{align*}

        \item[M2)] Seja $1 \in \real$ e
        \[
        A = \begin{pmatrix} a_{11} & a_{12}\\a_{21} & a_{22}\end{pmatrix} \in \cp{M}_{2 \times 2}(\real).
        \]
        Temos
        \begin{align*}
            1\cdot A & = \begin{pmatrix} a_{11} & a_{12}\\a_{21} & a_{22}\end{pmatrix}
            \\ &= \begin{pmatrix} 1\cdot a_{11} & 1\cdot a_{12}\\1\cdot a_{21} & 1\cdot a_{22}\end{pmatrix}
            \\ &= \begin{pmatrix} a_{11} & a_{12}\\a_{21} & a_{22}\end{pmatrix} = A.
        \end{align*}

        \item Sejam $\alpha \in \real$ e
        \[
        A = \begin{bmatrix} a_{11} & a_{12}\\a_{21} & a_{22}\end{bmatrix} \in \cp{M}_{2 \times 2}(\real),
        B = \begin{bmatrix} b_{11} & b_{12}\\b_{21} & b_{22}\end{bmatrix} \in \cp{M}_{2 \times 2}(\real).
        \]
        Temos
        \begin{align*}
            \alpha\cdot(A + B) & = \alpha\cdot\left(
            \begin{bmatrix} a_{11} & a_{12}\\a_{21} & a_{22}\end{bmatrix} +
            \begin{bmatrix} b_{11} & b_{12}\\b_{21} & b_{22}\end{bmatrix}\right)
            \\ &= \alpha\cdot\left(\begin{bmatrix} a_{11} + b_{11} & a_{12} + b_{12}\\a_{21} + b_{21} & a_{22} + b_{22}\end{bmatrix}\right)
            \\ &= \begin{bmatrix} \alpha(a_{11} + b_{11}) & \alpha(a_{12} + b_{12})\\\alpha(a_{21} + b_{21}) & \alpha(a_{22} + b_{22})\end{bmatrix}
            \\ &= \begin{bmatrix} \alpha a_{11} + \alpha b_{11} & \alpha a_{12} + \alpha b_{12}\\\alpha a_{21} + \alpha b_{21} & \alpha a_{22} + \alpha b_{22}\end{bmatrix}
            \\ &= \begin{bmatrix} \alpha a_{11} & \alpha a_{12}\\ \alpha a_{21} & \alpha a_{22}\end{bmatrix}
            + \begin{bmatrix} \alpha b_{11} & \alpha b_{12}\\ \alpha b_{21} & \alpha b_{22}\end{bmatrix}
            \\ &= \alpha\begin{bmatrix} a_{11} & a_{12}\\a_{21} & a_{22}\end{bmatrix} +
            \alpha\begin{bmatrix} b_{11} & b_{12}\\b_{21} & b_{22}\end{bmatrix}
            \\ &= \alpha\cdot A + \alpha\cdot B
        \end{align*}

        \item[D2)] Sejam $\alpha$, $\beta \in \real$ e
        \[
        A = \begin{bmatrix} a_{11} & a_{12}\\a_{21} & a_{22}\end{bmatrix} \in \cp{M}_{2 \times 2}(\real).
        \]
        Temos
        \begin{align*}
            (\alpha + \beta)\cdot A & = (\alpha + \beta)\cdot\begin{bmatrix} a_{11} & a_{12}\\a_{21} & a_{22}\end{bmatrix}
            \\ &= \begin{bmatrix} (\alpha + \beta)a_{11} & (\alpha + \beta)a_{12}\\(\alpha + \beta)a_{21} & (\alpha + \beta)a_{22}\end{bmatrix}
            \\ &= \begin{bmatrix} \alpha a_{11} + \beta a_{11} & \alpha a_{12} + \beta a_{12}\\\alpha a_{21} + \beta a_{21} & \alpha a_{22} + \beta a_{22}\end{bmatrix}
            \\ &= \begin{bmatrix} \alpha a_{11} & \alpha a_{12}\\ \alpha a_{21} & \alpha a_{22}\end{bmatrix}
            + \begin{bmatrix} \beta a_{11} & \beta a_{12}\\ \beta a_{21} & \beta a_{22}\end{bmatrix}
            \\ &= \alpha\cdot\begin{bmatrix} a_{11} & a_{12}\\a_{21} & a_{22}\end{bmatrix}  + \beta\cdot\begin{bmatrix} a_{11} & a_{12}\\a_{21} & a_{22}\end{bmatrix}
            \\ &= \alpha\cdot A + \beta\cdot A.
        \end{align*}
    \end{itemize}
    Assim $V = \cp{M}_{2 \times 2}(\real)$ é um espaço vetorial sobre $\real$.
\end{document}
