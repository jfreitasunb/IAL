%!TEX program = xelatex
%!TEX root = IAL.tex

\chapter{Transformações Lineares}

Em todo esse capítulo $\cp{K}$ denotará um corpo.

\section{Conceitos Básicos}

\begin{definicao}
  Sejam $(V, +, \cdot)$ e $(W, \oplus, \otimes)$ espaços vetoriais sobre um corpo $\cp{K}$. Uma função $T \colon V \to W$ é uma \textbf{transformação linear}\index{Transformação linear} se
  \begin{enumerate}[label={\roman*})]
    \item $T(u_1 + u_2) = T(u_1) \oplus T(u_2)$ para todos $u_1$, $u_2 \in V$;
    \item $T(\lambda \cdot u) = \lambda \otimes T(u)$ para todo $\lambda \in \cp{K}$ e todo $u \in V$.
  \end{enumerate}
\end{definicao}

\begin{observacoes}
    \begin{enumerate}[label={\roman*})]
        \item Se $T \colon V \to V$ é uma transformação linear de $V$ em $V$, então $T$ é chamada de um \textbf{operador linear}.\index{Transformação linear!operador linear}

        \item Para simplificar a notação, vamos adotar os mesmos símbolos para indicar a soma e o produto por escalar nos espaços vetoriais que aparecerão no decorrer do conteúdo. No entanto, deve-se estar ciente que estes símbolos podem ter significados diferentes, dependendo do espaço vetorial em questão.
    \end{enumerate}
\end{observacoes}

\begin{lema}
  Sejam $V$ e $W$ espaços vetoriais sobre $\cp{K}$ e $T \colon V \to W$ uma transformação linear. Então:
  \begin{enumerate}[label={\roman*})]\label{transformacao_linear_propriedades_basicas}
    \item $T(0_V) = 0_W$, onde $0_V$ e $0_W$ denotam os vetores nulos de $V$ e $W$, respectivamente.

    \item $T(-u) = -T(u)$, para cada $u \in V$.

    \item $T(\alpha_1u_1 + \alpha_2u_2 + \cdots + \cdots \alpha_mu_m) = \alpha_1 T(u_1) + \alpha_2 T(u_2) + \cdots + \alpha_mT(u_m)$, onde $\alpha_i \in \cp{K}$ e $u_i \in V$ para $i = 1$, \dots, $m$.
  \end{enumerate}
\end{lema}
\begin{prova}
  \begin{enumerate}[label={\roman*})]
    \item Note que
    \[
      0_W + T(0_V) = T(0_V + 0_V) = T(0_V) + T(0_V),
    \]
    ou seja, $T(0_V) = 0_W$.

    \item Basta observar que $-u = (-1_\cp{K})u$ e daí
    \[
      T(-u) = T((-1_\cp{K})u) = -1_\cp{K}T(u) = -T(u).
    \]

    \item Por indução em $m$. Se $m = 2$, então
    \[
      T(\alpha_1u_1 + \alpha_2u_2) = T(\alpha_1u_1) + T(\alpha_2u_2) = \alpha_1T(u_1) + \alpha_2T(u_2).
    \]
    Suponha que para $m = p$ tenhamos
    \[
      T(\alpha_1u_1 + \alpha_2u_2 + \cdots + \alpha_pu_p) = \alpha_1T(u_1) + \alpha_2T(u_2) + \cdots + \alpha_pT(u_p).
    \]
    Vamos mostra que é válido para $m = p + 1$. De fato,
    \begin{align*}
      T(\alpha_1u_1 + \alpha_2u_2 + \cdots + \alpha_{p + 1}u_{p + 1}) &= T([\alpha_1u_1 + \alpha_2u_2 + \cdots + \alpha_pu_p] + \alpha_{p + 1}u_{p + 1}) \\ &= T(\alpha_1u_1 + \alpha_2u_2 + \cdots + \alpha_{p + 1}u_{p + 1}) + T(\alpha_{p+1}u_{p+1}) \\ &= \alpha_1T(u_1) + \alpha_2T(u_2) + \cdots + \alpha_pT(u_p) + \alpha_{p+1}T(u_{p+1}).
    \end{align*}
  \end{enumerate}
\end{prova}

\begin{lema}\label{condicaoalternativaTL}
    Sejam $V$ e $W$ espaços vetoriais sobre $\cp{K}$. Então uma função $T \colon V \to W$ é uma transformação linear se, e somente se,
    \[
    T(\lambda u_1 + u_2) = \lambda T(u_1) + T(u_2),
    \]
    para todos $u_1$, $u_2 \in V$ e todo $\lambda \in \cp{K}$.
\end{lema}
\begin{prova}
    Deixada a cargo do leitor.
\end{prova}

\begin{exemplo}
  \begin{enumerate}[label={\arabic*})]
    \item Sejam $V$ e $W$ $\cp{K}$-espaços vetoriais. A função $T \colon V \to W$ dada por $T(u) = 0_W$ para todo $u \in V$ é uma transformação linear.

    \item Seja $V$ um $\cp{K}$-espaço vetorial. A função $T \colon V \to V$ dada por $T(u) = u$ para todo $u \in V$ é uma transformação linear.

    \item Considere $\real$ como um $\real$-espaço vetorial. Dado $a \in \real$, defina $T_a \colon \real \to \real$ por $T_a(x) = ax$ para todo $x \in \real$. Então $T_a$ é uma transformação linear.

    \item Seja $T \colon \real \to \real$ dada por $T(x) = e^x$. Então $T$ não é uma transformação linear pois $T(0) \ne 0$.

    \item Considere a função $T \colon \real^2 \to \real^2$ definida por $T(x, y) = (2x, 2y)$. Então $T$ é uma transformação linear.
    \begin{solucao}
        Sejam $u_1 = (a, b)$, $u_2 = (c, d) \in \real^2$ e $\lambda \in \real$. Temos:
        \begin{itemize}
            \item $T(u_1 + u_2) = T((a, b) + (c, d)) = T(a + c, b + d) = (2(a + c), 2(b + d)) = (2a + 2c, 2b + 2d) = (2a, 2b) + (2c, 2d) = T(a, b) + T(c, d) = T(u_1) + T(u_2)$

            \vspace*{.3cm}
            \item $T(\lambda u_1) = T(\lambda (a, b)) = T(\lambda a, \lambda b) = (2\lambda a, 2\lambda b) = \lambda(2a, 2b) = \lambda T(a, b) = \lambda T(u_1)$
        \end{itemize}

        Logo $T$ é uma transformação linear.
    \end{solucao}

    \item No caso geral, seja $k \in \real$ um número fixo. Considere a função $T_k \colon \real^2 \to \real^2$ definida por $T_k(x, y) = (kx, ky)$. Então $T$ é uma transformação linear.
    \begin{solucao}
        Sejam $u_1 = (a, b)$, $u_2 = (c, d) \in \real^2$ e $\lambda \in \real$. Temos:
        \begin{itemize}
            \item $T_k(u_1 + u_2) = T_k((a, b) + (c, d)) = T_k(a + c, b + d) = (k(a + c), k(b + d)) = (ka + kc, kb + kd) = (ka, kb) + (kc, kd) = T_k(a, b) + T_k(c, d) = T_k(u_1) + T_k(u_2)$

            \vspace*{.3cm}

            \item $T_k(\lambda u_1) = T_k(\lambda (a, b)) = T_k(\lambda a, \lambda b) = (k\lambda a, k\lambda b) = \lambda(ka, kb) = \lambda T_k(a, b) = \lambda T_k(u_1)$
        \end{itemize}

        Logo $T_k$ é uma transformação linear.
    \end{solucao}

    \item Seja $T \colon M_2(\real) \to M_{2 \times 3}(\real)$ dada por
    \[
        T\left(\begin{bmatrix}a & b\\c & d\end{bmatrix}\right) = \begin{bmatrix}ad & 1 & 0\\0 & 0 & 0\end{bmatrix}.
    \]
    Então $T$ não é transformação linear.
    \begin{solucao}
        Nesse caso como
        \[
            T\left(\begin{bmatrix}0 & 0\\0 & 0\end{bmatrix}\right) = \begin{bmatrix}0 & 1 & 0\\0 & 0 & 0\end{bmatrix} \ne \begin{bmatrix}0 & 0 & 0\\0 & 0 & 0\end{bmatrix}
        \]
        então $T$ não é uma transformação linear.
    \end{solucao}

    \item Seja $T \colon \real^3 \to \real^2$ dada por $T(x, y, z) = (x^2, 3y, 0)$. Então $T$ não é uma transformação linear.
    \begin{solucao}
        Nesse caso tome $u_1 = (1, 0, 1)$ e $u_2 = (1, 1, 0) \in \real^3$. Temos
        \begin{align*}
            T(u_1 + u_2) = T((1, 0, 1) + (1, 1, 0)) = T(2, 1, 1) = (4, 3, 0)\\
            T(u_1) + T(u_2) = (1, 3, 0) + (1, 3, 0) = (2, 6, 0).
        \end{align*}
        Logo $T(u_1 + u_2) \ne T(u_1) + T(u_2)$. Observe que nesse caso não podemos usar o vetor nulo uma vez que $T(0, 0, 0) = (0, 0, 0)$.
    \end{solucao}

    \item Considere a função $T \colon \real^2 \to \real^3$ dada por $T(x, y) = (x, 2x - y, 3x + 4y)$. Então $T$ é uma transformação linear.
    \begin{solucao}
        Sejam $u_1 = (a, b)$, $u_2 = (c, d) \in \real^2$ e $\lambda \in \real$. Temos
        \begin{align*}
            T(u_1 + u_ 2) &= T((a, b) + (c, d)) = T(a + c, b + d) \\ &= (a + c, 2(a + c) - (b + d), 3(a + c) + 4(b + d)) \\ &= (a + c, 2a - b + 2c - d, 3a + 4b + 3c + 4d)\\
            T(u_1) + T(u_2) &= T(a, b) + T(c, d) = (a, 2a - b, 3a + 4b) + (c, 2c - d, 3c + 4d) \\ &= (a + c, 2a - b + 2c - d, 3a + 4b + 3c + 4d)
        \end{align*}
        Logo, $T(u_1 + u_2) = T(u_1) + T(u_2)$.

        Agora
        \begin{align*}
            T(\lambda u_1) &= T(\lambda (a, b)) = T(\lambda a, \lambda b) = (\lambda a, 2\lambda a - \lambda b, 3\lambda a + 4\lambda 4) \\ &= \lambda (a, 2a - b, 3a + 4b) = \lambda T(a, b) \\ &= \lambda T(u_1).
        \end{align*}

        Logo, $T$ é uma transformação linear.
    \end{solucao}

    \item Sejam $\cp{K}^3$ e $\cp{M}_2(\cp{K})$ $\cp{K}$-espaços vetoriais. Defina $T \colon \cp{K}^3 \to \cp{M}_2(\cp{K})$ por
    \[
      T(a, b, c) = \begin{bmatrix}
        a+b & 0_\cp{K}\\
        0_\cp{K} & c - b
      \end{bmatrix}.
    \]
    Então $T$ é uma transformação linear.
    \begin{solucao}
        Vamos usar o Lema \ref{condicaoalternativaTL}. Para isso, sejam $(a, b, c)$, $(d, e, f) \in \cp{K}^3$ e $\lambda \in \cp{K}$. Temos:
        \begin{align*}
            T(\lambda(a,b,c) + (d,e,f)) &= T(\lambda a + d, \lambda b + e, \lambda c + f) \\ &= \begin{bmatrix}
            (\lambda a + d) + (\lambda b + e) & 0_\cp{K}\\
            0_\cp{K} & (\lambda c + f) - (\lambda b + e)
            \end{bmatrix} \\ &= \begin{bmatrix}
            \lambda a + \lambda b & 0_\cp{K}\\
            0_\cp{K} & \lambda c - \lambda b
            \end{bmatrix} + \begin{bmatrix}
            d + e & 0_\cp{K}\\
            0_\cp{K} & f - e
        \end{bmatrix} \\ &= \lambda T(a,b,c) + T(d,e,f).
        \end{align*}
        Logo, $T$ é uma transformação linear.
    \end{solucao}

    \item Seja $T \colon \mathcal{P}_n(\real) \to \mathcal{P}_{n + 1}(\real)$ dada por $T(p(x)) = xp(x)$. Então $T$ é uma transformação linear.
    \begin{solucao}
        Aqui também vamos usar o Lema \ref{condicaoalternativaTL}. Para isso, sejam $p(x)$, $q(x) \in \mathcal{P}_n(\real)$ e $\lambda \in \real$. Temos:
        \begin{align*}
            T(\lambda p(x) + q(x)) &= x(\lambda p(x) + q(x)) = x\lambda p(x) + xq(x) \\ &= \lambda xp(x) + xq(x) \\ &= \lambda T(p(x)) + T(q(x)).
        \end{align*}
        Logo, $T$ é uma transformação linear.
    \end{solucao}

    \item Seja $\mathcal{P}(\complex)$ um $\complex$-espaço vetorial e considere $D \colon \mathcal{P}(\complex) \to \mathcal{P}(\complex)$ dado por
    \[
      D(a_0 + a_1x + a_2x^2 + \cdots + a_nx^n) = a_1 + 2a_2x + \cdots + na_nx^{n-1}.
    \]
    Então $D$ é uma transformação linear.

    \item Seja $\mathcal{C}([a,b], \real) = \{ f : [a,b] \to \real \mid f \mbox{ é uma função contínua}\}$. é imediato verificar que $\mathcal{C}([a,b], \real)$ é um $\real$-espaço vetorial. Defina $T \colon \mathcal{C}([a,b], \real) \to \real$ por
    \[
      T(f(x)) = \int_a^bf(x)dx.
    \]
    Então $T$ é uma transformação linear.
  \end{enumerate}
\end{exemplo}

Sejam $V$ e $W$ espaços vetoriais sobre um corpo $\cp{K}$. Considere $\mathcal{B} = \{v_1, v_2, \dots, v_n\}$ uma base de $V$ e $T \colon V \to W$ uma transformação linear.

Como $\mathcal{B}$ é um base de $V$, então para todo $u \in V$ existem únicos escalares $\alpha_1$, $\alpha_2$, \dots, $\alpha_n \in \cp{K}$ tais que
\[
    u = \alpha_1v_1 + \alpha_2v_2 + \cdots + \alpha_nv_n.
\]
Assim
\begin{align*}
    T(u) &= T(\alpha_1v_1 + \alpha_2v_2 + \cdots + \alpha_nv_n) = T(\alpha_1v_1) + T(\alpha_2v_2) + \cdots + T(\alpha_nv_n) \\ &= \alpha_1T(v_1) + \alpha_2T(v_2) + \cdots + \alpha_nT(v_n),
\end{align*}
isto é, $T(u) \in Span(T(v_1), T(v_2), \dots, T(v_n))$. Portanto para conhecer a transformação $T$ basta conhecer seus valores em uma base de $V$. Mais ainda, temos o seguinte resultado:

\begin{teorema}\label{existencia_de_transformacao_unica_dado_valores}
  Sejam $V$ e $W$ $\cp{K}$-espaços vetoriais. Se $\{u_1, \dots, u_n\}$ é uma base de $V$ e se $\{w_1, \dots, w_n\} \subseteq W$, então existe uma única transformação linear $T \colon V \to W$ tal que $T(u_i) = w_i$ para cada $i = 1$, \dots, $n$.
\end{teorema}

\begin{exemplos}
    \begin{enumerate}[label={\arabic*})]
        \item Seja $\mathcal{B} = \{v_1 = (0, 1, 0); v_2 = (1, 0, 1); v_3 = (1, 1, 0)\}$ uma base de $\real^3$. Dada $T \colon \real^3 \to \real^2$ tal que
        \begin{align*}
            T(v_1) &= (1, -2)\\
            T(v_2) &= (3, 0)\\
            T(v_3) &= (1, 1)
        \end{align*}
        Determine $T(x, y, z)$ para todo $(x, y, z) \in \real^3$ e depois calcule $T(1, 2, 3)$.
        \begin{solucao}
            Como $\mathcal{B}$ é base de $\real^3$, então para todo $(x, y, z) \in \real^3$ existem $a$, $b$, $c \in \real$ tais que:
            \begin{align*}
                (x, y, z) = a(0, 1, 0) + b(1, 0, 1) + c(1, 1, 0)
            \end{align*}
            daí obtemos
            \[
                \begin{cases}
                    b + c = x\\
                    a + c = y\\
                    b = z
                \end{cases}
            \]
            Assim $a = y - x + z$, $b = z$ e $c = x - z$. Logo
            \[
                (x, y, z) = (y - x + z)v_1 + zv_2 + (x - z)v_3.
            \]
            Agora, aplicando a transformação $T$ na expressão anterior:
            \begin{align*}
                T(x, y, z) &= T((y - x + z)v_1 + zv_2 + (x - z)v_3) \\ &= (y - x + z)T(v_1) + zT(v_2) + (x - z)T(v_3) \\ &= (y - x + z)(1, -2) + z(3, 0) + (x - z)(1, 1) \\ &= (y - x + z, -2y - 2z + 2x) + (3z, 0) + (x - z, x - z) \\ &= (y + 3z, 3x - 2y - 3z) \in \real^2.
            \end{align*}

            Assim
            \begin{align*}
                T(1, 2, 3) = (2 + 3\cdot3, 3\cdot 1 - 2 \cdot 2 - 3 \cdot 3) = (11, -10).
            \end{align*}
        \end{solucao}
        \item Os vetores $v_1 = (1,2)$ e $v_2 = (3,4)$ são L.I em $\real^2$ e assim formam uma base de $\real^2$. Assim pelo Teorema \ref{existencia_de_transformacao_unica_dado_valores}, sabemos que existe uma única transformação linear $T \colon \real^2 \to \real^3$ tal que
        \begin{align*}
            T(v_1) &= T(1,2) = (3,2,1)\\
            T(v_2) &= T(3,4) = (6,5,4).
        \end{align*}
        Determine  $T(x, y)$ e depois calcule $T(1,0)$.
        \begin{solucao}
            Inicialmente vamos encontrar $\alpha$, $\beta \in \real$ tais que:
            \[
              (x, y) = \alpha(1,2) + \beta(3,4).
            \]
            Isso resulta no sistema
            \[
              \begin{cases}
                \alpha + 3\beta = x\\
                2\alpha + 4\beta = y
              \end{cases}
            \]
            cuja solução é:
            \[
              \alpha = \dfrac{3y - 4x}{2}\quad \mbox{e}\quad \beta = \dfrac{2x - y}{2}.
            \]
            Assim
            \begin{align*}
              (x, y) &= \dfrac{3y - 4x}{2}(1, 2) + \dfrac{2x - y}{3}(3, 4)\\
              T(x, y) &= T\left(\dfrac{3y - 4x}{2}(1, 2) + \dfrac{2x - y}{3}(3, 4)\right)\\
              T(x, y) &= \left(\dfrac{3y - 4x}{2}\right)T(1, 2) + \left(\dfrac{2x - y}{3}\right)T(3, 4)\\
              T(x, y) &= \left(\dfrac{3y - 4x}{2}\right)(3, 2, 1) + \left(\dfrac{2x - y}{3}\right)(6, 5, 4)\\
              T(x, y) &= \left(\dfrac{3y}{2}, \dfrac{y + 2x}{2}, \dfrac{4x - y}{2}\right)
            \end{align*}
            Daí
            \[
              T(1,0) =(0,1,2).
            \]
        \end{solucao}
    \end{enumerate}
\end{exemplos}

\begin{definicao}
  Sejam $V$ e $W$ $\cp{K}$-espaços vetoriais e $T \colon V \to W$ uma transformação linear.
  \begin{enumerate}[label={\roman*})]
    \item O conjunto
    \[
      \ker(T) = \{u \in V \mid T(u) = 0_W\}
    \]
    é chamado de \textbf{kernel}\index{Transformação linear!kernel} ou \textbf{núcleo}\index{Transformação linear!núcleo} de $T$. (O núcleo de $T$ também pode ser denotado por $Nuc(T)$.)

    \item O conjunto
    \[
      \im(T) = \{u \in W \mid \mbox{ existe } v \in V \mbox{ tal que } T(v) = u\}
    \]
    é chamado de \textbf{imagem}\index{Transformação linear!imagem} de $T$.
  \end{enumerate}
\end{definicao}

\begin{proposicao}
  Sejam $V$ e $W$ $\cp{K}$-espaços vetoriais e $T \colon V \to W$ uma transformação linear. Então:
  \begin{enumerate}[label={\roman*})]
    \item $\ker(T)$ é um subespaço de $V$;
    \item $\im(T)$ é um subespaço de $W$.
  \end{enumerate}
\end{proposicao}
\begin{prova}
  \begin{enumerate}[label={\roman*})]
    \item Inicialmente $\ker(T) \ne \emptyset$ pois $T(0_V) = 0_W$, ou seja, $0_V \in \ker(T)$. Agora, sejam $u_1$, $u_2 \in \ker(T)$ e $\lambda \in \cp{K}$. Precisamos mostrar que $\lambda u_1 + u_2 \in \cp{K}$, isto é, precisamos mostrar que $\lambda u_1 + u_2 \in \ker(T)$. Temos
    \[
      T(\lambda u_1 + u_2) = \lambda T(u_1) + T(u_2) = 0_W.
    \]
    Logo, $\ker(T)$ é um subespaço de $V$.

    \item Inicialmente $0_W \in \im(T)$ pois $0_W = T(0_V)$ e daí $\im(T) \ne \emptyset$. Sejam $w_1$, $w_2 \in \im(T)$ e $\lambda \in \cp{K}$. Então existem $u_1$, $u_2 \in V$ tais que $w_1 = T(u_1)$ e $w_2 = T(u_2)$. Assim
    \[
      \lambda w_1 + w_2 = \lambda T(u_1) + T(u_2) = T(\lambda u_1) + T(u_2) = T(\lambda u_1 + u_2)
    \]
    e então $\lambda w_1 + w_2 \in \im(T)$. Portanto, $\im(T)$ é um subespaço de $W$.
  \end{enumerate}
\end{prova}

\begin{lema}\label{transformacao_gera_imagem}
  Sejam $V$ e $W$ $\cp{K}$-espaços vetoriais e $T \colon V \to W$ uma transformação linear. Se $\mathcal{B} = \{u_1, \dots, u_n\}$ é uma base de $V$, então $\{T(u_1), \dots, T(u_n)\}$ gera $\im(T)$.
\end{lema}
\begin{prova}
  Seja $w \in \im T$. Por definição, existe $u \in V$ tal que $T(u) = w$. Como $\mathcal{B}$ é uma base de $V$, então existem $\alpha_1$, \dots, $\alpha_n \in \cp{K}$ tais que $u = \alpha_1u_1 + \cdots + \alpha_nu_n$. Daí
  \[
    w = T(u) = \alpha_1T(u_1) + \cdots + \alpha_nT(u_n),
  \]
  ou seja, todo vetor de $\im T$ é uma combinação linear de $T(u_1)$, \dots, $T(u_n)$. Portanto $\im(T) = [T(u_1), \dots, T(u_n)]$ como queríamos.
\end{prova}

\begin{exemplos}
  \begin{enumerate}[label={\arabic*})]
    \item Seja $T \colon \real^3 \to \cp{M}_2(\real)$ dada por
    \[
      T(a,b,c) = \begin{bmatrix}
        a + b & 0\\
        0 & c - b
      \end{bmatrix}.
    \]
    Determine $\ker(T)$ e $\im(T)$.
    \begin{solucao}
      Temos
      \[
        T(a,b,c) = \begin{bmatrix}
          0 & 0\\
          0 & 0
        \end{bmatrix}
      \]
      se, e s\'o se, $a = -b$ e $c = b$. Daí
      \[
        \ker(T) = \{(a,b,c) \in \real^3 \mid a = -b, c = b\} = \{(-b,b,b) \mid b \in \real\}.
      \]
      Note que $\{(1,1,1)\}$ é uma base de $\ker(T)$, ou seja, $\dim_\real \ker(T) = 1$.

      Agora,
      \begin{align*}
        \im(T) &= \{v \in \cp{M}_2(\real) \mid \mbox{ existe } u \in \real^3 \mbox{ de modo que } T(u) = v\}\\
        \im(T) &= \left\{ \begin{bmatrix}
          a + b & 0\\
          0 & c - b
        \end{bmatrix} \mid a, b, c \in \real\right\}.
      \end{align*}
      Assim temos
      \begin{align*}
        \begin{bmatrix}
          a + b & 0\\
          0 & c - b
        \end{bmatrix} &= (a + b) \begin{bmatrix}
          1 & 0\\ 0 & 0
        \end{bmatrix} + (c - b) \begin{bmatrix}
          0 & 0\\ 0 & 1
        \end{bmatrix}
      \end{align*}
      e é fácil ver que
      \[
        \mathcal{B}' = \left\{ \begin{bmatrix}
          1 & 0\\ 0 & 0
        \end{bmatrix}; \begin{bmatrix}
          0 & 0\\ 0 & 1
        \end{bmatrix}\right\}
      \]
      é um conjunto gerador de $\im(T)$ e é L.I., ou seja, é uma base de $\im(T)$, com isso $\dim_\real\im(T) = 2$. Observe que
      \[
        \dim_\real\ker(T) + \dim_\real\im(T) = 3 = \dim_\real\real^3.
      \]
    \end{solucao}

    \item Seja $T \colon \real^2 \to \real$ dada por $T(x,y) = x + y$. Determine $\ker(T)$ e $\im(T)$.
    \begin{solucao}
      Temos
      \begin{align*}
        \ker(T) &= \{(x,y) \in \real^2 \mid T(x,y) = 0\} = \{(x,y) \in \real^2 \mid x + y = 0\}\\
        \ker(T) &= \{(x,-x) \in \real^2 \mid x \in \real\}.
      \end{align*}
      Assim $\{(1,-1)\}$ é uma base de $\ker(T)$, ou seja, $\dim_\real\ker(T) = 1$.

      Agora
      \[
        \im(T) = \{w \in \real \mid \mbox{ existe } (x,y) \in \real^2 \mbox{ tal que } T(x,y) = w\}.
      \]
      Assim dado $w \in \real$ um número real qualquer, tome o elemento $(w,0) \in \real^2$. Temos $T(w,0) = w + 0 = w$. Logo $\im(T) = \real$ e então $\dim_\real\im(T) = 1$.

      Novamente temos
      \[
        \dim_\real\ker(T) + \dim_\real\im(T) = 2 = \dim_\real\real^2.
      \]
    \end{solucao}
  \end{enumerate}
\end{exemplos}

\begin{teorema}[Teorema do Núcleo e da Imagem]\label{teorema_do_nucleo_e_da_imagem}
  Sejam $V$ e $W$ $\cp{K}$-espaços vetoriais com $\dim_\cp{K}V$ finita. Seja $T : V \to W$ uma transformação linear. Então
  \[
    \dim_\cp{K}V = \dim_\cp{K}\ker(T) + \dim_\cp{K}\im(T).
  \]
\end{teorema}

\begin{definicao}
  Sejam $V$ e $W$ $\cp{K}$-espaços vetoriais e $T \colon V \to W$ uma transformação linear.
  \begin{enumerate}[label={\roman*})]
    \item Dizemos que $T$ é \textbf{injetora}\index{Transformação linear!injetora} se dados $u_1$, $u_2 \in V$ tais que $T(u_1) = T(u_2)$, então $u_1 = u_2$. De modo equivalente, se $u_1$, $u_2 \in V$ são tais que $u_1 \ne u_2$, então $T(u_1) \ne T(u_2)$.

    \item Dizemos que $T$ é \textbf{sobrejetora}\index{Transformação linear!sobrejetora} se $\im T = W$. Em outras palavras, $T$ é \textbf{sobrejetora} se para todo $w \in W$, existe $u \in V$ tal que $T(u) = w$.
  \end{enumerate}
\end{definicao}

\begin{exemplos}
  \begin{enumerate}[label={\arabic*})]
    \item A transformação linear $T \colon \real^2 \to \real$ dada por $T(x,y) = x + y$ é sobrejetora, mas não é injetora.

    \item A transformação linear $T \colon V \to V$ dada por $T(u) = u$ é injetora e sobrejetora.
  \end{enumerate}
\end{exemplos}

\begin{proposicao}\label{caracteriza_transformacao_injetora}
  Sejam $V$ e $W$ espaços vetoriais sobre um corpo $\cp{K}$ e $T \colon V \to W$ uma transformação linear.
    \begin{enumerate}[label={\roman*})]
      \item $T$ é injetora se, e somente se, $\ker(T) = \{0_V\}$.

      \item Se $\dim V = \dim W$ e $T$ é injetora, então $T$ transforma bases em bases, isto é, se $\mathcal{B} = \{v_1, v_2, \dots, v_n\}$ é uma base de $V$, então $\{T(v_1), T(v_2), \dots, T(v_n)\}$ é uma base de $W$.

      \item Se $\dim V = \dim W$, então $T$ é injetora se, e somente se, $T$ é sobrejetora.
    \end{enumerate}
\end{proposicao}

\begin{observacoes}
  Seja $T \colon V \to W$ uma transformação linear, onde $\dim V < \infty$ e $\dim W < \infty$.
  \begin{enumerate}[label={\roman*})]
    \item Se $\dim V > \dim W$, então $T$ não pode ser injetora, isto é, $\ker(T) \ne \emptyset$.

    \item Se $\dim V < \dim W$, então $T$ não pode ser sobrejetora, isto é, $\im(T) \ne W$.
  \end{enumerate}
\end{observacoes}

\begin{definicao}
  Se $T \colon V \to W$ é uma transformação linear \textbf{injetora e sobrejetora}, então dizemos que $T$ é uma transformação linear \textbf{bijetora}\index{Transformação linear!bijetora}, ou que $T$ é um \textbf{isomorfismo}\index{Transformação linear!isomorfismo} do espaço $V$ no espaço $W$. Neste caso, dizemos que o espaço $V$ é isomorfo ao espaço $W$ e denotamos isso, escrevendo $V \cong W$.
\end{definicao}

\begin{teorema}
  Seja $V$ um espaço vetorial sobre $\cp{K}$  com $\dim V = n$, então $V \cong \cp{K}^n$.
\end{teorema}

\begin{observacoes}
  Seja $V$ um espaço vetorial sobre $\cp{K}$ com $\dim V = n$. Então do teorema anterior temos:
  \begin{enumerate}[label={\roman*})]
    \item Se $\cp{K} = \rac$, então $V \cong \rac^n$.

    \item Se $\cp{K} = \real$, então $V \cong \real^n$.

    \item Se $\cp{K} = \complex$, então $V \cong \complex^n$.
  \end{enumerate}
\end{observacoes}

\begin{exemplo}
  Seja $T \colon \cp{K}^4 \to \cp{M}_2(\cp{K})$ dada por
  \begin{align*}
    T(1,0,0,0) &= \begin{bmatrix}
      1 & 0\\
      0 & 0
    \end{bmatrix}; T(0,1,0,0) = \begin{bmatrix}
      0 & 1\\
      0 & 0
    \end{bmatrix}\\
    T(0,0,1,0) &= \begin{bmatrix}
      0 & 0\\
      1 & 0
    \end{bmatrix}; T(0,0,0,1) = \begin{bmatrix}
      0 & 0\\
      0 & 1
    \end{bmatrix}\\
  \end{align*}
  é fácil ver que $\ker(T) = \{(0,0,0,0)\}$. Além disso,
  \[
    \im(T) = \left[\begin{bmatrix}
      1 & 0\\
      0 & 0
    \end{bmatrix};\begin{bmatrix}
      0 & 1\\
      0 & 0
    \end{bmatrix};\begin{bmatrix}
      0 & 0\\
      1 & 0
    \end{bmatrix};\begin{bmatrix}
      0 & 0\\
      0 & 1
    \end{bmatrix}\right] = \cp{M}_2(\cp{K}).
  \]
  Daí $\dim_\cp{K}\im(T) = 4$ e novamente
  \[
    \dim_\cp{K}\ker(T) + \dim_\cp{K}\im(T) = 4 = \dim_\cp{K}\cp{M}_2(\cp{K}).
  \]
\end{exemplo}

\begin{exemplo}
  Considere $\complex^2$ e $\real^3$ como $\real$-espaços vetoriais e seja $T : \complex^2 \to \real^3$ dada por
  \[
    T(a+bi, c+di) = (a - c, b + 2d, a + b - c + 2d)
  \]
  onde $a$, $b$ e $c \in \real$. é fácil ver que $T$ é uma transformação linear. Seja $\{(1,0);(i,0);(0,1);(0,i)\}$ uma base de $\complex^2$. Temos pelo Lema \ref{transformacao_gera_imagem} que $\{T(1,0);T(i,0);T(0,1);T(0,i)\}$ gera $\im(T)$. Agora
  \begin{align*}
    T(1,0) &= (1,0,1);\quad T(i,0) = (0,1,1)\\
    T(0,1) &= (-1,0,-1);\quad T(0,i) = (0,2,2)
  \end{align*}
  e temos
  \begin{align*}
    T(1,0) &= -T(0,1)\\
    T(i,0) &= 2T(0,i).
  \end{align*}
  Observe que $T(1,0)$ não é múltiplo de $T(0,i)$. Logo o conjunto $\{T(1,0); T(0,i)\}$ é uma base de $\im(T)$.
\end{exemplo}

\begin{definicao}
  Dadas $S \colon V \to W$  e $T \colon W \to U$ transformações lineares,  definimos a \textbf{composição de $T$ com $S$}\index{Transformação linear!composição}  como a aplicação:
  \begin{align*}
    T \circ S \colon &V \to U\\
                     &v \mapsto (T\circ S)(u) = T(S(v))
  \end{align*}
\end{definicao}

\begin{proposicao}
  Se $S \colon V \to W$ e $T \colon W \to U$ são transformações lineares,  então a composição de $T\circ S$ também é uma transformação linear.
\end{proposicao}

\begin{exemplos}
  \begin{enumerate}
    \item Sejam $T \colon \real^2 \to \real^3$ e $S \colon \real^3 \to \real$ transformações lineares dadas por
      \begin{align*}
        T(x, y) &= (2x, y - x, x + 2y)\\
        S(x, y, z) &= 3x - 2y + z
      \end{align*}
      Então a composta $S \circ T$ existe e $S \circ T \colon \real^2 \to \real$ é dada por
      \begin{align*}
        (S \circ T)(x, y) = S(T(x, y)) = S(2x, y - x, x + 2y) = 3(2x) - 2(y - x) + (x + 2y) = 9x
      \end{align*}
      Observe que nesse caso não é possível fazer a composta de $T$ com $S$.

    \item Agora sejam $S \colon \real^2 \to \real^2$ e $T \colon \real^2 \to \real^2$ dadas por
      \begin{align*}
        S(x, y) = (y, 2x - y)\\
        T(x, y) = (x + 3y, y - x)
      \end{align*}
      Nesse caso existe tanto a composição $S \circ T$ quanto $T \circ S$ e
      \begin{align*}
        (S \circ T)(x, y) &= S(T(x, y)) \\ &= S(x + 3y, y - x) \\ &= (y - x, 2(x + 3y) - (y - x)) \\ &= (y - x, 3x + 5y)\\
        (T \circ S)(x, y) &= T(S(x, y)) \\ &= T(y, 2x - y) \\ &= (y + 3(2x - y), 2x - y - y) \\ &= (6x - 2y, 2x - 2y)
      \end{align*}
      Assim temos $S \circ T \ne T \circ S$.
  \end{enumerate}
\end{exemplos}

\begin{definicao}
  Dado $T \colon V \to V$ um operador linear,  dizemos que $T$ é \textbf{invertível}\index{Transformação linear!invertível} se existe outra aplicação $T^{-1} \colon V \to V$  tal que para qualquer $u \in V$,  $T^{-1}(T(u)) = u$.
\end{definicao}

\begin{exemplo}
  Seja $T \colon \real^2 \to \real^2$ dada por $T(x, y) = (4x - 3y, -2x + 2y)$. Tomo $T^{-1} \colon \real^2 \to \real^2$ dada por $T^{-1}(a, b) = (a + 3b/2, a + 2b)$. Então $T^{-1}$ é o operador inverso de $T$. De fato,
  \begin{align*}
    T^{-1}(T(x, y)) &= T^{-1}(4x - 3y, -2x + 2y) \\ &= (4x - 3y + 3(-2x + 2y)/2, 4x - 3y + 2(-2x + 2y)) \\ &= (x, y)
  \end{align*}
\end{exemplo}
\begin{observacoes}
  \begin{enumerate}[label={\roman*})]
    \item Seja $I = I_V = Id = Id_V \colon V \to V$  a transformação linear tal que $I(u) = Id(u) = u$  para todo $u \in V$.  Então $I = Id$ é chamado de \textbf{operador identidade}\index{Transformação linear!operador identidade}.

    \item Se $T \colon V \to V$ é uma transformação linear invertível,  então a aplicação $T^{-1} \colon V \to V$ também é uma transformação linear.

    \item Se $T \colon V \to V$ é um operador invertível,  então $T^{-1} \colon V \to V$ é chamado de \textbf{operador inverso de $T$}\index{Transformação linear!operador inverso}.
  \end{enumerate}
\end{observacoes}

\begin{proposicao}
  Seja $T \colon V \to V$ um operador linear.  Então:
  \begin{enumerate}[label={\roman*})]
    \item Se $T$ é invertível  e $T^{-1} \colon V \to V$ é o seu inverso,  então
        \[
            (T^{-1} \circ T) =  (T \circ T^{-1}) =  Id.
        \]

    \item Se $V$ tem dimensão finita,  então $T$ é invertível  se, e somente se,  $\ker(T) = \{0_V\}$.

    \item Se $T$ é um operador invertível,  então dada uma base $\{v_1, v_2, \dots, v_n\}$ de $V$,  o conjunto $\{T(v_1),  T(v_2),  \dots, T(v_n)\}$  é uma base de $V$.
  \end{enumerate}
\end{proposicao}

\begin{proposicao}
  Se $S \colon V \to V$ e $T \colon V \to V$  são operadores invertíveis,  então $S \circ T$ é invertível e
  \[
    (S \circ T)^{-1}  = T^{-1}  \circ S^{-1}.
  \]
\end{proposicao}

\section{Transformações Lineares e Matrizes} % (fold)
\label{sec:transformacoes_lineares_e_matrizes}

Sejam $V$ um espaço vetorial sobre $\cp{K}$ de dimensão $n \ge 1$ e $\mathcal{B} = \{v_1,\dots,v_n\}$ uma base de $V$. Sabemos que cada elemento de $V$ se escreve de modo único como combinação linear dos elementos de $\mathcal{B}$, isto é, dado $u \in V$ existem únicos escalares $\alpha_1$, \dots, $\alpha_n \in \cp{K}$ tais que
\[
  u = \alpha_1v_1 + \cdots + \alpha_nv_n.
\]

Assim vamos fixar uma ordem para os elementos da base $\mathcal{B}$ e por isso vamos chamá-la de \textbf{base ordenada}\index{Espaço vetorial!base ordenada} de $V$. Pela unicidade dos elementos $\alpha_1$, \dots, $\alpha_n$ acima, podemos denotar o vetor $u$ por
\[
  [u]_\mathcal{B} = \begin{bmatrix}
    \alpha_1\\
    \vdots\\
    \alpha_n
  \end{bmatrix}_\mathcal{B}
\]
e dizemos que $\alpha_1$, \dots, $\alpha_n$ são as \textbf{coordenadas de $u$ em relação \`a base ordenada $\mathcal{B}$}.

\begin{exemplos}
  \begin{enumerate}[label={\arabic*})]
    \item Considere $V = \complex^2$ como um $\complex$-espaço vetorial e seja $\mathcal{B} = \{(1,i);(i,0)\}$ uma base de $\complex^2$ \textit{(Verifique!)}. Dado o vetor $v = (i, 2+i) \in \complex^2$, quais suas coordenadas em relação a tal base?
    \begin{solucao}
      As coordenadas de $v$ em relação \`a base $\mathcal{B}$ serão $[v]_\mathcal{B} = (\alpha_1,\alpha_2)_\mathcal{B}$ onde $\alpha_1$, $\alpha_2 \in \complex^2$ são tais que
      \[
        v = \alpha_1(1,i) + \alpha_2(i,0).
      \]
      Resolvendo o sistema resultante obtemos $\alpha_1 = 1 - 2i$ e $\alpha_2 = 3 + i$. Assim
      \[
        [v]_\mathcal{B} = \begin{bmatrix}
          1 - 2i\\
          3 + i
        \end{bmatrix}_\mathcal{B}.
      \]
    \end{solucao}

    \item Agora considere $V = \complex^2$ como um $\real$-espaço vetorial e seja $\mathcal{A} = \{(1,1);(i,0);(1,i);(0,1)\}$ uma base de $\complex^2$ \textit{(Verifique!)}. Dado o vetor $v = (i,2+i) \in \complex^2$, quais suas coordenadas em relação a tal base?
    \begin{solucao}
      As coordenadas de $v$ em relação \`a base $\mathcal{A}$ serão $[v]_\mathcal{A} = (\alpha_1,\alpha_2,\alpha_3,\alpha_4)_\mathcal{A}$ onde $\alpha_1$, $\alpha_2$, $\alpha_3$, $\alpha_4 \in \complex$ são tais que
      \[
        v = \alpha_1(1,1) + \alpha_2(i,0) + \alpha_3(1,i) + \alpha_4(0,1).
      \]
      Resolvendo o sistema resultante obtemos $\alpha_1 = -1$, $\alpha_2 = 1$, $\alpha_3 = 1$ e $\alpha_4 = 3$. Assim
      \[
        [v]_\mathcal{A} = \begin{bmatrix}
          -1\\
          \phantom{-}1\\
          \phantom{-}1\\
          \phantom{-}3
        \end{bmatrix}_\mathcal{A}.
      \]
    \end{solucao}
  \end{enumerate}
\end{exemplos}

Agora, sejam $V$ e $W$ espaços vetoriais sobre $\cp{K}$ tais que $\dim_\cp{K}V = p \ge 1$ e $\dim_\cp{K} W = q \ge 1$. Vamos fixar bases ordenadas $\mathcal{B}_1 = \{u_1,\dots,u_p\}$ e $\mathcal{B}_2 = \{w_1,\dots,w_q\}$ de $V$ e $W$, respectivamente. Seja $T \colon V \to W$ uma transformação linear. Sabemos pelo Teorema \ref{existencia_de_transformacao_unica_dado_valores} que $T$ fica completamente determinada se conhecermos seus valores na base de $V$. Assim
\begin{align*}
  T(u_1) &= a_{11}w_1 + a_{21}w_2 + \cdots + a_{q1}w_q\\
  T(u_2) &= a_{12}w_1 + a_{22}w_2 + \cdots + a_{q2}w_q\\
  &\vdots\\
  T(u_p) &= a_{1p}w_1 + a_{2p}w_2 + \cdots + a_{qp}w_q
\end{align*}
para certos $a_{ij} \in \cp{K}$ onde $1 \le i \le q$ e $1 \le j \le p$.

Agora, seja $x \in V$. Então
\[
  x = \alpha_1u_1 + \alpha_2u_2 + \cdots + \alpha_pu_p
\]
onde $\alpha_1$, $\alpha_2$, \dots, $\alpha_p \in \cp{K}$. Daí
\begin{align*}
  T(x) &= T(\alpha_1u_1 + \alpha_2u_2 + \cdots + \alpha_pu_p) = \alpha_1T(u_1) + \alpha_2T(u_2) + \cdots + \alpha_pT(u_p)\\
  &= \alpha_1(a_{11}w_1 + a_{21}w_2 + \cdots + a_{q1}w_q) + \alpha_2(a_{12}w_1 + a_{22}w_2 + \cdots + a_{q2}w_q) + \cdots \\
  &+ \alpha_p(a_{1p}w_1 + a_{2p}w_2 + \cdots + a_{qp}w_q)\\
  &= (a_{11}\alpha_1 + a_{12}\alpha_2 + \cdots + a_{1p}\alpha_p)w_1 + (a_{21}\alpha_1 + a_{22}\alpha_2 + \cdots + a_{2p}\alpha_p)w_2 + \\ &\cdots + (a_{q1}\alpha_1 + a_{q2}\alpha_2 + \cdots + a_{qp}\alpha_p)w_q\\
\end{align*}
Note que os escalares $a_{11}\alpha_1 + a_{12}\alpha_2 + \cdots + a_{1p}\alpha_p$, $a_{21}\alpha_1 + a_{22}\alpha_2 + \cdots + a_{2p}\alpha_p$, \dots, $a_{q1}\alpha_1 + a_{q2}\alpha_2 + \cdots + a_{qp}\alpha_p$ são as coordenadas do vetor $T(x)$ em relação \`a base $\mathcal{B}_2$, isto é,
\[
  [T(x)]_{\mathcal{B}_2} = \begin{bmatrix}
    a_{11}\alpha_1 + a_{12}\alpha_2 + \cdots + a_{1p}\alpha_p\\
    a_{21}\alpha_1 + a_{22}\alpha_2 + \cdots + a_{2p}\alpha_p\\
    \vdots\\
    a_{q1}\alpha_1 + a_{q2}\alpha_2 + \cdots + a_{qp}\alpha_p
  \end{bmatrix}.
\]
Por outro lado,
\[
  \underbrace{\begin{bmatrix}
    a_{11} & a_{12} & \dots & a_{1p}\\
    a_{21} & a_{22} & \dots & a_{2p}\\
    \vdots & & \ddots & \vdots\\
    a_{q1} & a_{q2} & \dots & a_{qp}
  \end{bmatrix}}_{A}\begin{bmatrix}
    \alpha_1\\
    \alpha_2\\
    \vdots\\
    \alpha_p
  \end{bmatrix} = \begin{bmatrix}
    a_{11}\alpha_1 + a_{12}\alpha_2 + \cdots + a_{1p}\alpha_p\\
    a_{21}\alpha_1 + a_{22}\alpha_2 + \cdots + a_{2p}\alpha_p\\
    \vdots \\
    a_{q1}\alpha_1 + a_{q2}\alpha_2 + \cdots + a_{qp}\alpha_p
  \end{bmatrix}
\]
e daí podemos escrever
\[
  [T(x)]_{\mathcal{B}_2} = A[x]_{\mathcal{B}_1}
\]
onde $A \in \cp{M}_{q\times p}(\cp{K})$.

Agora seja $B \in \cp{M}_{q\times p}(\cp{K})$ onde
\[
  B = \begin{bmatrix}
    b_{11} & b_{12} & \dots & b_{1p}\\
    b_{21} & b_{22} & \dots & b_{2p}\\
    \vdots & & \ddots & \vdots\\
    b_{q1} & b_{q2} & \dots & b_{qp}
  \end{bmatrix}.
\]

Seja $V_1$ um espaço vetorial sobre $\cp{K}$ de dimensão $p \ge 1$ e $W_1$ um espaço vetorial sobre $\cp{K}$ de dimensão $q \ge 1$. Tome $\mathcal{B}_1 = \{v_1,\dots,v_p\}$ e $\mathcal{B}_2 = \{l_1,\dots,l_q\}$ bases ordenadas de $V_1$ e $W_1$, respectivamente. Defina $G \colon V_1 \to W_1$ por
\begin{align*}
  G(v_1) &= b_{11}l_1 + b_{21}l_2 + \cdots + b_{q1}l_q\\
  &\vdots\\
  G(v_p) &= b_{1p}l_1 + b_{2p}l_2 + \cdots + b_{qp}l_q.
\end{align*}
Então $G$ é uma transformação linear  tal que
\[
  [G(x)]_{\mathcal{B}_2} = B[x]_{\mathcal{B}_1}
\]
para todo $x \in V$.

Assim provamos o seguinte teorema:
\begin{teorema}\label{teorema_toda_transformacao_matriz}
  Sejam $V$ e $W$ $\cp{K}$-espaços vetoriais de dimensões $p \ge 1$ e $q \ge 1$, respectivamente. Sejam $\mathcal{B}_V$ e $\mathcal{B}_W$ bases ordenadas de $V$ e $W$, respectivamente. Então para cada transformação linear $T \colon V \to W$, existe uma matriz $A \in \cp{M}_{q\times p}(\cp{K})$ tal que
  \[
    [T(x)]_{\mathcal{B}_W} = A[x]_{\mathcal{B}_V}
  \]
  para todo vetor $x \in V$. Além disso, a cada matriz $A \in \cp{M}_{q\times p}(\cp{K})$ corresponde uma transformação linear $T \colon V \to W$ tal que
  \[
    [T(x)]_{\mathcal{B}_W} = A[x]_{\mathcal{B}_V}
  \]
  para todo $x \in V$.
\end{teorema}

\begin{definicao}
  A matriz $A \in \cp{M}_{q\times p}(\cp{K})$ do Teorema \ref{teorema_toda_transformacao_matriz} é chamada de \textbf{matriz da transformação linear}\index{Transformação linear!matriz da transformação} $T$ com respeito \`as bases ordenadas $\mathcal{B}_V$ e $\mathcal{B}_W$ e será denotada por
  \[
    A = [T]_{\mathcal{B}_{W},\mathcal{B}_{V}}.
  \]
  No caso em que $V = W$ e $\mathcal{B}_V = \mathcal{B}_W = \mathcal{B}$, denotaremos $[T]_{\mathcal{B}_{V},\mathcal{B}_{W}}$ simplesmente por $[T]_\mathcal{B}$.
\end{definicao}

\begin{exemplos}
  \begin{enumerate}[label={\arabic*})]
    \item Seja $T \colon \real^2 \to \real^2$ a transformação linear definida por
    \[
      T(x,y) = (x,0).
    \]
    Considere $\real^2$ com a base can\^onica $\mathcal{B} = \{e_1=(1,0);e_2=(0,1)\}$. Encontre $[T]_\mathcal{B}$.
    \begin{solucao}
      Temos
      \begin{align*}
        T(e_1) &= (1,0) = 1(1,0) + 0(0,1)\\
        T(e_2) &= (0,0) = 0(1,0) + 0(0,1).
      \end{align*}
      Logo a matriz de $T$ é
      \[
        [T]_\mathcal{B} = \begin{bmatrix}
          1 & 0\\
          0 & 0
        \end{bmatrix}.
      \]
      Além disso, dado $(x,y) \in \real^2$ temos
      \[
        (x,y) = xe_1 + ye_2
      \]
      e então
      \[
        [(x,y)]_\mathcal{B} = \begin{bmatrix}
          x\\y
        \end{bmatrix}.
      \]
      Assim podemos escrever
      \[
        [T(x,y)]_\mathcal{B} = [T]_\mathcal{B}[(x,y)]_\mathcal{B} = \begin{bmatrix}
          1 & 0\\
          0 & 0
        \end{bmatrix}\begin{bmatrix}
          x\\y
        \end{bmatrix}
      \]
      para todo $(x,y) \in \real^2$.
    \end{solucao}

    \item Seja $T \colon \real^3 \to \real^2$ tal que
    \[
      T(x,y,z) = (2x+y-z,3x-2y+4z).
    \]
    Considere as bases $\mathcal{B}_1 = \{(1,1,1);(1,1,0);(1,0,0)\}$ e $\mathcal{B}_2 = \{(1,3);(1,4)\}$. Encontre $[T]_{\mathcal{B}_{1},\mathcal{B}_{2}}$.
    \begin{solucao}
      Temos
      \begin{align*}
        T(1,1,1) &= (2,5) = 3(1,3) - 1(1,4)\\
        T(1,1,0) &= (3,1) = 11(1,3) - 8(1,4)\\
        T(1,0,0) &= (2,3) = 5(1,5) - 3(1,4)
      \end{align*}
      e daí
      \[
        [T]_{\mathcal{B}_{1},\mathcal{B}_{2}} = \begin{bmatrix}
          \phantom{-}3 & \phantom{-}11 & \phantom{-}5\\
          -1 & -8 & -3
        \end{bmatrix}.
      \]
      Agora considerando as bases $\mathcal{B}_3 = \{(1,0,0);(0,1,0);(0,0,1)\}$ e $\mathcal{B}_4 = \{(1,0);(0,1)\}$ obtemos
      \begin{align*}
        T(1,0,0) &= (2,3) = 2(1,0) + 3(0,1)\\
        T(0,1,0) &= (1,-2) = 1(1,0) - 2(0,1)\\
        T(0,0,1) &= (-1,4) = -1(1,0) + 4(0,1)
      \end{align*}
      e assim a matriz de $T$ será
      \[
        [T]_{\mathcal{B}_{3},\mathcal{B}_{4}} = \begin{bmatrix}
          \phantom{-}2 & \phantom{-}1 & -1\\
          \phantom{-}3 & -2 & \phantom{-}4
        \end{bmatrix}.
      \]
    \end{solucao}
  \end{enumerate}
\end{exemplos}

\begin{teorema}\label{matriz_da_composicao_de_transformacoes}
  Sejam $F \colon U \to V$ e $G \colon V \to W$ duas transformações lineares onde $U$, $V$ e $W$ são $\cp{K}$-espaços vetoriais de dimensões $n$, $m$ e $r$, respectivamente. Fixe bases ordenadas $\mathcal{B}_U$, $\mathcal{B}_V$ e $\mathcal{B}_W$ para $U$, $V$ e $W$, respectivamente. Então $(G \circ F) \colon U \to W$ dada por $(G\circ F)(v) = G(F(v))$ é uma transformação linear e
  \[
    [(G \circ F)]_{{\mathcal{B}_W},{\mathcal{B}_U}} = [G]_{{\mathcal{B}_W},{\mathcal{B}_V}}[F]_{{\mathcal{B}_V},{\mathcal{B}_U}}.
  \]
\end{teorema}

\begin{teorema}
  Sejam $T \colon V \to V$ é um operador linear e $\mathcal{B}$ é uma base ordenada de $V$. Então as seguintes afirmações são equivalentes:
  \begin{enumerate}[label={\roman*})]
    \item $T$ é injetor
    \item $[T]_\mathcal{B}$ é invertível
  \end{enumerate}
  Além disso, se valem essas condições equivalentes, então
  \[
    [T^{-1}]_\mathcal{B} = [T]_\mathcal{B}^{-1}.
  \]
\end{teorema}

Vamos tratar, principalmente, com a representação por matrizes de transformações lineares de um espaço em si mesmo. Lembre-se que esta matriz muda de acordo com a escolha da base. Assim, deve-se prestar atenção sempre \`a base escolhida. Assim como um espaço vetorial pode ter várias bases distintas, o que acontecerá com a matriz representante de uma transformação linear quando mudamos a base ordenada?

Vamos considerar $T \colon V \to V$ uma transformação linear sobre o $\cp{K}$-espaço vetorial de dimensão finita $V$ e sejam
\[
  \mathcal{B}_1 = \{v_1,\dots,v_n\}, \quad \mathcal{B}_2 = \{w_1,\dots,w_n\}
\]
bases ordenadas de $V$. Qual a relação entre as matrizes $[T]_{\mathcal{B}_1}$ e $[T]_{\mathcal{B}_2}$?

Para responder a esta questão, primeiro vamos determinar uma relação entre $[x]_{\mathcal{B}_1}$ e $[x]_{\mathcal{B}_2}$, para todo $x \in V$. Como $\mathcal{B}_1$ é uma base de $V$, então existem $\alpha_{ij} \in \cp{K}$, $1 \le i,\ j \le n$ tais que
\begin{align*}
  w_1 &= \alpha_{11}v_1 + \cdots + \alpha_{n1}v_n\\
  &\vdots\\
  w_n &= \alpha_{1n}v_1 + \cdots + \alpha_{nn}v_n.
\end{align*}
Agora, dado $x \in V$, existem $\beta_1$, \dots, $\beta_n \in \cp{K}$ tais que
\[
  x = \beta_1w_1 + \cdots + \beta_nw_n.
\]
Daí
\begin{align*}
  x &= \beta_1(\alpha_{11}v_1 + \cdots + \alpha_{n1}v_n) + \cdots + \beta_n(\alpha_{1n}v_1 + \cdots + \alpha_{nn}v_n)\\
  &= (\beta_1\alpha_{11} + \beta_2\alpha_{12} + \cdots + \beta_n\alpha_{1n})v_1 + \cdots + (\beta_1\alpha_{n1} + \beta_2\alpha_{n2} + \cdots + \beta_n\alpha_{nn})v_n
\end{align*}
Assim as coordenadas de $x$ em relação \`a base $\mathcal{B}_1$ são
\[
  [x]_{\mathcal{B}_1} = \begin{bmatrix}
    \beta_1\alpha_{11} + \beta_2\alpha_{12} + \cdots + \beta_n\alpha_{1n}\\
    \vdots\\
    \beta_1\alpha_{n1} + \beta_2\alpha_{n2} + \cdots + \beta_n\alpha_{nn}
  \end{bmatrix}.
\]
Seja $P$ a matriz cuja entrada $(i,j)$ é o escalar $\alpha_{ij}$, isto é,
\[
  P = \begin{bmatrix}
    \alpha_{11} & \cdots & \alpha_{1n}\\
    \vdots & & \vdots\\
    \alpha_{n1} & \cdots & \alpha_{nn}
  \end{bmatrix}.
\]
Então podemos escrever
\[
  [x]_{\mathcal{B}_1} = P \begin{bmatrix}
    \beta_1\\
    \vdots\\
    \beta_n
  \end{bmatrix} = P[x]_{\mathcal{B}_2}.
\]
Mais ainda, como $\mathcal{B}_1$ e $\mathcal{B}_2$ são bases de $V$, então $[x]_{\mathcal{B}_1} = [0_V]_{\mathcal{B}_1}$ se, e somente se, $[x]_{\mathcal{B}_2} = [0_V]_{\mathcal{B}_2}$. Logo $P$ possui inversa. Assim mostramos o seguinte teorema:

\begin{teorema}\label{teorema_mudanca_base}
  Sejam $V$ um $\cp{K}$-espaço vetorial de dimensão $n \ge 1$, $\mathcal{B}_1$ e $\mathcal{B}_2$ bases ordenadas de $V$. Então existe uma única matriz $P \in \cp{M}_n(\cp{K})$, necessariamente invertível tal que
  \begin{enumerate}[label={\roman*})]
    \item $[x]_{\mathcal{B}_1} = P[x]_{\mathcal{B}_2}$
    \item $[x]_{\mathcal{B}_2} = P^{-1}[x]_{\mathcal{B}_1}$
  \end{enumerate}
  para todo $x \in V$. As colunas de $P = \begin{bmatrix}
    P_1 & P_2 & \cdots & P_n
  \end{bmatrix}$ são dadas por
  \[
    P_j = [w_j]_{\mathcal{B}_1}
  \]
  para $j = 1$, \dots, $n$.
\end{teorema}

Além disso, também temos o seguinte teorema:

\begin{teorema}\label{teorema_matriz_mudanca_base}
  Seja $P \in \cp{M}_n(\cp{K})$ um matriz invertível. Sejam $V$ um $\cp{K}$-espaço vetorial de dimensão $n \ge 1$ e $\mathcal{B}_1$ uma base ordenada de $V$. Então existe uma única base ordenada $\mathcal{B}_2$ de $V$ tal que
  \begin{enumerate}[label={\roman*})]
    \item $[x]_{\mathcal{B}_1} = P[x]_{\mathcal{B}_2}$
    \item $[x]_{\mathcal{B}_2} = P^{-1}[x]_{\mathcal{B}_1}$
  \end{enumerate}
  para todo $x \in V$.
\end{teorema}

\begin{definicao}
  A matriz $P \in \cp{M}_n(\cp{K})$ do Teorema \ref{teorema_matriz_mudanca_base} é chamada de \textbf{matriz de mudança de base} e é denotada por $P = [I]_{{\mathcal{B}_1},{\mathcal{B}_2}}$.
\end{definicao}

Agora, sejam $T \colon V \to V$ uma transformação linear, $\mathcal{B}_1 = \{v_1,\dots,v_n\}$ e $\mathcal{B}_2 = \{w_1,\dots,w_n\}$ bases ordenadas de $V$. Sabemos que existe $P \in \cp{M}_n(\cp{K})$ invertível tal que
\begin{equation}\label{equacao_mudanca_base}
  [x]_{\mathcal{B}_1} = P[x]_{\mathcal{B}_2}
\end{equation}
para todo $x \in V$. Mas, para todo $x \in V$ temos
\begin{equation}\label{equacao_coordenadas_imagem}
  [T(x)]_{\mathcal{B}_1} = [T]_{\mathcal{B}_1}[x]_{\mathcal{B}_1}.
\end{equation}
Aplicando \eqref{equacao_mudanca_base} ao vetor $T(x)$ obtemos
\begin{equation}\label{equacao_mudanca_base_imagem}
  [T(x)]_{\mathcal{B}_1} = P[T(x)]_{\mathcal{B}_2}.
\end{equation}
Combinando \eqref{equacao_mudanca_base}, \eqref{equacao_coordenadas_imagem} e \eqref{equacao_mudanca_base_imagem}:
\begin{align*}
  [T]_{\mathcal{B}_1}[x]_{\mathcal{B}_1} &= [T(x)]_{\mathcal{B}_1} = P[T(x)]_{\mathcal{B}_2}\\
  [T]_{\mathcal{B}_1}P[x]_{\mathcal{B}_2} &= P[T(x)]_{\mathcal{B}_2}\\
  P^{-1}[T]_{\mathcal{B}_1}P[x]_{\mathcal{B}_2} &= [T(x)]_{\mathcal{B}_2} = [T]_{\mathcal{B}_2}[x]_{\mathcal{B}_2}.
\end{align*}
Daí
\[
  P^{-1}[T]_{\mathcal{B}_1}P = [T]_{\mathcal{B}_2}
\]
o que responde nossa pergunta.

Por outro lado, sabemos que existe uma única transformação linear $G \colon V \to V$ tal que
\[
  G(v_i) = w_i
\]
para $i = 1$, \dots, $n$. Mais ainda, tal transformação linear é um isomorfismo. Afirmamos que a matriz $P$ acima é exatamente a matriz de $G$ em relação \`a base $\mathcal{B}_1$. De fato, as entradas de $P$ são definidas por
\[
  w_i = \alpha_{1j}v_1 + \alpha_{2j}v_2 + \cdots + \alpha_{nj}v_n
\]
e como $G(v_i) = w_i$ podemos escrever
\[
  G(v_i) = w_i = \alpha_{1j}v_1 + \alpha_{2j}v_2 + \cdots + \alpha_{nj}v_n
\]
e então por definição $[G]_{\mathcal{B}_1} = P$. Desse modo, temos o seguinte teorema:

\begin{teorema}
  Sejam $V$ um espaço vetorial sobre $\cp{K}$ de dimensão finita, $\mathcal{B}_1 = \{v_1,\dots,v_n\}$ e $\mathcal{B}_2 = \{w_1,\dots,w_n\}$ bases ordenadas de $V$. Suponha que $T \colon V \to V$ é uma transformação linear. Se $P = \begin{bmatrix}P_1 & P_2 & \dots & P_n
  \end{bmatrix} \in \cp{M}_n(\cp{K})$ é uma matriz com colunas
  \[
    P_j = [w_j]_{\mathcal{B}_1}
  \]
  então
  \[
    [T]_{\mathcal{B}_2} = P^{-1}[T]_{\mathcal{B}_1}P.
  \]
  Alternativamente, se $G$ é o isomorfismo de $V$ definido por $G(v_i) = w_i$, $i = 1$, \dots, $n$ então
  \[
    [T]_{\mathcal{B}_2} = ([G]_{\mathcal{B}_1})^{-1}[T]_{\mathcal{B}_1}[G]_{\mathcal{B}_1}.
  \]
\end{teorema}

\begin{definicao}
  Sejam $A$, $B \in \cp{M}_n(\cp{K})$. Dizemos que $B$ é \textbf{semelhante} a $A$ sobre $\cp{K}$ se existe $P \in \cp{M}_n(\cp{K})$ invertível tal que
  \[
    B = P^{-1}AP.
  \]
\end{definicao}

\begin{exemplos}
  \begin{enumerate}[label={\arabic*})]
    \item Seja $T \colon \real^2 \to \real^2$ dada por $T(x,y) = (x,0)$. Sabemos que com relação \`a base $\mathcal{B}_1 = \{e_1=(1,0);e_2=(0,1)\}$ a matriz de $T$ é
    \[
      [T_{\mathcal{B}_1} = \begin{bmatrix}
        1 & 0\\
        0 & 0
      \end{bmatrix}.
    \]
    Agora, seja $\mathcal{B}_2$ a base de $\real^2$ formada por
    \[
      \mathcal{B}_2 = \{w_1 = (1,1);w_2=(2,1)\}.
    \]
    Então
    \begin{align*}
      w_1 &= e_1 + e_2\\
      w_2 &= 2e_1 + e_2.
    \end{align*}
    Assim $P$ é a matriz
    \[
      P = \begin{bmatrix}
        1 & 2\\
        1 & 1
      \end{bmatrix}
    \]
    cuja inversa é
    \[
      P^{-1} = \begin{bmatrix}
        -1 & \phantom{-}2\\
        \phantom{-}1 & -1
      \end{bmatrix}.
    \]
    Logo
    \begin{align*}
      [T]_{\mathcal{B}_2} &= P^{-1}[T]_{\mathcal{B}_1}P\\
      [T]_{\mathcal{B}_2} &= \begin{bmatrix}
        -1 & -2\\
        \phantom{-}1 & \phantom{-}2
      \end{bmatrix}.
    \end{align*}

    \item Seja $\mathcal{D} \colon \mathcal{P}_3(\real) \to \mathcal{P}_3(\real)$ a transformação derivada. Considere a base $\mathcal{B}_1 = \{f_1 = 1; f_2 = x; f_3 = x^2; f_4 = x^3\}$. Tome $t \in \real$ e defina
    \[
      \mathcal{B}_2 = \{g_1 = 1; g_2 = (x + t); g_3 = (x + t)^2; g_4 = (x + t)^3\}.
    \]
    Assim
    \begin{align*}
      g_1 &= 1f_1 + 0f_2 + 0f_3 + 0f_4\\
      g_2 &= tf_1 + xf_2 + 0f_3 + 0f_4\\
      g_3 &= t^2f_1 + 2tf_2 + 1f_3 + 0f_4\\
      g_4 &= t^3f_1 + 3t^2f_2 + 3tf_3 + 1f_4\\
    \end{align*}
    e então
    \[
      P = \begin{bmatrix}
        1 & t & t^2 & t^3\\
        0 & 1 & 2t & 3t^2\\
        0 & 0 & 1 & 3t\\
        0 & 0 & 0 & 1
      \end{bmatrix}
    \]
    é invertível com
    \[
      P^{-1} = \begin{bmatrix}
        1 & -t & \phantom{-}t^2 & -t^3\\
        0 & \phantom{-}1 & -2t & \phantom{-}3t^2\\
        0 & \phantom{-}0 & \phantom{-}1 & -3t\\
        0 & \phantom{-}0 & \phantom{-}0 & \phantom{-}1
      \end{bmatrix}.
    \]
    Agora
    \begin{align*}
      D(f_1) &= 0\\
      D(f_2) &= 1\\
      D(f_3) &= 2x\\
      D(f_4) &= 3x^2
    \end{align*}
    e então
    \[
      [D]_{\mathcal{B}_1} = \begin{bmatrix}
        0 & 1 & 0 & 0\\
        0 & 0 & 2 & 0\\
        0 & 0 & 0 & 3\\
        0 & 0 & 0 & 0
      \end{bmatrix}.
    \]
    Logo
    \[
      [D]_{\mathcal{B}_2} = P^{-1}[D]_{\mathcal{B}_1}P = \begin{bmatrix}
        0 & 1 & 0 & 0\\
        0 & 0 & 2 & 0\\
        0 & 0 & 0 & 3\\
        0 & 0 & 0 & 0
      \end{bmatrix}.
    \]
  \end{enumerate}
\end{exemplos}
%% section transformacoes_lineares_e_matrizes (end)
