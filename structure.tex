%!TEX program = xelatex
%!TEX root = IAL.tex


\usepackage{ccicons} %ícones creative commons

\usepackage{amstext}

\usepackage[brazil]{babel}

\usepackage{multicol}

\usepackage{gauss}

\usepackage{tikz}

\usepackage{tikz-3dplot}

\usetikzlibrary{circuits.ee.IEC}

\usepackage{circuitikz}

\usetikzlibrary{tikzmark}

\usepackage{pgfplots}

\pgfplotsset{compat=1.15}

\usepackage{mathrsfs}

\usetikzlibrary{arrows}

\usepackage{graphicx}

\graphicspath{{/home/jfreitas/GitHub/IAL/Images/}{C:\\Users\\josea\\GitHub\\IAL\\Images\\}{/home/jfreitas/GitHub/IAL/figuras_geogebra/}{C:\\Users\\josea\\GitHub\\IAL\\figuras_geogebra\\}}

\usepackage{caption}

\usepackage{subcaption}

%\captionsetup[equ]{labelformat=empty}

%----------------------------------------------------------------------------------------
%	MATH COMMANDS
%----------------------------------------------------------------------------------------


\newcommand{\n}{\mathbb{N}}
\newcommand{\z}{\mathbb{Z}}
\newcommand{\rac}{\mathbb{Q}}
\newcommand{\dom}{{\rm dom\,}}
\newcommand{\im}{{\rm Im\,}}
\newcommand{\aut}{{\rm Aut\,}}
\newcommand{\cp}[1]{\mathbb{#1}}
\newcommand{\sub}{\subseteq}
\newcommand{\real}{\mathbb{R}}
\newcommand{\complex}{\mathbb{C}}
\newcommand{\lap}[1]{\mathcal{L}\left\{#1\right\}}
\newcommand{\lapi}[1]{\mathcal{L}^{-1}\left\{#1\right\}}
\newcommand{\se}[1]{\displaystyle\sum_{n = 1}^\infty{#1}}
\newcommand{\dlim}[2]{\displaystyle\lim_{#1\rightarrow #2}}
\newcommand{\slim}{\displaystyle\lim_{n \rightarrow \infty}}
\newcommand{\seq}[1]{\{{#1_n\}}}
\newcommand{\seg}[1]{\displaystyle\sum_{n = 1}^\infty{#1_n}}
\newcommand{\sei}[2]{\displaystyle\sum_{#1}^\infty{#2}}
\newcommand{\sepc}[3]{\displaystyle\sum_{#1}^\infty{#2(x - #3)^n}}
\newcommand{\imp}[3]{\displaystyle\int_{#1}^{+\infty}{#3}{d #2}}
\newcommand{\dint}[4]{\displaystyle\int_{#1}^{#2}{#4}{d#3}}
\newcommand{\inti}[2]{\displaystyle\int{#1}{d#2}}
\newcommand{\norma}[1]{\left\lVert#1\right\rVert}
\newcommand{\flim}[1]{\displaystyle\lim_{#1\rightarrow \infty}}
\newcommand{\tr}[1]{{\rm tr\,(#1)}}
\renewcommand{\sin}{{\rm sen\,}}
\renewcommand{\tan}{{\rm tg\,}}
\renewcommand{\csc}{{\rm cossec\,}}
\renewcommand{\cot}{{\rm cotg\,}}
\renewcommand{\sinh}{{\rm senh\,}}
\newcommand\T{\rule{0pt}{2.6ex}}
\newcommand{\p}[1]{\mbox{\textrm{posto(#1)}}}

%-------------amatrix
% Augmented matrix.  Usage (note the argument does not count the aug col):
% \begin{amatrix}{2}
%   1  2  3 \\  4  5  6
% \end{amatrix}
\newenvironment{amatrix}[1]{%
  \left[\begin{array}{@{}*{#1}{r}|r@{}}
}{%
  \end{array}\right]
}


\usepackage{array}

\makeatletter
\newcounter{elimination@steps}
\newcolumntype{R}[1]{>{\raggedleft\arraybackslash$}p{#1}<{$}}
\def\elimination@num@rights{}
\def\elimination@num@variables{}
\def\elimination@col@width{}
\newenvironment{elimination}[4][0]
{
    \setcounter{elimination@steps}{0}
    \def\elimination@num@rights{#1}
    \def\elimination@num@variables{#2}
    \def\elimination@col@width{#3}
    \renewcommand{\arraystretch}{#4}
    \start@align\@ne\st@rredtrue\m@ne
}
{
    \endalign
    \ignorespacesafterend
}
\newcommand{\eliminationstep}[2]
{
    \ifnum\value{elimination@steps}>0\sim\quad\fi
    \left[
        \ifnum\elimination@num@rights>0
            \begin{array}
            {@{}*{\elimination@num@variables}{R{\elimination@col@width}}
            |@{}*{\elimination@num@rights}{R{\elimination@col@width}}}
        \else
            \begin{array}
            {@{}*{\elimination@num@variables}{R{\elimination@col@width}}}
        \fi
            #1
        \end{array}
    \right]
    &
    \begin{array}{l}
        #2
    \end{array}
    &%                                    moved second & here
    \addtocounter{elimination@steps}{1}
}
\makeatother
% Defines the theorem text style for each type of theorem to one of the three styles above
\newcounter{dummy}
\numberwithin{dummy}{section}
\theoremstyle{ocrenumbox}
\newtheorem{teoremaT}[dummy]{Teorema}
\newtheorem{problema}{Problema}[chapter]
\newtheorem{exercicio}{Exercício}[chapter]
\theoremstyle{blacknumex}
\newtheorem{exemplo}{Exemplo}[chapter]
\newtheorem{exemplos}{Exemplos}[chapter]
\newtheorem{observacao}{Observaçao}[chapter]
\newtheorem{observacoes}{Observações}[chapter]
\newtheorem{nota}{Nota}[chapter]
\theoremstyle{blacknumbox}
\newtheorem{definicaoT}{Definição}[section]
\newtheorem{definicoesT}{Definições}[section]
\newtheorem{corolarioT}[dummy]{Corolário}
\theoremstyle{ocrenum}
\newtheorem{proposicao}[dummy]{Proposição}
\newtheorem{lema}[dummy]{Lema}
\newenvironment{prova}[1][Prova]{\noindent\textbf{#1:} }{\qedsymbol}%{\ \rule{0.5em}{0.5em}}


% Creates an environment for each type of theorem and assigns it a theorem text style from the "Theorem Styles" section above and a colored box from above
\newenvironment{teorema}{\begin{tBox}\begin{teoremaT}}{\end{teoremaT}\end{tBox}}
\newenvironment{definicao}{\begin{dBox}\begin{definicaoT}}{\end{definicaoT}\end{dBox}}
\newenvironment{definicoes}{\begin{dBox}\begin{definicoesT}}{\end{definicoesT}\end{dBox}}
\newenvironment{corolario}{\begin{cBox}\begin{corolarioT}}{\end{corolarioT}\end{cBox}}
\newtheorem*{solucao}{Solu{\c c}{\~a}o:}
\newtheorem{notacao}{Nota\c{c}\~ao}[section]
\newtheorem{propriedades}{Propriedades}[section]