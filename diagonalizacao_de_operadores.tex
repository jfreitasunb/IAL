%!TEX program = xelatex
%!TEX root = IAL.tex

\chapter{Diagonalização de Operadores Lineares}

\section{Operadores Diagonaliz\'aveis} % (fold)
\label{sec:operadores_diagonalizaveis}

Considere o operador linear $T \colon \real^3 \to \real^3$ dado por
\begin{align}\label{exemplodiagonalizacao}
    T(x, y, z) = (x + 2y - z, -2x -3y + z, 2x + 2y - 2z).
\end{align}

Quando consideramos a base ordenada canônica de $\real^3$, $\mathcal{B} = \{e_1 = (1, 0, 0); e_2 = (0, 1, 0); e_3 = (0, 0, 1)\}$, a matriz desse operador com respeito à essa base é
\[
    [T]_\mathcal{B} = \begin{bmatrix}
        \phantom{-}1 & \phantom{-}2 & -1\\
        -2 & -3 & \phantom{-}1\\
        \phantom{-}2 & \phantom{-}2 & \phantom{-}2
    \end{bmatrix}.
\]

Agora, se tomamos a base ordenada $\mathcal{B}_1 = \{w_1 = (1, 0, 2); w_2 = (0, 1, 2); w_3 = (1, -1, 1)\}$ a matriz de $T$ com respeito à essa base é
\[
    [T]_{\mathcal{B}_1} = \begin{bmatrix}
        -1 & \phantom{-}0 & \phantom{-}0\\
        \phantom{-}0& -1 & \phantom{-}0\\
        \phantom{-}0& \phantom{-}0 & -2
    \end{bmatrix}.
\]

Analisando a matriz $[T]_{\mathcal{B}_1}$ é fácil ver que $T$ é um operador invertível e também é fácil achar $T^{-1}$. Nesse caso, a matriz de $T^{-1}$ com respeito à base $\mathcal{B}_1$ é:
\[
[T^{-1}]_{\mathcal{B}_1} = \begin{bmatrix}
    -1 & \phantom{-}0 & \phantom{-}0\\
    \phantom{-}0& -1 & \phantom{-}0\\
    \phantom{-}0& \phantom{-}0 & -1/2
\end{bmatrix}.
\]

Note que para a base $\mathcal{B}_1$ temos
\begin{align*}
    T(w_1) &= -1w_1\\
    T(w_2) &= -1w_2\\
    T(w_3) &= -2w_3\\
\end{align*}

No caso geral, seja $T : V \to V$ um operador linear e suponha que exista uma base $\mathcal{B} = \{v_1,\dots,v_n\}$ de $V$ tal que
\begin{align}\label{formadiagonal}
    [T]_\mathcal{B} = \begin{bmatrix}
        \lambda_1 & 0 & 0 & \dots & 0\\
        0 & \lambda_2 & 0 & \dots & 0\\
        \vdots & & \ddots & & \vdots\\
        0 & 0 & 0 & \dots & \lambda_n
    \end{bmatrix}
\end{align}
com $\lambda_i \in \cp{K}$ para $i = 1$, \dots, $n$. Assim
\[
    [T(v_i)]_\mathcal{B} = [T]_\mathcal{B}[v_i]_\mathcal{B} = [T]_\mathcal{B}\begin{bmatrix}
        0\\
        0\\
        \vdots\\
        0\\
        1\\
        0\\
        \vdots\\
        0
    \end{bmatrix}_\mathcal{B} = \begin{bmatrix}
        0\\
        0\\
        \vdots\\
        0\\
        \lambda_i\\
        0\\
        \vdots\\
        0
    \end{bmatrix}_\mathcal{B}
\]
para $i = 1$, \dots, $n$. Isto \'e,
\[
    T(v_1) = \lambda_1 v_1,\ T(v_2) = \lambda_2 v_2,\dots, T(v_n) = \lambda_n v_n.
\]

\begin{definicao}
    Seja $T : V \to V$ um operador linear.
    \begin{enumerate}[label={\roman*})]
        \item Um \textbf{autovalor} de $T$ \'e um elemento $\lambda \in \cp{K}$ tal que existe um vetor n\~ao nulo $u \in V$ com $T(u) = \lambda u$.\index{Autovalor}
        \item Se $\lambda$ \'e um autovalor de $T$, ent\~ao todo vetor n\~ao nulo $u \in V$ tal que
        \[
            T(u) = \lambda u
        \]
        \'e chamado de \textbf{autovetor} de $T$ \textbf{associado} ao autovalor $\lambda$. Denotaremos por $Aut_T(\lambda)$ o subespa\c{c}o gerado por todos os autovetores associados a $\lambda$. Assim\index{Autovetor}
        \[
            \aut_T(\lambda) = \{u \in V \mid T(u) = \lambda u\}.
        \]
        \item Suponha que $\dim V = n < \infty$. Dizemos que $T$ \'e \textbf{diagonaliz\'avel} se existir uma base $\mathcal{B}$ de $V$ tal que $[T]_\mathcal{B}$ \'e diagonal, isto \'e, tem a forma \eqref{formadiagonal}. Tal fato equivale a dizer que existe uma base formada por autovetores.\index{Transforma\c{c}\~ao Linear!Diagonaliz\'avel}
    \end{enumerate}
\end{definicao}

\begin{exemplo}
    No caso da transformação linear $T$ dada em \eqref{exemplodiagonalizacao}, $-1$ é um autovalor de $T$ e os autovetores associados ao autovalor $-1$ são $w_1 = (1, 0, 1)$ e $w_2 = (0, 1, 2)$. Nesse caso
    \[
        Aut_T(-1) = [w_1, w_2].
    \]
    Observer que se $u = \alpha_1w_1 + \alpha_2w_2 \in Aut_T(-1)$, então
    \begin{align*}
        T(u) &= T(\alpha_1w_1 + \alpha_2w_2) \\ &= \alpha_1T(w_1) + \alpha_2T(w_2) \\ &= -\alpha_1w_1 - \alpha_2w_2 \\ &= -(\alpha_1w_1 + \alpha_2w_2) \\ &= -u
    \end{align*}

    Além disso, $-2$ é um autovalor de $T$ e o autovetor associado a esse autovalor é $w_3 = (1, -1, 1)$. Assim
    \[
        Aut_T(-2) = [w_3].
    \]
    Aqui também temos que se $u = \alpha w_3 \in Aut_T(-2)$, então
    \begin{align*}
        T(u) &= T(\alpha w_3) \\ &= \alpha T(w_3) \\ &= -2\alpha w_3 \\ &= -2u
    \end{align*}
\end{exemplo}

Temos aqui duas questões importantes:
\begin{enumerate}
    \item Dado um operador $T \colon V \to V$, existem autovetores para $T$?

    \item Se sim, existe um número suficiente de autovetores para forma uma base de $V$?
\end{enumerate}

Os seguintes exemplos respondem a essas perguntas:

\begin{exemplos}
    \begin{enumerate}
        \item Se $T \colon \real^2 \to \real^2$ é o operador linear dado pela matriz
        \[
            [T]_\mathcal{B} = \begin{bmatrix}0 & -1\\1 & \phantom{-}0\end{bmatrix}
        \]
        onde $\mathcal{B}$ é a base canônica, então não existe $u \in \real$ tal que $T(u) = \lambda_u$, com $\lambda_\in \real$.

        \item Se $T \colon \real^3 \to \real^3$ é o operador linear dado pela matriz
        \[
        [T]_\mathcal{B} = \begin{bmatrix}3 & -3 & -4\\0 & \phantom{-}3 & \phantom{-}5\\0 & \phantom{-}0 & \phantom{-}1\end{bmatrix}
        \]
        onde $\mathcal{B}$ é a base canônica. Então os únicos vetores de $u \in \real^3$ tais que $T(u) = \lambda_u$, para algum $\lambda_\in \real$ são $u = (1, 0, 0)$ e $v = (-7/4, -5/2, 1)$.

        Nessa caso o conjunto $\{u, v\}$ é linearmente independente mas não é uma base para $\real^3$.
    \end{enumerate}
\end{exemplos}

Aqui surgem duas questões:
\begin{enumerate}
    \item Como determinar se uma transforma\c{c}ão linear $T \colon V \to V$ possui autovetores?

    \item A existência de autovetores de $T$ depende da base escolhida para $V$?
\end{enumerate}

As respostas para essas perguntas são dadas pelos seguintes teoremas:

\begin{teorema}
    As seguintes afirma\c{c}ões são equivalentes:
    \begin{enumerate}
        \item o escalar $\lambda \in \cp{K}$ é um autovalor do operador linear $T \colon V \to V$;

        \item o operador $T - \lambda Id$ não é invertível;

        \item dada uma base ordenada $\mathcal{B}$ qualquer de $V$ temos
            \[
                \det([T]_\mathcal{B} - \lambda[Id]_\mathcal{B}) = 0.
            \]
    \end{enumerate}
\end{teorema}

\begin{teorema}
    Se $A \in M_n(\cp{K})$ for uma matriz $n \times n$, então $\lambda \in \cp{K}$ é um autovalor de $A$ se, e somente se, $\lambda$ satisfaz a equa\c{c}ão
    \[
        \det(A - \lambda I_n) = 0.
    \]
    Essa equa\c{c}ão é chamada de \textbf{equa\c{c}ão característica} de $A$.
\end{teorema}

Observe que nos dois teoremas anteriores aparecem o determinante de uma matriz. Vamos então relembrar algumas propriedades dos determinantes.

\section{Determinantes}

Seja $A \in M_n(\cp{K})$ onde $n \ge 1$. Vamos definir o \textbf{determinante da matriz $A$}, denotado por $\det(A)$, de modo indutivo.

Se $n = 1$, ent\~ao a matriz $A \in \cp{M}_1(\cp{K})$ \'e da forma
\[
	A = (a_{11})
\]
e neste caso definimos
\[
	\det(A) = a_{11} \in \cp{K}.
\]
Suponha que $n > 1$ e que $\det(B)$ esteja definido para todas as matrizes $B \in \cp{M}_p(\cp{K})$ com $p < n$ e seja $A \in \cp{M}_n(\cp{K})$. Para cada $(i,j)$, defina a matriz $A_{ij}$ formada a partir de $A$ retirando-se a sua $i$-\'esima linha e a sua $j$-\'esima coluna. \'E claro que $A \in \cp{M}_{n - 1}(\cp{K})$ e portanto $\det(A_{ij})$ est\'a definido. Defina ent\~ao
\begin{align*}
	\det(A) &= \sum_{j = 1}^n(-1)^{i + j}a_{ij}\det(A_{ij}) \\ &= (-1)^{i + 1}a_{i1}\det(A_{i1}) + (-1)^{i + 2}a_{i2}\det(A_{i2}) \\ & + (-1)^{i + 3}a_{i3}\det(A_{i3}) + \cdots + (-1)^{i + n}a_{in}\det(A_{in}) \in \cp{K}
\end{align*}

\begin{exemplo}
	\begin{enumerate}[label={\arabic*})]
		\item Seja
		\[
			A =
			\begin{bmatrix}
				a & b\\
				c & d
			\end{bmatrix} \in \cp{M}_2(\cp{K}).
		\]
		Fixada a linha 1, temos
		\begin{align*}
			\det(A) &= \sum_{j = 1}^2(-1)^{1 + j}a_{1j}\det(A_{1j}) = (-1)^{1 + 1}a_{11}\det(A_{11}) + (-1)^{1 + 2}a_{12}\det(A_{12})\\
			\det(A) &= ad - bc.
	    \end{align*}
		Obter{\'\i}amos o mesmo resultado se consider\'assemos a linha 2.

		\item Seja
		\[
			A =
			\begin{bmatrix}
				a_{11} & a_{12} & a_{13}\\
				a_{21} & a_{22} & a_{23}\\
				a_{31} & a_{32} & a_{33}
			\end{bmatrix} \in \cp{M}_2(\cp{K}).
		\]
		Fixada a linha 2, temos
		\begin{align*}
			\det(A) &= \sum_{j = 1}^3(-1)^{2 + j}a_{2j}\det(A_{2j}) \\ &= (-1)^{2 + 1}a_{21}\det(A_{21}) + (-1)^{2 + 2}a_{22}\det(A_{22}) + (-1)^{2 + 3}a_{32}\det(A_{32})\\ &= -a_{21}\det\left(\begin{bmatrix}a_{12} & a_{13}\\a_{32} & a_{33}\end{bmatrix}\right) + a_{22}\det\left(\begin{bmatrix}a_{11} & a_{12}\\a_{31} & a_{33}\end{bmatrix}\right) - a_{23}\det\left(\begin{bmatrix}a_{11} & a_{12}\\a_{31} & a_{32}\end{bmatrix}\right).
		\end{align*}
		Da{\'\i}
		\[
			\det(A) = a_{11}a_{22}a_{33} + a_{12}a_{23}a_{31} + a_{13}a_{21}a_{32} - a_{13}a_{22}a_{31} - a_{12}a_{21}a_{33} - a_{11}a_{23}a_{32}.
		\]

	    \item Seja
	        \[
	            A = \begin{bmatrix}
	                \phantom{-}6 & -3 & \phantom{-}1 & 0\\
	                \phantom{-}0 & \phantom{-}3 & \phantom{-}1 & 0\\
	                -6 & \phantom{-}6 & \phantom{-}0 & 0\\
	                -3 & \phantom{-}3 & -2 & 3
	            \end{bmatrix} \in M_4(\real).
	        \]
	        Para calcular $\det(A)$ podemos usar a terceira linha dessa matriz, pois nessa linha $a_{33} = a_{34} = 0$. Assim a expressão
	        \[
	            \det(A) = \sum_{j = 1}^4(-1)^{3 + i}a_{3j}\det(A_{3j})
	        \]
	        torna-se simplesmente
	        \begin{align*}
	            \det(A) &= (-1)^{3 + 1}a_{31}\det(A_{31}) + (-1)^{3 + 2}a_{32}\det(A_{32}) \\ &=
	            -6\cdot\det\left(\begin{bmatrix}-3 & \phantom{-}1 & 0\\\phantom{-}0 & \phantom{-}1 & 0\\\phantom{-}3 & -2 & 0\end{bmatrix}\right) -6\cdot\det\left(\begin{bmatrix}\phantom{-}6 & \phantom{-}1 & 0\\\phantom{-}0 & \phantom{-}1 & 0\\-3 & -2 & 3\end{bmatrix}\right)\\
	            \det(A) &= -6\cdot 18 = -108
	        \end{align*}
	\end{enumerate}
\end{exemplo}

\begin{proposicao}
	Sejam $A$, $B \in \cp{M}_n(\cp{K})$ e $\lambda \in \cp{K}$. Temos:
	\begin{enumerate}[label={\roman*})]
		\item $\det(AB) = \det(A) \det(B)$,
		\item $\det(\lambda A) = \lambda^n \det(A)$,
		\item Se $A$ é invertível, então $\det(A^{-1}) = [(\det(A)]^{-1}$.
	\end{enumerate}
\end{proposicao}


\begin{proposicao}
	Seja $A$ uma matriz $n \times n$ com entradas num corpo $\cp{K}$.
	\begin{enumerate}[label={\roman*})]
		\item Se $B$ \'e a matriz resultante da permuta\c{c}\~ao de duas linhas de $A$, ent\~ao $\det (B) = -\det (A)$.
		\item Se $B$ \'e a matriz resultante da multiplica\c{c}\~ao de uma linha de $A$ por um escalar n\~ao nulo $\alpha \in \cp{K}$, ent\~ao $\det(B) = \alpha\det(A)$.
		\item Se $B$ \'e a matriz resultante da soma da linha $i$ de $A$ com um m\'ultiplo n\~ao nulo $\alpha \in \cp{K}$ da linha $j$ de $A$, ent\~ao $\det(B) = \det(A)$.
	\end{enumerate}
\end{proposicao}

\begin{observacao}
	\'E poss{\'\i}vel mostrar que o determinante tamb\'em pode ser definido a partir das colunas de uma matriz $A \in \cp{M}_n(\cp{K})$.
\end{observacao}

\begin{teorema}
	Uma matriz $A \in \cp{M}_n(\cp{K})$ \'e invert{\'\i}vel se, e somente se, $\det(A) \ne 0$.
\end{teorema}

\section{Diagonaliza\c{c}ão}
Dado $V$ um espaćo vetorial sobre $\cp{K}$, com $\dim_\cp{K} V = n < \infty$ e um operador linear $T \colon V \to V$, queremos tentar encontrar, se existir, uma base ordenada $\mathcal{D}$ de $V$ formada por autovetores. Para isso devemos proceder da seguinte maneira:
\begin{enumerate}
    \item Se não foi dada, encontre a matriz de $T$ com rela\c{c}ão a alguma base ordenada $\mathcal{B}$ de $V$. Pode-se usar a base canônica $\mathcal{B}$ de $V$ para facilitar o cálculo de $[T]_\mathcal{B}$.

    \item Uma vez encontrada $[T]_\mathcal{B}$ calcule
        \[
            [T]_\mathcal{B} - \lambda I_n = [T]_\mathcal{B} - \begin{bmatrix}\lambda & 0 & 0 & \cdots & 0\\0 & \lambda & 0 & \cdots & 0\\0 & 0 & \lambda & \cdots & 0\\\vdots & \vdots & \vdots & \cdots & \vdots\\0 & 0 & 0 & \cdots & \lambda\end{bmatrix}.
        \]

    \item\label{passo4diag} Determine todos os valores, se existem, de $\lambda \in \cp{K}$ tais que
        \[
            \det([T]_\mathcal{B} - \lambda I_n) = 0.
        \]

    \item Se não existir $\lambda \in \cp{K}$ no passo anterior, então $T$ não é diagonalizável.

    \item Para cada $\lambda_1$, $\lambda_2$, \dots, $\lambda_r$ encontrando em \eqref{passo4diag} resolva o sistema linear homogêneo
        \[
            ([T]_\mathcal{B} - \lambda_jI_n)X = 0.
        \]
        O subespa\c{c}o $Aut_T(\lambda_j)$ será dada pelas solu\c{c}ões desse sistema.

    \item Encontre uma base para cada $Aut_T(\lambda_j)$, com $j = 1$, 2, \dots, $r$.

    \item Se o conjunto $\mathcal{D}$ resultante a união de todas as bases encontradas no passo anterior possuir $\dim_\cp{K}V$ elementos, então $\mathcal{D}$ será uma base de $V$ tal que a matriz de $T$ com rela\c{c}ão à essa base será diagonal. Se $\mathcal{D}$ possuir menos elementos que $\dim_\cp{K}V$ então $T$ não é diagonalizável.
\end{enumerate}

\begin{exemplo}
    \begin{enumerate}[label={\arabic*})]
        \item Seja $T : \real^2 \to \real^2$ o operador linear dado por $T(x,y) = (-y,x)$. O operador $T$ é diagonalizável?
        \begin{solucao}
            Como não foi dada a matriz de $T$ vamos primeiro encontrá-la. Para isso podemos considerar a base can\^onica de $\real^2$ dada por $\mathcal{B} = \{e_1 = (1,0); e_2 = (0,1)\}$. Temos
            \begin{align}
                T(1,0) &= (0,1) = 0(1,0) + 1(1,0)\\
                T(0,1) &= (-1,0) = -1(1,0) + 0(1,0).
            \end{align}
            Da{\'\i}
            \[
                [T]_\mathcal{B} = \begin{bmatrix}0 & -1\\ 1 & \phantom{-}0\end{bmatrix}.
            \]
            Agora calculamos
            \begin{align*}
                [T]_\mathcal{B} - \lambda I_2 &= \begin{bmatrix}0 & -1\\ 1 & \phantom{-}0\end{bmatrix} - \lambda\begin{bmatrix}1 & 0\\0 & 1\end{bmatrix} = \begin{bmatrix}0 & -1\\ 1 & \phantom{-}0\end{bmatrix} - \begin{bmatrix}\lambda & 0\\0 & \lambda\end{bmatrix}\\ &= \begin{bmatrix}-\lambda & -1\\ 1 & -\lambda\end{bmatrix}.
            \end{align*}
            Com isso
            \begin{align*}
                \det([T]_\mathcal{B} - \lambda I_2]) = \det\left(\begin{bmatrix} -\lambda & -1\\\phantom{-}1 & -\lambda\end{bmatrix}\right) = \lambda^2 + 1.
            \end{align*}
            Nesse caso, não existe $\lambda \in \real$ tal que $\lambda^2 + 1 = 0$. Logo $T$ não é diagonalizável.
        \end{solucao}

        \item Seja $T : \complex^2 \to \complex^2$ o operador linear dado por $T(x,y) = (-y,x)$. Encontre os autovalores de $T$ e os autoespa\c{c}os associados, se existirem, considerando $\complex^2$ com um $\complex$-espa\c{c}o vetorial.
        \begin{solucao}
            Considere a base can\^onica $\mathcal{B} = \{e_1 = (1, 0); e_2 = (0, 1)\}$ de $\complex^2$. Temos
            \begin{align}
                T(1,0) &= (0,1) = 0(1,0) + 1(1,0)\\
                T(0,1) &= (-1,0) = -1(1,0) + 0(1,0).
            \end{align}
            Da{\'\i}
            \[
                [T]_\mathcal{B} = \begin{bmatrix}0 & -1\\ 1 & \phantom{-}0\end{bmatrix}.
            \]
            Agora calculamos
            \begin{align*}
                [T]_\mathcal{B} - \lambda I_2 &= \begin{bmatrix}0 & -1\\ 1 & \phantom{-}0\end{bmatrix} - \lambda\begin{bmatrix}1 & 0\\0 & 1\end{bmatrix} = \begin{bmatrix}0 & -1\\ 1 & \phantom{-}0\end{bmatrix} - \begin{bmatrix}\lambda & 0\\0 & \lambda\end{bmatrix}\\ &= \begin{bmatrix}-\lambda & -1\\ 1 & -\lambda\end{bmatrix}.
            \end{align*}
            Com isso
            \begin{align*}
                \det([T]_\mathcal{B} - \lambda I_2]) = \det\left(\begin{bmatrix} -\lambda & -1\\\phantom{-}1 & -\lambda\end{bmatrix}\right) = \lambda^2 + 1.
            \end{align*}

            Agora, como $\lambda \in \complex$, então tomando $\lambda = -i$ ou $\lambda = i$ teremos $\lambda^2 + 1 = 0$. Vamos então determinar $\aut_T(-i)$ e $\aut_T(i)$:
            \begin{itemize}
                \item Para $\lambda = -i$ temos:
                \[
                    [T + iI_2]_\mathcal{B} = \begin{bmatrix} i & -1\\1 & \phantom{-}i\end{bmatrix}
                \]
                e assim $(x,y) \in \aut_T(-i)$ se, e s\'o se,
                \[
                    \begin{bmatrix} i & -1\\1 & \phantom{-}i\end{bmatrix} \begin{bmatrix} x\\y\end{bmatrix}     = \begin{bmatrix} 0\\0\end{bmatrix},
                \]
                Resolvendo esse sistema obtemos $x = -iy$. Logo
                \[
                    \aut_T(-i) = \{(-iy,y) \in \complex^2 \mid y \in \complex\} = [(-i,1)].
                \]
                Assim, $\mathcal{B}_2 = \{(-i,1)\}$ \'e uma base de $\aut_T(-i)$ e então $\dim_\complex\aut_T(-i) = 1$.

                \item para $\lambda = i$ temos:
                \[
                    [T - iI_2]_\mathcal{B} = \begin{bmatrix} -i & -1\\\phantom{-}1 & -i\end{bmatrix}
                \]
                e assim $(x,y) \in \aut_T(i)$ se, e s\'o se,
                \[
                    \begin{bmatrix} -i & -1\\\phantom{-}1 & -i\end{bmatrix} \begin{bmatrix} x\\y\end{bmatrix}     = \begin{bmatrix} 0\\0\end{bmatrix}.
                \]
                Resolvendo esse sistema, obtemos $x = iy$. Logo
                \[
                    \aut_T(i) = \{(iy,y) \in \complex^2 \mid y \in \complex\} = [(i,1)].
                \]
                Assim, $\mathcal{B}_1 = \{(i,1)\}$ \'e uma base de $\aut_T(i)$ e então $\dim_\complex\aut_T(i) = 1$.
            \end{itemize}
            Agora o conjunto $\mathcal{D} = \mathcal{B}_1 \cup \mathcal{B}_2 = \{(-i,1); (i,1)\}$ \'e uma base de $\complex^2$ e nesta base temos
            \[
                [T]_\mathcal{B} = \begin{bmatrix} -i & 0\\\phantom{-}0 & i\end{bmatrix}.
            \]
        \end{solucao}
        \item Seja $T : \real^3 \to \real^3$ o operador linear tal que
        \[
            [T]_\mathcal{B} = \begin{bmatrix}
                                3 & -3 & -4\\
                                0 & \phantom{-}3 & \phantom{-}5\\
                                0 & \phantom{-}0 & \phantom{-}1
                            \end{bmatrix}
        \]
        onde $\mathcal{B}$ \'e a base ordenada canônica de $\real^3$. Determine, casa exista, uma base de $\real^3$ tal que o operador $T$ seja diagonaliz\'avel.
        \begin{solucao}
            Aqui como já temos a matriz da transforma\c{c}ão vamos passar para a etapa 2:
            \begin{align*}
                [T]_\mathcal{B} - \lambda I_3 = \begin{bmatrix}3 & -3 & -4\\0 & \phantom{-}3 & \phantom{-}5\\0 & \phantom{-}0 & \phantom{-}1\end{bmatrix} - \begin{bmatrix} \lambda & 0 & 0\\0 & \lambda & 0\\0 & 0 & \lambda\end{bmatrix} = \begin{bmatrix}3 - \lambda & -3 & -4\\0 & 3 - \lambda & 5\\0 & 0 &1 - \lambda\end{bmatrix}
            \end{align*}
            Usando a terceira linha dessa matriz para calcular o determinante obtemos:
            \begin{align*}
                \det([T]_\mathcal{B} - \lambda I_3) &= (-1)^{3 + 3}a_{33}\det(A_{33}) = (1 - \lambda)\det\left(\begin{bmatrix}3 - \lambda & -3\\0 & 3 - \lambda\end{bmatrix}\right) \\ &= (1 - \lambda)(3 - \lambda)^2
            \end{align*}
            Assim $\det([T]_\mathcal{B} - \lambda I_3) = 0$ quando $(1 - \lambda)(3 - \lambda)^2 = 0$, ou seja, quando $\lambda = 1$ ou $\lambda = 3$.

            Vamos determinar $\aut_T(3)$ e $\aut_T(1)$:
            \begin{itemize}
                \item para $\lambda = 3$ temos que $(x,y,z) \in \aut_T(3)$ se, e s\'o se,
                \[
                    \begin{bmatrix}
                        0 & -3 & -4\\
                        0 & \phantom{-}0 & \phantom{-}5\\
                        0 & \phantom{-}0 & -2
                    \end{bmatrix}\begin{bmatrix}
                        x\\y\\z
                    \end{bmatrix} = \begin{bmatrix}
                        0\\0\\0
                    \end{bmatrix}.
                \]
                Cuja solu\c{c}ão é $z = y = 0$. Assim
                \[
                    \aut_T(3) = \{(x,0,0) \mid x \in \real\} = [(1,0,0)]
                \]
                e ent\~ao $\mathcal{B}_1 = \{(1,0,0)\}$ \'e uma base de $\aut_T(3)$ e $\dim_\real\aut_T(3) = 1$.
                \item para $\lambda = 1$ temos que $(x,y,z) \in \aut_T(1)$ se, e s\'o se,
                \[
                    \begin{bmatrix}
                        2 & -3 & -4\\
                        0 & \phantom{-}2 & \phantom{-}5\\
                        0 & \phantom{-}0 & \phantom{-}0
                    \end{bmatrix}\begin{bmatrix}
                        x\\y\\z
                    \end{bmatrix} = \begin{bmatrix}
                        0\\0\\0
                    \end{bmatrix}.
                \]
                Cuja solu\c{c}ão é $y = -5z/2$ e $x = -7z/4$. Assim
                \[
                    \aut_T(1) = \{(-7z/4,-5z/2,z) \mid z \in \real\} = [(-7/4,-5/2,1)]
                \]
                e ent\~ao $\mathcal{B}_2 = \{(-7/4,-5/2,1)\}$ \'e uma base de $\aut_T(1)$ e $\dim_\real\aut_T(1) = 1$.
            \end{itemize}
            Note que o conjunto $\mathcal{D} = \mathcal{B}_1 \cup \mathcal{B}_2$ \'e L.I. mas n\~ao \'e uma base de $\real^3$. Neste caso o operador $T$ n\~ao \'e diagonaliz\'avel.
        \end{solucao}
        \item Seja $T : \real^3 \to \real^3$ o operador tal que
        \[
            [T]_\mathcal{B} = \begin{bmatrix}
                                \phantom{-}1 & \phantom{-}2 & -1\\
                                -2 & -3 & \phantom{-}1\\
                                \phantom{-}2 & \phantom{-}2 & -2
                            \end{bmatrix}
        \]
        onde $\mathcal{B}$ \'e uma base qualquer de $\real^3$. Determinar se $T$ \'e diagonaliz\'avel.
        \begin{solucao}
            Aqui como já temos a matriz da transforma\c{c}ão vamos passar para a etapa 2:
            \begin{align*}
                [T]_\mathcal{B} - \lambda I_3 = \begin{bmatrix}1 & 2 & -1\\-2 & -3 & 1\\2 & 2 & -2\end{bmatrix} - \begin{bmatrix} \lambda & 0 & 0\\0 & \lambda & 0\\0 & 0 & \lambda\end{bmatrix} = \begin{bmatrix}1 - \lambda & 2 & -1\\-2 & -3 - \lambda & 1\\2 & 2 &-2 - \lambda\end{bmatrix}
            \end{align*}
            Usando a primeira linha dessa matriz para calcular o determinante obtemos:
            \begin{align*}
                \det([T]_\mathcal{B} - \lambda I_3) &= (-1)^{1 + 1}a_{11}\det(A_{11}) + (-1)^{1 + 2}a_{12}\det(A_{12}) + (-1)^{1 + 3}a_{13}\det(A_{13}) \\ &= (1 - \lambda)\det\left(\begin{bmatrix}-3 - \lambda & 1\\2 & -2 - \lambda\end{bmatrix}\right) - 2\det\left(\begin{bmatrix}-2 & 1\\2 & -2 - \lambda\end{bmatrix}\right) \\ &-\det\left(\begin{bmatrix}-2 & -3 - \lambda\\2 & 2\end{bmatrix}\right)\\ &= (1 - \lambda)[(-3 - \lambda)(-2 - \lambda) - 2] - 2[(-2)(-2 - \lambda) - 2]  \\ &- [(-2)2 - 2(-3 - \lambda)] \\ &= (1 - \lambda)(6 + 3\lambda + 2\lambda + \lambda^2 - 2) - 2(4 + 2\lambda - 2) - (-4 + 6 + 2\lambda) \\ &= (1 - \lambda)(4 + 5\lambda + \lambda^2) - 6 - 6\lambda \\ &= 4 + \lambda + \lambda^2 - 4\lambda - 5\lambda^2 - \lambda^3 - 6 - 6\lambda \\ &= -\lambda^3 - 4\lambda^2 - 5\lambda - 2 \\ &= -(\lambda^3 + 4\lambda^2 + 5\lambda + 2) \\ &= -(\lambda + 1)^2(\lambda + 2) 
            \end{align*}
            Assim $\det([T]_\mathcal{B} - \lambda I_3) = 0$ quando $-(\lambda + 1)^2(\lambda + 2) = 0$, ou seja, quando $\lambda = -1$ ou $\lambda = -2$.
            Vamos determinar $\aut_T(-1)$ e $\aut_T(-2)$:
            \begin{itemize}
                \item para $\lambda = -1$ temos que $(x,y,z) \in \aut_T(-1)$ se, e s\'o se,
                \[
                    \begin{bmatrix}
                        2 & 2 & -1\\
                        -2 & -2 & 1\\
                        2 & 2 & -1
                    \end{bmatrix}\begin{bmatrix}
                        x\\y\\z
                    \end{bmatrix} = \begin{bmatrix}
                        0\\0\\0
                    \end{bmatrix}.
                \]
                Cuja solu\c{c}ão é $z = 2x + 2y$. Assim
                \[
                    \aut_T(-1) = \{(x,y,2x + 2y) \mid x,y \in \real\} = \{(x,0,2x) + (0, y, 2y) \mid x,y \in \real\} = [(1,0,2); (0, 1, 2)]
                \]
                e ent\~ao $\mathcal{B}_1 = \{(1,0,2); (0, 1, 2)\}$ \'e uma base de $\aut_T(-1)$ e $\dim_\real\aut_T(3) = 2$.

                \item para $\lambda = -2$ temos que $(x,y,z) \in \aut_T(-2)$ se, e s\'o se,
                \[
                    \begin{bmatrix}
                        3 & 2 & -1\\
                        -2 & \phantom{-}1 & \phantom{-}1\\
                        2 & \phantom{-}2 & \phantom{-}0
                    \end{bmatrix}\begin{bmatrix}
                        x\\y\\z
                    \end{bmatrix} = \begin{bmatrix}
                        0\\0\\0
                    \end{bmatrix}.
                \]
                Cuja solu\c{c}ão é $y = x$ e $z = x$. Assim
                \[
                    \aut_T(-2) = \{(x,-x,x) \mid z \in \real\} = [(1,-1,1)]
                \]
                e ent\~ao $\mathcal{B}_2 = \{(1,-1,1)\}$ \'e uma base de $\aut_T(-2)$ e $\dim_\real\aut_T(-2) = 1$.
            \end{itemize}
            \'E f\'acil verificar que o conjunto $\mathcal{D} = \{(1,0,2); (0,1,2); (1,-1,1)\}$ \'e L.I, logo uma base de $\real^3$. Nesta base temos
            \[
                [T]_\mathcal{D} = \begin{bmatrix}
                                -1 & \phantom{-}0 & \phantom{-}0\\
                                \phantom{-}0 & -1 & \phantom{-}0\\
                                \phantom{-}0 & \phantom{-}0 & -2
                            \end{bmatrix}.
            \]
            Logo $T$ \'e diagonaliz\'avel.
        \end{solucao}

        \item Decida se o operador $T \colon \complex^3 \to \complex^3$ dado pela matriz
            \[
                [T]_\mathcal{B} = \begin{bmatrix}\phantom{-}1 & 3 & 3\\\phantom{-}0 & 4 & 0\\-3 & 3 & 1\end{bmatrix}
            \]
            onde $\mathcal{B}$ é uma base qualquer de $\complex^3$ é diagonalizável.
            \begin{solucao}
                Temos
                \[
                    [T]_\mathcal{B} - \lambda I_3 = \begin{bmatrix}1 - \lambda & 3 & 3\\\phantom{-}0 & 4 - \lambda & 0\\-3 & 3 & 1 - \lambda\end{bmatrix}.
                \]
                Daí, usando a segunda linha dessa matriz para calcular o determinante temos:
                \begin{align*}
                    \det([T]_\mathcal{B} - \lambda I_3) &= \det\left(\begin{bmatrix}1 - \lambda & 3 & 3\\0 & 4 - \lambda & 0\\-3 & 3 & 1 - \lambda\end{bmatrix}\right) \\ &= (-1)^{2 + 2}a_22\det(A_{22}) = (4 - \lambda)\det\left(\begin{bmatrix}1 - \lambda & 3\\-3 & 1 - \lambda\end{bmatrix}\right) \\ &= (4 - \lambda)[(1 - \lambda)^2 + 9] \\ &= (4 - \lambda)(\lambda^2 - 2\lambda + 10).
                \end{align*}
                Assim $\det([T]_\mathcal{B} - \lambda I_3) = 0$ se, e sé se, $(4 - \lambda)(\lambda^2 - 2\lambda + 10) = 0$. Como $\lambda \in \complex$, então $\lambda = 4$ ou $\lambda = 1 - 3i$ ou $\lambda = 1 + 3i$.

                Vamos determinar $\aut_T(4)$, $\aut_T(1 - 3i)$ e $\aut_T(1 + 3i)$:
                \begin{itemize}
                    \item para $\lambda = 4$ temos que $(x,y,z) \in \aut_T(4)$ se, e s\'o se,
                    \[
                        \begin{bmatrix}-3 & 3 & \phantom{-}3\\\phantom{-}0 & 0 & \phantom{-}0\\\phantom{-}0 & 3 & -3\end{bmatrix}
                        \begin{bmatrix}x\\y\\z\end{bmatrix} = \begin{bmatrix}0\\0\\0\end{bmatrix}.
                    \]
                    Cuja solu\c{c}ão é $y = z$ e $x = 2z$. Assim
                    \[
                        \aut_T(4) = \{(2z,z,z) \mid z \in \complex\} = [(2,1,1)]
                    \]
                e ent\~ao $\mathcal{B}_1 = \{(2,1,1))\}$ \'e uma base de $\aut_T(4)$ e $\dim_\complex\aut_T(4) = 1$.

                \item para $\lambda = 1 - 3i$ temos que $(x,y,z) \in \aut_T(1 - 3i)$ se, e s\'o se,
                \[
                    \begin{bmatrix}
                        3i & 3 & \phantom{-}3\\
                        \phantom{-}0 & 3 + 3i & \phantom{-}0\\
                        -3 & 3 & 3i
                    \end{bmatrix}\begin{bmatrix}
                        x\\y\\z
                    \end{bmatrix} = \begin{bmatrix}
                        0\\0\\0
                    \end{bmatrix}.
                \]
                Cuja solu\c{c}ão é $x = iz$ e $y = 0$. Assim
                \[
                    \aut_T(1 - 3i) = \{(iz,0,z) \mid z \in \complex\} = [(i,0,1)]
                \]
                e ent\~ao $\mathcal{B}_2 = \{(i,0,1)\}$ \'e uma base de $\aut_T(1 - 3i)$ e $\dim_\complex\aut_T(1 - 3i) = 1$.

                \item para $\lambda = 1 + 3i$ temos que $(x,y,z) \in \aut_T(1 + 3i)$ se, e s\'o se,
                \[
                    \begin{bmatrix}
                        3i & 3 & \phantom{-}3\\
                        \phantom{-}0 & 3 - 3i & \phantom{-}0\\
                        -3 & 3 & -3i
                    \end{bmatrix}\begin{bmatrix}
                        x\\y\\z
                    \end{bmatrix} = \begin{bmatrix}
                        0\\0\\0
                    \end{bmatrix}.
                \]
                Cuja solu\c{c}ão é $x = -iz$ e $y = 0$. Assim
                \[
                    \aut_T(1 - 3i) = \{(-iz,0,z) \mid z \in \complex\} = [(-i,0,1)]
                \]
                e ent\~ao $\mathcal{B}_2 = \{(-i,0,1)\}$ \'e uma base de $\aut_T(1 + 3i)$ e $\dim_\complex\aut_T(1 + 3i) = 1$.
            \end{itemize}

            Logo o conjunto $\mathcal{D} = \{(2, 1, 1); (i, 0, 1); (-i, 0, 1)\}$ é uma base de $\complex^3$ e $T$ é diagonalizável. Com rela\c{c}ão à base $\mathcal{D}$ temos:
            \[
                [T]_\mathcal{D} = \begin{bmatrix}4 & 0 & 0\\0 & 1 - 3i & 0\\0 & 0 & 1 + 3i\end{bmatrix}.
            \]
            \end{solucao}

        \item Decida se o operador $T \colon \real^4 \to \real^4$ dado pela matriz
            \[
                [T]_\mathcal{B} = \begin{bmatrix}2 & 0 & 0 &\phantom{-}0\\0 & 1 & 0 & \phantom{-}0\\0 & 2 & 0 & \phantom{-}0\\1 & 0 & 0 & -2\end{bmatrix}
            \]
            onde $\mathcal{B}$ é uma base qualquer de $\real^4$ é diagonalizável.
            \begin{solucao}
                Aqui temos
                \[
                    [T]_\mathcal{B} - \lambda I_4 = \begin{bmatrix}2 - \lambda & 0 & \phantom{-}0 & 0\\0 & 1 - \lambda & \phantom{-}0 & 0\\0 & 2 & -\lambda & 0\\1 & 0 & \phantom{-}0 & -2 - \lambda\end{bmatrix}.
                \]
                Usando a primeira linha para calcular o determinante:
                \begin{align*}
                    \det([T]_\mathcal{B} - \lambda I_4) &= \det\left(\begin{bmatrix}2 - \lambda & 0 & 0 & 0\\0 & 1 - \lambda & 0 & 0\\0 & 2 & -\lambda & 0\\1 & 0 & 0 & -2 - \lambda\end{bmatrix}\right) \\ &= (-1)^{1 + 1}a_{11}\det(A_{11}) \\ &= (2 - \lambda)\det\left(\begin{bmatrix}1 - \lambda & 0 & 0\\2 & -\lambda & 0\\0 & 0 & -2 - \lambda\end{bmatrix}\right) \\ &= (2 - \lambda)(1 - \lambda)(-\lambda)(-2 - \lambda)
                \end{align*}
                Assim $\det([T]_\mathcal{B} - \lambda I_4) = 0$ se, e sé se, $(2 - \lambda)(1 - \lambda)(-\lambda)(-2 - \lambda) = 0$. Então $\lambda = 2$ ou $\lambda = 1$ ou $\lambda = 0$ ou $\lambda = -2$.

                Vamos determinar $\aut_T(2)$, $\aut_T(1)$, $\aut_T(0)$ e $\aut_T(-2)$:
                \begin{itemize}
                    \item para $\lambda = 2$ temos que $(x,y,z,t) \in \aut_T(2)$ se, e s\'o se,
                        \[
                            \begin{bmatrix}0 & \phantom{-}0 & \phantom{-}0 & \phantom{-}0\\0 & -1 & \phantom{-}0 & \phantom{-}0\\0 & \phantom{-}2 & -2 & \phantom{-}0\\1 & \phantom{-}0 & \phantom{-}0 & -4\end{bmatrix}
                            \begin{bmatrix}x\\y\\z\\t\end{bmatrix} = \begin{bmatrix}0\\0\\0\\0\end{bmatrix}.
                        \]
                        Cuja solu\c{c}ão é $y = z = 0$ e $x = 4t$. Assim
                        \[
                            \aut_T(2) = \{(4t,0,0,t) \mid t \in \real\} = [(4,0,0,1)]
                        \]
                        e ent\~ao $\mathcal{B}_1 = \{(4,0,0,1))\}$ \'e uma base de $\aut_T(2)$ e $\dim_\real\aut_T(2) = 1$.

                    \item para $\lambda = 1$ temos que $(x,y,z,t) \in \aut_T(1)$ se, e s\'o se,
                        \[
                            \begin{bmatrix}1 & 0 & \phantom{-}0 & \phantom{-}0\\0 & 0 & \phantom{-}0 & \phantom{-}0\\0 & 2 & -1 & \phantom{-}0\\1 & 0 & \phantom{-}0 & -3\end{bmatrix}
                            \begin{bmatrix}x\\y\\z\\t\end{bmatrix} = \begin{bmatrix}0\\0\\0\\0\end{bmatrix}.
                        \]
                        Cuja solu\c{c}ão é $x = t = 0$ e $z = 2y$. Assim
                        \[
                            \aut_T(1) = \{(0,y,2y,0) \mid y \in \real\} = [(0,1,2,0)]
                        \]
                        e ent\~ao $\mathcal{B}_2 = \{(0,1,2,0))\}$ \'e uma base de $\aut_T(1)$ e $\dim_\real\aut_T(1) = 1$.

                    \item para $\lambda = 0$ temos que $(x,y,z,t) \in \aut_T(0)$ se, e s\'o se,
                        \[
                            \begin{bmatrix}2 & 0 & 0 & \phantom{-}0\\0 & 1 & 0 & \phantom{-}0\\0 & 2 & 0 & \phantom{-}0\\1 & 0 & 0 & -2\end{bmatrix}
                            \begin{bmatrix}x\\y\\z\\t\end{bmatrix} = \begin{bmatrix}0\\0\\0\\0\end{bmatrix}.
                        \]
                        Cuja solu\c{c}ão é $x = y = t = 0$. Assim
                        \[
                            \aut_T(0) = \{(0,0,z,0) \mid z \in \real\} = [(0,0,1,0)]
                        \]
                        e ent\~ao $\mathcal{B}_3 = \{(0,0,1,0))\}$ \'e uma base de $\aut_T(0)$ e $\dim_\real\aut_T(0) = 1$.

                    \item para $\lambda = -2$ temos que $(x,y,z,t) \in \aut_T(-2)$ se, e s\'o se,
                        \[
                            \begin{bmatrix}4 & 0 & 0 & 0\\0 & 3 & 0 & 0\\0 & 2 & 2 & 0\\1 & 0 & 0 & 0\end{bmatrix}
                            \begin{bmatrix}x\\y\\z\\t\end{bmatrix} = \begin{bmatrix}0\\0\\0\\0\end{bmatrix}.
                        \]
                        Cuja solu\c{c}ão é $x = y = z = 0$. Assim
                        \[
                            \aut_T(-2) = \{(0,0,0,t) \mid t \in \real\} = [(0,0,0,1)]
                        \]
                        e ent\~ao $\mathcal{B}_4 = \{(0,0,0,1))\}$ \'e uma base de $\aut_T(-2)$ e $\dim_\real\aut_T(-2) = 1$.
                \end{itemize}

                Logo o conjunto $\mathcal{D} = \{(4, 0, 0, 1); (0, 1, 2, 0); (4, 0, 0, 1); (0, 0, 0, 1)\}$ é uma base de $\real^4$ e $T$ é diagonalizável. Com rela\c{c}ão à base $\mathcal{D}$ temos:
                \[
                    [T]_\mathcal{D} = \begin{bmatrix}2 & 0 & 0 & \phantom{-}0\\0 & 1 & 0 & \phantom{-}0\\0 & 0 & 0 & \phantom{-}0\\0 & 0 & 0 & -2\end{bmatrix}.
                \]
            \end{solucao}

        \item Decida se o operador $T \colon \real^3 \to \real^3$ dado pela matriz
            \[
                [T]_\mathcal{B} = \begin{bmatrix}\phantom{-}1 & 0 & 2\\-1 & 0 & 1\\\phantom{-}1 & 1 & 2\end{bmatrix}
            \]
            onde $\mathcal{B}$ é uma base qualquer de $\real^3$ é diagonalizável.
            \begin{solucao}
                Aqui como já temos a matriz da transforma\c{c}ão vamos passar para a etapa 2:
                \begin{align*}
                    [T]_\mathcal{B} - \lambda I_3 = \begin{bmatrix}\phantom{-}1 & 0 & 2\\-1 & 0 & 1\\\phantom{-}1 & 1 & 2\end{bmatrix} - \begin{bmatrix} \lambda & 0 & 0\\0 & \lambda & 0\\0 & 0 & \lambda\end{bmatrix} = \begin{bmatrix}1 - \lambda & \phantom{-}0 & 2\\-1 & -\lambda & 1\\\phantom{-}1 & \phantom{-}1 &2 - \lambda\end{bmatrix}
                \end{align*}
                Usando a primeira linha dessa matriz para calcular o determinante obtemos:
                \begin{align*}
                    \det([T]_\mathcal{B} - \lambda I_3) &= (-1)^{1 + 1}a_{11}\det(A_{11}) + (-1)^{1 + 2}a_{12}\det(A_{12}) + (-1)^{1 + 3}a_{13}\det(A_{13}) \\ &= (1 - \lambda)\det\left(\begin{bmatrix}-\lambda & 1\\1 & 2 - \lambda\end{bmatrix}\right) + 2\det\left(\begin{bmatrix}-1 & -\lambda\\1 & 1\end{bmatrix}\right) \\ &= (1 - \lambda)[(-\lambda)(2 - \lambda) - 1] + 2[(-1)1 - (-\lambda)1] \\ &= (1 - \lambda)(-2\lambda + \lambda^2 - 1) - 2(1 - \lambda) \\ &= (1 - \lambda)(\lambda^2 - 2\lambda - 1 - 2) \\ &= ( 1 -\lambda)(\lambda^2 - 2\lambda - 3) \\ &= (1 - \lambda)(\lambda + 1)(\lambda - 3) 
                \end{align*}
                Assim $\det([T]_\mathcal{B} - \lambda I_3) = 0$ quando $(1 - \lambda)(\lambda + 1)(\lambda - 3) = 0$, ou seja, quando $\lambda = -1$ ou $\lambda = 1$ ou $\lambda = 3$.

                Vamos determinar $\aut_T(-1)$, $\aut_T(1)$ e $\aut_T(3)$:
                \begin{itemize}
                    \item para $\lambda = -1$ temos que $(x,y,z) \in \aut_T(-1)$ se, e s\'o se,
                        \[
                            \begin{bmatrix}
                                \phantom{-}2 & 0 & 2\\
                                -1 & 1 & 1\\
                                \phantom{-}1 & 1 & 3
                            \end{bmatrix}\begin{bmatrix}
                                x\\y\\z
                            \end{bmatrix} = \begin{bmatrix}
                                0\\0\\0
                            \end{bmatrix}.
                        \]
                        Cuja solu\c{c}ão é $x = -5z$ e $y = 2z$. Assim
                        \[
                            \aut_T(-1) = \{(-5z,2z,z) \mid z \in \real\} = [(-5,2,1)]
                        \]
                        e ent\~ao $\mathcal{B}_1 = \{(-5,2,1)\}$ \'e uma base de $\aut_T(-1)$ e $\dim_\real\aut_T(-1) = 1$.

                    \item para $\lambda = 1$ temos que $(x,y,z) \in \aut_T(1)$ se, e s\'o se,
                        \[
                            \begin{bmatrix}
                                \phantom{-}0 & \phantom{-}0 & 2\\
                                -1 & -1 & 1\\
                                \phantom{-}1 & \phantom{-}1 & 1
                            \end{bmatrix}\begin{bmatrix}
                                x\\y\\z
                            \end{bmatrix} = \begin{bmatrix}
                                0\\0\\0
                            \end{bmatrix}.
                        \]
                        Cuja solu\c{c}ão é $x = -y$ e $z = 0$. Assim
                        \[
                            \aut_T(1) = \{(-y,y,0) \mid y \in \real\} = [(-1,1,0)]
                        \]
                        e ent\~ao $\mathcal{B}_2 = \{(-1,1,0)\}$ \'e uma base de $\aut_T(1)$ e $\dim_\real\aut_T(1) = 1$.

                    \item para $\lambda = 3$ temos que $(x,y,z) \in \aut_T(3)$ se, e s\'o se,
                        \[
                            \begin{bmatrix}
                                -2 & \phantom{-}0 & \phantom{-}2\\
                                -1 & -3 & \phantom{-}1\\
                                \phantom{-}1 & \phantom{-}1 & -1
                            \end{bmatrix}\begin{bmatrix}
                                x\\y\\z
                            \end{bmatrix} = \begin{bmatrix}
                                0\\0\\0
                            \end{bmatrix}.
                        \]
                    Cuja solu\c{c}ão é $x = z$ e $y = 0$. Assim
                    \[
                        \aut_T(3) = \{(z,0,z) \mid z \in \real\} = [(1,0,1)]
                    \]
                    e ent\~ao $\mathcal{B}_3 = \{(1,0,1)\}$ \'e uma base de $\aut_T(3)$ e $\dim_\real\aut_T(3) = 1$.
            \end{itemize}

             Logo o conjunto $\mathcal{D} = \{(-5, 2, 1); (-1, 1, 0); (1, 0, 1)\}$ é uma base de $\real^3$ e $T$ é diagonalizável. Com rela\c{c}ão à base $\mathcal{D}$ temos:
                \[
                    [T]_\mathcal{D} = \begin{bmatrix}-1 & 0 & 0\\\phantom{-}0 & 1 & 0\\\phantom{-}0 & 0 & 3\end{bmatrix}.
                \]
        \end{solucao}
    \end{enumerate}
\end{exemplo}

\begin{teorema}
    Seja $T : V \to V$ um operador linear onde $V$ \'e um $\cp{K}$-espa\c{c}o vetorial de dimens\~ao finita e sejam $\lambda_1$, \dots, $\lambda_r$, $r \ge 1$, autovalores de $T$, dois a dois distintos.
    \begin{enumerate}[label={\roman*})]
        \item\label{autovetorLI} Se $u_1 + \cdots + u_r = 0_V$ com $u_i \in \aut_T(\lambda_i)$; $i = 1$, \dots, $r$; ent\~ao $u_i = 0_V$ para todo $i$.
        \item Para cada $i = 1$, \dots, $r$ seja $\mathcal{B}_i$ um conjunto linearmente independente contido em $\aut_T(\lambda_i)$. Ent\~ao $\mathcal{B}_1 \cup \cdots \cup \mathcal{B}_r$ \'e L.I. em $V$.
    \end{enumerate}
\end{teorema}

\begin{corolario}
    Seja $T : V \to V$ um operador linear, onde $V$ \'e um $\cp{K}$-espa\c{c}o vetorial de dimens\~ao finita. Se $\lambda_1$, \dots, $\lambda_r$ s\~ao todos os autovalores de $T$, ent\~ao $T$ \'e diagonaliz\'avel se, e somente se,
    \[
        \dim_\cp{K} V = \sum_{i = i}^r\dim_\cp{K}\aut_T(\lambda_i).
    \]
\end{corolario}
