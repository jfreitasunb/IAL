%!TEX program = xelatex
%!TEX root = IAL.tex

\chapter{Diagonalização de Operadores Lineares}

\section{Operadores Diagonaliz\'aveis} % (fold)
\label{sec:operadores_diagonalizaveis}

Considere o operador linear $T \colon \real^3 \to \real^3$ dado por
\begin{align}\label{exemplodiagonalizacao}
    T(x, y, z) = (x + 2y - z, -2x -3y + z, 2x + 2y - 2z).
\end{align}

Quando consideramos a base ordenada canônica de $\real^3$, $\mathcal{B} = \{e_1 = (1, 0, 0); e_2 = (0, 1, 0); e_3 = (0, 0, 1)\}$, a matriz desse operador com respeito à essa base é
\[
    [T]_\mathcal{B} = \begin{bmatrix}
        \phantom{-}1 & \phantom{-}2 & -1\\
        -2 & -3 & \phantom{-}1\\
        \phantom{-}2 & \phantom{-}2 & \phantom{-}2
    \end{bmatrix}.
\]

Agora, se tomamos a base ordenada $\mathcal{B}_1 = \{w_1 = (1, 0, 2); w_2 = (0, 1, 2); w_3 = (1, -1, 1)\}$ a matriz de $T$ com respeito à essa base é
\[
    [T]_{\mathcal{B}_1} = \begin{bmatrix}
        -1 & \phantom{-}0 & \phantom{-}0\\
        \phantom{-}0& -1 & \phantom{-}0\\
        \phantom{-}0& \phantom{-}0 & -2
    \end{bmatrix}.
\]

Analisando a matriz $[T]_{\mathcal{B}_1}$ é fácil ver que $T$ é um operador invertível e também é fácil achar $T^{-1}$. Nesse caso, a matriz de $T^{-1}$ com respeito à base $\mathcal{B}_1$ é:
\[
[T^{-1}]_{\mathcal{B}_1} = \begin{bmatrix}
    -1 & \phantom{-}0 & \phantom{-}0\\
    \phantom{-}0& -1 & \phantom{-}0\\
    \phantom{-}0& \phantom{-}0 & -1/2
\end{bmatrix}.
\]

Note que para a base $\mathcal{B}_1$ temos
\begin{align*}
    T(w_1) &= -1w_1\\
    T(w_2) &= -1w_2\\
    T(w_3) &= -2w_3\\
\end{align*}

No caso geral, seja $T : V \to V$ um operador linear e suponha que exista uma base $\mathcal{B} = \{v_1,\dots,v_n\}$ de $V$ tal que
\begin{align}\label{formadiagonal}
    [T]_\mathcal{B} = \begin{bmatrix}
        \lambda_1 & 0 & 0 & \dots & 0\\
        0 & \lambda_2 & 0 & \dots & 0\\
        \vdots & & \ddots & & \vdots\\
        0 & 0 & 0 & \dots & \lambda_n
    \end{bmatrix}
\end{align}
com $\lambda_i \in \cp{K}$ para $i = 1$, \dots, $n$. Assim
\[
    [T(v_i)]_\mathcal{B} = [T]_\mathcal{B}[v_i]_\mathcal{B} = [T]_\mathcal{B}\begin{bmatrix}
        0\\
        0\\
        \vdots\\
        0\\
        1\\
        0\\
        \vdots\\
        0
    \end{bmatrix}_\mathcal{B} = \begin{bmatrix}
        0\\
        0\\
        \vdots\\
        0\\
        \lambda_i\\
        0\\
        \vdots\\
        0
    \end{bmatrix}_\mathcal{B}
\]
para $i = 1$, \dots, $n$. Isto \'e,
\[
    T(v_1) = \lambda_1 v_1,\ T(v_2) = \lambda_2 v_2,\dots, T(v_n) = \lambda_n v_n.
\]

\begin{definicao}
    Seja $T : V \to V$ um operador linear.
    \begin{enumerate}[label={\roman*})]
        \item Um \textbf{autovalor} de $T$ \'e um elemento $\lambda \in \cp{K}$ tal que existe um vetor n\~ao nulo $u \in V$ com $T(u) = \lambda u$.\index{Autovalor}
        \item Se $\lambda$ \'e um autovalor de $T$, ent\~ao todo vetor n\~ao nulo $u \in V$ tal que
        \[
            T(u) = \lambda u
        \]
        \'e chamado de \textbf{autovetor} de $T$ \textbf{associado} ao autovalor $\lambda$. Denotaremos por $Aut_T(\lambda)$ o subespa\c{c}o gerado por todos os autovetores associados a $\lambda$. Assim\index{Autovetor}
        \[
            \aut_T(\lambda) = \{u \in V \mid T(u) = \lambda u\}.
        \]
        \item Suponha que $\dim V = n < \infty$. Dizemos que $T$ \'e \textbf{diagonaliz\'avel} se existir uma base $\mathcal{B}$ de $V$ tal que $[T]_\mathcal{B}$ \'e diagonal, isto \'e, tem a forma \eqref{formadiagonal}. Tal fato equivale a dizer que existe uma base formada por autovetores.\index{Transforma\c{c}\~ao Linear!Diagonaliz\'avel}
    \end{enumerate}
\end{definicao}

\begin{exemplo}
    No caso da transformação linear $T$ dada em \eqref{exemplodiagonalizacao}, $-1$ é um autovalor de $T$ e os autovetores associados ao autovalor $-1$ são $w_1 = (1, 0, 1)$ e $w_2 = (0, 1, 2)$. Nesse caso
    \[
        Aut_T(-1) = [w_1, w_2].
    \]
    Observer que se $u = \alpha_1w_1 + \alpha_2w_2 \in Aut_T(-1)$, então
    \begin{align*}
        T(u) &= T(\alpha_1w_1 + \alpha_2w_2) \\ &= \alpha_1T(w_1) + \alpha_2T(w_2) \\ &= -\alpha_1w_1 - \alpha_2w_2 \\ &= -(\alpha_1w_1 + \alpha_2w_2) \\ &= -u
    \end{align*}

    Além disso, $-2$ é um autovalor de $T$ e o autovetor associado a esse autovalor é $w_3 = (1, -1, 1)$. Assim
    \[
        Aut_T(-2) = [w_3].
    \]
    Aqui também temos que se $u = \alpha w_3 \in Aut_T(-2)$, então
    \begin{align*}
        T(u) &= T(\alpha w_3) \\ &= \alpha T(w_3) \\ &= -2\alpha w_3 \\ &= -2u
    \end{align*}
\end{exemplo}

Temos aqui duas questões importantes:
\begin{enumerate}
    \item Dado um operador $T \colon V \to V$, existem autovetores para $T$?

    \item Se sim, existe um número suficiente de autovetores para forma uma base de $V$?
\end{enumerate}

Os seguintes exemplos respondem a essas perguntas:

\begin{exemplos}
    \begin{enumerate}
        \item Se $T \colon \real^2 \to \real^2$ é o operador linear dado pela matriz
        \[
            [T]_\mathcal{B} = \begin{bmatrix}0 & -1\\1 & \phantom{-}0\end{bmatrix}
        \]
        onde $\mathcal{B}$ é a base canônica, então não existe $u \in \real$ tal que $T(u) = \lambda_u$, com $\lambda_\in \real$.

        \item Se $T \colon \real^3 \to \real^3$ é o operador linear dado pela matriz
        \[
        [T]_\mathcal{B} = \begin{bmatrix}3 & -3 & -4\\0 & \phantom{-}3 & \phantom{-}5\\0 & \phantom{-}0 & \phantom{-}1\end{bmatrix}
        \]
        onde $\mathcal{B}$ é a base canônica. Então os únicos vetores de $u \in \real^3$ tais que $T(u) = \lambda_u$, para algum $\lambda_\in \real$ são $u = (1, 0, 0)$ e $v = (-7/4, -5/2, 1)$.

        Nessa caso o conjunto $\{u, v\}$ é linearmente independente mas não é uma base para $\real^3$.
    \end{enumerate}
\end{exemplos}

Aqui surgem duas questões:
\begin{enumerate}
    \item Como determinar se uma transforma\c{c}ão linear $T \colon V \to V$ possui autovetores?

    \item A existência de autovetores de $T$ depende da base escolhida para $V$?
\end{enumerate}

As respostas para essas perguntas são dadas pelos seguintes teoremas:

\begin{teorema}
    As seguintes afirma\c{c}ões são equivalentes:
    \begin{enumerate}
        \item o escalar $\lambda \in \cp{K}$ é um autovalor do operador linear $T \colon V \to V$;

        \item o operador $T - \lambda Id$ não é invertível;

        \item dada uma base ordenada $\mathcal{B}$ qualquer de $V$ temos
            \[
                \det([T]_\mathcal{B} - \lambda[Id]_\mathcal{B}) = 0.
            \]
    \end{enumerate}
\end{teorema}

\begin{teorema}
    Se $A \in M_n(\cp{K})$ for uma matriz $n \times n$, então $\lambda \in \cp{K}$ é um autovalor de $A$ se, e somente se, $\lambda$ satisfaz a equa\c{c}ão
    \[
        \det(A - \lambda I_n) = 0.
    \]
    Essa equa\c{c}ão é chamada de \textbf{equa\c{c}ão característica} de $A$.
\end{teorema}

Observe que nos dois teoremas anteriores aparecem o determinante de uma matriz. Vamos então relembrar algumas propriedades dos determinantes.

\section{Determinantes}

Seja $A \in M_n(\cp{K})$ onde $n \ge 1$. Vamos definir o \textbf{determinante da matriz $A$}, denotado por $\det(A)$, de modo indutivo.

Se $n = 1$, ent\~ao a matriz $A \in \cp{M}_1(\cp{K})$ \'e da forma
\[
	A = (a_{11})
\]
e neste caso definimos
\[
	\det(A) = a_{11} \in \cp{K}.
\]
Suponha que $n > 1$ e que $\det(B)$ esteja definido para todas as matrizes $B \in \cp{M}_p(\cp{K})$ com $p < n$ e seja $A \in \cp{M}_n(\cp{K})$. Para cada $(i,j)$, defina a matriz $A_{ij}$ formada a partir de $A$ retirando-se a sua $i$-\'esima linha e a sua $j$-\'esima coluna. \'E claro que $A \in \cp{M}_{n - 1}(\cp{K})$ e portanto $\det(A_{ij})$ est\'a definido. Defina ent\~ao
\begin{align*}
	\det(A) &= \sum_{j = 1}^n(-1)^{i + j}a_{ij}\det(A_{ij}) \\ &= (-1)^{i + 1}a_{i1}\det(A_{i1}) + (-1)^{i + 2}a_{i2}\det(A_{i2}) \\ & + (-1)^{i + 3}a_{i3}\det(A_{i3}) + \cdots + (-1)^{i + n}a_{in}\det(A_{in}) \in \cp{K}
\end{align*}

\begin{exemplo}
	\begin{enumerate}[label={\arabic*})]
		\item Seja
		\[
			A =
			\begin{bmatrix}
				a & b\\
				c & d
			\end{bmatrix} \in \cp{M}_2(\cp{K}).
		\]
		Fixada a linha 1, temos
		\begin{align*}
			\det(A) &= \sum_{j = 1}^2(-1)^{1 + j}a_{1j}\det(A_{1j}) = (-1)^{1 + 1}a_{11}\det(A_{11}) + (-1)^{1 + 2}a_{12}\det(A_{12})\\
			\det(A) &= ad - bc.
	    \end{align*}
		Obter{\'\i}amos o mesmo resultado se consider\'assemos a linha 2.

		\item Seja
		\[
			A =
			\begin{bmatrix}
				a_{11} & a_{12} & a_{13}\\
				a_{21} & a_{22} & a_{23}\\
				a_{31} & a_{32} & a_{33}
			\end{bmatrix} \in \cp{M}_2(\cp{K}).
		\]
		Fixada a linha 2, temos
		\begin{align*}
			\det(A) &= \sum_{j = 1}^3(-1)^{2 + j}a_{2j}\det(A_{2j}) \\ &= (-1)^{2 + 1}a_{21}\det(A_{21}) + (-1)^{2 + 2}a_{22}\det(A_{22}) + (-1)^{2 + 3}a_{32}\det(A_{32})\\ &= -a_{21}\det\left(\begin{bmatrix}a_{12} & a_{13}\\a_{32} & a_{33}\end{bmatrix}\right) + a_{22}\det\left(\begin{bmatrix}a_{11} & a_{12}\\a_{31} & a_{33}\end{bmatrix}\right) - a_{23}\det\left(\begin{bmatrix}a_{11} & a_{12}\\a_{31} & a_{32}\end{bmatrix}\right).
		\end{align*}
		Da{\'\i}
		\[
			\det(A) = a_{11}a_{22}a_{33} + a_{12}a_{23}a_{31} + a_{13}a_{21}a_{32} - a_{13}a_{22}a_{31} - a_{12}a_{21}a_{33} - a_{11}a_{23}a_{32}.
		\]

	    \item Seja
	        \[
	            A = \begin{bmatrix}
	                \phantom{-}6 & -3 & \phantom{-}1 & 0\\
	                \phantom{-}0 & \phantom{-}3 & \phantom{-}1 & 0\\
	                -6 & \phantom{-}6 & \phantom{-}0 & 0\\
	                -3 & \phantom{-}3 & -2 & 3
	            \end{bmatrix} \in M_4(\real).
	        \]
	        Para calcular $\det(A)$ podemos usar a terceira linha dessa matriz, pois nessa linha $a_{33} = a_{34} = 0$. Assim a expressão
	        \[
	            \det(A) = \sum_{j = 1}^4(-1)^{3 + i}a_{3j}\det(A_{3j})
	        \]
	        torna-se simplesmente
	        \begin{align*}
	            \det(A) &= (-1)^{3 + 1}a_{31}\det(A_{31}) + (-1)^{3 + 2}a_{32}\det(A_{32}) \\ &=
	            -6\cdot\det\left(\begin{bmatrix}-3 & \phantom{-}1 & 0\\\phantom{-}0 & \phantom{-}1 & 0\\\phantom{-}3 & -2 & 0\end{bmatrix}\right) -6\cdot\det\left(\begin{bmatrix}\phantom{-}6 & \phantom{-}1 & 0\\\phantom{-}0 & \phantom{-}1 & 0\\-3 & -2 & 3\end{bmatrix}\right)\\
	            \det(A) &= -6\cdot 18 = -108
	        \end{align*}
	\end{enumerate}
\end{exemplo}

\begin{proposicao}
	Sejam $A$, $B \in \cp{M}_n(\cp{K})$ e $\lambda \in \cp{K}$. Temos:
	\begin{enumerate}[label={\roman*})]
		\item $\det(AB) = \det(A) \det(B)$,
		\item $\det(\lambda A) = \lambda^n \det(A)$,
		\item Se $A$ é invertível, então $\det(A^{-1}) = [(\det(A)]^{-1}$.
	\end{enumerate}
\end{proposicao}


\begin{proposicao}
	Seja $A$ uma matriz $n \times n$ com entradas num corpo $\cp{K}$.
	\begin{enumerate}[label={\roman*})]
		\item Se $B$ \'e a matriz resultante da permuta\c{c}\~ao de duas linhas de $A$, ent\~ao $\det (B) = -\det (A)$.
		\item Se $B$ \'e a matriz resultante da multiplica\c{c}\~ao de uma linha de $A$ por um escalar n\~ao nulo $\alpha \in \cp{K}$, ent\~ao $\det(B) = \alpha\det(A)$.
		\item Se $B$ \'e a matriz resultante da soma da linha $i$ de $A$ com um m\'ultiplo n\~ao nulo $\alpha \in \cp{K}$ da linha $j$ de $A$, ent\~ao $\det(B) = \det(A)$.
	\end{enumerate}
\end{proposicao}

\begin{observacao}
	\'E poss{\'\i}vel mostrar que o determinante tamb\'em pode ser definido a partir das colunas de uma matriz $A \in \cp{M}_n(\cp{K})$.
\end{observacao}

\begin{teorema}
	Uma matriz $A \in \cp{M}_n(\cp{K})$ \'e invert{\'\i}vel se, e somente se, $\det(A) \ne 0$.
\end{teorema}

\section{Diagonaliza\c{c}ão}
Dado $V$ um espaćo vetorial sobre $\cp{K}$, com $\dim_\cp{K} V = n < \infty$ e um operador linear $T \colon V \to V$, queremos tentar encontrar, se existir, uma base ordenada $\mathcal{D}$ de $V$ formada por autovetores. Para isso devemos proceder da seguinte maneira:
\begin{enumerate}
    \item Se não foi dada, encontre a matriz de $T$ com rela\c{c}ão a alguma base ordenada $\mathcal{B}$ de $V$. Pode-se usar a base canônica $\mathcal{B}$ de $V$ para facilitar o cálculo de $[T]_\mathcal{B}$.

    \item Uma vez encontrada $[T]_\mathcal{B}$ calcule
        \[
            [T]_\mathcal{B} - \lambda I_n = [T]_\mathcal{B} - \begin{bmatrix}\lambda & 0 & 0 & \cdots & 0\\0 & \lambda & 0 & \cdots & 0\\0 & 0 & \lambda & \cdots & 0\\\vdots & \vdots & \vdots & \cdots & \vdots\\0 & 0 & 0 & \cdots & \lambda\end{bmatrix}.
        \]

    \item\label{passo4diag} Determine todos os valores, se existem, de $\lambda \in \cp{K}$ tais que
        \[
            \det([T]_\mathcal{B} - \lambda I_n) = 0.
        \]

    \item Se não existir $\lambda \in \cp{K}$ no passo anterior, então $T$ não é diagonalizável.

    \item Para cada $\lambda_1$, $\lambda_2$, \dots, $\lambda_r$ encontrando em \eqref{passo4diag} resolva o sistema linear homogêneo
        \[
            ([T]_\mathcal{B} - \lambda_jI_n)X = 0.
        \]
        O subespa\c{c}o $Aut_T(\lambda_j)$ será dada pelas solu\c{c}ões desse sistema.

    \item Encontre uma base para cada $Aut_T(\lambda_j)$, com $j = 1$, 2, \dots, $r$.

    \item Se o conjunto $\mathcal{D}$ resultante a união de todas as bases encontradas no passo anterior possuir $\dim_\cp{K}V$ elementos, então $\mathcal{D}$ será uma base de $V$ tal que a matriz de $T$ com rela\c{c}ão à essa base será diagonal. Se $\mathcal{D}$ possuir menos elementos que $\dim_\cp{K}V$ então $T$ não é diagonalizável.
\end{enumerate}

\begin{exemplo}
    \begin{enumerate}[label={\arabic*})]
        \item Seja $T : \real^2 \to \real^2$ o operador linear dado por $T(x,y) = (-y,x)$. Encontre os autovalores de $T$ e os autoespa\c{c}os associados, se existirem.
        \begin{solucao}
            Vamos considerar a base can\^onica de $\real^2$ dada por $\mathcal{A} = \{e_1 = (1,0); e_2 = (0,1)\}$. Temos
            \begin{align}
                T(1,0) &= (0,1) = 0(1,0) + 1(1,0)\\
                T(0,1) &= (-1,0) = -1(1,0) + 0(1,0).
            \end{align}
            Da{\'\i}
            \[
                [T]_\mathcal{A} = \begin{bmatrix}0 & -1\\ 1 & \phantom{-}0\end{bmatrix}
            \]
            e ent\~ao
            \begin{align*}
                p_T(x) = \det([T - xId]_\mathcal{A}) = \det\begin{bmatrix} -x & -1\\\phantom{-}1 & -x\end{bmatrix} = x^2 + 1.
            \end{align*}
            Como $p_T(x)$ n\~ao possui ra{\'\i}zes em $\real$, segue que $T$ n\~ao possui autovalores.
        \end{solucao}
        \item Seja $T : \complex^2 \to \complex^2$ o operador linear dado por $T(x,y) = (-y,x)$. Encontre os autovalores de $T$ e os autoespa\c{c}os associados, se existirem, considerando $\complex^2$ com um $\complex$-espa\c{c}o vetorial.
        \begin{solucao}
            Considere a base can\^onica $\mathcal{A}$ de $\complex^2$. \'E imediato verificar que o polin\^omio caracter{\'\i}stico de $T$ \'e $p_T(x) = x^2 + 1$, cujas ra{\'\i}zes s\~ao $\pm i$. Assim $T$ possui 2 autovalores distintos e para cada um deles vamos encontrar o autoespa\c{c}o associado.
            \begin{itemize}
                \item Para $\lambda_1 = i$ temos:
                \[
                    [T - iId]_\mathcal{A} = \begin{bmatrix} -i & -1\\\phantom{-}1 & -i\end{bmatrix}
                \]
                e assim $(x,y) \in \aut_T(i)$ se, e s\'o se,
                \[
                    \begin{bmatrix} -i & -1\\\phantom{-}1 & -i\end{bmatrix} \begin{bmatrix} x\\y\end{bmatrix}     = \begin{bmatrix} 0\\0\end{bmatrix},
                \]
                isto \'e, $x = iy$. Logo
                \[
                    \aut_T(i) = \{(iy,y) \in \complex^2 \mid y \in \complex\} = [(i,1)].
                \]
                Assim, $\mathcal{B}_1 = \{(i,1)\}$ \'e uma base de $\aut_T(i)$ e da{\'\i} $\dim_\complex\aut_T(i) = 1$.
                \item Para $\lambda_1 = -i$ temos:
                \[
                    [T + iId]_\mathcal{A} = \begin{bmatrix} i & -1\\1 & \phantom{-}i\end{bmatrix}
                \]
                e assim $(x,y) \in \aut_T(-i)$ se, e s\'o se,
                \[
                    \begin{bmatrix} i & -1\\1 & \phantom{-}i\end{bmatrix} \begin{bmatrix} x\\y\end{bmatrix}     = \begin{bmatrix} 0\\0\end{bmatrix},
                \]
                isto \'e, $x = -iy$. Logo
                \[
                    \aut_T(-i) = \{(-iy,y) \in \complex^2 \mid y \in \complex\} = [(-i,1)].
                \]
                Assim, $\mathcal{B}_2 = \{(-i,1)\}$ \'e uma base de $\aut_T(-i)$ e da{\'\i} $\dim_\complex\aut_T(-i) = 1$.
            \end{itemize}
            Agora o conjunto $\mathcal{B} = \mathcal{B}_1 \cup \mathcal{B}_2 = \{(i,1);(-i,1)\}$ \'e uma base de $\complex^2$ e nesta base temos
            \[
                [T]_\mathcal{B} = \begin{bmatrix} i & \phantom{-}0\\0 & -i\end{bmatrix}.
            \]
        \end{solucao}
        \item Seja $T : \real^3 \to \real^3$ o operador linear tal que
        \[
            [T]_\mathcal{A} = \begin{bmatrix}
                                3 & -3 & -4\\
                                0 & \phantom{-}3 & \phantom{-}5\\
                                0 & \phantom{-}0 & \phantom{-}1
                            \end{bmatrix}
        \]
        onde $\mathcal{A}$ \'e uma base qualquer de $\real^3$. Determine, casa exista, uma base de $\real^3$ tal que o operador $T$ seja diagonaliz\'avel.
        \begin{solucao}
            Temos
            \[
                p_T(x) = \det([T - xId]_\mathcal{A}) = (3 - x)^2(1 - x)
            \]
            e assim os autovalores de $T$ s\~ao $\lambda_1 = 3$ e $\lambda_2 = 1$.
            \begin{itemize}
                \item Para $\lambda_1 = 3$ temos que $(x,y,z) \in \aut_T(3)$ se, e s\'o se,
                \[
                    \begin{bmatrix}
                        0 & -3 & -4\\
                        0 & \phantom{-}0 & \phantom{-}5\\
                        0 & \phantom{-}0 & -2
                    \end{bmatrix}\begin{bmatrix}
                        x\\y\\z
                    \end{bmatrix} = \begin{bmatrix}
                        0\\0\\0
                    \end{bmatrix}.
                \]
                Assim
                \[
                    \aut_T(3) = \{(x,0,0) \mid x \in \real\} = [(1,0,0)]
                \]
                e ent\~ao $\mathcal{B}_1 = \{(1,0,0)\}$ \'e uma base de $\aut_T(3)$ e $\dim_\real\aut_T(3) = 1$.
                \item Para $\lambda_2 = 1$ temos que $(x,y,z) \in \aut_T(1)$ se, e s\'o se,
                \[
                    \begin{bmatrix}
                        2 & -3 & -4\\
                        0 & \phantom{-}2 & \phantom{-}5\\
                        0 & \phantom{-}0 & \phantom{-}0
                    \end{bmatrix}\begin{bmatrix}
                        x\\y\\z
                    \end{bmatrix} = \begin{bmatrix}
                        0\\0\\0
                    \end{bmatrix}.
                \]
                Assim
                \[
                    \aut_T(1) = \{(-7z/4,-5z/2,z) \mid z \in \real\} = [(-7/4,-5/2,1)]
                \]
                e ent\~ao $\mathcal{B}_2 = \{(-7/4,-5/2,1)\}$ \'e uma base de $\aut_T(1)$ e $\dim_\real\aut_T(1) = 1$.
            \end{itemize}
            Note que o conjunto $\mathcal{B} = \mathcal{B}_1 \cup \mathcal{B}_2$ \'e L.I. mas n\~ao \'e uma base de $\real^3$. Neste caso o operador $T$ n\~ao \'e diagonaliz\'avel.
        \end{solucao}
        \item Seja $T : \real^3 \to \real^3$ o operador tal que
        \[
            [T]_\mathcal{A} = \begin{bmatrix}
                                \phantom{-}1 & \phantom{-}2 & -1\\
                                -2 & -3 & \phantom{-}1\\
                                \phantom{-}2 & \phantom{-}2 & -2
                            \end{bmatrix}
        \]
        onde $\mathcal{A}$ \'e uma base qualquer de $\real^3$. Determinar se $T$ \'e diagonaliz\'avel.
        \begin{solucao}
        Temos
        \[
                p_T(x) = \det([T - xId]_\mathcal{A}) = -(x + 1)^2(x + 2)
            \]
            e assim os autovalores de $T$ s\~ao $\lambda_1 = -1$ e $\lambda_2 = -2$. C\'alculos simples mostram que
            \begin{align*}
                \aut_T(-1) = [(1,0,2); (0,1,2)]\\
                \aut_T(-2) = [(1,-1,1)].
            \end{align*}
            \'E f\'acil verificar que o conjunto $\mathcal{B} = \{(1,0,2); (0,1,2); (1,-1,1)\}$ \'e L.I, logo uma base de $\real^3$. Nesta base temos
            \[
                [T]_\mathcal{B} = \begin{bmatrix}
                                -1 & \phantom{-}0 & \phantom{-}0\\
                                \phantom{-}0 & -1 & \phantom{-}0\\
                                \phantom{-}0 & \phantom{-}0 & -2
                            \end{bmatrix}.
            \]
            Logo $T$ \'e diagonaliz\'avel.
        \end{solucao}
    \end{enumerate}
\end{exemplo}

\begin{teorema}
    Seja $T : V \to V$ um operador linear onde $V$ \'e um $\cp{K}$-espa\c{c}o vetorial de dimens\~ao finita e sejam $\lambda_1$, \dots, $\lambda_r$, $r \ge 1$, autovalores de $T$, dois a dois distintos.
    \begin{enumerate}[label={\roman*})]
        \item\label{autovetorLI} Se $u_1 + \cdots + u_r = 0_V$ com $u_i \in \aut_T(\lambda_i)$; $i = 1$, \dots, $r$; ent\~ao $u_i = 0_V$ para todo $i$.
        \item Para cada $i = 1$, \dots, $r$ seja $\mathcal{B}_i$ um conjunto linearmente independente contido em $\aut_T(\lambda_i)$. Ent\~ao $\mathcal{B}_1 \cup \cdots \cup \mathcal{B}_r$ \'e L.I. em $V$.
    \end{enumerate}
\end{teorema}
\begin{prova}
    \begin{enumerate}[label={\roman*})]
        \item A prova ser\'a por indu\c{c}\~ao em $r \ge 1$. Se $r = 1$, nada h\'a a fazer. Seja $r > 1$ e suponha que o teorema seja v\'alido para todo $j < r$. Vamos mostrar que tamb\'em \'e v\'alido para $j = r$. Temos
        \begin{equation}\label{equacaoauxiliar1}
            u_1 + u_2 + \cdots + u_r = 0_V
        \end{equation}
        com $u_i \in \aut_T(\lambda_i)$.

        Aplicando $T$ em \eqref{equacaoauxiliar1} obtemos
        \begin{equation}\label{equacaoauxiliar2}
            0_V = T(u_1) + T(u_2) + \cdots + T(u_r) = \lambda_1u_1 + \lambda_2u_2 + \cdots + \lambda_ru_r.
        \end{equation}
        Agora multiplicando \eqref{equacaoauxiliar1} por $\lambda_1$ e subtraindo de \eqref{equacaoauxiliar2} obtemos
        \begin{align*}
            \lambda_1u_2 + \lambda_1u_r - \lambda_2u_2 - \cdots - \lambda_ru_r = 0_V\\
            (\lambda_1 - \lambda_2)u_2 - \cdots - (\lambda_1 - \lambda_r)u_r = 0_V.
        \end{align*}
        Mas por hip\'otese de indu\c{c}\~ao, segue que $(\lambda_1 - \lambda_i)u_i = 0$ para $i = 2$, \dots, $r$. Como $\lambda_i \ne \lambda_j$ se $i \ne j$, ent\~ao $\lambda_1 - \lambda_i \ne 0_\cp{K}$ e ent\~ao $u_i = 0_V$ para $i = 2$, \dots, $r$. Logo $u_1 = 0_V$ e o resultado est\'a provado.
        \item Para cada $i$, seja $\mathcal{B}_i = \{u_{i1}, \dots, u_{in_i}\}$. Vamos mostrar que o subconjunto de $V$ dado por $\mathcal{B} = \{u_{11}, \dots, u_{1n_1}, u_{21}, \dots, u_{2n_2}, \dots, u_{r1}, \dots, u_{rn_r}\}$ \'e L.I. em $V$. Para isso sejam $\alpha_{in_i} \in \cp{K}$, $i = 1$, \dots, $r$ tais que
        \[
            \alpha_{11}u_{11} + \cdots + \alpha_{1n_1}u_{1n_1} + \cdots + \alpha_{r1}u_{r1} + \cdots + \alpha_{rn_r}u_{rn_r} = 0_V.
        \]
        Mas
        \[
            \sum_{j = 1}^{n_i}\alpha_{ij}u_{ij} \in \aut_T(\lambda_i)
        \]
        para $i = 1$, \dots, $r$. Da{\'\i} segue do item \ref{autovetorLI} que
        \[
            \sum_{j = 1}^{n_i}\alpha_{ij}u_{ij} = 0_V
        \]
        para $i = 1$, \dots, $r$. Como $\mathcal{B}_i$ \'e L.I. para $i = 1$, \dots, $r$, ent\~ao $\alpha_{ij} = 0_\cp{K}$ para $i = 1$, \dots, $r$ e $j = 1$, \dots, $n_i$. Portanto $\mathcal{B}_1 \cup \cdots \cup \mathcal{B}_r$ \'e L.I. em $V$.
    \end{enumerate}
\end{prova}

\begin{corolario}
    Seja $T : V \to V$ um operador linear, onde $V$ \'e um $\cp{K}$-espa\c{c}o vetorial de dimens\~ao finita. Se $\lambda_1$, \dots, $\lambda_r$ s\~ao todos os autovalores de $T$, ent\~ao $T$ \'e diagonaliz\'avel se, e somente se,
    \[
        \dim_\cp{K} V = \sum_{i = i}^r\dim_\cp{K}\aut_T(\lambda_i).
    \]
\end{corolario}

\begin{definicao}
    Seja $\lambda$ um autovalor de um operador linear $T : V \to V$ onde $V$ \'e um $\cp{K}$-espa\c{c}o vetorial de dimens\~ao finita e suponhamos que
    \[
        p_T(x) = (x - \lambda)^mq(x)
    \]
    com $q(\lambda) \ne 0$, seja o polin\^omio caracter{\'\i}stico de $T$.
    \begin{enumerate}[label={\roman*})]
        \item O n\'umero $m$ \'e chamada de \textbf{multiplicidade alg\'ebrica} de $\lambda$ e o denotamos por $ma(\lambda)$.\index{Multiplicidade!Alg\'ebrica}
        \item Chamamos de \textbf{multiplicidade geom\'etrica} de $\lambda$ \`a dimens\~ao do subespa\c{c}o $\aut_T(\lambda)$ e indicamos tal n\'umero por $mg(\lambda)$.\index{Multiplicidade!Geom\'etrica}
    \end{enumerate}
\end{definicao}

\begin{observacao}
    A multiplicidade alg\'ebrica de um autovalor $\lambda$ \'e o maior {\'\i}ndice $j$ tal que
    \[
        p_T(x) = (x - \lambda)^jq(x)
    \]
    com $q(\lambda) \ne 0$;
\end{observacao}

\begin{exemplo}
    \begin{enumerate}[label={\arabic*})]
        \item $p_T(x) = (x - 2)(x^2 - 5x + 6)$, $ma(2) = 2$
        \item $p_T(x) = (x + 1)^3(x - 2)$, $ma(2) = 1$, $ma(-1) = 3$.
    \end{enumerate}
\end{exemplo}

\begin{proposicao}
    Seja $\lambda$ um autovalor de $T : V \to V$, onde $V$ \'e um $\cp{K}$-espa\c{c}o vetorial de dimens\~ao finita. Ent\~ao $mg(\lambda) \le ma(\lambda)$.
\end{proposicao}
\begin{prova}
    Seja $W = \aut_T(\lambda)$ e assuma que $\dim_\cp{K}W = r$. Sejam $\mathcal{B}_W = \{w_1,\dots,w_r\}$ uma base de $W$ e $\mathcal{B}_V = \{w_1,\dots,w_r,u_{r + 1},\dots,u_n\}$ uma base de $V$ contendo $\mathcal{B}_W$. Como $T(w_i) = \lambda w_i$ para $i = 1$, \dots, $r$; podemos escrever $[T]_{\mathcal{B}_V}$ na forma
    \[
        [T]_{\mathcal{B}_V} = \begin{bmatrix}
            \lambda & 0_\cp{K} & \dots & 0_\cp{K} & a_{1r+1} & \dots & a_{1n}\\
            0_\cp{K} & \lambda &  \dots & 0_\cp{K} & a_{2r+1} & \dots & a_{2n}\\
            \vdots\\
            0_\cp{K} & 0_\cp{K} & \dots & \lambda & a_{rr+1} & \dots & a_{rn}\\
            0_\cp{K} & 0_\cp{K} & \dots & 0_\cp{K} & b_{r+1r+1} & \dots & b_{r+1n}\\
            \vdots\\
            0_\cp{K} & 0_\cp{K} & \dots & 0_\cp{K} & b_{nr+1} & \dots & b_{nn}\\
        \end{bmatrix}.
    \]
    Assim
    \[
        p_T(x) = \det([T - xId]_{\mathcal{B}_V}) = (x - \lambda)^r\det(A_2).
    \]
    Por defini\c{c}\~ao, $ma(\lambda)$ \'e o maior {\'\i}ndice $j$ tal que $(x - \lambda)^j$ divide $p_T(x)$. Portanto, $mg(\lambda) = r \le ma(\lambda)$.
\end{prova}

Seja $T : V \to V$ um operador linear onde $V$ \'e um $\cp{K}$-espa\c{c}o vetorial de dimens\~ao finita. Suponha que $p_T(x) = (x - \lambda_1)^{n_1}\dots(x - \lambda_r)^{n_r}$, onde $\lambda_1$, \dots, $\lambda_r \in \cp{K}$ s\~ao distintos. Da defini\c{c}\~ao de $p_T(x)$ temos que
\[
    \dim_\cp{K}V = n_1 + \cdots + n_r.
\]
Assim, pela proposi\c{c}\~ao anterior,
\[
    \dim_\cp{K}V = \sum_{i = 1}^r \dim_\cp{K}\aut_T(\lambda_i)
\]
se, e somente se, $mg(\lambda_i) = ma(\lambda_i)$ para $i = 1$, \dots, $r$. Assim temos o seguinte teorema:
\begin{teorema}
    Seja $T : V \to V$ um operador linear onde $V$ \'e um $\cp{K}$-espa\c{c}o vetorial de dimens\~ao finita e sejam $\lambda_1$, \dots, $\lambda_r \in \cp{K}$ seus autovalores distintos. As seguintes afirma\c{c}\~oes s\~ao equivalentes:
    \begin{enumerate}[label={\roman*})]
        \item $T$ \'e diagonaliz\'avel.
        \item $p_T(x) = (x - \lambda_1)^{n_1}\dots(x - \lambda_r)^{n_r}$, $n_i \ge 1$ e $mg(\lambda_i) = ma(\lambda_i)$ para cada $i = 1$, \dots, $r$.
        \item $\dim_\cp{K}V = \displaystyle\sum_{i = 1}^r \dim_\cp{K}\aut_T(\lambda_i)$.
    \end{enumerate}
\end{teorema}
% section operadores_diagonaliz\'aveis (end)

\section{Subespa\c{c}os T-invariantes} % (fold)
\label{sec:subespacos_T-invariantes}
Seja $T : \real^3 \to \real^3$ o operador tal que
        \[
    [T]_\mathcal{A} = \begin{bmatrix}
            \phantom{-}1 & \phantom{-}2 & -1\\
            -2 & -3 & \phantom{-}1\\
            \phantom{-}2 & \phantom{-}2 & -2
        \end{bmatrix}
        \]
onde $\mathcal{A}$ \'e uma base qualquer de $\real^3$. Sabemos que $T$ \'e diagonaliz\'avel e os autoespa\c{c}os de $T$ s\~ao
\begin{align*}
    \aut_T(-1) = [(1,0,2); (0,1,2)]\\
    \aut_T(-2) = [(1,-1,1)].
\end{align*}
Seja $u \in \aut_T(-1)$. Assim existem $\alpha$, $\beta \in \real$ tais que
\[
    u = \alpha(1,0,2) + \beta(0,1,2)
\]
e da{\'\i}
\[
    T(u) = \alpha T(1,0,2) + \beta T(0,1,2) = -\alpha(1,0,2) - \beta(0,1,2) \in \aut_T(-1).
\]
Logo, para todo $u \in \aut_T(-1)$, $T(u) \in \aut_T(-1)$. Em outras palavras
\[
    T(\aut_T(-1)) \sub \aut_T(-1).
\]
Analogamente, para todo $u \in \aut_T(-2)$, $T(u) \in \aut_T(-2)$. Em outras palavras
\[
    T(\aut_T(-2)) \sub \aut_T(-2).
\]
Agora, seja $W = [(1,0,0)]$. Primeiramente, podemos escrever
\[
    (1,0,0) = \alpha(1,0,2) + \beta(0,1,2) + \gamma(1,-1,1)
\]
tomando $\alpha = 3$ e $\beta = \gamma = -2$. Da{\'\i}
\[
    T(1,0,0) = (1,-2,2) \notin W
\]
e ent\~ao $T(W) \varsubsetneq W$.

\begin{definicao}
    Seja $T : V \to V$ um operador linear onde $V$ \'e um $\cp{K}$-espa\c{c}o vetorial e seja $W \sub V$ um subespa\c{c}o de $V$. Dizemos que $W$ \'e um \textbf{subespa\c{c}o $T$-invariante} de $V$ se $T(W) \sub W$, isto \'e, $T(u) \in W$ para todo $u \in W$.\index{Subespa\c{c}o!$T$-invariante}
\end{definicao}

\begin{exemplo}
Seja $T : V \to V$ um operador linear onde $V$ \'e um $\cp{K}$-espa\c{c}o vetorial.
    \begin{enumerate}[label={\arabic*})]
        \item Os subespa\c{c}os triviais de $V$ s\~ao $T$-invariantes.
        \item Os subespa\c{c}os $\ker T$ e $\im T$ s\~ao $T$-invariantes. De fato, se $u \in \ker T$, ent\~ao $T(u) = 0_V \in \ker T$. Assim $T(\ker T) \sub \ker T$. Agora, se $w \in \im T$, ent\~ao existe $u \in V$ tal que $T(u) = w$. Assim $T(w) = T(T(u))$, logo $T(w) \in \im T$ para todo $w \in \im T$.
        \item Se $\lambda$ for um autovalor de $T$, ent\~ao $\aut_T(\lambda)$ \'e um subespa\c{c}o $T$-invariante.
        \item Se $W$ \'e um subespa\c{c}o $T$-invariante, ent\~ao $T : W \to W$ \'e um operador linear.
        \item Seja $T : \real^2 \to \real^2$ um operador linear cuja matriz em rela\c{c}\~ao \`a base can\^onica $\mathcal{B}$ de $\real^2$ \'e dada por
        \[
            [T]_\mathcal{B} = \begin{bmatrix}
                0 & -1\\
                1 & \phantom{-}0
            \end{bmatrix}.
        \]
        Ent\~ao os \'unicos subespa\c{c}os $T$-invariantes s\~ao os triviais. De fato, qualquer outro espa\c{c}o $T$-invariante teria dimens\~ao 1, isto \'e, se $W$ \'e um subespa\c{c}o de $\real^2$, $T$-invariante ent\~ao $W = [v]$. Da{\'\i} $v$ seria um autovetor de $T$. Mas
        \[
            p_T(x) = x^2 + 1
        \]
        que n\~ao possui ra{\'\i}zes em $\real$. Logo, $T$ n\~ao possui subespa\c{c}o $T$-invariante n\~ao trivial.
    \end{enumerate}
\end{exemplo}

\begin{definicao}
    Sejam $W_1$ e $W_2$ dois subespa\c{c}os vetoriais de um $\cp{K}$-espa\c{c}o vetorial $V$. Dizemos que $W_1 + W_2$ \'e uma \textbf{soma direta} se $W_1 \cap W_2 = \{0_V\}$. Neste caso escreveremos $W_1 \oplus W_2$.\index{Soma direta}
\end{definicao}

\begin{exemplo}
    \begin{enumerate}[label={\arabic*})]
        \item Sejam $W_1$ e $W_2$ dois subespa\c{c}os de $\complex^4$ com bases $\{(1,2,0,i);(i,0,0,1)\}$ e $\{(0,0,3,1)\}$, respectivamente. Seja $(z_1,z_2,z_3,z_4) \in W_1 \cap W_2$. Temos
        \[
            (z_1,z_2,z_3,z_4) = \alpha(1,2,0,i)    + \beta(i,0,0,1) = \gamma(0,0,3,1)
        \]
        donde $\alpha = \beta = \gamma = 0$. Logo $W_1 \cap W_2 = \{(0,0,0,0)\}$ e portanto $W_1 + W_2$ \'e uma soma direta e escrevemos $W_1 \oplus W_2$.
        \item Sejam $W_1 = [(0,1)]$ e $W_2 = [(1,1)]$ subespa\c{c}os de $\real^2$. Temos que se $(x,y) \in W_1 \cap W_2$, ent\~ao
        \[
            (x,y) = \alpha(0,1) = \beta(1,1)
        \]
        e da{\'\i} $\alpha = \beta = 0$. Logo, $W_1 \cap W_2 =\{(0,0)\}$ e ent\~ao $W_1 + W_2$ \'e uma soma direta. Mais ainda
        \[
            \dim_\real(W_1 \oplus W_2) = \dim_\real W_1 + \dim_\real W_2 = 2
        \]
        e portanto,
        \[
            W_1 \oplus W_2 = \real^2.
        \]
    \end{enumerate}
\end{exemplo}

\begin{definicao}
    Seja $V$ um $\cp{K}$-espa\c{c}o vetorial e sejam $W_1$ e $W_2$ dois subespa\c{c}os de $V$. Dizemos que $V$ \'e a \textbf{soma direta} de $W_1$ e $W_2$ se
    \begin{enumerate}[label={\roman*})]
        \item $W_1 \cap W_2 = \{0_V\}$;
        \item $W_1 + W_2 = V$.
    \end{enumerate}
    Neste caso escrevemos
    \[
        V = W_1 \oplus W_2.
    \]
\end{definicao}

\begin{proposicao}
    Sejam $V$ um $\cp{K}$-espa\c{c}o vetorial e $W_1$, $W_2$ dois subespa\c{c}os de $V$. Ent\~ao $V = W_1 \oplus W_2$ se, e s\'o, se
    cada elemento $u \in V$ se escreve de maneira \'unica como uma soma $x_1 + x_2$, onde $x_1 \in W_1$ e $x_2 \in W_2$.
\end{proposicao}
\begin{prova}
    \begin{itemize}
        \item[($\Rightarrow$)] Vamos supor que $V = W_1 \oplus W_2.$ Segue ent\~ao que cada elemento $u \in V$ se escreve
        como soma de um elemento de $W_1$ com um elemento de $W_2$. Suponha agora que $u = x_1 + x_2 = y_1 + y_2$ onde
        $x_1$, $y_1 \in W_1$ e $x_2$, $y_2 \in W_2$. Da{\'\i}
        \[
            x_1 - y_1 = y_2 - x_2 \in W_1 \cap W_2
        \]
        pois $x_1 - y_1 \in W_1$ e $y_2 - x_2 \in W_2$. Mas $W_1 \cap W_2 = \{0_V\}$, logo $x_1 = y_1$ e $x_2 = y_2$,
        como quer{\'\i}amos.
        \item[($\Leftarrow$)] Como cada elemento de $V$ \'e uma soma de elementos de $W_1$ com elementos de $W_2$, logo
        $V = W_1 + W_2$. Seja $u \in W_1 + W_2$ com $u \ne 0_V$. Assim como $0_V \in W_1$ temos
        \[
            u = 0_V + u
        \]
        considerando $u \in W_2$. Por outro lado, $0_V \in W_2$, da{\'\i}
        \[
            u = u + 0_V
        \]
        considerando $u \in W_1$. Logo $u$ pode ser escrito de duas maneiras distintas, o que contradiz nossa hip\'otese.
        Logo $u = 0_V$, isto \'e, $W_1 \cap W_2 = \{0_V\}$. Portanto, $V = W_1 \oplus W_2$.
    \end{itemize}
\end{prova}

\begin{proposicao}
    Sejam $V$ um $\cp{K}$-espa\c{c}o vetorial n\~ao nulo e de dimens\~ao finita e $W_1$ um subespa\c{c}o n\~ao nulo de $V$. Ent\~ao
    existe um subespa\c{c}o $W_2$ de $V$ tal que $V = W_1 \oplus W_2$.
\end{proposicao}
\begin{prova}
    Se $V = W_1$ n\~ao h\'a nada a fazer, pois basta escolher $W_2 = \{0_V\}$. Suponha ent\~ao que $V \ne W_1$. Seja
    $\mathcal{B}_1 = \{w_1, \dots, w_m\}$ uma base de $W_1$. Sabemos que podemos estender $\mathcal{B}_1$ a uma base
    $\mathcal{B}$ de $V$. Seja $\mathcal{B} = \{w_1, \dots, w_m, u_1, \dots, u_n\}$ uma base de $V$ contendo
    $\mathcal{B}_1$. Defina
    \[
        W_2 = [u_1, \dots, u_n]
    \]
    o subespa\c{c}o gerado por $u_1$, \dots, $u_n$. Como $\mathcal{B}$ gera $V$, ent\~ao $V = W_1 + W_2$. Seja $v \in W_1 \cap
    W_2$. Ent\~ao existem escalares $\alpha_1$, \dots, $\alpha_m$, $\beta_1$, \dots, $\beta_n$ tais que
    \begin{align*}
        v = \alpha_1w_1 + \cdots + \alpha_mw_m\\
        v = \beta_1u_1 + \cdots + \beta_nu_n,
    \end{align*}
    isto \'e,
    \[
        \alpha_1w_1 + \cdots + \alpha_mw_m - \beta_1u_1 - \cdots - \beta_nu_n = 0_V
    \]
    e ent\~ao $\alpha_1 = \dots = \alpha_m = \beta_1 = \dots = \beta_n = 0_\cp{K}$ pois $\mathcal{B}$ \'e L.I.. Assim $W_1
    \cap W_2 = \{0_V\}$ e portanto $V = W_1 \oplus W_2$.
\end{prova}

\begin{exemplo}
    \begin{enumerate}[label={\arabic*})]
        \item Sejam $V = \real^3$ e $W_1 = [(1,0,0)]$. Ent\~ao podemos tomar $W_2 = [(0,1,0); (0,0,1)]$ e teremos $V = W_1 \oplus W_2$. Tamb\'em podemos tomar $W_3 = [(1,1,1); (0,0,1)]$ e assim $V = W_1
        \oplus W_3$.
        \item Sejam $V = \cp{M}_2(\real)$ e $W_1 = \left\{\begin{bmatrix}
            a & b\\ 0 & 0
        \end{bmatrix} \mid a,\ b \in \real\right\}$. Tomando
        \[
            W_2 = \left\{\begin{bmatrix}
            0 & 0\\ c & d
        \end{bmatrix} \mid c,\ d \in \real\right\}
        \]
        temos $V = W_1 \oplus W_2$.
    \end{enumerate}
\end{exemplo}

% section subespacos_T-invariantes (end)

\section{Polin\^omio Minimal} % (fold)
\label{sec:polinomio_minimal}

Seja $T : V \to V$ um operador linear sobre um $\cp{K}$-espa\c{c}o vetorial de dimens\~ao $n \ge 1$. Para cada $i \ge 0$ se definirmos
\[
    T^i = \begin{cases}
        \underbrace{T\circ T \circ \dots \circ T}_{i \mbox{ vezes}}, & i \ge 1\\
        Id, & i = 0,
    \end{cases}
\]
ent\~ao $T^i \in \mathcal{L}(V,V)$. Mas, $\dim_\cp{K} \mathcal{L}(V,V) = n^2$, assim existe $r \ge 1$ tal que o conjunto $\{T^0, T, T^2, \dots, T^{r - 1}\}$ \'e L.I., mas $\{T^0, T, T^2, \dots, T^r\}$ \'e L.D.. Logo existem escalares $a_0$, $a_1$, \dots, $a_{r - 1}$ tais que
\[
    T^r = a_0T^0 + a_1T^1 + \cdots + a_{r - 1}T^{r - 1},
\]
ou seja,
\[
    T^r = \sum_{i = 0}^{r - 1}a_iT^i.
\]
Assim
\[
    T^r(u) = \sum_{i = 0}^{r - 1}a_iT^i(u)
\]
para todo $u \in V$.

Defina
\[
    m_T(x) = x^r - a_{r - 1}x^{r - 1} - \cdots - a_1x - a_0.
\]
Do exposto anteriormente segue que
\[
    m_T(T(u)) = 0_V
\]
para todo $u \in V$.

\begin{definicao}
    O \textbf{polin\^omio minimal} de um operador linear $T$ em $\mathcal{L}(V,V)$ \'e o polin\^omio m\^onico $m_T(x)$ de menor grau tal que $m_T(T(u)) = 0_V$ para todo $u \in V$. Assim, se o grau de $m_T(x)$ \'e $r$, ent\~ao o coeficiente de $x^r$ \'e 1.\index{Polin\^omio!Minimal}
\end{definicao}

\begin{exemplo}
    Seja $T \in \mathcal{L}(\complex^3, \complex^3)$ dado por
    \[
        T(a, b, c) = (a, a + b, c).
    \]
    Temos $T \ne \lambda Id$ e
    \begin{align*}
        T^2(a,b,c) = T(T(a,b,c)) = T(a,a+b,c) = (a,2a+b,c) = 2(a,a+b,c) - (a,b,c)
    \end{align*}
    isto \'e,
    \[
        T^2(a,b,c) = 2T(a,b,c) - Id(a,b,c).
    \]
    Assim o polin\^omio minimal de $T$ \'e
    \[
        m_T(x) = (x - 1)^2.
    \]
\end{exemplo}

\begin{teorema}
    Seja $T \in \mathcal{L}(V,V)$, onde $\dim_\cp{K} V < \infty$. Os polin\^omios caracter{\'\i}stico e minimal de $T$ possuem exatamente as mesmas ra{\'\i}zes, a menos de multiplicidade.
\end{teorema}

\begin{exemplo}
    Seja $T$ o operador sobre $\real^n$ representado em rela\c{c}\~ao \`a base can\^onica pela matriz $A$ dada. Encontre o polin\^omio minimal de $T$.
    \begin{enumerate}[label={\arabic*})]
        \item $A = \begin{bmatrix}
            \phantom{-}5 & -6 & -6\\
            -1 & \phantom{-}4 & \phantom{-}2\\
            \phantom{-}3 & -6 & -4
        \end{bmatrix}$, $n = 3$
        \begin{solucao}
            O polin\^omio caracter{\'\i}stico de $T$ \'e
            \[
                p_T(x) = (x - 1)(x - 2)^2.
            \]
            Assim os poss{\'\i}veis candidatos a polin\^omio minimal s\~ao
            \[
                (x - 1)(x - 2), \ (x -1)(x - 2)^2.
            \]
            Temos
            \[
                A - I_3 = \begin{bmatrix}
                    \phantom{-}4 & -6 & -6\\
                    -1 & \phantom{-}3 & \phantom{-}2\\
                    \phantom{-}3 & -6 & -5
                \end{bmatrix},
                A - 2I_3 = \begin{bmatrix}
                    \phantom{-}3 & -6 & -6\\
                    -1 & \phantom{-}2 & \phantom{-}2\\
                    \phantom{-}3 & -6 & -6
                \end{bmatrix}
            \]
            e assim
            \[
                (A - I_3)(A - 2I_3) = [0]_{3 \times 3}.
            \]
            Logo o polin\^omio minimal de $T$ \'e
            \[
                m_T(x) = (x - 1)(x - 2).
            \]
            Note que $p_T(x) = m_T(x)(x - 2)$ e assim $p_T(A) = [0]_{3 \times 3}$.
        \end{solucao}
        \item $A = \begin{bmatrix}
            2 & 1 & \phantom{-}0 & 0\\
            0 & 2 & \phantom{-}0 & 0\\
            0 & 0 & \phantom{-}1 & 1\\
            0 & 0 & -2 & 4
        \end{bmatrix}$, $n = 4$.
        \begin{solucao}
            O polin\^omio caracter{\'\i}stico de $T$ \'e
            \[
                p_T(x) = (x - 3)(x - 2)^3.
            \]
            Assim os poss{\'\i}veis candidatos a polin\^omio minimal s\~ao
            \[
                f(x) = (x - 3)(x - 2), \ g(x) = (x -3)(x - 2)^2, h(x) = (x - 3)(x - 2)^3.
            \]
            Temos
            \[
                f(A) = (A - 3I_4)(A - 2I_4) = \begin{bmatrix}
                    -1 & \phantom{-}1 & \phantom{-}0 & 0\\
                    \phantom{-}0 & -1 & \phantom{-}0 & 0\\
                    \phantom{-}0 & \phantom{-}0 & -2 & 1\\
                    \phantom{-}0 & \phantom{-}0 & -2 & 1
                \end{bmatrix}
                \begin{bmatrix}
                    0 & 1 & \phantom{-}0 & 0\\
                    0 & 0 & \phantom{-}0 & 0\\
                    0 & 0 & -1 & 1\\
                    0 & 0 & -2 & 2
                \end{bmatrix} = \begin{bmatrix}
                    0 & 1 & 0 & 0\\
                    0 & 0 & 0 & 0\\
                    0 & 0 & 0 & 0\\
                    0 & 0 & 0 & 0
                \end{bmatrix}.
            \]
            Agora
            \[
                g(A) = (A - 3I_4)(A - 2I_4)^2 = \begin{bmatrix}
                    -1 & \phantom{-}1 & \phantom{-}0 & 0\\
                    \phantom{-}0 & -1 & \phantom{-}0 & 0\\
                    \phantom{-}0 & \phantom{-}0 & -2 & 1\\
                    \phantom{-}0 & \phantom{-}0 & -2 & 1
                \end{bmatrix}
                \begin{bmatrix}
                    0 & 1 & \phantom{-}0 & 0\\
                    0 & 0 & \phantom{-}0 & 0\\
                    0 & 0 & -1 & 1\\
                    0 & 0 & -2 & 2
                \end{bmatrix}^2 = \begin{bmatrix}
                    0 & 0 & 0 & 0\\
                    0 & 0 & 0 & 0\\
                    0 & 0 & 0 & 0\\
                    0 & 0 & 0 & 0
                \end{bmatrix}.
            \]
            Logo o polin\^omio minimal de $T$ \'e
            \[
                m_T(x) = (x - 3)(x - 2)^2.
            \]
            Note que $p_T(x) = m_T(x)(x - 2)$ e assim $p_T(A) = [0]_{4 \times 4}$.
        \end{solucao}
        \item $A = \begin{bmatrix}
            \lambda & a\\
            0 & \lambda
        \end{bmatrix}$ onde $a \ne 0$, $n = 2$.
        \begin{solucao}
            O polin\^omio caracter{\'\i}stico de $T$ \'e
            \[
                p_T(x) = (x - \lambda)^2.
            \]
            Assim o polin\^omio minimal ser\'a da forma
            \[
                (x - \lambda),\ (x - \lambda)^2;
            \]
            Como $T \ne \lambda Id$, ent\~ao segue que $m_T(x) = p_T(x)$ e da{\'\i} $p_T(A) = [0]_{2 \times 2}$.
        \end{solucao}
    \end{enumerate}
\end{exemplo}

\begin{teorema}[Cayley-Hamilton]\label{TeoremaCayley-Hamilton}
    Seja $T$ um operador linear sobre um $\cp{K}$-espa\c{c}o vetorial $V$ de dimens\~ao finita. Se $p_T(x)$ \'e o polin\^omio caracter{\'\i}stico de $T$, ent\~ao $p_T(T(u)) = 0_V$ para todo $u \in V$. Em particular, o polin\^omio caracter{\'\i}stico $p_T(x)$ \'e um m\'ultiplo do polin\^omio minimal $m_T(x)$ de $T$.
\end{teorema}

Seja $T : \cp{M}_2(\real) \to \cp{M}_2(\real)$ o operador linear dado por
\[
    T \begin{bmatrix}
        a & b\\
        c & d
    \end{bmatrix} = \begin{bmatrix}
        d & c\\
        0 & a
    \end{bmatrix}.
\]
Considere tamb\'em os seguintes subespa\c{c}os de $\cp{M}_2(\real)$:
\begin{align*}
    W_1 = \left[e_1 = \begin{bmatrix}
        0 & 1\\
        0 & 0
    \end{bmatrix}; e_2 = \begin{bmatrix}
        0 & 0\\
        1 & 0
    \end{bmatrix}\right]\\
    W_2 = \left[e_3 = \begin{bmatrix}
        1 & 0\\
        0 & 0
    \end{bmatrix}; e_4 = \begin{bmatrix}
        0 & 0\\
        0 & 1
    \end{bmatrix}\right].
\end{align*}

Temos
\[
    T(e_1) = \begin{bmatrix}
        0 & 0\\
        0 & 0
    \end{bmatrix} \in W_1,
    T(e_2) = \begin{bmatrix}
        0 & 1\\
        0 & 0
    \end{bmatrix} \in W_1
\]
assim $W_1$ \'e um subespa\c{c}o $T$-invariante. Seja ent\~ao $T_1 = T : W_1 \to W_1$ e $\mathcal{B}_1 = \{e_1, e_2\}$ uma base de $W_1$. Temos
\[
    [T_1]_{\mathcal{B}_1} = \begin{bmatrix}
        0 & 1\\
        0 & 0
    \end{bmatrix}.
\]
Agora,
\[
    T(e_3) = \begin{bmatrix}
        0 & 0\\
        0 & 1
    \end{bmatrix} \in W_2,
    T(e_4) = \begin{bmatrix}
        1 & 0\\
        0 & 0
    \end{bmatrix} \in W_2
\]
assim $W_2$ \'e um subespa\c{c}o $T$-invariante. Seja ent\~ao $T_2 = T : W_2 \to W_2$ e $\mathcal{B}_2 = \{e_3, e_4\}$ uma base de $W_2$. Temos
\[
    [T_2]_{\mathcal{B}_2} = \begin{bmatrix}
        0 & 1\\
        1 & 0
    \end{bmatrix}.
\]
Al\'em disso, \'e f\'acil ver que $\cp{M}_2(\real) = W_1 \oplus W_2$ e assim $\mathcal{B} = \mathcal{B}_1 \cup \mathcal{B}_2$ \'e uma base de $\cp{M}_2(\real)$. Assim
\[
    [T]_\mathcal{B} = \begin{bmatrix}
        0 & 1 & 0 & 0\\
        0 & 0 & 0 & 0\\
        0 & 0 & 0 & 1\\
        0 & 0 & 1 & 0
    \end{bmatrix} = \begin{bmatrix}
        [T_1]_{\mathcal{B}_1} & 0\\
        0 & [T_2]_{\mathcal{B}_2}
    \end{bmatrix}.
\]

Neste caso dizemos que $T$ \'e a \textbf{soma direta} dos operadores $T_1$ e $T_2$ e escrevemos
\[
    T = T_1 \oplus T_2.
\]

Al\'em disso temos
\begin{align*}
    T_1^2 (e_1) = T_1(T(e_1)) = \begin{bmatrix}
        0 & 0\\
        0 & 0
    \end{bmatrix}\\
    T_1^2 (e_2) = T_1(T(e_2)) = \begin{bmatrix}
        0 & 0\\
        0 & 0
    \end{bmatrix}
\end{align*}
assim $T_1^2 = 0$.

Note tamb\'em que $T_2$ \'e invert{\'\i}vel pois leva uma base de $W_2$ em uma base de $W_2$. Desse modo o operador $T$ pode ser escrito como a soma direta
\[
    T = T_1 \oplus T_2
\]
onde $T_1^2 = 0$ e $T_2$ \'e invert{\'\i}vel. O operador $T_1$ \'e chamado de \textbf{nilpotente} de {\'\i}ndice de nilpot\^encia 2.

\begin{definicao}
    Seja $V = W_1 \oplus W_2 \oplus \cdots \oplus W_r$ onde $\dim_\cp{K} V < \infty$. Seja $T : V \to V$ um operador linear e suponha que $W_i$ seja $T$-invariante para $i = 1$, \dots, $r$. Sejam $\mathcal{B}_1$, \dots, $\mathcal{B}_r$ bases de $W_1$, \dots, $W_r$, respectivamente. Como $\mathcal{B} = \mathcal{B}_1 \cup \dots \cup \mathcal{B}_r $ \'e uma base de $V$ ent\~ao
    \[
        [T]_\mathcal{B} = \begin{bmatrix}
        [T_1]_{\mathcal{B}_1} & 0 & 0 & \dots & 0\\
        0 & [T_2]_{\mathcal{B}_2} & 0 & \dots & 0\\
        \vdots\\
        0 & 0 & 0 & \dots & [T_r]_{\mathcal{B}_r}
    \end{bmatrix}
    \]
    onde $T_i = T : W_i \to W_i$, $i = 1$, \dots, $r$. Neste caso escrevemos
    \[
        T = T_1 \oplus T_2 \oplus \cdots \oplus T_r
    \]
    e dizemos que $T$ \'e a \textbf{soma direta dos operadores} $T_1$, \dots, $T_r$.
\end{definicao}

\begin{definicao}
    Um operador linear $T : V \to V$ \'e chamado de \textbf{nilpotente} se existir um $r > 0$ tal que $T^r = 0$. O \textbf{{\'\i}ndice de nilpot\^encia} de um operador nilpotente ser\'a o menor inteiro $i$ tal que $T^i = 0$.\index{Operador Linear!Nilpotente}
\end{definicao}

\begin{observacao}
    Se $T : V \to V$ \'e um operador linear nilpotente, ent\~ao $\ker T \ne \{0_V\}$. De fato, se $T$ \'e nilpotente de {\'\i}ndice $i \ge 1$, ent\~ao existe $u \in V$ tal que $T^i(u) = 0_V$ e $T^{i - 1}(u) \ne 0_V$. Assim
    \[
        0_V = T^i(u) = T(T^{i - 1}(u)),
    \]
    isto \'e, $T^{i - 1}(u) \in \ker T$.
\end{observacao}

\begin{exemplo}
    \begin{enumerate}[label={\arabic*})]
        \item Seja $D : \mathcal{P}_3(\real) \to \mathcal{P}_3(\real)$ o operador deriva\c{c}\~ao. \'E f\'acil ver que $D$ \'e nilpotente de {\'\i}ndice de nilpot\^enica 4.
        \item Seja $D : \mathcal{P}_n(\real) \to \mathcal{P}_n(\real)$ o operador deriva\c{c}\~ao. \'E f\'acil ver que $D$ \'e nilpotente de {\'\i}ndice de nilpot\^enica $n + 1$.
        \item Seja $T : \real^2 \to \real^2$ o operador linear tal que
        \[
            [T] = \begin{bmatrix}
                0 & 0\\
                1 & 0
            \end{bmatrix}.
        \]
        \'E f\'acil ver que $T$ \'e nilpotente de {\'\i}ndice de nilpot\^encia 2.
    \end{enumerate}
\end{exemplo}

\begin{teorema}\label{decomposicaonilpotente}
    Seja $T : V \to V$ um operador linear, onde $V$ \'e um $\cp{K}$-espa\c{c}o vetorial de dimens\~ao finita. Ent\~ao $T$ \'e a soma direta de um operador nilpotente e um operador invert{\'\i}vel. Al\'em disso, tal decomposi\c{c}\~ao \'e essencialmente \'unica.
\end{teorema}

\begin{observacao}
    O Teorema \ref{decomposicaonilpotente} n\~ao vale para espa\c{c}os vetoriais de dimens\~ao infinita. Por exemplo, seja $T : \mathcal{P}(\real) \to \mathcal{P}(\real)$ o operador linear dado por $T(p(x)) = xp(x)$. Suponha que  $T = T_1 \oplus T_2$, onde $T_1$ \'e nilpotente e $T_2$ \'e invert{\'\i}vel. Primeiro note que para todo $l \ge 1$, $\ker T^l = \{0\}$, logo $T$ n\~ao \'e nilpotente. Assim $T_2 \ne 0$. Seja $W_2$ um subespa\c{c}o $T$-invariante tal que $T_2 = T : W_2 \to W_2$ seja invert{\'\i}vel. Logo $T_2$ \'e sobrejetora. Tome $q(x) \in W_2$ um polin\^omio m\^onico de menor grau poss{\'\i}vel. Como $T_2$ \'e sobrejetora, ent\~ao existe $p(x) \in W_2$ tal que
    \[
        xp(x) = T_2(p(x)) = q(x).
    \]
    Mas o grau de $xp(x)$ \'e maior que o grau de $q(x)$, o que \'e um absurdo. Logo $T_2$ n\~ao \'e sobrejetora, ou seja, $T$ n\~ao pode ser decomposta com uma soma de um operador nilpotente com um invert{\'\i}vel.
\end{observacao}

\begin{proposicao}\label{basenilpotente}
    Seja $T : V \to V$ um operador linear nilpotente de {\'\i}ndice de nilpot\^encia $r \ge 1$ e $V$ um $\cp{K}$-espa\c{c}o vetorial de dimens\~ao finita. Se $u \in V$ \'e tal que $T^{r - 1}(u) \ne 0_V$, ent\~ao
    \begin{enumerate}[label={\roman*})]
        \item O conjunto $\{u, T(u), \dots, T^{r - 1}(u)\}$ \'e L.I..
        \item Existe um subespa\c{c}o $T$-invariante $W$ de $V$ tal que $V = U \oplus W$, onde $U$ \'e o $\cp{K}$-espa\c{c}o vetorial dado por $U = [u,T(u),\dots, T^{r - 1}(u)]$.
    \end{enumerate}
\end{proposicao}

Seja $T : V \to V$ um operador linear, onde $V$ \'e um $\cp{K}$-espa\c{c}o vetorial de dimens\~ao finita $n \ge 1$. Suponha que $T$ seja nilpotente de {\'\i}ndice de nilpot\^encia $r \ge 1$. \'E imediato que $r \le n$. Al\'em disso, como $T^{r - 1}\ne 0$, existe $u \in V$, $u \ne 0_V$ tal que $T^{r - 1}(u) \ne 0_V$. Da{\'\i}, se $r = n$, ent\~ao pela Proposi\c{c}\~ao \ref{basenilpotente}, o conjunto $\mathcal{B} = \{u, T(u), \dots, T^{n - 1}(u)\}$ \'e uma base de $V$. Com rela\c{c}\~ao \`a essa base temos
\begin{align*}
    T(u) = 0u + 1T(u) + 0T^2(u) + \cdots + 0T^{n - 1}(u)\\
    T(T(u)) = 0u + 0T(u) + 1T^2(u) + \cdots + 0T^{n - 1}(u)\\
    \vdots\\
    T^n(u) = 0u + 0T(u) + 0T^2(u) + \cdots + 0T^{n - 1}(u)\\
\end{align*}
e assim
\[
    [T]_\mathcal{B} = \begin{bmatrix}
        0 & 0 & 0 & \cdots & 0 & 0\\
        1 & 0 & 0 & \cdots & 0 & 0\\
        0 & 1 & 0 & \cdots & 0 & 0\\
        \vdots\\
        0 & 0 & 0 & \cdots & 1 & 0
    \end{bmatrix}
\]

Al\'em disso, se o polin\^omio caracter{\'\i}stico de $T$ \'e $p_T(x) = (x - \lambda)^n$, ent\~ao pelo Teorema de Cayley-Hamilton, \ref{TeoremaCayley-Hamilton}, o operador $T - \lambda Id$ \'e nilpotente. Se o seu {\'\i}ndice de nilpot\^encia for $n$, ent\~ao existir\'a uma base $\mathcal{B}$ de $V$ tal que $[T]_\mathcal{B}$ ter\'a a forma
\[
    [T]_\mathcal{B} = \begin{bmatrix}
        \lambda & 0 & 0 & \cdots & 0 & 0\\
        1 & \lambda & 0 & \cdots & 0 & 0\\
        0 & 1 & \lambda & \cdots & 0 & 0\\
        \vdots\\
        0 & 0 & 0 & \cdots & 1 & \lambda
    \end{bmatrix}
\]

\begin{definicao}
    Um \textbf{bloco de Jordan} $r \times r$ em $\lambda$ \'e a matrix $J_r(\lambda)$ em $\cp{M}_n(\cp{K})$ que tem $\lambda$ na diagonal principal e 1 na diagonal abaixo da principal, isto \'e,
    \[
    J_r(\lambda) = \begin{bmatrix}
        \lambda & 0 & 0 & \cdots & 0 & 0\\
        1 & \lambda & 0 & \cdots & 0 & 0\\
        0 & 1 & \lambda & \cdots & 0 & 0\\
        \vdots\\
        0 & 0 & 0 & \cdots & 1 & \lambda
    \end{bmatrix}_{r \times r}.
\]\index{Jordan!Bloco de }
\end{definicao}

\begin{teorema}\label{operador_nilpotente}
    Seja $T : V \to V$ um operador linear nilpotente com {\'\i}ndice de nilpot\^encia $r \ge 1$, onde $V$ \'e um $\cp{K}$-espa\c{c}o vetorial de dimens\~ao finita. Ent\~ao existem n\'umeros positivos $p$, $m_1$, \dots, $m_p$ e vetores $u_1$, \dots, $u_p$ tais que
    \begin{enumerate}[label=({\roman*})]
        \item\label{propriedade_i_operado_nilpotente} $r = m_1 \ge m_2 \ge \cdots \ge m_p$.
        \item\label{propriedade_ii_operado_nilpotente} O conjunto $\mathcal{B} = \{u_1, T(u_1), \dots, T^{m_1 - 1}(u_1); u_2, T(u_2), \dots, T^{m_2 - 1}(u_2); \dots; u_p, T(u_p), \dots, \linebreak T^{m_p - 1}(u_p)\}$ \'e uma base de $V$.
        \item\label{propriedade_iii_operado_nilpotente} $T^{m_i}(u_i) = 0_V$ para cada $i = 1$, \dots, $p$.
        \item\label{propriedade_iv_operado_nilpotente} Se $S$ for um operador linear em um $\cp{K}$-espa\c{c}o vetorial $W$ de dimens\~ao finita, ent\~ao os inteiros $p$, $m_1$, \dots, $m_p$ associados a $S$ e a $T$ s\~ao iguais se, e somente se, existir um isomorfismo $\Phi : V \to W$ com $\Phi T \Phi^{-1} = S$.
    \end{enumerate}
\end{teorema}

\begin{teorema}\label{forma_de_jordan}
    Seja $T : V \to V$ um operador linear, onde $V$ \'e um $\cp{K}$-espa\c{c}o vetorial de dimens\~ao finita. Suponha que
    \[
        p_T(x) = (x - \lambda_1)^{m_1}\dots(x - \lambda_r)^{m_r}
    \]
    onde $m_i \ge 1$ e $\lambda_i \ne \lambda_j$ se $i \ne j$. Ent\~ao $V = W_1 \oplus \cdots \oplus W_r$ onde para cada $i = 1$, \dots, $r$ temos:
    \begin{enumerate}[label=({\roman*})]
        \item\label{propriedade_i_forma_de_jordan} $\dim_\cp{K} W_i = m_i$
        \item\label{propriedade_ii_forma_de_jordan} O subespa\c{c}o $W_i$ \'e $T-invariante$
        \item\label{propriedade_iii_forma_de_jordan} A restri\c{c}\~ao do operador $T - \lambda_i Id$ \`a $W_i$ \'e nilpotente.
    \end{enumerate}
\end{teorema}

Seja $T : V \to V$ um operador linear sobre um $\cp{K}$-espa\c{c}o vetorial de dimens\~ao finita tal que seu polin\^omio caracter{\'\i}stico seja dado por
\[
    p_T(x) = (x - \lambda_1)^{m_1}\dots(x - \lambda_1)^{m_r}
\]
com $r \ge 1$ e $\lambda_i \ne \lambda_j$ se $i \ne j$. Pelo Teorema \ref{forma_de_jordan}, existe uma decomposi\c{c}\~ao $V = W_1 \oplus \cdots \oplus W_r$ satisfazendo as propriedades \ref{propriedade_i_forma_de_jordan}, \ref{propriedade_ii_forma_de_jordan} e \ref{propriedade_iii_forma_de_jordan} de seu enunciado. Agora, para cada $i = 1$, \dots, $r$ considere $T_i = T - \lambda_i Id : W_i \to W_i$. Pelo item \ref{propriedade_iii_forma_de_jordan} do Teorema \ref{forma_de_jordan}, $T_i$ \'e nilpotente. Seja $\alpha_i$ o {\'\i}ndice de nilpot\^encia de $T_i$. Assim
\begin{align*}
    T_i^{\alpha_i} = 0\\
    (T - \lambda_i Id)^{\alpha_i} = 0
\end{align*}
e ent\~ao $T_i$ \'e raiz do polin\^omio $(x - \lambda_i)^{\alpha_i}$ para $i = 1$, \dots, $r$. Seja $f(x) = (x - \lambda_1)^{\alpha_1} (x - \lambda_2)^{\alpha_2}\dots (x - \lambda_r)^{\alpha_r}$. Pela defini\c{c}\~ao de {\'\i}ndice de nilpot\^encia, $f(x)$ \'e o polin\^omio de menor grau tal que suas ra{\'\i}zes s\~ao $T_1$, \dots, $T_r$. Logo $T$ \'e uma raiz de $f(x)$, isto \'e, $f(x)$ \'e o polin\^omio minimal de $T$. Portanto o {\'\i}ndice de nilpot\^encia de $T_i$ \'e determinado pelo polin\^omio minimal $m_T(x)$.

Como $T_i$ \'e nilpotente, pelo item \ref{propriedade_ii_operado_nilpotente} do Teorema \ref{operador_nilpotente}, existe uma base $\mathcal{B}_i$ de $W_i$ e n\'umeros $p_i$, $m_{i_1} \ge m_{i_2} \cdots \ge m_{i_{p_i}}$ tais que
\[
    [T_i]_{\mathcal{B}_i} = \begin{bmatrix}
        J_{m_{i_1}}(\lambda_i)\\
        & J_{m_{i_2}}(\lambda_i)\\
         & & \ddots\\
        & & & J_{m_{i_{p_i}}}(\lambda_i)
    \end{bmatrix}
\]
onde para cada $i = 1$, \dots, $r$ e $j = 1$, \dots, $p_i$
\[
    J_{m_{ij}}(\lambda_i) = \begin{bmatrix}
        \lambda_i & 0 & 0 & \dots & 0\\
        1 & \lambda_i & 0 & \dots & 0\\
        0 & 1 & \lambda_i & \dots & 0\\
        \vdots& & & \ddots & \vdots\\
        0 & 0 & \dots & 1 & \lambda_i
    \end{bmatrix}
\]
\'e o bloco de Jordan correspondente. Como $V = W_1 \oplus \cdots \oplus W_r$, ent\~ao $\mathcal{B} = \mathcal{B}_1 \cup \mathcal{B}_2 \cup \dots \cup \mathcal{B}_r$ \'e uma base de $V$. Em rela\c{c}\~ao \`a essa base temos
\begin{equation}\label{matriznaformadeJordan}
    [T]_\mathcal{B} = \begin{bmatrix}
        [T_1]_{\mathcal{B}_1}\\
        & [T_2]_{\mathcal{B}_2}\\
        & & \ddots\\
        & & & [T_r]_{\mathcal{B}_r}
    \end{bmatrix}.
\end{equation}
A matriz \eqref{matriznaformadeJordan} \'e chamada \textbf{forma de Jordan} associada ao operador linear $T$. Os n\'umeros $p_i$, $m_{ij}$, $i = 1$, \dots, $r$ e $j = 1$, \dots, $p_i$ s\~ao completamente determinados por $T$. Mais ainda, pelo item \ref{propriedade_iv_operado_nilpotente} do Teorema \ref{operador_nilpotente}, dois operadores lineares $S \in \mathcal{L}(V,V)$ e $T \in \mathcal{L}(W,W)$ t\^em a mesma forma de Jordan se, e somente se, existir um isomorfismo $\Phi : V \to W$ tal que $\Phi^{-1}S\Phi = T$.\index{Jordan!Forma de}

\begin{exemplo}
    \begin{enumerate}[label={\arabic*})]
        \item Seja $T : \cp{K}^7 \to \cp{K}^7$ um operador linear tal que seu polin\^omio caracter{\'\i}stico \'e $p_T(x) = (x - 2)^4(x - 3)^3$. Encontre a(s) poss{\'\i}vel(is) forma(s) de Jordan associadas a $T$.
        \begin{solucao}
            Como $p_T(x) = (x - 2)^4(x - 3)^3$, ent\~ao $V = W_1 \oplus W_2$ onde $\dim_\cp{K}W_1 = 4$ e $\dim_\cp{K}W_2 = 3$. Assim $T = T_1 \oplus T_2$, onde $T_1 = T : W_1 \to W_1$ e $T_2 = T : W_2 \to W_2$. Agora, $(T_1 - 2I_4)$ \'e nilpotente de {\'\i}ndice de nilpot\^encia $r \le 4$.
            Se $r = 1$, ent\~ao
            \[
                [T_1]_{\mathcal{B}_1} = \left[\begin{tabular}{cccc}
                     2 & 0 & 0 & 0\\
                     0 & 2 & 0 & 0\\
                     0 & 0 & 2 & 0 \\
                     0 & 0 & 0 & 2
                \end{tabular}
                \right].
            \]
            Se $r = 2$, ent\~ao
            \[
                [T_1]_{\mathcal{B}_1} = \left[\begin{tabular}{cc|cc}
                     2 & 0 &  & \\
                     1 & 2 &  & \\ \cline{1-4}
                     &  & 2 & 0 \\
                     &  & 1 & 2
                 \end{tabular}
                \right]\quad \mbox{ou}\quad [T_1]_{\mathcal{B}_1} = \begin{bmatrix}
                     2 & 0 &  & \\
                     1 & 2 &  & \\
                     &  & 2 &  \\
                     &  &  & 2
                 \end{bmatrix}.
            \]
            Se $r = 3$, ent\~ao
            \[
                [T_1]_{\mathcal{B}_1} = \left[\begin{tabular}{ccc|c}
                         2 & 0 & 0 & \\
                         1 & 2 & 0 & \\
                         0 & 1 & 2 & \\ \cline{1-4}
                         &  &  & 2
                    \end{tabular}
                \right].
            \]
            Se $r = 4$, ent\~ao
            \[
                [T_1]_{\mathcal{B}_1} = \left[\begin{tabular}{ccccc}
                         2 & 0 & 0 & 0\\
                         1 & 2 & 0 & 0\\
                         0 & 1 & 2 & 0\\
                         0 & 0 & 1 & 2
                    \end{tabular}
                \right].
            \]
            O operador $(T_2 - 3I_3)$ \'e nilpotente de {\'\i}ndice de nilpot\^encia $r \le 3$.

            Se $r = 1$, ent\~ao
            \[
                [T_2]_{\mathcal{B}_2} = \left[\begin{tabular}{ccc}
                     3 & 0 & 0\\
                     0 & 3 & 0\\
                     0 & 0 & 3
                \end{tabular}
                \right].
            \]
            Se $r = 2$, ent\~ao
            \[
                [T_1]_{\mathcal{B}_1} = \left[\begin{tabular}{cc|c}
                     3 & 0 & \\
                     1 & 3 & \\ \cline{1-3}
                     &  & 3
                 \end{tabular}
                \right].
            \]
            Se $r = 3$, ent\~ao
            \[
                [T_1]_{\mathcal{B}_1} = \left[\begin{tabular}{ccc}
                         3 & 0 & 0 \\
                         1 & 3 & 0 \\
                         0 & 1 & 3
                    \end{tabular}
                \right].
            \]
            Logo existem 15 poss{\'\i}veis formas de Jordan para $T$.
        \end{solucao}
        \item Seja $T : \complex^5 \to \complex^5$ tal que $p_T(x) = (x + 1)^3(x - 2)^2$.
        \begin{solucao}
            Neste caso como n\~ao conhecemos o polin\^omio minimal de $T$ precisamos analizar o que acontece para cada possibilidade do polin\^omio minimal. Assim:
            \begin{enumerate}
                \item Se $m_T(x) = (x + 1)^3(x - 2)^2$, ent\~ao existe uma base $\mathcal{B}$ de $\complex^5$ tal que
                \[
                    [T]_\mathcal{B} = \begin{bmatrix}
                    -1 & \phantom{-}0 & \phantom{-}0 & &\\
                    \phantom{-}1 & -1 & \phantom{-}0 & & \\
                    \phantom{-}0 & \phantom{-}1 & -1 & &\\
                    & & & 2 & 0\\
                    & & & 1 & 2
                \end{bmatrix}
                \]
                \item Se $m_T(x) = (x + 1)^3(x - 2)$, ent\~ao existe uma base $\mathcal{B}$ de $\complex^5$ tal que
                \[
                    [T]_\mathcal{B} = \begin{bmatrix}
                    -1 & \phantom{-}0 & \phantom{-}0 & &\\
                    \phantom{-}1 & -1 & \phantom{-}0 & & \\
                    \phantom{-}0 & \phantom{-}1 & -1 & &\\
                    & & & 2 & \\
                    & & & & 2
                \end{bmatrix}
                \]
                \item Se $m_T(x) = (x + 1)^2(x - 2)^2$, ent\~ao existe uma base $\mathcal{B}$ de $\complex^5$ tal que
                \[
                    [T]_\mathcal{B} = \begin{bmatrix}
                    -1 & \phantom{-}0 &\\
                    \phantom{-}1 & -1 &\\
                    & & -1 & & \\
                    & & & 2 & 0\\
                    & & & 1 & 2
                \end{bmatrix}
                \]
                \item Se $m_T(x) = (x + 1)^2(x - 2)$, ent\~ao existe uma base $\mathcal{B}$ de $\complex^5$ tal que
                \[
                    [T]_\mathcal{B} = \begin{bmatrix}
                    -1 & \phantom{-}0 & & &\\
                    \phantom{-}1 & -1 & & & \\
                    & & -1 & &\\
                    & & & 2 & \\
                    & & & & 2
                \end{bmatrix}
                \]
                \item Se $m_T(x) = (x + 1)(x - 2)^2$, ent\~ao existe uma base $\mathcal{B}$ de $\complex^5$ tal que
                \[
                    [T]_\mathcal{B} = \begin{bmatrix}
                    -1 & & & &\\
                    & -1 & & & \\
                    & & -1 & &\\
                    & & & 2 & 0\\
                    & & & 1 & 2
                \end{bmatrix}
                \]
                \item Se $m_T(x) = (x + 1)(x - 2)$, ent\~ao existe uma base $\mathcal{B}$ de $\complex^5$ tal que
                \[
                    [T]_\mathcal{B} = \begin{bmatrix}
                    -1 & & & &\\
                    & -1 & & & \\
                    & & -1 & &\\
                    & & & 2 &\\
                    & & & & 2
                \end{bmatrix}
                \]
            \end{enumerate}

        \end{solucao}
        \item Seja $T : \cp{K}^4 \to \cp{K}^4$ tal que em rela\c{c}\~ao \`a base can\^onica, $T$ seja representado pela matriz
        \[
            [T] = \begin{bmatrix}
                3 & -1 & \phantom{-}1 & -7\\
                9 & -3 & -7 & -1\\
                0 & \phantom{-}0 & \phantom{-}4 & -8\\
                0 & \phantom{-}0 & \phantom{-}2 & -4
            \end{bmatrix}.
        \]
        Encontre a forma de Jordan de $T$.
        \begin{solucao}
            Primeiramente temos que
            \[
                p_T(x) = x^4.
            \]
            C\'alculos simples mostram que $m_T(x) = x^2$. Assim a forma de Jordan de $T$ possui um bloco de Jordan de tamanho 2 associado a 0. Agora,
            \begin{align*}
                \begin{gmatrix}[b]
                      3 & -1 & \phantom{-}1 & -7\\
                    9 & -3 & -7 & -1\\
                    0 & \phantom{-}0 & \phantom{-}4 & -8\\
                    0 & \phantom{-}0 & \phantom{-}2 & -4
                    \rowops
                       \add[-3]{0}{1}
                       \add[-1/2]{2}{3}
                 \end{gmatrix}\leadsto\begin{gmatrix}[b]
                      3 & -1 & \phantom{-}1 & -7\\
                    0 & \phantom{-}0 & -10 & -20\\
                    0 & \phantom{-}0 & \phantom{-}4 & -8\\
                    0 & \phantom{-}0 & \phantom{-}0 & \phantom{-}0
                    \rowops
                       \add[-0,4]{1}{2}
                 \end{gmatrix}\leadsto\begin{bmatrix}
                      3 & -1 & \phantom{-}1 & -7\\
                    0 & \phantom{-}0 & -10 & -20\\
                    0 & \phantom{-}0 & \phantom{-}0 & \phantom{-}0\\
                    0 & \phantom{-}0 & \phantom{-}0 & \phantom{-}0
                 \end{bmatrix}
            \end{align*}
            e assim $\dim_\cp{K}\aut_T(0) = 2$, isto \'e, existem dois blocos de Jordan associados ao autovalor 0. Portanto, existe uma base $\mathcal{B}$ de $V$ tal que
            \[
                [T]_{B} = \left[\begin{tabular}{cc|cc}
                     0 & 0 &  & \\
                     1 & 0 &  & \\ \cline{1-4}
                     &  & 0 & 0 \\
                     &  & 1 & 0
                 \end{tabular}
                \right].
            \]
        \end{solucao}
        \item Seja $T : \real^6 \to \real^6$ o operador linear representado pela matriz
        \[
            [T] = \begin{bmatrix}
                -1 & \phantom{-}1 & -1 & -3 & -1 & \phantom{-}7\\
                \phantom{-}0 & -1 & \phantom{-}1 & \phantom{-}2 & \phantom{-}3 & \phantom{-}2\\
                \phantom{-}0 & \phantom{-}0 & -1 & \phantom{-}0 & -2 & \phantom{-}1\\
                \phantom{-}0 & \phantom{-}0 & \phantom{-}0 & -1 & \phantom{-}1 & -2\\
                \phantom{-}0 & \phantom{-}0 & \phantom{-}0 & \phantom{-}0 & -1 & \phantom{-}3\\
                \phantom{-}0 & \phantom{-}0 & \phantom{-}0 & \phantom{-}0 & \phantom{-}0 & -4
            \end{bmatrix}.
        \]
        Encontre a forma de Jordan de $T$.
        \begin{solucao}
            Inicialmente, como $[T]$ \'e uma matriz triangular, seu polin\^omio caracter{\'\i}stico ser\'a
            \[
                p_T(x) = (x + 1)^5(x + 4).
            \]
            Um c\'alculo simples, mostra que
            \[
                m_T(x) = (x + 1)^3(x + 4)
            \]
            logo a forma de Jordan de $T$ possui um bloco de Jordan de tamanho 3, associado \`a $-1$ e um bloco de Jordan de tamanho 1, associado \`a $4$. Desse modo a forma de Jordan de $T$ ser\'a
            \[
                \begin{bmatrix}
                    -1\\
                    \phantom{-}1 & -1\\
                    \phantom{-}0 & \phantom{-}1 & -1\\
                    & & & -1 & \\
                    & & & \phantom{-}1 & -1\\
                    & & & & & \phantom{-}4
                \end{bmatrix}\quad \mbox{ou} \quad\begin{bmatrix}
                    -1\\
                    \phantom{-}1 & -1\\
                    \phantom{-}0 & \phantom{-}1 & -1\\
                    & & & -1 & \\
                    & & & & -1\\
                    & & & & & \phantom{-}4
                \end{bmatrix}.
            \]
            Vamos determinar agora, $\aut_T(-1)$ para decidir qual ser\'a a forma de Jordan de $T$. Temos
            \begin{align*}
                \begin{gmatrix}[b]
                      0 & 1 & -1 & -3 & -1 & \phantom{-}7\\
                    0 & 0 & \phantom{-}1 & \phantom{-}2 & \phantom{-}3 & \phantom{-}2\\
                    0 & 0 & \phantom{-}0 & \phantom{-}0 & -2 & \phantom{-}1\\
                    0 & 0 & \phantom{-}0 & \phantom{-}0 & \phantom{-}1 & -2\\
                    0 & 0 & \phantom{-}0 & \phantom{-}0 & \phantom{-}0 & \phantom{-}3\\
                    0 & 0 & \phantom{-}0 & \phantom{-}0 & \phantom{-}0 & -3
                    \rowops
                       \mult5{\times -1/3}
                 \end{gmatrix}\leadsto\\\begin{gmatrix}[b]
                      0 & 1 & -1 & -3 & -1 & \phantom{-}7\\
                    0 & 0 & \phantom{-}1 & \phantom{-}2 & \phantom{-}3 & \phantom{-}2\\
                    0 & 0 & \phantom{-}0 & \phantom{-}0 & -2 & \phantom{-}1\\
                    0 & 0 & \phantom{-}0 & \phantom{-}0 & \phantom{-}1 & -2\\
                    0 & 0 & \phantom{-}0 & \phantom{-}0 & \phantom{-}0 & \phantom{-}3\\
                    0 & 0 & \phantom{-}0 & \phantom{-}0 & \phantom{-}0 & \phantom{-}1
                    \rowops
                       \add[-3]{5}{4}
                       \add[2]{5}{3}
                       \add[-1]{5}{2}
                       \add[-2]{5}{1}
                       \add[-7]{5}{0}
                 \end{gmatrix}\leadsto\\\begin{gmatrix}[b]
                      0 & 1 & -1 & -3 & -1 & 0\\
                    0 & 0 & \phantom{-}1 & \phantom{-}2 & \phantom{-}3 & 0\\
                    0 & 0 & \phantom{-}0 & \phantom{-}0 & -2 & 0\\
                    0 & 0 & \phantom{-}0 & \phantom{-}0 & \phantom{-}1 & 0\\
                    0 & 0 & \phantom{-}0 & \phantom{-}0 & \phantom{-}0 & 0\\
                    0 & 0 & \phantom{-}0 & \phantom{-}0 & \phantom{-}0 & 1
                    \rowops
                       \add[2]{3}{2}
                       \add[-3]{3}{1}
                       \add{3}{0}
                 \end{gmatrix}\leadsto\begin{bmatrix}
                      0 & 1 & -1 & -3 & 0 & 0\\
                    0 & 0 & \phantom{-}1 & \phantom{-}2 & 0 & 0\\
                    0 & 0 & \phantom{-}0 & \phantom{-}0 & 0 & 0\\
                    0 & 0 & \phantom{-}0 & \phantom{-}0 & 1 & 0\\
                    0 & 0 & \phantom{-}0 & \phantom{-}0 & 0 & 0\\
                    0 & 0 & \phantom{-}0 & \phantom{-}0 & 0 & 1
                 \end{bmatrix}
             \end{align*}
             donde obtemos que $\aut_T(-1) = \{(x_1, x_2, -2x_2, x_2, 0, 0) \mid x_1, x_2 \in \real\}$. Logo obtemos $\dim_\real \aut_T(-1) = 2$, com isso existem dois blocos de Jordan associados ao autovalor -1 e portanto a forma de Jordan de $T$ \'e
             \[
                 \begin{bmatrix}
                    -1\\
                    \phantom{-}1 & -1\\
                    \phantom{-}0 & \phantom{-}1 & -1\\
                    & & & -1 & \\
                    & & & \phantom{-}1 & -1\\
                    & & & & & \phantom{-}4
                \end{bmatrix}.
             \]
        \end{solucao}
    \end{enumerate}
\end{exemplo}

\begin{observacao}
    \begin{enumerate}[label={\arabic*})]
        \item  Se um operador linear $T : V \to V$, onde $\dim_\cp{K} V < \infty$, \'e tal que
            \[
                p_T(x) = (x - \lambda_1)^{m_1}\dots(x - \lambda_1)^{m_r}
            \]
        com $r \ge 1$ e $\lambda_i \ne \lambda_j$ se $i \ne j$. Se $T - \lambda_i Id$ \'e nilpotente de {\'\i}ndice de nilpot\^encia $\alpha_i$, ent\~ao existe um bloco de Jordan de tamanho $\alpha_i$, $i = 1$, \dots, $r$.
        \item A dimens\~ao de $\aut_T(\lambda_i)$ \'e igual ao n\'umero de blocos de Jordan $J_c(\lambda_i)$ associados ao autovalor $\lambda_i$ que aparece em $T$.
        \item A base que gera a forma de Jordan \'e chamada de \textbf{base de Jordan}.\index{Jordan!Base de}
    \end{enumerate}
\end{observacao}

% section polinomio_minimal (end)

\section{Como encontrar a base de Jordan} % (fold)
\label{sec:base_de_jordan}

Seja $T : \real^3 \to \real^3$ um operador linear tal que sua forma de Jordan seja
\[
    [T]_\mathcal{B} = \begin{bmatrix}
        3 & 0 & 0\\
        1 & 3 & 0\\
        0 & 1 & 3
    \end{bmatrix}.
\]
Assim se $\mathcal{B} = \{v_1, v_2, v_3\}$ \'e a base de Jordan, temos
\begin{align*}
    T(v_1) = 3v_1 + v_2\\
    T(v_2) = 3v_2 + v_3\\
    T(v_3) = 3v_3,
\end{align*}
isto \'e,
\begin{align*}
    (T- 3Id)(v_1) = v_2\\
    (T - 3Id)(v_2) = v_3\\
    (T - 3Id)(v_3) = 0.
\end{align*}

Assim para achar a base de Jordan precisamos de:
\begin{enumerate}[label={\roman*})]
    \item Achar todos os autovetores correspondentes a um certo autovalor, isto \'e, encontrar $\aut_T(\lambda)$.
    \item O n\'umero de autovetores L.I. associados ao autovalor $\lambda$ \'e igual ao n\'umero de blocos de Jordan associados \`a $\lambda$.
    \item Resolver a equa\c{c}\~ao $(T - \lambda Id)(u) = v_\lambda$ onde $v_\lambda$ \'e o autovetor associado ao autovalor $\lambda$ para cada autovetor diferente.
\end{enumerate}

Vamos aplicar este m\'etodo para o caso de matrizes $3 \times 3$. Neste caso temos tr\^es situa\c{c}\~oes para analisar:

\begin{enumerate}[label={\arabic*})]
    \item Existem 3 autovetores L.I.

    Por exemplo, para o operador representado pela matriz
    \[
        A = \begin{bmatrix}
            4 & 0 & 1\\
            2 & 3 & 2\\
            1 & 0 & 4
        \end{bmatrix}.
    \]
    Encontre a forma de Jordan de $A$.
    \begin{solucao}
        O polin\^omio caracter{\'\i}stico \'e
        \[
            p_A(x) = (x - 5)(x - 3)^2
        \]
        e da{\'\i}
        \begin{align*}
            \aut_A(5) = [e_1 = (1,2,1)]\\
            \aut_A(3) = [e_2 = (0,1,0); e_3 = (-1,0,1)].
        \end{align*}
        Assim a forma de Jordan de $A$ \'e
        \[
            \begin{bmatrix}
                5 & 0 & 0\\
                0 & 3 & 0\\
                0 & 0 & 3
            \end{bmatrix},
        \]
        onde $J_1(5) = [5]$ e $J_1(3) = [3]$.
    \end{solucao}
    \item Existem 2 autovetores L.I.

    Encontre a forma de Jorda e a base correspondente para o operador representado pela matriz
    \[
        A = \begin{bmatrix}
            1 & 1 & 1\\
            0 & 1 & 0\\
            0 & 0 & 1
        \end{bmatrix}.
    \]
    \begin{solucao}
        Temos
        \[
            p_A(x) = (x - 1)^3 \quad \mbox{e}\quad m_A(x) = (x - 1)^2.
        \]
        Assim existem dois bloco de Jordan de tamanho 2. Al\'em disso,
        \[
            \aut_A(1) = [e_1 = (1,0,0); e_2 = (0,1,-1)]
        \]
        e da{\'\i} temos duas poss{\'\i}veis formas de Jordan para $A$, a saber:
        \[
            B = \begin{bmatrix}
                1 & 0 &\\
                1 & 1 &\\
                & & 1
            \end{bmatrix},\quad
            C = \begin{bmatrix}
                 1 &  &\\
                 & 1 & 0\\
                 & 1 & 1
            \end{bmatrix}.
        \]
        Vamos procurar uma base que gere a matriz $B$. Se $\mathcal{B} = \{v_1, v_2, v_3\}$ \'e a base de Jordan que produz a matriz $B$, ent\~ao temos
        \begin{align*}
            A(v_1) = v_1 + v_2\\
            A(v_2) = v_2\\
            A(v_3) = v_3
        \end{align*}
        e assim
        \begin{align*}
            (A - I_3)(v_1) = v_2\\
            (A - I_3)(v_2) = 0\\
            (A - I_3)(v_3) = 0.
        \end{align*}
        Deste modo podemos escolher $v_2$ e $v_3$ como autovetores de $A$. Digamos que
        \[
            v_2 = (0,1,-1) \quad \mbox{e} \quad v_3 = (1,0,0).
        \]
        Precisamos encontrar $v_1 = (x,y,z)$ tal que
        \[
            (A - I_3)(v_1) = v_2.
        \]
        O sistema associado \'e
        \[
            \begin{bmatrix}
                y + z\\
                0\\
                0
            \end{bmatrix} = \begin{bmatrix}
                0\\
                1\\
                1
            \end{bmatrix}
        \]
        que \'e imposs{\'\i}vel. Assim vamos tomar
        \[
            v_2 = (1,0,0) \quad \mbox{e} \quad v_3 = (0,1,-1).
        \]
        Precisamos encontrar $v_1 = (x,y,z)$ tal que
        \[
            (A - I_3)(v_1) = v_2.
        \]
        Resolvendo o sistema associado encontramos
        \[
                v_1 = x(1,0,0) + y(0,1,-1) + (0,0,1).
        \]
        Tomando $x = y = 0$ o vetor procurado \'e $v_1 = (0,0,1)$. Assim na base $\mathcal{B} = \{v_1 = (0,0,1); v_2 = (1,0,0); v_3 = (0,1,-1)\}$ a forma de Jordan de $A$ ser\'a como dado por $B$.

        Se escolhermos a ordem $\mathcal{B}_1 = \{v_3, v_1, v_2\}$ para a base, obtemos
                \[
                    [A]_{\mathcal{B}_1} = \begin{bmatrix}
                             1 &  &\\
                             & 1 & 0\\
                             & 1 & 1
                        \end{bmatrix}.
                \]

    \end{solucao}
    \item Existe 1 autovalor L.I.

    Encontre a forma de Jordan e a base correspondente para o operador representado pela matriz
    \[
        A = \begin{bmatrix}
            -1 & -1 & \phantom{-}0\\
            \phantom{-}0 & -1 & -2\\
            \phantom{-}0 & \phantom{-}0 & -1
        \end{bmatrix}.
    \]
    \begin{solucao}
    Aqui
    \[
        p_A(x) = (x + 1)^3 = m_A(x).
    \]
    Deste modo existe um bloco de Jordan de tamanho 3. Al\'em disso
    \[
        \aut_A(-1) = [e_1 = (1,0,0)].
    \]
    Assim a \'unica possibilidade para a forma de Jordan \'e
    \[
        \begin{bmatrix}
            -1 & & \\
            \phantom{-}1 & -1 &\\
            \phantom{-}0 & \phantom{-}1 & -1
        \end{bmatrix}.
    \]
    Assim se $\mathcal{B} = \{v_1, v_2, v_3\}$ \'e a base de Jordan, ent\~ao
    \begin{align*}
        (A + I_3)(v_1) = v_2\\
        (A + I_3)(v_2) = v_3\\
        (A + I_3)(v_3) = 0.
    \end{align*}
    Tome $v_3 = (1,0,0)$ e seja $v_2 = (x,y,z)$. Temos
    \[
        \begin{bmatrix}
            0 & -1 & \phantom{-}0\\
            0 & \phantom{-}0 & -2\\
            0 & \phantom{-}0 & \phantom{-}0
        \end{bmatrix}\begin{bmatrix}
            x\\
            y\\
            z
        \end{bmatrix} = \begin{bmatrix}
            1\\
            0\\
            0
        \end{bmatrix}
    \]
    donde
    \[
        v_2 = x(1,0,0) + (0,-1,0).
    \]
    Podemos ent\~ao tomar $v_2 = (0,-1,0)$.

    Agora, precisamos encontrar $v_1 = (\alpha, \beta, \gamma)$ tal que $(A + I_3)(v_1) = v_2$. Do sistema
    \[
        \begin{bmatrix}
            0 & -1 & \phantom{-}0\\
            0 & \phantom{-}0 & -2\\
            0 & \phantom{-}0 & \phantom{-}0
        \end{bmatrix}\begin{bmatrix}
            \alpha\\
            \beta\\
            \gamma
        \end{bmatrix} = \begin{bmatrix}
            \phantom{-}0\\
            -1\\
            \phantom{-}0
        \end{bmatrix}
    \]
    encontramos
    \[
        v_1 = x(1,0,0) + (0,0,1/2).
    \]
    Tomando $v_1 = (0,0,1/2)$ a base de Jordan ser\'a
    \[
        \mathcal{B} = \{v_1 = (0,0,1/2); v_2 = (0,-1,0); v_3 = (1,0,0)\}.
    \]
    \end{solucao}
\end{enumerate}




% section base_de_jordan (end)
